\documentclass[10pt]{amsart}

\title[Commutative Character Sheaves]{Commutative Character Sheaves}
%with an application to geometric types for supercuspidal representations
\author{Clifton Cunningham}
\address{Department of Mathematics and Statistics, University of Calgary, 2500 University Drive Northwest, Calgary, AB, Canada, {T2N~1N4}.}
\email{cunning@math.ucalgary.ca}
\author{David Roe}
\address{Department of Mathematics, University of Pittsburgh, 301 Thackeray Hall, Pittsburgh, PA , United States, 15260.}
\email{roed.math@gmail.com}

\subjclass[2010]{14F05 (primary), 14L15 (secondary), 22E50 (tertiary)}
\keywords{character sheaves}

\usepackage{amssymb}
\usepackage{amsrefs}
% Fonts
\usepackage{mathrsfs}
% Enumitem
\usepackage{enumitem}
% Hyperrefs
\usepackage{hyperref}

\usepackage{tikz}
\usetikzlibrary{shapes,arrows,calc,matrix}
\usepackage{tikz-cd}

%%%%%%%%%%%%%%% THEOREM STYLES %%%%%%%%%%%%%%%
\theoremstyle{plain}
      \newtheorem{theorem}{Theorem}[section]
      \newtheorem*{theorem*}{Theorem}
      \newtheorem{proposition}[theorem]{Proposition}
      \newtheorem{lemma}[theorem]{Lemma}
      \newtheorem{corollary}[theorem]{Corollary}

      \theoremstyle{definition}
      \newtheorem{definition}[theorem]{Definition}

      %\theoremstyle{remark}
      \newtheorem{remark}[theorem]{Remark}
      \newtheorem{example}[theorem]{Example}
      
%%%%%%%%%%%%%%% RINGS AND GROUPS %%%%%%%%%%%%%%%
\newcommand{\FF}{{\mathbb{F}}}
\newcommand{\ZZ}{{\mathbb{Z}}}
\newcommand{\NN}{{\mathbb{N}}}
\newcommand{\CC}{{\mathbb{C}}}
\newcommand{\QQ}{{\mathbb{Q}}}
\newcommand{\RR}{{\mathbb{R}}}
\newcommand{\EE}{\mathbb{\bar Q}_\ell}
\newcommand{\OK}{\mathcal{O}_K}
\newcommand{\OL}{\mathcal{O}_L}
\newcommand{\OO}[1]{\mathcal{O}_{#1}}
\newcommand{\bFq}{\bar{k}}
\newcommand{\Fq}{k}
\newcommand{\Fqm}{k_m}
\newcommand{\EEx}{\EE^\times}
\newcommand{\ZEx}{\mathbb{\bar Z}_\ell^\times}
\newcommand{\Weil}[1]{\mathcal{W}_{#1}}
\newcommand{\m}{{\mathfrak{m}}}
%%%%%%%%%%%%%%% ALGEBRAIC GROUPS %%%%%%%%%%%%%%%
\newcommand{\Gm}[1]{\mathbb{G}_{\hskip-1pt\textbf{m},#1}}
\DeclareMathOperator{\GL}{GL}
\newcommand{\comp}{\Pi} % Component group
\newcommand{\G}{\textbf{G}}
%%%%%%%%%%%%%%% NAMED OPERATORS %%%%%%%%%%%%%%%
\DeclareMathOperator{\Gal}{Gal}
\newcommand{\Frob}[1]{\operatorname{Fr}_{#1}}
\DeclareMathOperator{\Aut}{Aut}
\DeclareMathOperator{\Hom}{Hom}
\DeclareMathOperator{\ord}{ord}
\DeclareMathOperator{\coker}{coker}
\DeclareMathOperator{\Gr}{Gr}
\DeclareMathOperator{\Irrep}{Irrep}
\DeclareMathOperator{\id}{id}
\DeclareMathOperator{\Ext}{Ext}
\DeclareMathOperator{\Hh}{H}
\DeclareMathOperator{\Res}{Res}
\DeclareMathOperator{\Nm}{Nm}
\DeclareMathOperator{\trace}{Tr}
\DeclareMathOperator{\obj}{obj}
\DeclareMathOperator{\mor}{mor}
\DeclareMathOperator{\Lang}{Lang}
\DeclareMathOperator{\image}{im}
\DeclareMathOperator{\Loc}{Loc}
\DeclareMathOperator{\Tot}{Tot}
\DeclareMathOperator{\Tor}{Tor}
\DeclareMathOperator{\SL}{SL}
\DeclareMathOperator{\PGL}{PGL}
\newcommand{\gal}[1]{{\operatorname{Gal}\hskip-1pt\left( {\bar #1}/#1 \right)}}
\newcommand{\Spec}[1]{{\operatorname{Spec}(#1)}}
\newcommand{\op}{_{\operatorname{op}}}
\newcommand{\der}{_{\operatorname{der}}}
\newcommand{\ab}{_{\operatorname{ab}}}

%%%%%%%%%%%% MISCELLANEOUS OPERATORS %%%%%%%%%%%%
\newcommand{\sheafHom}{{\mathscr{H}\hskip-4pt{\it o}\hskip-2pt{\it m}}}
\newcommand{\abs}[1]{{\vert #1 \vert}}
\newcommand{\ceq}{{\, :=\, }}
\newcommand{\tq}{{\ \vert\ }}
\newcommand{\iso}{{\ \cong\ }}
\newcommand{\trFrob}[1]{t_{#1}}
\newcommand{\TrFrob}[1]{\operatorname{Tr}_{#1}}
%% Limits
\newcommand{\invlim}[1]{\lim\limits_{\overleftarrow{#1}}}
\newcommand{\dirlim}[1]{\lim\limits_{\overrightarrow{#1}}}
\newcommand{\limit}[1]{\mathop{\textsc{lim}}\limits_{#1}}
\newcommand{\colimit}[1]{\mathop{\textsc{colim}}\limits_{#1}}
%% Fonts for quasicharacter sheaves
\newcommand{\cs}[1]{{\mathcal{#1}}}
\newcommand{\gcs}[1]{{\mathcal{\bar #1}}}
\newcommand{\dualgcs}[1]{\gcs{#1}^\dagger}
\newcommand{\dualcs}[1]{\cs{#1}^\dagger}
%% Categories
\newcommand{\CS}{{\mathcal{C\hskip-0.8pt S}}}
\newcommand{\CCS}{{\mathcal{C\hskip-.8pt C\hskip-0.8pt S}}}
\newcommand{\bCS}{{\CS_0}}
\newcommand{\catname}[1]{\normalfont{\textsf{#1}}}
\newcommand{\Sch}[1]{{\catname{Sch}_{/#1}}}
\newcommand{\QCS}{{\mathcal{QC\hskip-0.8pt S}}}
\newcommand{\CSiso}[1]{\CS(#1)_{/\text{iso}}}
\newcommand{\bCSiso}[1]{\bCS(#1)_{/\text{iso}}}
\newcommand{\QCSiso}[1]{\QCS(#1)_{/\text{iso}}}
\newcommand{\CCSiso}[1]{\CCS(#1)_{/\text{iso}}}
%% Labeled items
\makeatletter
\newcommand{\labitem}[2]{
\def\@itemlabel{\textbf{#1}}
\item
\def\@currentlabel{#1}\label{#2}}
\makeatother
%% Shorthand for bars
\renewcommand{\bf}{\bar{f}}
\newcommand{\bg}{{\bar{g}}}
\newcommand{\bm}{\bar{m}}
\newcommand{\bG}{\bar{G}}
\newcommand{\bH}{\bar{H}}
\newcommand{\brho}{{\bar\rho}}
\newcommand{\bx}{{\bar{x}}}
%% Spacing control
\newcommand{\tight}[3]{\hspace{-#1pt}{#2}\hspace{-#3pt}}
\newcommand{\GxG}{\text{$G \tight{1}{\times}{1} G$}}
\newcommand{\bGxG}{\text{$\bar{G} \tight{1}{\times}{1} \bar{G}$}}
\newcommand{\bfxf}{\text{$\bar{f} \tight{1}{\times}{1} \bar{f}$}}
\newcommand{\GxxG}{\text{$G \tight{1}{\times}{1} G$}}
\newcommand{\LxL}{\text{$\gcs{L} \tight{0}{\boxtimes}{0} \gcs{L}$}}


%% Hyphenation override
\hyphenation{quasi-character}
%% Number one equation of many
\newcommand\numberthis{\addtocounter{equation}{1}\tag{\theequation}}

%%%%%%%%%%%% BEGIN DOCUMENT %%%%%%%%%%%
\usepackage{todonotes}

\begin{document}

\begin{abstract}
We generalize the function-sheaf dictionary to arbitrary smooth group schemes over finite fields.  In previous work,
we considered the case of smooth commutative group schemes and found that the standard definition of
character sheaves produced a dictionary with a nontrivial kernel.  In this paper, we give a modification of
the category of character sheaves that remedies this defect, and is also extensible to non-commutative groups.
We use these \emph{commutative character sheaves} to geometrize types for supercuspidal representations of
$p$-adic groups. 
\end{abstract}

\maketitle

\tableofcontents

\section*{Introduction}

In \cite{cunningham-roe:13a}, we generalized the function-sheaf dictionary from connected commutative algebraic groups
over a finite field $k$ to smooth commutative group schemes $G$ over $k$.  We described a category $\CS(G)$ of
\emph{character sheaves} on $G$ and a surjective homomorphism $\TrFrob{G} : \CSiso{G} \to \Hom(G(k), \EEx)$.
In contrast to the connected case, $\TrFrob{G}$ may have nontrivial kernel; we gave an explicit description of
its kernel as $\Hh^2(\pi_0(\bG), \EEx)^{\Frob{}}$ \cite{cunningham-roe:13a}*{Thm. 3.6}.  

The first aim of this paper is to repair this defect in the function-sheaf dictionary by describing a full subcategory
$\CCS(G)$ of $\CS(G)$ so that $\TrFrob{G}$ restricts to an isomorphism $\CCSiso{G} \to \Hom(G(k), \EEx)$.
We refer to the resulting objects as \emph{commutative character sheaves}, since the passage from $\CS(G)$
to $\CCS(G)$ involves a condition that exchanges the inputs to the multiplication morphism on $G$
(see Definition \ref{def:CCScom}).  When $G$ is connected, all character sheaves on $G$ are commutative.
This category clarifies several questions about character sheaves on $G$: invisible character sheaves
\cite{cunningham-roe:13a}*{Def. 2.8} are precisely those that are not commutative, and the inverse map
$\Hom(G(k), \EEx) \to \CCSiso{G}$ provides a canonical splitting of $\TrFrob{G}$ \cite{cunningham-roe:13a}*{Rem. 3.7}.

Next, we broaden our scope further to encompass smooth group schemes $G$ that are not necessarily commutative.
In this paper, we do not consider Lusztig's geometrization of arbitrary characters of the non-abelian group $G(k)$, where $G$ is reductive,
using character sheaves \cite{lusztig:85a}*{Def.~2.10}, but focus on the case of $1$-dimensional characters
$\Hom(G(k), \EEx)$ assuming only that $G$ is smooth. The category $\CS(G)$ has a straightforward generalization to this case, but again
there are more linear character sheaves than there are characters, as pointed out by Kamgarpour \cite{kamgarpour:09a}*{(1.1)}.
We define a category $\CCS(G)$ and a forgetful functor to $\CS(G)$ so that $\TrFrob{G} : \CCSiso{G} \to \Hom(G\ab(k), \EEx)$
is an isomorphism.  In fact, pullback along the quotient $q : G \to G\ab$ defines an equivalence of categories
$\CCS(G\ab) \to \CCS(G)$.  Note that the functor $\CCS(G) \to \CS(G)$ is not essentially surjective, missing
the kinds of linear character sheaves highlighted by Kamgarpour.  
Since $\Hom(G\ab(k), \EEx)$ surjects
onto $\Hom(G(k), \EEx)$, with a kernel described by Galois cohomology, this paper shows that, for every linear character $\chi : G(k) \to \EEx$ there is a commutative character sheaf $\cs{L}$ on $G$ such that $\TrFrob{G}(\cs{L})\circ q = \chi$.

Finally, we use the category $\CCS(G)$ to give a geometric description of generic types for supercuspidal representations
of connected reductive linear algebraic groups over local fields.  In this way, we provide all of the ingredients
needed to parametrize supercuspidal representations of arbitrary depth in the same category:
perverse sheaves on group schemes over finite fields.

\todo{Add summary of sections}

\section{Recollections and definitions} \label{sec:defs}

Let $G$ be a smooth group scheme over a finite field $\Fq$; that is, let $G$ be a group scheme over $\Fq$
for which the structure morphism $G \to \Spec{\Fq}$ is smooth in the sense of \todo{include reference for this definition}.
This implies $G \to \Spec{\Fq}$ is locally of finite type, but not that it is of finite type.
We remark that the identity component $G^0$ of $G$ is of finite type over $\Fq$, while the component group scheme
$\pi_0(G)$ of $G$ is an \'etale group scheme over $\Fq$, and both are smooth over $\Fq$.

In this paper we use a common formalism for Weil sheaves, writing $\cs{L}$ for the pair $(\gcs{L},\phi)$ and writing $\alpha : \cs{L} \to \cs{L}'$ for a morphism $\alpha : \gcs{L} \to \gcs{L}$ such that 
\[
\begin{tikzcd}
\Frob{}^* \gcs{L} \arrow{d}[swap]{\phi} \arrow{r}{\Frob{}^*\alpha} &  \Frob{}^* \gcs{L} \arrow{d}{\phi'}\\
\gcs{L} \arrow{r}{\alpha} & \gcs{L}
\end{tikzcd}
\]
commutes, where $\cs{L}'$ refers to $(\gcs{L}',\phi')$.
While this simplifies notation considerably, it is, unfortunately, not consistent with our earlier paper.

We write $m : G \times G \to G$ for the multiplication morphism, and $G(k)^*$ for $\Hom(G(k), \EEx)$.
Define $\theta : G\times G \to G\times G$ by $\theta(g,h) = (h,g)$.

When $G$ is commutative, a \emph{character sheaf} on $G$ is a triple $(\gcs{L}, \mu, \phi)$,
where $\gcs{L}$ is a rank-one $\ell$-adic local system on $\bG$, 
$\mu : \bm^* \gcs{L} \to \LxL$ is an isomorphism
of sheaves on $\bG \times \bG$, 
and $\phi : \Frob{G}^* \gcs{L} \to \gcs{L}$ is an isomorphism of sheaves on $\bG$;
the triple $(\gcs{L}, \mu, \phi)$ is required to satisfy certain conditions \cite{cunningham-roe:13a}*{Def. 1.1}.
Write $\CS(G)$ for the category of character sheaves on $G$.

Even when $G$ is not commutative, the category $\CS(G)$, defined as in \cite{cunningham-roe:13a}*{Def. 1.1},
still makes sense.  In order to distinguish the resulting objects from
the character sheaves of Lusztig, we will refer to the former as \emph{linear character sheaves}
(to evoke the one-dimensional character sheaves of \cite{kamgarpour:09a}).

\section{Commutative character sheaves on commutative groups}\label{sec:comcom}

We consider first the case that $G$ is commutative, which will be used for general smooth $G$.
Let $\cs{L}$ be a character sheaf on $G$.  Since $m = m \circ \theta$ in this case,
there is a canonical isomorphism $\xi : m^* \cs{L} \to \theta^* m^* \cs{L}$.
There is also an isomorphism $\vartheta : \cs{L}\boxtimes\cs{L} \to \theta^*(\cs{L}\boxtimes\cs{L})$
given on stalks by the canonical map $\gcs{L}_{g} \otimes \gcs{L}_{h} \to \gcs{L}_{h} \otimes \gcs{L}_{g}$.

\begin{definition}\label{def:CCScom}
A character sheaf $(\cs{L}, \mu)$ on a smooth commutative group scheme $G$ is \emph{commutative}
if the following diagram of Weil sheaves on $G \times G$ commutes.
  \[
  \begin{tikzcd}[row sep=30]
   m^*\cs{L} \arrow{d}[swap]{\xi}{m= m\circ\theta} \arrow{r}{\mu} & \cs{L}\boxtimes\cs{L} \arrow{d}{\vartheta}\\
   \theta^*(m^*\cs{L}) \arrow{r}{{\theta}^*\mu} &  \theta^*(\cs{L}\boxtimes\cs{L})
  \end{tikzcd}
  \]
We write $\CCS(G)$ for the full subcategory of $\CS(G)$ consisting of commutative character sheaves.
 \end{definition}

\subsection{Eliminating Invisible Character Sheaves}\label{ssec:defect}

In \cite{cunningham-roe:13a}*{Thm. 3.6}, we showed that $\TrFrob{G} : \CSiso{G} \to G(k)^*$ is surjective and
explicitly computed its kernel.  In this section, we show that the corresponding map
$\TrFrob{G} : \CCSiso{G} \to G(k)^*$ for commutative character sheaves is an isomorphism.
We begin by reinterpreting Definition \ref{def:CCScom} in terms of cocycles.

Let $G$ be a commutative \'etale group scheme over $k$. For a character sheaf $\cs{L}$ on $G$, recall
\cite{cunningham-roe:13a}*{\S 2.3} that $S_G : \CSiso{G} \to \Hh^2(E_G^\bullet)$ is an isomorphism mapping
$[\cs{L}]$ to $[\alpha \oplus \beta]$, where $E_G^\bullet$ is the total space of the zeroth page
of the Hochschild-Serre spectral sequence, $\alpha \in C^0(\Weil{}, C^2(\bG, \EEx))$ is obtained from $\mu$ and
$\beta \in C^1(\Weil{}, C^1(\bG, \EEx))$ is obtained from $\phi$.

If $M$ is a $\bG$-module, then we say a class in the Galois cohomology group $\Hh^2(\bG, M)$
is \emph{symmetric} if it is represented by a symmetric $2$-cocycle.  Since every coboundary is symmetric,
either all cocycles in a given class will be symmetric or none will.  The connection between
commutative character sheaves and symmetric classes is given in the following lemma.
\todo{Define symmetric for element of $\Hh^2(E_G^\bullet)$.}

\begin{lemma} \label{lem:symccslink}
Suppose $G$ is a smooth commutative group scheme, and let $\cs{L}$ be a character sheaf on $G$.
Then $\cs{L}$ is commutative if and only if $S_G(\cs{L})$ is symmetric.
\end{lemma}
\begin{proof}
The symmetry of $S_G(\cs{L})$ is a direct consequence of the commutativity of the diagram in Definition \ref{def:CCScom}
after choosing bases for each stalk.
\end{proof}

\begin{lemma} \label{lem:symtriv}
Let $\bG$ be a commutative group.  Then the only symmetric class in $\Hh^2(\bG, \EEx)$ is the trivial class.
\end{lemma}

\begin{proof}
By the universal coefficient theorem,
\[
0 \to \Ext^1_\ZZ(\Hh_{n-1}(\bG, \ZZ), \EEx) \to \Hh^n(\bG, \EEx) \to \Hom(\Hh_n(\bG, \ZZ), \EEx) \to 0
\]
is exact for all $n > 0$.  When $n = 2$, using the fact that $\bG$ is commutative, we have that $\Hh_1(\bG, \ZZ) \cong \bG$
and that $\Hh_2(\bG, \ZZ) \cong \wedge^2 \bG$. We get
\[
0 \to \Ext^1_\ZZ(\bG, \EEx) \to \Hh^2(\bG, \EEx) \to \Hom(\wedge^2 \bG, \EEx) \to 0.
\]
The map $\Hh^2(\bG, \EEx) \to \Hom(\wedge^2 \bG, \EEx)$ maps a $2$-cocycle $f$ to the alternating function
\[
(x,y) \mapsto \frac{f(x,y)}{f(y,x)}.
\]
Thus the cohomology classes represented by symmetric cocycles are precisely those in the image of $\Ext^1_\ZZ(\bG, \EEx)$.
But $\Ext^1_\ZZ(-, \EEx)$ vanishes because $\EEx$ is divisible.
\end{proof}



\begin{lemma} \label{lem:conncomm}
If $G$ is a connected commutative algebraic group over $\Fq$ then every character sheaf on $G$ is commutative.
\end{lemma}

\begin{proof}
We can use \'etale descent to see that pullback by the Lang isogeny defines an equivalence
of categories between local systems on $G$ and $G(\Fq)$-equivariant local systems on $G$.  Thus every character
sheaf on $G$ arises through the Lang isogeny; since the covering group is $G$ itself, which is commutative,
every character sheaf on $G$ is commutative.
\end{proof}

\begin{theorem} \label{thm:trfrobiso}
If $G$ is a smooth commutative group scheme over $\Fq$ then $\TrFrob{G} : \CCSiso{G} \to G(\Fq)^*$ is an isomorphism.
\end{theorem}

\begin{proof}
Suppose first that $G$ is \'etale.  Consider the isomorphism of short exact sequences
\[
\begin{tikzcd}
 0 \arrow{r} & \ker \TrFrob{G} \arrow{d} \arrow{r} & \CSiso{G}\arrow{d}{S_G} \arrow{r}{\TrFrob{G}} \arrow{r} & G(\Fq)^* \arrow{d} \arrow{r} & 0\\
  0 \arrow{r} & \Hh^0(\Weil{},\Hh^2(\bG,\EEx)) \arrow{r} & \Hh^2(E^\bullet_G) \arrow{r} & \Hh^1(\Weil{},\Hh^1(\bG,\EEx)) \arrow{r} & 0
 \end{tikzcd}
 \]
from \cite{cunningham-roe:13a}*{Prop. 2.7}.

Suppose that $\cs{L}$ is a commutative character sheaf
with $\trFrob{\cs{L}} = 1$.  Then $S_G([\cs{L}])$ is in the image of $\Hh^2(\bG, \EEx)^\Weil{}$, so is cohomologous to
$[\alpha', 0]$.  Since $\alpha$ is symmetric and coboundaries are symmetric, $\alpha'$ is symmetric as well.
So by Lemma \ref{lem:symtriv}, $\alpha'$ is cohomologically trivial, and thus $[\cs{L}]$ is trivial as well.

To see that $\TrFrob{G}$ is still surjective on $\CCSiso{G}$, note that the character sheaf constructed in the proof of
\cite{cunningham-roe:13a}*{Prop. 2.6} has trivial $\alpha$, and is thus commutative.

For general smooth commutative group schemes, we use Lemma \ref{lem:conncomm} and the snake lemma, as in the proof of
\cite{cunningham-roe:13a}*{Thm. 3.6}
\end{proof}

\begin{remark}
Since $\Hh^0(\Weil{},\Hh^2(\bG,\EEx))$ is not necessarily trivial \cite{cunningham-roe:13a}*{Ex. 2.10}, the functor
$\CCS(G) \to \CS(G)$ is not an equivalence of categories in general.
\end{remark}

\section{Commutative character sheaves on non-commutative groups}\label{sec:noncom}

We now consider the case of a smooth group scheme without the commutativity assumption.  We start
by relating character sheaves on $G$ to character sheaves on its abelianization.

If $\chi \in G(k)^*$ is a character, it must vanish on the derived subgroup $G\der(k)$.
Kamgarpour gives an example \cite{kamgarpour:09a}*{(1.1)} of a character sheaf
that does not vanish on $G\der$, defined by the extension
\[
1 \to \mu_n \to \SL_n \to \PGL_n \to 1.
\]
In order to obtain a relationship between character sheaves on $G$ and characters of $G(k)$,
he opts to give a different definition of commutator.  Since we already need to adapt our
notion of character sheaf, even in the commutative case, we instead add restrictions
to the definition of commutative character sheaf and leave the definition of $G\der$ unchanged.

\subsection{Definition}\label{ssec:noncomdef}

In order to get character sheaves that correspond to characters in $G(k)^*$, we must discard those
character sheaves whose restriction to the derived subgroup is nontrivial.
The following proposition will be proven in Section \ref{ssec:proof1}.
\begin{proposition} \label{prop:Gder_triv}
Suppose $G$ is a smooth group scheme and $\cs{L} \in \CS(G)$ is a linear character sheaf on $G$.
Then the restriction of $\cs{L}$ to $G\der$ is trivial if and only if $\cs{L} \cong q^*(\cs{L}\ab)$ for some
character sheaf $\cs{L}\ab$ on $G\ab$.
\end{proposition}

We may now define commutative character sheaves on $G$.
%
Suppose $(\cs{L},\mu)$ is a linear character sheaf on $G$ such that its pull-back along
$j: G\der \hookrightarrow G$ is trivial;
let $\beta : \cs{L}\vert_{G\der} \to (\EE)_{G\der}$ be an isomorphism in $\CS(G\der)$.
Let $\CS'(G)$ be the category of such triples, $(\cs{L},\mu,\beta)$, in which a morphism $(\cs{L},\mu,\beta)\to (\cs{L}',\mu',\beta')$ is a morphism $\alpha : (\cs{L},\mu)\to (\cs{L}',\mu')$ in $\CS(G)$ such that $\beta = \beta' \circ \alpha\vert_{G\der}$.
%

Every $\beta : \cs{L}\vert_{G\der} \to (\EE)_{G\der}$ determines an isomorphism $\gamma : m^*\cs{L} \to \theta^*m^*\cs{L}$ as follows.
Let $c : G\times G\to G\der$ be the commutator map, defined by $c(x,y)= xyx^{-1}y^{-1}$.
Then $c$ is a smooth morphism of $\Fq$ schemes.
Set $m' = i \circ m \circ \theta$; then $j\circ c = m \circ (m \times m')$.
Then, $\beta : \cs{L}\vert_{G\der} \to (\EE)_{G\der}$ determines the isomorphism $\gamma' : m^*\cs{L} \otimes \theta^* m^* i^*\cs{L} \to (\EE)_{G\times G}$ by the diagram of isomorphisms, below.
\begin{equation}
\begin{tikzcd}
\arrow[equal]{d} c^* (\cs{L}\vert_{G\der}) \arrow{r}{c^*(\beta)} 
	&  c^*((\EE)_{G\der}) \arrow[equal]{d} \\
\arrow{d}[swap]{j\circ c = m \circ (m \times m')} c^* j^* \cs{L} 
	&   (\EE)_{G\times G} \\
(m \times m')^* m^* \cs{L} \arrow{d}[swap]{(m \times m')^*(\mu)} 
	&  m^*\cs{L} \otimes \theta^* m^* i^* \cs{L}  \arrow[dashed]{u}[swap]{\gamma'} \\
(m \times m')^* (\cs{L} \boxtimes \cs{L}) \arrow[equal]{r} 
	& m^*\cs{L} \otimes (m')^*\cs{L} \arrow{u}[swap]{m' = i\circ m\circ \theta} 
\end{tikzcd}
\end{equation}
In the diagram above, the arrows labeled with equations come from canonical isomorphisms of functors on Weil sheaves derived from the equations; so, for example, the middle left isomorphism comes from $(m\times m')^* m^* \iso c^* j^*$ since $j\circ c = m \circ (m \times m')$.
Using the monoidal structure of the category of Weil local systems on $G\times G$, the isomorphism $\gamma' : m^*\cs{L} \otimes \theta^* m^* i^*\cs{L} \to (\EE)_{G\times G}$ defines an isomorphism
\[
m^*\cs{L} \to (\theta^* m^* i^*\cs{L})^\vee.
\]
Since $(\theta^* m^* i^*\cs{L})^\vee \iso \theta^* m^* i^* (\cs{L}^\vee)$ canonically, and since $\cs{L}^\vee \iso i^*\cs{L}$, canonically, this defines the promised isomorphism
\[
\begin{tikzcd}
\gamma : m^*\cs{L} \arrow{r} & \theta^* m^* \cs{L}.
\end{tikzcd}
\]

\begin{definition}\label{def:CCS}
The category $\CCS(G)$ of commutative character sheaves on $G$ is the full subcategory of $\CS'(G)$ consisting of triples $(\cs{L},\mu,\beta)$ such that the following diagram of Weil sheaves on $G \times G$ commutes,
  \[
  \begin{tikzcd}[row sep=30]
   m^*\cs{L} \arrow{d}[swap]{\gamma} \arrow{r}{\mu} & \cs{L}\boxtimes\cs{L} \arrow{d}{\vartheta}\\
   \theta^*(m^*\cs{L}) \arrow{r}{{\theta}^*\mu} &  \theta^*(\cs{L}\boxtimes\cs{L})
  \end{tikzcd}
  \]
where $\gamma : m^*\cs{L} \to \theta^* m^* \cs{L}$ is the isomorphism built from $\beta : \cs{L}\vert_{G\der} \to (\EE)_{G\der}$, above.
\end{definition}

Suppose $G$ is commutative, so $G\der = 1$. 
Suppose $(\cs{L},\mu,\beta)$ is an object in $\CS'(G)$.
Then $\beta : \cs{L}_1\to \EE$ is an isomorphism in $\CS(1)$, and unique by \cite{cunningham-roe:13a}*{Thm 3.9}.
Tracing through the construction of $\gamma : m^*\cs{L} \to \theta^*m^*\cs{L}$ from $\beta : \cs{L}_1\to \EE$, we find \todo{I'm having some trouble seeing this.} that $\gamma : m^*\cs{L} \to \theta^*m^*\cs{L}$ is the canonical isomorphism coming from the equation $m = m \circ \theta$. 
Thus, when $G$ is commutative, Definition~\ref{def:CCS} recovers Definition~\ref{def:CCScom}.
The next result generalizes this observation.


\begin{theorem}\label{thm:Gab}
Pull-back along the abelianization $q : G \to G\ab$
defines an equivalence of categories
\[
\CCS(G\ab) \to \CCS(G).
\]
\end{theorem}
%
Sections~\ref{ssec:equivariant1}, \ref{ssec:equivariant2}, \ref{ssec:descent} and \ref{ssec:closed} set up machinery needed to prove Theorem~\ref{thm:Gab}.
The precise definition of the functor $\CCS(G\ab) \to \CCS(G)$ will be given with the proof of Theorem~\ref{thm:Gab}, in Section~\ref{ssec:proof2}.



\subsection{Objects and maps in commutative character sheaves}

Theorem~\ref{thm:Gab} shows that $\CCS(G)$ is a categorical solution to the problem that linear character sheaves on $G$ need not be trivial on $G\der$, as discussed at the beginning of Section~\ref{sec:noncom}; at the same time, it is the categorical solution to the problem, discussed in Section~\ref{ssec:defect}, that the dictionary between linear character sheaves on $G\ab$ and characters of $G\ab$, need not be perfect.
%
But a description of $\CCS(G)$ requires, at the very least, a description of isomorphism classes of objects, and maps. 
%

\begin{proposition}
The trace of Frobenius $\TrFrob{} : \CCSiso{G}\to G(\Fq)^*$ fits into the following diagram,
\[
\begin{tikzcd}
\ & & \CCSiso{G\ab} \arrow{d}{\TrFrob{}}[swap]{\iso} \arrow{r}{\iso} & \CCSiso{G} \arrow{d}{\TrFrob{}}& \\
1 \arrow{r} & \Delta^* \arrow{r} & G\ab(k)^* \arrow{r} & G(k)^* \arrow{r} & 1,
\end{tikzcd}
\]
where $\Delta$ denotes the image of the connecting homomorphism $G\ab(\Fq) \to \Hh^1(\Fq, G\der)$.
In particular, isomorphism classes of objects in $\CCS(G)$ are canonically identified with the group $G\ab(\Fq)^*$.
\end{proposition}

\begin{proof}
Let $\Delta$ be the image of the boundary homomorphism $\delta$ in the sequence
\[
1 \to G\der(k) \to G(k) \to G\ab(k) \xrightarrow{\delta} \Hh^1(k, G\der),
\]
yielding a short exact sequence
\[
1 \to G(k) / G\der(k) \to G\ab(k) \to \Delta \to 1.
\]
Applying $\Hom(-, \EEx)$ and using the fact that every homomorphism $G(k) \to \EEx$ vanishes on $G\der(k)$, we get
\[
1 \to \Delta^* \to G\ab(k)^* \to G(k)^* \to 1.
\]
By Theorem~\ref{thm:Gab}, the trace of Frobenius map  . . . .\todo{Finish this.}
\end{proof}

\begin{remark}
Note that when $\Hh^1(k,G\der) = 0$, as is the case when $G\der(\bFq)$ ``is simply connected,'' then $\CCSiso{G} \iso G(\Fq)^*$, so
we succeed in geometrizing characters of $G(k)$ on the nose.
\end{remark}

\begin{proposition}
Every map in $\CCS(G)$ is either trivial or an isomorphism. 
For every commutative character sheaf on $G$, its automorphism group is
$
((\pi_0(G\ab))^*)^{\Frob{}}.
$
\end{proposition}

\begin{proof}
By Theorem~\ref{thm:Gab}, $\Aut_{\CCS(G)}(\cs{L},\mu,\beta) = \Aut_{\CCS(G\ab)}(\cs{L}\ab,\mu\ab)$. 
By \cite{cunningham-roe:13a}*{Thm 3.9},  $\Aut_{\CCS(G\ab)}(\cs{L}\ab,\mu\ab) = ((\pi_0(G\ab))^*)^{\Frob{}}$.
\end{proof}

%\todo{Masound's definition is designed to match the definition coming from central extensions, while we have one that matches Ext1 in Ab, in the connected case. So, I think the modification is easy: add mop to diagram 2.4 in Masoud's paper. \\ -CC}


%\section{Descent to the Abelianization} \label{sec:descent}

%In this section we use descent along the abelianization $q : G \to G\ab$ to prove Propositions~1.2 and 1.4, and Theorem~\ref{thm:Gab}.
%This approach requires introducing equivariant linear character sheaves, as defined in Section~[].

\subsection{Proofs for results from Section \ref{ssec:noncomdef}} \label{ssec:proofs}

\subsubsection{Equivariant Weil local systems}\label{ssec:equivariant1}

Let $G$ be a smooth group scheme over $k$, as above.
Let $H$ be a group scheme and write $n : H \times H \to H$ for the multiplication morphism.
Let $a : H \times G\to G$ be a group action compatible with the group structure on $G$ and write $p : H\times G \to G$ for projection. 
Consider the morphisms
\[
\begin{tikzcd}
H \times H \times G 
\arrow[shift left=4]{r}{b_1, b_2, b_3} 
\arrow{r}{}
\arrow[shift right=4]{r}{} 
& H \times G 
\arrow[shift left=2]{r}{a}
\arrow[shift right=2]{r}[swap]{p}
 & G 
\end{tikzcd}
\]
defined by 
\begin{align*}
b_1(h_1,h_2,g) &= (h_1h_2,g) \\
b_2(h_1,h_2,g) &= (h_1,h_2g) \\
b_3(h_1,h_2,g) &= (h_2,g).
\end{align*}
Note that
\begin{align*}
a\circ b_1 &= a\circ b_2 \\
a\circ b_3 &= p\circ b_2 \numberthis \label{eqn:bap}\\
p\circ b_1 &= p\circ b_3.
\end{align*}
Define $s : G\to H\times G$ by $s(g) = (1,g)$.
An $H$-equivariant Weil local sytem on $G$ is a Weil local system $\cs{L}$ on $G$ together with an isomorphism  
\[
\nu : a^*\cs{L} \to p^*\cs{L}
\] 
of Weil local systems on $H\times G$ such that 
\begin{equation}\label{E1}
s^*(\nu) = \id_{\cs{L}}
\end{equation}
and the following diagram of isomorphisms of local systems on $H\times H\times G$ commutes.
\begin{equation}\label{E2}
\begin{tikzcd}
\ &  \arrow{dl}{a\circ b_1 = a\circ b_2}  b_2^*  a^*\, \mathcal{L} \arrow{rr}{b_2^*(\nu)} && b_2^* p^*\, \mathcal{L} \arrow{dr}[swap]{p\circ b_2 = a\circ b_3} & \\
b_1^* a^*\, \mathcal{L} \arrow{dr}{b_1^*(\nu)} &&&&   \arrow{dl}[swap]{b_3^*(\nu)} b_3^* a^*\, \mathcal{L} \\
& b_1^* p^*\, \mathcal{L}  && \arrow{ll}[swap]{p\circ b_3 = p\circ b_1} b_3^* p^*\, \mathcal{L} & 
\end{tikzcd}
\end{equation}
%In the diagram above, the arrows labeled with equations come from canonical isomorphisms of functors on Weil sheaves derived from equations \eqref{eqn:bap}; so, for example, the top left isomorphism comes from $b_1^* a^* \iso b_2^* a^*$ since $a\circ b_1 = a\circ b_2$.
%On stalks, this condition is ...
%\[
%\begin{tikzcd}
%\ &  \cs{L}_{h_1\cdot(h_2\cdot g)} \arrow{r}{\nu_{h_1,h_2\cdot g}} & \cs{L}_{h_2\cdot g} & \\
%\cs{L}_{(h_1 h_2) \cdot g} \arrow{dr}[swap]{\nu_{(h_1h_2),g}} \arrow[equal]{ur} &&& \arrow[equal]{ul}  \arrow{dl}{\nu_{h_2,g}} \cs{L}_{h_2\cdot g} \\
%& \cs{L}_g \arrow[equal]{r} & \cs{L}_g & 
%\end{tikzcd}
%\]
Morphisms of $H$-equivariant Weil local systems $(\cs{L},\nu)\to (\cs{L}',\nu')$ are morphisms of Weil local systems $\alpha: \cs{L}\to \cs{L}'$ for which the diagram
\begin{equation}\label{E3}
\begin{tikzcd}
\arrow{d}[swap]{\nu} a^*\cs{L} \arrow{r}{a^*(\alpha)} & a^*\cs{L}' \arrow{d}{\nu'} \\
p^*\cs{L} \arrow{r}{p^*(\alpha)} & p^*\cs{L}'
\end{tikzcd}
\end{equation}
commutes.
This defines $\Loc_H(G)$, the category of $H$-equivariant Weil local systems on $G$.

\subsubsection{Equivariant linear character sheaves}\label{ssec:equivariant2}

An $H$-equivariant linear character sheaf on $G$ is a triple $(\cs{L},\mu, \nu)$ where $(\cs{L},\mu)$ is a linear character sheaf and $(\cs{L},\nu)$ is an $H$-equivariant local system.  We require that $\mu$ be compatible with $\nu$ in the following sense.
We define morphisms:
\begin{align*}
c_0 : H\times G \times G &\to H\times G \times H \times G \\
(h,g_1,g_2) &\mapsto (h,g_1,h,g_2); \\
c_1 : H \times G\times G &\to G\times G \\
(h, g_1, g_2) &\mapsto (hg_1,hg_2); \\
c_2 : H \times G\times G &\to H\times G \\
(h,g_1,g_2) &\mapsto (h, g_1g_2); \\
c_3 : H \times G\times G &\to G\times G \\
(h, g_1, g_2) &\mapsto (g_1, g_2).
\end{align*}
We require that the following diagram of Weil local systems on $H \times G\times G$ commutes:
\begin{equation}\label{ECS1}
\begin{tikzcd}[column sep=30]
\ & \arrow{dl}{c_2\circ a = m\circ c_1}  c_2^* a^* \cs{L} \arrow{r}{c_2^*(\nu)} & c_2^* p^* \cs{L} \arrow{dr}[swap]{p\circ c_2 = m\circ c_3}  &  \\
\arrow{d}[swap]{c_1^*(\mu)} c_1^* m^* \cs{L} &&& c_3^* m^*\cs{L} \arrow{d}{c_3^*(\mu)} \\
c_1^*(\cs{L}\boxtimes \cs{L}) \arrow{dr}{p_i\circ c_1 = a\circ p_i\circ c_0} &&& \arrow{dl}[swap]{p\circ p_i\circ c_0 = p_i \circ c_3} c_3^*(\cs{L}\boxtimes \cs{L}) \\
 & c_0^*(a^*\cs{L}\boxtimes a^*\cs{L}) \arrow{r}{c_0^*(\nu\boxtimes \nu)} & c_0^*(p^*\cs{L} \boxtimes p^*\cs{L}) & 
\end{tikzcd}
\end{equation}

%On stalks, this condition is
%\[
%\begin{tikzcd}
%\arrow{d}[swap]{\mu_{(h\cdot g_1, h\cdot g_2)}} \cs{L}_{(h\cdot g_1)(h\cdot g_2)} & \arrow[equal]{l}  \cs{L}_{h\cdot(g_1 g_2)} \arrow{r}{\nu_{(h, g_2 g_2)}} & \cs{L}_{g_1 g_2} \arrow[equal]{r} & \cs{L}_{g_1 g_2} \arrow{d}{\mu_{(g_1,g_2)}} \\
%\cs{L}_{h\cdot g_1}\otimes \cs{L}_{h\cdot g_2} \arrow[equal]{r} & \cs{L}_{h\cdot g_1}\otimes \cs{L}_{h\cdot g_2} \arrow{r}{\nu_{(h,g_1)}\otimes \nu_{(h,g_2)}} & \cs{L}_{g_1}\otimes \cs{L}_{g_2} & \arrow[equal]{l} \cs{L}_{g_1} \otimes \cs{L}_{g_2},
%\end{tikzcd}
%\]
%for all $(h,g_1,g_2)\in H\times G\times G$.

A morphism of $H$-equivariant linear character sheaves $(\cs{L},\mu,\nu) \to (\cs{L}',\mu',\nu')$ is a morphism of $H$-equivariant Weil sheaves $\alpha : \cs{L}\to \cs{L}'$ which is also a morphism of linear character sheaves.
%so
%\begin{equation}\label{CS4}
%\begin{tikzcd}
%\arrow{d}[swap]{\mu} m^*\cs{L} \arrow{r}{m^*(\alpha)} & m^*\cs{L}' \arrow{d}{\mu'} \\
%\cs{L}\boxtimes \cs{L} \arrow{r}{\alpha\boxtimes\alpha} & \cs{L}'\boxtimes\cs{L}' 
%\end{tikzcd}
%\end{equation}
%commutes; see also \cite{cunningham-roe:13a}, CS.4].
%
Let $\CS_H(G)$ be the category of $H$-equivariant linear character sheaves on $G$.

%Is $\CS_H(G)$ a rigid monoidal category?

%If $(\cs{L},\mu,\nu)$ and $(\cs{L},\mu,\nu')$ in $\CS_H(G)$, then $\nu = \nu'$?



\begin{lemma}\label{lem:HH}
If $(\cs{L},\mu,\nu)$ is an $H$-equivariant linear character sheaf on $G$ then $\mu : m^*\cs{L} \to \cs{L}\boxtimes \cs{L}$ and $\vartheta : \cs{L}\boxtimes \cs{L} \to \theta^*(\cs{L}\boxtimes \cs{L})$ are morphisms of $H\times H$-equivariant Weil local systems on $G\times G$. 
\end{lemma}

\begin{proof} 
Define
\begin{align*}
d : H\times H\times G\times G &\to H\times G\times H\times G \\
(h_1,h_2,g_1,g_2) &\mapsto (h_1, g_1, h_2, g_2)\\
a_2 : H\times G\times H\times G &\to G\times G \\
(h_1,g_1,h_2,g_2) &\mapsto (h_1g_1, h_2g_2) \\
p_2 : H\times G\times H\times G &\to G\times G \\
(h_1,g_1,h_2,g_2) &\mapsto ( g_1,g_2).
\end{align*}
The following diagram defines the isomorphisms needed to see that both $m^*\cs{L}$ and $\cs{L}\boxtimes\cs{L}$ are $H\times H$-equivariant Weil local systems.
\[
\begin{tikzcd}[column sep=40]
\arrow{d}[swap]{a_2(\mu)} a_2^* (m^*\cs{L}) \arrow[dashed]{r} 
	& \arrow{d}{p_2^*(\mu)} p_2^*(m^*\cs{L})\\
\arrow{d}[swap]{a_2= (a\times a)\circ d} a_2^*(\cs{L}\boxtimes\cs{L}) \arrow[dashed]{r} 
	& \arrow{d}{p_2= (p\times p)\circ d} p_2^*(\cs{L}\boxtimes\cs{L}) \\
%d^* (a\times a)^* (\cs{L}\boxtimes\cs{L}) & d^* (p\times p)^* (\cs{L}\boxtimes\cs{L}) \\
d^* (a^*\cs{L}\boxtimes a^*\cs{L}) \arrow{r}{d^*(\nu\boxtimes \nu)}& d^*(p^*\cs{L}\boxtimes p^*\cs{L}) 
\end{tikzcd}
\]
The dashed arrows both satisfy \eqref{E1} and \eqref{E2} as they apply here.
This diagram also shows that $\mu : m^*\cs{L} \to \cs{L}\boxtimes \cs{L}$ is a morphism of equivariant sheaves, since it satisfies \eqref{E3} as it applies here.
The proof that $\theta^*(\cs{L}\boxtimes\cs{L})$ is also straightforward, since $a_2\circ \theta = \theta_2 \circ a_2$ and $p_2\circ \theta = \theta_2 \circ p_2$ for the obvious $\theta_2$.
Let $\nu_2 : a_2^*(\cs{L}\boxtimes\cs{L}) \to p_2^*(\cs{L}\boxtimes\cs{L})$ be the middle horizontal isomorphism of Weil local systems, above. 
Then, to see that
\[
\begin{tikzcd}
a_2^*(\cs{L}\boxtimes\cs{L}) \arrow{r}{a_2^*(\vartheta)} \arrow{d}[swap]{\nu_2} 
	& a_2^*\theta^*(\cs{L}\boxtimes\cs{L}) \arrow{d}{\theta^*(\nu_2)} \\
p_2^*(\cs{L}\boxtimes\cs{L}) \arrow{r}{p_2^*(\vartheta)} 
	& p_2^*\theta^*(\cs{L}\boxtimes\cs{L})
\end{tikzcd}
\]
commutes, consider the commuting diagram of stalks, below.
\[
\begin{tikzcd}
\cs{L}_{h_1g_1}\otimes\cs{L}_{h_2g_2} \arrow{r}{\vartheta}  \arrow{d}[swap]{\nu_2} 
	& \cs{L}_{h_2g_2} \otimes\cs{L}_{h_1g_1}  \arrow{d}{\nu_2} \\
\cs{L}_{g_1}\otimes\cs{L}_{g_2} \arrow{r}{\vartheta}  
	& \cs{L}_{g_2} \otimes\cs{L}_{g_1}
\end{tikzcd}
\]
\end{proof}



\subsubsection{Descent along a torsor}\label{ssec:descent}

Now suppose $q : G \to G_0$ is a regular epimorphism of smooth group schemes with kernel pair $(a,p)$
\[
\begin{tikzcd}
H\times G
 \arrow[shift left=2]{r}{a}
  \arrow[shift right=2,swap]{r}{p}
&
G 
\arrow{r}{q}
& 
G_0.
\end{tikzcd}
\]
Let $\cs{L}_0$ be a linear character sheaf on $G_0$.
Consider the functor
\[
q^* : \CS(G_0) \to \CS(G)
\]
given on objects by $(\cs{L}_0,\mu_0) \mapsto (q^*\cs{L}_0, (q\times q)^* \mu_0)$; see \cite{cunningham-roe:13a}*{Lem 1.4}.
%To see that $(q^*\cs{L}_0, (q\times q)^* \mu_0)$ is indeed a linear character sheaf on $G$, verify \cite{cunningham-roe:13a}, CS.3], arguing as in \cite{cunningham-roe:13a}].
This linear character sheaf $(\cs{L},\mu)\ceq (q^*\cs{L}_0, (q\times q)^* \mu_0)$ on $G$ comes equipped with the canonical isomorphism $\nu : a^* \cs{L} \to p^* \cs{L}$ defined by the following diagram of canonical isomorphisms.
\[
\begin{tikzcd}[column sep=40]
a^*\cs{L} \arrow[equal]{d} \arrow[dashed]{r}{\nu} &  \arrow[equal]{d} p^*\cs{L} \\
a^* ( q^*\cs{L}_0) \arrow{r}{q\circ a = q\circ p}
& p^* (q^*\cs{L}_0)
\end{tikzcd}
\]
Since this isomorphism satisfies \eqref{ECS1}, it follows that $(\cs{L}, \mu,\nu)$ is an $H$-equivariant linear character sheaf on $G$.
If $\alpha_0 : (\cs{L}_0,\mu_0) \to (\cs{L}_0',\mu_0')$ is a morphism in $\CS(G_0)$, 
then $q^*(\alpha_0) : (\cs{L},\mu) \to (\cs{L}',\mu')$ satisfies \cite{cunningham-roe:13a}*{CS4}, so $\alpha$ is a morphism in $\CS(G)$.
%\todo{I find myself having to make many references to \cite{cunningham-roe:13a}] and promising that the results there adapt to the noncommutative group case. It makes me uneasy, but the alternative is to actually repeat the proofs here, with the weaker hypothesis on $G$, which would be ugly. I'm just agonizing out loud, I guess.} 
These simple observations define the comparison functor
\[
q_H^* : \CS(G_0) \to CS_H(G)
\]
and show that the functor $q^* : \CS(G_0) \to \CS(G)$ factors according to the following commuting diagram of functors
\begin{equation}\label{qH}
\begin{tikzcd}
\CS(G) &\arrow{l}[swap]{q^*} 
\CS(G_0) \arrow{dl}{q_H^*}\\
\arrow{u}{\text{forget}} \CS_{H}(G). & 
\end{tikzcd}
\end{equation}
The definition of $q_H^* : \CS(G_0)\to \CS_H(G)$ will be revisited in the proof of the following result.

\begin{lemma}\label{lem:torsor}
If $q : G \to G_0$ is an $H$-torsor in the fppf topology 
then $q_H^*: \CS(G_0) \to \CS_H(G)$ is an equivalence.
\end{lemma}

\begin{proof}
First we recall that the comparison functor 
$\Loc(G_0) \to \Loc_H(G)$ is an equivalence.
To see this, observe first that $\Loc$ is a stack on schemes over $\Fq$ in the Zariski topology.
Then, observe that the comparison functor for Weil local systems is an equivalence for flat surjective morphisms of affine schemes over $\Fq$.
It follows that $\Loc$ is a stack over schemes over $\Fq$ in the fppf topology [Vistoli].
(This is a slight variation on the argument showing that quasicoherent sheaves are a stack over schemes in the fppf topology.)
Since $q : G\to G_0$ is an $H$-torsor in the fppf topology, is now follows from descent theory that the comparison functor $\Loc(G_0) \to \Loc_H(G)$ is an equivalence [Vistoli]. \todo{Need more specific references}

For use below, we write $L_1 : \Loc(G_0) \to \Loc_H(G)$ for the comparison functor, $R_1 : \Loc_H(G) \to \Loc(G_0)$ for its adjoint, and $(\epsilon_1, \eta_1)$ for the counit and unit of the adjunction.
%
Arguing as above, we also see that pull-back along $q\times q$ determines an equivalence $L_2 : \Loc(G_0\times G_0) \to \Loc_{H\times H}(G\times G)$.
For use below, we write $R_2 : \Loc_{H\times H}(G\times G) \to \Loc(G_0\times G_0)$ for its adjoint and $(\epsilon_2, \eta_2)$ for the counit and unit of the adjunction.

We may now revisit the definition of the functor $q_H^*: \CS(G_0)\to \CS_H(G)$:
on objects, $q_H^*: \CS(Q)\to \CS_H(G)$ is given by $q_H^*(\cs{L}_0,\mu_0) = (L_1(\cs{L}_0), L_2(\mu_0))$; on maps, $q_H^*: \CS(Q)\to \CS_H(G)$ is given by $q_H^*(\alpha_0) = L_1(\alpha_0)$.
Direct calculation confirms \cite{cunningham-roe:13a}*{CS1, CS2, CS3}, \eqref{E1}, \eqref{E2} and \eqref{ECS1}, and therefore that $(L_1(\cs{L}_0), L_2(\mu_0))$ is an object of $\CS_H(G)$; 
likewise, $L_1(\alpha_0)$ is a map in $\CS_H(G)$ after checking \cite{cunningham-roe:13a}*{CS4} and \eqref{E3}.
%
With this description of $q_H^*: \CS(Q)\to \CS_H(G)$ is is easy to see that it is an equivalence.
Its adjoint is given on objects by $(\cs{L},\mu,\nu) \mapsto (R_1(\cs{L},\nu), R_2(\nu))$, making use of Lemma~\ref{lem:HH}, and on morphisms by $\alpha \mapsto R_1(\alpha)$.
The adjunction is built from the adjunctions $(\epsilon_1, \eta_1)$ and $(\epsilon_2, \eta_2)$ for $(L_1,R_1)$ and $(L_2, R_2)$.
\end{proof}


\subsubsection{Quotient by a closed subgroup}\label{ssec:closed}

We now suppose that $j : H \hookrightarrow G$ is a closed subgroup scheme over $\Fq$ and that the action $a : H\times G\to G$ is obtained by restricting the action $m : G \times G\to G$ to $H\times G$.
%Then we have the diagram
%\[
%\begin{tikzcd}
%\ &  \arrow{dl}[swap]{a} H \times G \arrow{dr}{p} \arrow{dd}[swap]{f} & \\
%G & & G \\
%& \arrow{ul}{m} G \times G \arrow{ur}[swap]{p_2} & 
%\end{tikzcd}
%\]
Define $f : H\times G \to G\times G$ by $f(h,g) = (j(h),g)$ and note that $a = m\circ f$.
Every $H$-equivariant linear character sheaf $(\cs{L},\mu, \nu)$ uniquely determines an isomorphism 
\[
\beta : \cs{L}\vert_H \to (\EE)_{H}
\]
by 
\begin{equation}\label{betanu}
\nu = (\beta \boxtimes \id) \circ  f^*\mu,
\end{equation}
or more precisely, by the following diagram of isomorphisms, where we write $p_1: G\times G\to G$ for projection to the first component and $p_2 : G\times G \to G$ for projection to the second.
\[
\begin{tikzcd}[column sep=30]
\ & a^*\cs{L} \arrow{dl}{m\circ f = a} \arrow{r}{\nu} & p^*\cs{L} & \\
f^* m^*\cs{L} \arrow{dr}{f^*\mu} &&& (\EE)_{H} \boxtimes \cs{L} \arrow{ul} \\ 
& f^*(\cs{L}\boxtimes \cs{L}) \arrow{r}{p_1 \circ f = j}[swap]{p_2\circ f = \id}  & 
\cs{L}\vert_{H} \boxtimes \cs{L} \arrow{ur}{\beta\boxtimes \id} & 
\end{tikzcd}
\]
In this way we see we may replace $(\cs{L},\mu,\nu)$ with a triple, $(\cs{L},\mu,\beta)$, from which $(\cs{L},\mu,\nu)$ may be recovered.
The next lemma makes that statement more precise.


\begin{lemma}\label{lem:beta}
If $H\hookrightarrow G$ is a closed subgroup scheme over $\Fq$ then
$\CS_H(G)$ is equivalent to the category $\CS^H(G)$ of triples $(\cs{L},\mu,\beta)$ where $(\cs{L},\mu)\in \CS(G)$ and $\beta : \cs{L}\vert_H \to (\EE)_H$ is an isomorphism in $\CS(H)$; a morphism $(\cs{L},\mu,\beta) \to (\cs{L}',\mu',\beta')$ in $\CS^H(G)$ is a morphism $\alpha: \cs{L} \to \cs{L}'$ in $\CS(G)$ such that $\beta = \beta' \circ \alpha\vert_{H}$.
\end{lemma}

\begin{proof}
We have seen how every $(\cs{L},\mu,\nu)\in \CS_H(G)$ determines an isomorphism of Weil local systems $\beta : \cs{L}\vert_{H} \to (\EE)_{H}$.
This isomorphism is constrained by the conditions 
appearing in \eqref{E1}, \eqref{E2} and \eqref{ECS1}.
It is a straightforward, tedious exercise to show that those conditions are exactly equivalent to the condition
that the isomorphism of Weil local systems $\beta : \cs{L}\vert_{H} \to (\EE)_{H}$ is an isomorphism in the category of of linear character sheaves on $H$.
\end{proof}


We can now give the missing proofs from Section \ref{ssec:noncomdef}.

\subsubsection{Proof of Proposition~\ref{prop:Gder_triv}}\label{ssec:proof1}

%\begin{proof}[Proof of Proposition~\ref{prop:Gder_triv}] 
The abelianization $G\ab\ceq G/G\der$ exists in schemes over $\Fq$ \cite{SGA3}*{Expos\'e {VI}$_A$, Thm. 3.2}.
Set $H=G\der$.
Let $j : H\hookrightarrow G$ be the inclusion and let $q : G \to G\ab$ be the quotient. 
With reference to \eqref{qH} and Section~\ref{ssec:closed}, consider the following diagram.
%\[
%\begin{tikzcd}
%{} & G\times G \arrow{d}{c} &&&\\
%1 \arrow{r} & H \arrow{r}{j} & G \arrow{r}{q} & G\ab \arrow{r} & 1
%\end{tikzcd}
%\]
%gives
\[
\begin{tikzcd}
\CS(H) & \arrow{l}[swap]{j^*} \CS(G) &\arrow{l}[swap]{q^*} 
\CS(G\ab) \arrow{dl}{\text{equiv}}[swap]{q_{H}^*} \\
 \CS'(G)  \arrow{r}{\beta \mapsto \nu}[swap]{\text{equiv}} &  \CS_{H}(G) \arrow{u}{\text{forget}}  & 
\end{tikzcd}
\]
%
Here we use the notation $\CS'(G)$ from Section~\ref{ssec:noncomdef} and the observation that $\CS'(G)$ is precisely the category $\CS^{H}(G)$ from Section~\ref{ssec:closed} for $H=G\der$.
The forget
Since the quotient $q : G \to G\ab$ is a trivial $H$-torsor in the fppf topology \cite{SGA3}*{Expos\'e {VI}$_A$, Thm. 3.2}, 
it now follows from Lemma~\ref{lem:torsor} that the comparison functor $\CS(G\ab)\to \CS_{H}(G)$ is an equivalence.
On the other hand, by Lemma~\ref{lem:beta}, $\CS'(G) \to \CS_{H}(G)$, defined by $(\cs{L},\mu,\beta)\mapsto (\cs{L},\mu, \nu)$, is an equivalence.
Since the essential image of the forgetful functor $\CS'(G)\to \CS(G)$ is precisely the category of $(\cs{L},\mu)\in \CS(G)$ such that $\cs{L}\vert_{H} \iso (\EE)_{H}$, this completes the proof of Proposition~\ref{prop:Gder_triv}.
%\end{proof}

\subsubsection{Proof of Theorem~\ref{thm:Gab}}\label{ssec:proof2}

%\begin{proof}[Proof of Theorem~\ref{thm:Gab}] 
Here we use notation from the proof of Proposition~\ref{prop:Gder_triv};
in particular, $H = G\der$.
By Definition~\ref{def:CCScom}, $\CCS(G\ab) \to \CS(G\ab)$ is a full subcategory.
Since the comparison functor $\CS(G\ab) \to \CS_{H}(G)$ is an equivalence, it determines a full subcategory $\CCS_{H}(G)$ which is equivalent to $\CCS(G\ab)$, as pictured below.
The proof of Theorem~\ref{thm:Gab} now reduces to the following claim: the essential image of the equivalence $\CS_{H}(G) \to \CS'(G)$ is $\CCS(G)$.
\[
\begin{tikzcd}[row sep=30, column sep=40]
\CS(H) 
& \arrow{l}[swap]{j^*} \CS(G) 
&\arrow{l}[swap]{q^*} 
\CS(G\ab) \arrow{dl}{\text{equiv}}[swap]{q_{H}^*} \\
\CS'(G)  \arrow{r}{\beta \mapsto \nu}[swap]{\text{equiv}} 
&  \CS_{H}(G) \arrow{u}{\text{forget}}  
&  %\arrow[bend left=35pt, dashed]{dll} 
\CCS(G\ab) \arrow{u}[swap]{\text{full sub}} \arrow{dl}{\text{equiv}} \\
\CCS(G) \arrow{u}{\text{full sub}} \arrow{r}{\beta \mapsto \nu}[swap]{\text{equiv}}
& \arrow{u}{\text{full sub}}  \CCS_{H}(G) 
& 
\end{tikzcd}
\]

Suppose $(\cs{L},\mu,\beta)\in \CS'(G)$ and $(\cs{L},\mu,\nu) = q^*(\cs{L}\ab,\mu\ab)$;
we must show that $(\cs{L},\mu,\beta)\in \CCS(G)$ if and only if $(\cs{L}\ab,\mu\ab)\in \CCS(G\ab)$.
Let $\xi : m\ab^*\cs{L}\ab \to \theta^* m\ab^*\cs{L}$ be the isomorphism attached to $(\cs{L}\ab,\mu\ab)\in \CS(G\ab)$ as in Section~\ref{sec:comcom}.
Let $\gamma : m^*\cs{L} \to \theta^* m^*\cs{L}$ be the isomorphism attached to $\beta : \cs{L}\vert_{H} \to (\EE)_{H}$ as in Section~\ref{ssec:noncomdef}.
Then the diagram in Definition~\ref{def:CCS} is precisely the result of applying the functor $(q\times q)^*$ to the diagram in Definition~\ref{def:CCScom}, as pictured below; 
in particular $\gamma = (q\times q)^* \xi$.
  \[
  \begin{tikzcd}[row sep=10, column sep = 20]
   m_0^*\cs{L}_0 \arrow{dd}[swap]{\xi} \arrow{r}{\mu_0} 
   & \cs{L}_0\boxtimes\cs{L}_0 \arrow{dd}{\vartheta}
&&&   m^*\cs{L} \arrow{dd}[swap]{\gamma} \arrow{r}{\mu} 
 & \cs{L}\boxtimes\cs{L} \arrow{dd}{\vartheta}\\
 && \arrow{r}{(q\times q)^*}  &\ & & &\\ 
   \theta^*(m_0^*\cs{L}_0) \arrow{r}{{\theta}^*\mu_0} 
 &  \theta^*(\cs{L}_0\boxtimes\cs{L}_0)
&&&   \theta^*(m^*\cs{L}) \arrow{r}{{\theta}^*\mu} 
&  \theta^*(\cs{L}\boxtimes\cs{L})
  \end{tikzcd}
\]
By Lemma~\ref{lem:HH}, we may interpret the diagram on the right, above, as a diagram in $\Loc_{H\times H}(G\times G)$.
By Lemma~\ref{lem:torsor}, this corresponds to a diagram in $\Loc(G\ab\times G\ab)$, necessarily the diagram on the left, above, and also that the diagram in Definition~\ref{def:CCS} commutes if and only if the diagram in Definition~\ref{def:CCScom} commutes. 
In other words, $(\cs{L},\mu,\beta)\in \CCS(G)$
if and only if $(\cs{L}\ab,\mu\ab)\in \CCS(G)$.
This completes the proof of Theorem~\ref{thm:Gab}.
%This also defines the functor $\CCS(G\ab) \to \CCS(G)$ by the dashed arrow in the diagram above.
%\end{proof}


\section{Application to type theory for $p$-adic groups}

\iffalse
\subsection{Linear Characters}

For non-commutative $G$, we aim to model homomorphisms from $G(k)$ to $\EEx$ using the category $\CCS(G)$.
The precise nature of the relationship between $G(k)^*$ and $\CCSiso{G}$ is described by Galois cohomology.
Let $\Delta$ be the image of the boundary homomorphism $\delta$ in the sequence
\[
1 \to G\der(k) \to G(k) \to G\ab(k) \xrightarrow{\delta} \Hh^1(k, G\der),
\]
yielding a short exact sequence
\[
1 \to G(k) / G\der(k) \to G\ab(k) \to \Delta \to 1.
\]
Applying $\Hom(-, \EEx)$ and using the fact that every homomorphism $G(k) \to \EEx$ vanishes on $G\der(k)$, we get
\[
1 \to \Delta^* \to G\ab(k)^* \to G(k)^* \to 1.
\]

Combining Theorems \ref{thm:trfrobiso} and \ref{thm:Gab} gives an isomorphism $\CCSiso{G} \to G\ab(k)^*$. \todo{Could we turn this into a declared proposition? It's a nice result.}
We may thus ask for a condition on a commutative character sheaf on $G$ that corresponds to having trivial
trace of Frobenius, with the expectation that the isomorphism classes of such commutative character sheaves
should be in natural bijection with $\Delta^*$.

Note that when $\Hh^1(k,G\der) = 0$, as is the case when $G\der(\bFq)$ ``is simply connected,'' then
we succeed in geometrizing characters of $G(k)$ on the nose.

% \begin{corollary}
% If $\Hh^1(\Fq,G\der)=0$ then
%the following diagram of isomorphisms commutes.
%\[
%\begin{tikzcd}
%\CCSiso{G} \arrow{r} & G(\Fq)^*\\
%\arrow{u}{q^*} \CCSiso{G\ab} \arrow{r} & \arrow{u} G\ab(\Fq)^* 
%\end{tikzcd}
%\]
% \end{corollary}
% 
% \begin{proof}
% By Theorem \ref{thm:trfrobiso}, $\CCSiso{G}\to G(\Fq)^*$ and $\CCSiso{G\ab}\to G\ab(\Fq)^*$ are isomorphisms.
%By Theorem~\ref{thm:Gab}, $\CCSiso{G} \iso \CCSiso{G\ab}$.
%Since $\Hh^1(\Fq,G\der)=0$, we have  $G\ab(\Fq) = G(\Fq)/G\der(\Fq)$.
%Thus, all characters of $G(\Fq)$ factor through
% $G\ab(\Fq)$. . . . 
% The isomorphisms are compatible because  . . . 
% \end{proof}


\subsection{Linear characters of certain $p$-adic groups}

Let $K$ be a local field with residue field $\Fq$.
Let $\underline{G}$ be a smooth group scheme over $K$.
Set $G_n = \Gr^R_n(\underline{G})$; this is a smooth group scheme over $\Fq$.
Set $G = \Gr_R(\underline{G})$; this is a group scheme over $\Fq$ equipped with $G = \varprojlim_{n\in \NN} G_n$.
Let $\CCS(\underline{G})$ be the category of commutative quasicharacter sheaves on $G$. . . .  as in \cite{cunningham-roe:13a}.

\begin{proposition}\label{prop:QCS}
$\underline{G}(\OK)^* \iso \CCSiso{\underline{G}}$.
\end{proposition}

\begin{proof}
That's not correct; it should be 
\[
\CCSiso{\underline{G}} \iso (\Gr(\underline{G})\ab(\Fq))^*.
\]
Is $\Gr(\underline{G})\ab = \Gr(\underline{G}\ab)?$ If so, then at least we get 
\[
\CCSiso{\underline{G}} \iso (\underline{G}\ab(\OK))^*.
\]
\end{proof}

\fi

%\subsection{Geometric types for supercuspidal representations}

\newcommand{\red}{^{\operatorname{red}}}


As proved in \cite{kim:07a}, all supercuspidal representations of unramified $p$-adic groups can be built from the data of \cite{yu:01a}*{\S 15}.
The type, in the sense of \cite{bushnell-kutzko:98a}, of the resulting supercuspidal representations can be constructed directly from the data. 
Let us refer to the types arising in this manner as {\it Yu types}.
In this section we show that every Yu type is geometrizable, in a sense which will be made clear below.


\subsection{Review of Yu's types}

A Yu type is a pair $(\,^\circ K,\,^\circ \rho)$, where $\,^\circ K$ is a compact group and $\,^\circ \rho$ is a representation of $\,^\circ K$, with certain properties. 
One of the main results of \cite{yu:models} is that there is a canonical affine smooth group scheme $\underline{G}$ of finite type over $\OK$ such that $\,^\circ K = \underline{G}(\OK)$ \cite{yu:models}*{Prop 10.2}.
As explained in \cite{yu:models}*{\S 10.4}, the
the group scheme $\underline{G}$ comes with the following auxillary data:
\begin{enumerate}
\item[(Ji)]
	affine smooth group schemes $\underline{J}^i$, for $i=0, 1, \ldots, d$, together with morphisms $\underline{J}^i \to \underline{G}$ such that \[\prod_{i=0}^d \underline{J}^i(\OK) = \underline{G}(\OK),\] where we identify $\underline{J}^i(\OK)$ with its image under $\underline{J}^i \to \underline{G}$, and such that $\prod_{i=j+1}^{d} \underline{J}^i(\OK)$ is normal in $\underline{G}(\OK)$, for $j=1, \ldots, d$;
\item[(Gj)] affine smooth group schemes $\underline{G}^{j}$, for $j=1, \ldots, d$, together with morphisms $\underline{G}^{j}\to \underline{G}$ such that 
\[
\underline{G}^{j}(\OK) = \prod_{i=0}^{j} \underline{J}^i(\OK),
\] 
where we identify $\underline{J}^i(\OK)$ with its image under $\underline{J}^i \to \underline{G}$ and $\underline{G}^{j}(\OK)$ with its image under $\underline{G}^j \to \underline{G}$.
\end{enumerate}

The representation $\,^\circ \rho$ of $\underline{G}(\OK)$ is build from the following data:
\begin{enumerate}
\item
	a representation $\sigma$ of $\underline{G}_s\red(\Fq)$;
\item
if $d \geq 2$, for $j=1,\ldots, d-1$, 
\begin{enumerate}
\item
a representation $\beta_j$ of $H^j(\Fq)$, where $H^j$ is Heisenberg group arising as a quotient $(\underline{J}^{j})_s \to H^j$,
\item
a representation $\kappa_j$ of $(\underline{G}^{j-1})_s\red(\Fq)$,
\item
a linear quasicharacter $\phi_{j-1}$ of $\underline{G}^{j-1}(\OK)$,
\end{enumerate}
subject to the following conditions:
\begin{enumerate}
\item
\end{enumerate}
\item
	a linear quasicharacter $\phi$ of $\underline{G}(\OK)$.
\end{enumerate}
From this data, the representation $\,^\circ \rho$ is built as follows, as explained in \cite{yu:models}*{\S 10.5}.
\begin{enumerate}
\item 
	Let $\dot{\sigma}$ be the representation of $\underline{G}_s(\Fq)$ obtained by pulling back $\sigma$ along $\underline{G}_s(\Fq) \to \underline{G}_s\red(\Fq)$. 	
	Let $\ddot{\sigma}$ be the representation of $\underline{G}(\OK)$ obtained by pullback along $\underline{G}(\OK) \to \underline{G}_s(\Fq)$.
\item For $j=1,\ldots, d-1$, 
\begin{enumerate}
\item
	let $\dot{\beta}_j$ be the representation of $\underline{J}^{j}_s(\Fq)$ obtained by pulling back $\beta_j$ along $\underline{J}^{j}_s(\Fq) \to H^j(\Fq)$ and let $\ddot{\beta}_j$ be . . . 
;
\item 
	let $\dot{\kappa}_j$ be the representation of $\underline{G}^{j-1}_s(\Fq)$ obtained by pulling back $\kappa_j$ along $\underline{G}^{j-1}_s(\Fq) \to (\underline{G}^{j-1})_s\red(\Fq)$ and let $\ddot{\kappa}_j$ . . ;
\item
	let $\phi_{j-1}$ be the linear quasicharacter of $\underline{G}^{j-1}(\OK)$ as above. 
\end{enumerate}
\item Let $\phi$ be the linear quasicharacter of $\underline{G}(\OK)$, as above.
\end{enumerate}
Using the decomposition $\underline{G}^j(\OK) = \underline{G}^{j-1}(\OK) \times \underline{J}^j(\OK)$, combine $\ddot{\beta}_j$, $\ddot{\kappa}_j$  and $\phi_j$ to form the representation $(\phi_{j-1}\otimes \ddot{\kappa}_j)\times \ddot{\beta}_j$ of $\underline{G}^{j}(\OK)$, for $j=1,\ldots, d-1$. 
	Since $\prod_{i=j+1}^{d} \underline{J}^i(\OK)$ is normal in $\underline{G}(\OK)$, this determines a representation $\delta_j$ of $\underline{G}(\OK)$, for $j=1,\ldots, d-1$.
Then 
\[
\,^\circ \rho = \ddot{\sigma} \otimes \delta_1 \cdots \otimes \delta_{d-1} \otimes \phi.
\]

Regarding this construction of the type $(\,^\circ  K,\,^\circ \rho)$, the paper \cite{yu:models} concludes with the following provocative observation.
\begin{quotation}
{\it Therefore, up to some linear characters, all the ingredient representations are on groups of
 the form $\underline{H}(\mathcal{O})$, where $\underline{H}$ is a smooth group scheme over $\mathcal{O}$, and the representations are inflated from $\underline{H}(\kappa)$. These results suggest that algebraic geometry and group schemes should play an
important role in the representation theory of $p$-adic groups.} \cite{yu:models}
\end{quotation}
%
The objective of this section is to push this observation further by geometrizing the representation $\,^\circ \rho$ of $\underline{G}$.

\subsection{Geometrization of types}

Let $(\underline{G}(\OK),\,^\circ \rho)$ be a Yu type.
Set $G = \Gr(\underline{G})$, the Greenberg transform of $\underline{G}$.
Recall that $G$ is a group scheme over $\Fq$ and comes equipped with 


\begin{proposition}\label{prop:types}
Let $(\underline{G}(\OK),\rho)$ be a Yu type.
With notation as above, suppose $\pi_0(\underline{G}_s\red)$ is cyclic and that $\pi_0((\underline{G}^{j-1})_s\red)$ is cyclic for $j=1,\ldots, d-1$.
Then there is a virtual admissible $\ell$-adic sheaf $\mathcal{F}$ on $G$ such that $\trFrob{\mathcal{F}} = \trace \rho$ under the canonical isomorphism $G(\Fq) = \underline{G}(\OK)$.
\end{proposition}

\begin{proof}
First, we make some observations concerning the auxillary data for $\underline{G}$. 

Introduce morphisms $\underline{G}\to \underline{M}^j$ and $\underline{M}^{j} \to \underline{M}^{j-1}$ and $\underline{M}^j \to \underline{J}^j$.
\[
\begin{tikzcd}
\underline{J}^j \arrow{r} & \underline{G}^{j} \arrow{r} 
& \underline{G} \arrow{r}  
& \underline{M}^{j} \arrow{d} \arrow{r} & \underline{J}^j \\
& \arrow{u} \underline{G}^{j-1} 
&
& \underline{M}^{j-1}
&
\end{tikzcd}
\]
\[
\begin{tikzcd}
\underline{G} \arrow{r}  
& \underline{M}^{j} \arrow{d} \arrow{r} & \underline{J}^j \\
\underline{G}_s \arrow{u} \arrow{d}
& \underline{M}^{j-1}
 & (\underline{J}^j)_s \arrow{u} \arrow{d} \\
\underline{G}_s\red & (\underline{M}^{j-1})_s \arrow{u} \arrow{d} & H^j \\
& (\underline{M}^{j-1})_s\red & 
\end{tikzcd}
\]

Now, we introduce the tools needed to geometrize all the ingredient representations.
\begin{enumerate}
\item
Let $(A_0,\varphi)$ be a Frobenius-stable character sheaf on $\underline{G}_s\red$, in the sense of [Lusztig].
Let $(A,\varphi)$ be the Frobenius-stable constructible sheaf on $\underline{G}_s$ obtained by pull-back.
\item For $j=1,\ldots, d-1$, 
\begin{enumerate}
\item
Let $(B_0^{j},\varphi)$ be a Frobenius-stable character sheaf on $H^j$ in the sense of [Boyarchenko].
Let $(B^{j},\varphi)$ be the Frobenius-stable constructible sheaf on $\underline{J}^j_s$ obtained by pull-back.
\item
Let $(C_0^{j-1},\varphi)$ be a Frobenius-stable character sheaf on $(\underline{M}^{j-1})_s\red$ in the sense of [Lusztig].
Let $(C^{j-1},\varphi)$ be the Frobenius-stable constructible sheaf on $\underline{M}^{j-1}_s$ obtained by pull-back.
\item
Let $\cs{E}_n$ be a commutative quasicharacter sheaf on $M_n^{j-1}$.
\end{enumerate}
\item
Let $\cs{L}_n$ be a commutative quasicharacter sheaf on $G_n$.
\end{enumerate}
\[
\begin{tikzcd}
G \arrow{r}  \arrow{d}
& M^{j} \arrow{r} \arrow{d}
& J^j \arrow{d} \\
G_n \arrow{r}  \arrow{d}
& M_n^{j} \arrow{r} \arrow{d}
& J_n^j \arrow{d} \\
\underline{G}_s \arrow{d}
& M_n^{j-1} \arrow{d}
& (\underline{J}^j)_s  \arrow{d} \\
\underline{G}_s\red 
& (\underline{M}^{j-1})_s \arrow{d} & H^j \\
& (\underline{M}^{j-1})_s\red 
& 
\end{tikzcd}
\]

We may now encorporate all these ingredients into one sheaf on $G_n$.
\begin{enumerate}
\item
Let $\cs{A}_n$ be sheaf on $G_n$ produced from $A$ by pullback along $G_n \to \underline{G}_s$.
\item
For $j=1, \ldots, d-1$,
\begin{enumerate}
\item let $\cs{B}^j$ be the sheaf on $J^j_n$ ...
\item let $\cs{C}^j$ be the sheaf on $M^{j-1}_n$ ...
\item let $\cs{L}^j$ be the sheaf on $M^{j-1}_n$ ...
\end{enumerate}
Combine: $\cs{B}^j$, $\cs{C}^j$ and $\cs{E}^j$ into $\cs{D}^j$ on $G_n$ . . . .
\item Let $\cs{L}$ be the sheaf on $G_n$ as above...
\end{enumerate}

\[
\cs{A} \otimes \cs{D}^1 \otimes \cdots \otimes \cs{D}^{d-1} \otimes \cs{L}.
\]

Interpret 
\begin{enumerate}
\item
	$\epsilon_j$ as a representation of $(\underline{M}^{j-1})_s\red(\Fq)$, for each $j=1,\ldots, d$;
\item
	$\phi_{j-1}$ as a linear quasicharacter of $\underline{M}^{j-1}(\OK)$, for $j=1,\ldots, d$;
\end{enumerate}
so that
\begin{enumerate}
\item 
	$\dot{\epsilon}_j$ is the representation of $\underline{M}^{j-1}(\OK)$ obtained by pulling back $\epsilon_j$ along 
\[\underline{M}^{j-1}(\OK) \to \underline{M}^{j-1}_s(\Fq) \to (\underline{M}^{j-1})_s\red(\Fq),\]
for $j=1,\ldots, d-1$.
\item
	$\dot{\eta}_j$ is a representation of $\underline{J}^{j}(\OK)$ obtained by pulling back  . . . .  $j=1,\ldots, d-1$;
\item
	$\tau_j$ is the representation of $\underline{G}(\OK)$ obtained by pulling back  $\phi_{j-1}$ and $\dot{\epsilon}_j$ and $\dot{\eta}_j$ from $\underline{M}^{j}(\OK)$ along $\underline{G}(\OK)\to \underline{M}^j(\OK)$, for $j=1,\ldots, d-1$. 
\end{enumerate}


\end{proof}


\iffalse

Let $K$ be a $p$-adic field. Let $\G$ be a connected reductive linear algebraic group over $K$. 
A generic datum is a triple $(\vec{\G}, \pi_0, \vec{\phi})$ where: $\vec{\G}$ is a tower of $p$-adic groups $\G^0(K) \subset \G^1(K) \subset \cdots  \subset \G^d(K) = \G(K)$ (proper containment); $\pi_0$ is a depth-zero representation of $\G^0(K)$, and $\vec{\phi} = (\phi_0, \ldots , \phi_{d-1})$ where each $\phi_i : \G^i(K) \to \EEx$ is a quasicharacter (continuous group homomorphism).  

%The generic datum $(\vec{\G}, \pi_0, \vec{\phi})$ determines a type (in the sense of \cite{bushnell-kutzko:98a}) $(C, \rho)$ for $\G$, as explained in \cite{yu:construction}*{\S 15}, where $C \subset \G(K)$ is compact mod centre subgroup and $\rho$ is a representation of $C$. In \cite{yu:smooth} it is shown that the $C$ admits a smooth integral model, in the sense that there is a smooth group scheme $\underline{C}$ over $\OK$ such that $\underline{C}(\OK) = C$.

In \cite{yu:03a} it is shown that, for $0 \leq i < d$, the $p$-adic group $\G^i(K)$ admits a smooth integral model, in the sense that there is \todo{is $\underline{G}^i$ canonically determined?} a smooth group scheme $\underline{G}^i$ over $\OK$ such that $\underline{G}^i(\OK) = \G^i(K)$.
Set $G^i_n \ceq \Gr^R_n(\underline{G}^i)$ and $G^i \ceq \Gr_R(\underline{G}^i)$; 
then \[G^i(\Fq) = \G^i(K)\] (canonical isomorphism of groups).
In this way, $\vec{\G}$ determines a tower $\vec{G}$ of group schemes $G^0 \subset G^1 \subset \cdots  \subset G^{d-1}$ (proper containment) over $\Fq$, from which the tower of $p$-adic groups
$\G^0(K) \subset \G^1(K) \subset \cdots  \subset \G^{d-1}(K)$ can be recovered.

Using Corollary~\ref{cor:QCS} \todo{broken reference} we see that each quasicharacter $\phi_i : G^i(\Fq) \to \EEx$ determines a quasicharacter sheaf $\cs{L}^i$, unique up to isomorphism, from which the quasicharacter $\phi_i$ can be recovered using the trace of Frobenius. 
In fact, more is true. Let $n_i$ be the smallest non-negative integer $n$ such that $\phi_i : G^i(\Fq) \to \EEx$ factors through $G^i(\Fq) \to G^i_n(\Fq)$.
Then $\phi_i$ determines a commutative character sheaf $\cs{L}_{n_i}^i$, unique up to isomorphism, from which the quasicharacter $\phi_i$ can be recovered.
Then $\cs{A}_i\ceq \cs{L}_{n_i}^i[\dim G_{n_i}^i]$ is a perverse sheaf on $G_{n_i}^i$ from which the quasicharacter $\phi_i$ can be recovered.
In this way,  $\vec{\phi} = (\phi_0, \ldots , \phi_{d-1})$ determines perverse sheaves $\vec{\cs{A}} = (\cs{A}_0, \ldots , \cs{A}_{d-1})$ from which $\vec{\phi}$ can be recovered. 

Finally, consider the depth-zero supercuspidal representation $\pi_0$ of $\G^0(K)$.
Then $\pi_0$ is produced by compact induction from representation $\sigma$ of a compact mod centre subgroup $\underline{C}_0(\OK) \subset \G^0(K)$. Moreover, $C_0$ admits a smooth integral model $\underline{C}$ ...

%\subsection{Toral supercuspidal representations of $p$-adic groups}

\fi

\bibliography{bibliography/Biblio}

\end{document}

Suppose there is some $f : H\times G\to G\times G$ such that the following diagram commutes.
\[
\begin{tikzcd}
\ &  \arrow{dl}[swap]{a} H \times G \arrow{dr}{p} \arrow{dd}[swap]{f} & \\
G & & G \\
& \arrow{ul}{m} G \times G \arrow{ur}[swap]{p_2} & 
\end{tikzcd}
\]
Then $f(h,g) = (a(h,g) g^{-1}, g)$.
Define $f_1: H\times G \to G$ by $f_1(h,g)\ceq a(h,g) g^{-1}$; then $f(h,g) = (f_1(h,g),g)$ and $f_1(h_1 h_2,g) = f_1(h_1, h_2\cdot g) f_1(h_2,g)$.

Every $H$-equivariant linear character sheaf $(\cs{L},\mu, \nu)$ uniquely determines an isomorphism 
\[
\beta_1 : f_1^*\cs{L} \to (\EE)_{H\times G}
\]
by $\beta \otimes \id_{p^*\cs{L}} = \nu \circ f^*\mu^{-1}$, or more precisely, by the following diagram of isomorphisms,
\[
\begin{tikzcd}
a^*\cs{L} \arrow{d}{m\circ f = a} \arrow{rrr}{\nu}
&&& p^*\cs{L}\\
f^* m^*\cs{L} \arrow{r}{f^*\mu} & 
f^*(\cs{L}\boxtimes \cs{L}) \arrow{d} & 
f_1^*\cs{L} \otimes p^*\cs{L} \arrow{r}{\beta\otimes \id_{p^*\cs{L}}} & 
(\EE)_{H\times G} \otimes p^*\cs{L} \arrow{u} \\
& f^*(p_1^*\cs{L}\otimes p_2^*\cs{L}) \arrow{r} & f^*p_1^*\cs{L}\otimes f^* p_2^*\cs{L} \arrow{u}{f_1 = p_1\circ f}[swap]{p = p_2\circ f} &
\end{tikzcd}
\]
In this way we see we may replace $(\cs{L},\mu,\nu)$ with an alternate triple, $(\cs{L},\mu,\beta_1)$, from which $(\cs{L},\mu,\nu)$ may be recovered.
The isomorphism $\beta_1 : f_1^*\cs{L} \to (\EE)_{H\times G}$ is constrained by two properties, corresponding to the diagrams in Sections [] and [].

\subsection{Quotient and restriction}

We now combine the material of the last two sections, by specializing to the following case: $H$ is a closed subgroup of $G$ and $q : G \to G/H$ is the fppf quotient, which exists in schemes by  . . .

\iffalse

\subsection{Change of action}

Now, let $a' : H'\times G \to G$ be another group action on $G$ and suppose $f: H \times G \to H'\times G$ makes the following diagram commute.
\[
\begin{tikzcd}
&  \arrow{dl}[swap]{a} H \times G \arrow{dr}{p} \arrow{dd}[swap]{f} & \\
G & & G \\
& \arrow{ul}{a'} H' \times G \arrow{ur}[swap]{p} & 
\end{tikzcd}
\]
Then pull-back along $f: H\times G \to H'\times G$ defines a monoidal functor 
\[
\CS_{H'}(G) \to \CS_{H}(G).
\]


Then $f : H\times G \to H'\times G$ is given by $f(h,g) = (k(h,g),g)$ for some $k : H\times G\to H'$. 
Define $d : H\to H'$ by $d(h) = k(h,1)$.




\begin{lemma}\label{lem:action}
The canonical functor
\[
\CS_{H}(G) \to \CS_{H'}(G)
\]
is an equivalence if ...
\end{lemma}
\begin{proof}
\end{proof}

\subsection{Conjugation}

We close this section with an alternate characterization of $\CS'(G)$, the category of linear character sheaves $\cs{L}$ on $G$ for which $\cs{L}\vert_{G\der} \iso (\EE)_{G\der}$.
Recall that $\CS'(G)$ is a full subcategory of $\CS(G)$ and that $\CCS(G)$ is a full subcategory of $\CS'(G)$.

Consider the following case of Section~\ref{ssec:action}.

\fi
