\documentclass[10pt]{amsart}

\title{Commutative Character Sheaves}
\author{Clifton Cunningham}
\address{Department of Mathematics and Statistics, University of Calgary, 2500 University Drive Northwest, Calgary, AB, Canada, {T2N~1N4}.}
\email{cunning@math.ucalgary.ca}
\author{David Roe}
\address{Department of Mathematics, University of Pittsburgh, 301 Thackeray Hall, Pittsburgh, PA , United States, 15260.}
\email{roed.math@gmail.com}

\subjclass[2010]{14F05 (primary), 14L15 (secondary), 22E50 (tertiary)}
\keywords{character sheaves}

\usepackage{amssymb}
\usepackage{amsrefs}
% Fonts
\usepackage{mathrsfs}
% Enumitem
\usepackage{enumitem}
% Hyperrefs
\usepackage{hyperref}

\usepackage{tikz}
\usetikzlibrary{shapes,arrows,calc,matrix}
\usepackage{tikz-cd}

%%%%%%%%%%%%%%% THEOREM STYLES %%%%%%%%%%%%%%%
\theoremstyle{plain}
      \newtheorem{theorem}{Theorem}[section]
      \newtheorem*{theorem*}{Theorem}
      \newtheorem{proposition}[theorem]{Proposition}
      \newtheorem{lemma}[theorem]{Lemma}
      \newtheorem{corollary}[theorem]{Corollary}

      \theoremstyle{definition}
      \newtheorem{definition}[theorem]{Definition}

      %\theoremstyle{remark}
      \newtheorem{remark}[theorem]{Remark}
      \newtheorem{example}[theorem]{Example}
      
%%%%%%%%%%%%%%% RINGS AND GROUPS %%%%%%%%%%%%%%%
\newcommand{\FF}{{\mathbb{F}}}
\newcommand{\ZZ}{{\mathbb{Z}}}
\newcommand{\NN}{{\mathbb{N}}}
\newcommand{\CC}{{\mathbb{C}}}
\newcommand{\QQ}{{\mathbb{Q}}}
\newcommand{\RR}{{\mathbb{R}}}
\newcommand{\EE}{\mathbb{\bar Q}_\ell}
\newcommand{\OK}{\mathcal{O}_K}
\newcommand{\OL}{\mathcal{O}_L}
\newcommand{\OO}[1]{\mathcal{O}_{#1}}
\newcommand{\bFq}{\bar{k}}
\newcommand{\Fq}{k}
\newcommand{\Fqm}{k_m}
\newcommand{\EEx}{\EE^\times}
\newcommand{\ZEx}{\mathbb{\bar Z}_\ell^\times}
\newcommand{\Weil}[1]{\mathcal{W}_{#1}}
\newcommand{\m}{{\mathfrak{m}}}
%%%%%%%%%%%%%%% ALGEBRAIC GROUPS %%%%%%%%%%%%%%%
\newcommand{\Gm}[1]{\mathbb{G}_{\hskip-1pt\textbf{m},#1}}
\DeclareMathOperator{\GL}{GL}
\newcommand{\comp}{\Pi} % Component group
%%%%%%%%%%%%%%% NAMED OPERATORS %%%%%%%%%%%%%%%
\DeclareMathOperator{\Gal}{Gal}
\newcommand{\Frob}[1]{\operatorname{Fr}_{#1}}
\DeclareMathOperator{\Aut}{Aut}
\DeclareMathOperator{\Hom}{Hom}
\DeclareMathOperator{\ord}{ord}
\DeclareMathOperator{\coker}{coker}
\DeclareMathOperator{\Gr}{Gr}
\DeclareMathOperator{\Irrep}{Irrep}
\DeclareMathOperator{\id}{id}
\DeclareMathOperator{\Ext}{Ext}
\DeclareMathOperator{\Hh}{H}
\DeclareMathOperator{\Res}{Res}
\DeclareMathOperator{\Nm}{Nm}
\DeclareMathOperator{\trace}{Tr}
\DeclareMathOperator{\obj}{obj}
\DeclareMathOperator{\mor}{mor}
\DeclareMathOperator{\Lang}{Lang}
\DeclareMathOperator{\image}{im}
\DeclareMathOperator{\Loc}{Loc}
\DeclareMathOperator{\Tot}{Tot}
\DeclareMathOperator{\Tor}{Tor}
\newcommand{\gal}[1]{{\operatorname{Gal}\hskip-1pt\left( {\bar #1}/#1 \right)}}
\newcommand{\Spec}[1]{{\operatorname{Spec}(#1)}}
%%%%%%%%%%%% MISCELLANEOUS OPERATORS %%%%%%%%%%%%
\newcommand{\sheafHom}{{\mathscr{H}\hskip-4pt{\it o}\hskip-2pt{\it m}}}
\newcommand{\abs}[1]{{\vert #1 \vert}}
\newcommand{\ceq}{{\, :=\, }}
\newcommand{\tq}{{\ \vert\ }}
\newcommand{\iso}{{\ \cong\ }}
\newcommand{\trFrob}[1]{t_{#1}}
\newcommand{\TrFrob}[1]{\operatorname{Tr}_{#1}}
%% Limits
\newcommand{\invlim}[1]{\lim\limits_{\overleftarrow{#1}}}
\newcommand{\dirlim}[1]{\lim\limits_{\overrightarrow{#1}}}
\newcommand{\limit}[1]{\mathop{\textsc{lim}}\limits_{#1}}
\newcommand{\colimit}[1]{\mathop{\textsc{colim}}\limits_{#1}}
%% Fonts for quasicharacter sheaves
\newcommand{\cs}[1]{{\mathcal{#1}}}
\newcommand{\gcs}[1]{{\mathcal{\bar #1}}}
\newcommand{\dualgcs}[1]{\gcs{#1}^\dagger}
\newcommand{\dualcs}[1]{\cs{#1}^\dagger}
%% Categories
\newcommand{\CS}{{\mathcal{C\hskip-0.8pt S}}}
\newcommand{\CCS}{{\mathcal{CC\hskip-0.8pt S}}}
\newcommand{\bCS}{{\CS_0}}
\newcommand{\catname}[1]{\normalfont{\textsf{#1}}}
\newcommand{\Sch}[1]{{\catname{Sch}_{/#1}}}
\newcommand{\QCS}{{\mathcal{QC\hskip-0.8pt S}}}
\newcommand{\CSiso}[1]{\CS(#1)_{/\text{iso}}}
\newcommand{\bCSiso}[1]{\bCS(#1)_{/\text{iso}}}
\newcommand{\QCSiso}[1]{\QCS(#1)_{/\text{iso}}}
\newcommand{\CCSiso}[1]{\CS(#1)_{/\text{iso}}}
%% Labeled items
\makeatletter
\newcommand{\labitem}[2]{
\def\@itemlabel{\textbf{#1}}
\item
\def\@currentlabel{#1}\label{#2}}
\makeatother
%% Shorthand for bars
\renewcommand{\bf}{\bar{f}}
\newcommand{\bg}{{\bar{g}}}
\newcommand{\bm}{\bar{m}}
\newcommand{\bG}{\bar{G}}
\newcommand{\bH}{\bar{H}}
\newcommand{\brho}{{\bar\rho}}
\newcommand{\bx}{{\bar{x}}}
%% Spacing control
\newcommand{\tight}[3]{\hspace{-#1pt}{#2}\hspace{-#3pt}}
\newcommand{\GxG}{\text{$G \tight{1}{\times}{1} G$}}
\newcommand{\bGxG}{\text{$\bar{G} \tight{1}{\times}{1} \bar{G}$}}
\newcommand{\bfxf}{\text{$\bar{f} \tight{1}{\times}{1} \bar{f}$}}
\newcommand{\GxxG}{\text{$G \tight{1}{\times}{1} G$}}
\newcommand{\LxL}{\text{$\gcs{L} \tight{0}{\boxtimes}{0} \gcs{L}$}}

%% Hyphenation override
\hyphenation{quasi-character}

%%%%%%%%%%%% BEGIN DOCUMENT %%%%%%%%%%%

\begin{document}

\maketitle

\section{Definitions and Recollections}

\begin{definition}\label{def:CCS}
A \emph{commutative character sheaf on $G$} is a triple
$\cs{L}\ceq (\gcs{L},\mu,\phi)$ where:
\begin{enumerate}
\labitem{(CCS.1)}{CS.1} $\gcs{L}$ is a rank-one $\ell$-adic local system on $\bG$, by which we mean a constructible
$\ell$-adic sheaf on $\bG$, {\it lisse} on each connected component of $\bG$, whose stalks are one-dimensional $\EE$-vector spaces;
\labitem{(CCS.2)}{CS.2} $\mu: \bm^* \gcs{L} \to \LxL$ is an isomorphism of
sheaves on $\bGxG$ such that the following diagrams commute,
  where $m_3 \ceq m\circ (m\tight{1}{\times}{2}\id) = m\circ (\id\tight{2}{\times}{1} m)$,
  \[
  \begin{tikzcd}[row sep=30]
  \bm_3^*\gcs{L} \arrow{rr}{(\bm \tight{1}{\times}{2} \id)^*\mu} \arrow[swap]{d}{(\id \tight{2}{\times}{1} \bm)^*\mu} 
    &&  \bm^*\gcs{L} \boxtimes \gcs{L} \dar{\mu \tight{0}{\boxtimes}{1} \id}  && \gcs{L} \arrow{r}{\mu} \arrow{dr}[swap]{\mu}& \gcs{L} \boxtimes \gcs{L} \arrow{d}{\iso}\\
    \gcs{L} \boxtimes \bm^* \gcs{L} \arrow{rr}{\id \boxtimes \mu}
    &&  \gcs{L} \tight{0}{\boxtimes}{0} \LxL && &  \gcs{L} \boxtimes \gcs{L}
  \end{tikzcd}
  \]
where 
$\LxL \to \LxL$ given by $x\otimes y \mapsto y \otimes x$, by which we meant the case of the sheaf isomorphism $\cs{L}_1 \boxtimes \cs{L}_2 \to \cs{L}_2\boxtimes \cs{L}_1$ when $\cs{L}_1 = \cs{L}_2 = \cs{L}$.
\labitem{(CCS.3)}{CS.3} $\phi : \Frob{G}^* \gcs{L} \to \gcs{L}$ is an
  isomorphism of constructible $\ell$-adic sheaves on $\bG$ compatible with
  $\mu$ in the sense that the following diagram commutes.
  \[
  \begin{tikzcd}[row sep=20]
  \Frob{\GxxG}^* \bm^* \gcs{L} \arrow{rr}{\Frob{\GxxG}^*\mu}
    && \Frob{\GxxG}^*(\LxL)\\
    \arrow[equal]{u} \bm^*  \Frob{G}^* \gcs{L} \arrow[swap]{d}{\bm^* \phi}
    && \Frob{G}^*\gcs{L}\boxtimes \Frob{G}^*\gcs{L} \dar{\phi\boxtimes \phi} \arrow[equal]{u} \\
    \bm^*\gcs{L} \arrow{rr}{\mu}
    && \LxL
  \end{tikzcd}
  \]
\end{enumerate}
Morphisms of character sheaves are defined in the natural way:
\begin{enumerate}
\labitem{(CCS.4)}{CS.4} if $\cs{L} = (\gcs{L},\mu,\phi)$ and
  $\cs{L'} = (\gcs{L'},\mu',\phi')$ are character sheaves on $G$ then
  a morphism $\rho : \cs{L} \to \cs{L}'$ is a map $\brho : \gcs{L} \to \gcs{L'}$
  of constructible $\ell$-adic sheaves on $\bG$ such that the following diagrams both commute.
  \[
  \begin{tikzcd}[column sep=40]
  \Frob{G}^* \gcs{L} \rar{\Frob{G}^* \brho} \arrow[swap]{d}{\phi} & \Frob{G}^* \gcs{L'} \dar{\phi'}
  & & \arrow[swap]{d}{\mu} \bm^* \gcs{L} \rar{\bm^* \brho} & \bm^* \gcs{L'} \dar{\mu'} \\
  \gcs{L} \rar{\brho} & \gcs{L'}
  & {} & \LxL \rar{\tight{1}{\rho\boxtimes \rho}{1}} & \gcs{L'} \tight{0}{\boxtimes}{0} \gcs{L'}
  \end{tikzcd}
  \]
\end{enumerate}
 The category of character sheaves on $G$ will be denoted by $\CCS(G)$.
 \end{definition}


With reference to \cite{Cunningham-Roe:QCS}*{Def. 1.1}, a we see that a character sheaf $\cs{L} = (\gcs{L},\mu, \phi)$ on $G$ is commutative if it satisfies only one additional condition: that
\[
\begin{tikzcd}
\gcs{L} \arrow{r}{\mu} \arrow{dr}[swap]{\mu}& \gcs{L} \boxtimes \gcs{L} \arrow{d}{\iso}\\
&  \gcs{L} \boxtimes \gcs{L}
\end{tikzcd}
\]
commutes.
Consequently, comparing with \cite{Cunningham-Roe:QCS}*{Def. 1.1} we find conditions ({\textbf{CS}.1), ({\textbf{CS}.3) and ({\textbf{CS}.4) appearing in the definition of the category $\CS(G)$, coincide exactly with conditions ({\textbf{CCS}.1), ({\textbf{CCS}.3) and ({\textbf{CCS}.4), above.

The additional condition distinguishing which character sheaves on $G$ are commutative, has several benefits, as we will see in this note.
\begin{enumerate}
\labitem{(1)}{b1}
A character sheaf $\cs{L}$ on $G$ is invisible if and only if $\cs{L}$ is not commutative.
Recall from \cite{Cunningham-Roe:QCS}*{Def. 2.8} that a character sheaf $\cs{L}$ in $G$ is {\it invisible} if $\cs{L}  \ncong \EE$ yet $\trFrob{\cs{L}} =1$.
These are exactly the non-trivial character sheaves in the kernel of the surjective group homomorphism 
\[
\TrFrob{G} : \CSiso{G} \to G(\Fq)^*
\]
 given by the trace of Frobenius.
\labitem{(2)}{b2}
This note shows that $\TrFrob{G} : \CSiso{G} \to G(\Fq)^*$ admits a canonical section, and thus a splitting of the short exact sequence 
\[
\begin{tikzcd}
0 \arrow{r} & \Hh^2(\pi_0(\bG),\EEx)^{\Weil{}} \arrow{r} & \CSiso{G} \arrow{r}{\TrFrob{G}} & G(\Fq)^* \arrow{r} & 0
\end{tikzcd}
\]
found in \cite{Cunningham-Roe:QCS}*{Thm. 3.6}.
\labitem{(3)}{b3}
Like the category $\CS(G)$ of character sheaves on $G$, which described completely in \cite{Cunningham-Roe:QCS}, $\CCS(G)$ is a rigid monoidal category.
Since $\CCS(G)$ is a full subcategory of $\CS(G)$, and since Item \ref{b2} above shows $\CCSiso{G} \iso G(\Fq)^*$, canonically, the category $\CCS(G)$ is now completely described by the following consequence of \cite{Cunningham-Roe:QCS}*{Thm. 3.9}: in $\CCS(G)$, every morphism $\cs{L} \to \cs{L}'$ is either trivial (zero on stalks) or and isomorphism, and
\[
\Aut(\cs{L}) \iso  \Hh^1(\pi_0(\bG), \EEx)^{\Weil{}}.
\]
Consequently, we have a complete description of category $\CCS(G)$.
\end{enumerate}


\section{Eliminating Invisible Character Sheaves}

If $M$ is a $\bG$-module, then we say a cohomology class in $\Hh^2(\bG, M)$ is \emph{symmetric} if it is represented
by a symmetric $2$-cocycle.  Since every coboundary is symmetric, either all cocycles in a given class will be symmetric
or none will.

\begin{lemma}
Let $\bG$ be a commutative group.  Then the only symmetric class in $\Hh^2(\bG, \EEx)$ is the trivial class.
\end{lemma}

\begin{proof}
By the universal coefficient theorem,
\[
0 \to \Ext^1_\ZZ(\Hh_{n-1}(\bG, \ZZ), \EEx) \to \Hh^n(\bG, \EEx) \to \Hom(\Hh_n(\bG, \ZZ), \EEx) \to 0
\]
for all $n > 0$.  When $n = 2$, using the fact that $\bG$ is commutative, we have that $\Hh_1(\bG, \ZZ) \cong \bG$
and that $\Hh_2(\bG, \ZZ) \cong \wedge^2 \bG$. We get
\[
0 \to \Ext^1_\ZZ(\bG, \EEx) \to \Hh^2(\bG, \EEx) \to \Hom(\wedge^2 \bG, \EEx) \to 0.
\]
The map $\Hh^2(\bG, \EEx) \to \Hom(\wedge^2 \bG, \EEx)$ maps a $2$-cocycle $f$ to the alternating function
\[
(x,y) \mapsto f(x,y) - f(y,x).
\]
Thus the cohomology classes represented by symmetric cocycles are precisely those in the image of $\Ext^1_\ZZ(\bG, \EEx)$.
But $\Ext^1_\ZZ(-, \EEx)$ vanishes because $\EEx$ is divisible.
\end{proof}

\begin{theorem}
Commutative.
\end{theorem}

\end{document}