%%------------------------------------------------------%%
%% This file should be the main file of your article.   %%
%% Please uncomment exactly one of the following lines. %%
%%------------------------------------------------------%%

%% \documentclass[CM,Submssn]{degruyter-crelle}       %% Equations numbered as (1), (2) etc.
 \documentclass[CM,Submssn,SecEq]{degruyter-crelle} %% Equations numbered as (1.1), (1.2) etc.

%% Do not alter the following line.
\firstpage{}\volume{}\volumeyear{}\copyrightyear{}\doiyear{}\doi{}\received{XXX}\revised{XXX}

\title[Shortened title for the running title]{Title}

\author{Firstname}{Lastname}{F.~Lastname}{City}
\author{Firstname}{Lastname}{F.~Lastname}{City}

\contact{Department, university, street, zip code, city, country}{e-mail}
\contact{Department, university, street, zip code, city, country}{e-mail}

\researchsupported{Please insert information concerning research grant support here.}

%% Define theorem-like environments as usual:

\theoremstyle{plain}
 \newtheorem{theorem}{Theorem}[section]

\theoremstyle{definition}
 \newtheorem{remark}[theorem]{Remark}


\begin{document}

\begin{abstract}
Insert your abstract here. Remember that online systems rely heavily on the content
of titles and abstracts to identify articles in electronic bibliographic databases
and search engines. We ask you to take great care in preparing the abstract
and to not use references to the bibliography.
\end{abstract}

%% \tableofcontents %% Just for papers exceeding 50 pages.


\section{Introduction}\label{sec:intro}

PLEASE INSERT YOUR MANUSCRIPT HERE.

\acknowl{Insert acknowledgments of the assistance of colleagues or similar notes of appreciation here.}


%-------------------------------------------------------------------------%
% GENERAL INSTRUCTIONS                                                    %
% Please keep formatting macros to a minimum.                             %
% Avoid redundant source code such as unused definitions or longer        %
% passages of comments (% or {comment}).                                  %
% Do not worry about bad page breaks, etc. and avoid adding extra space   %
% and using glue to improve the appearance of the manuscript.             %
% Please ensure that the material has been carefully proofread and that   %
% all files used are made available to the editors/publisher.             %
%-------------------------------------------------------------------------%

%-------------------------------------------------------------------------%
% PACKAGES                                                                %
% Missing packages can be downloaded at www.tug.org/ctan.html.            %
%-------------------------------------------------------------------------%

%-------------------------------------------------------------------------%
% MULTILINE EQUATIONS                                                     %
% For numbered/unnumbered multiline equations use align/align* instead of %
% eqnarray. (Other available environments for multiline displays are      %
% gather, multline, aligned, split, etc.)                                 %
%-------------------------------------------------------------------------%

%-------------------------------------------------------------------------%
% QED SYMBOL                                                              %
% The \qedhere command forces the QED symbol to go on the present line,   %
% which might be useful if the last part of a proof environment is, e.g., %
% a displayed equation or list environment.                               %
%-------------------------------------------------------------------------%

%-------------------------------------------------------------------------%
% SOME MORE MATH                                                          %
% -- Avoid blank lines before or after a display, unless you really want  %
%    to start a new paragraph.                                            %
% -- Use equation*, gather* or the bracket pair \[ and \] instead of      %
%    double dollar signs $$.                                              %
% -- Use \operatorname{...} for math operators that are not predefined    %
%    or define new operators with \DeclareMathOperator. lim-like          %
%    operators can be defined with \DeclareOperator*                      %
% -- Use \quad for spacings in displayed formulas.                        %
% -- Use \substack{... \\ ...} for multiline subscripts on, e.g., sums.   %
%-------------------------------------------------------------------------%

%-------------------------------------------------------------------------%
% ENUMERATIONS                                                            %
% By default, the labels of first level enumerations are (i), (ii), etc.  %
% You may change them to become, e.g., A., B. etc., using an optional     %
% argument: \begin{enumerate}[A.]                                         %
%-------------------------------------------------------------------------%

%-------------------------------------------------------------------------%
% FIGURES AND TABLES                                                      %
% You may include a figure writing                                        %
% \begin{figure}[t]                                                       %
%   \includegraphics[width=.8\textwidth]{xxx.yyy}                         %
%   \caption{Caption.}\label{fig:zzz}                                     %
% \end{figure}                                                            %
% A table is included similar using the table environment.                %
% Sub-captions are created as follows:                                    %
% \begin{figure}[t]                                                       %
%   \subfloat[Caption a]{\includegraphics[width=.45\textwidth]{...}}      %
%   \hfill                                                                %
%   \subfloat[Caption b]{\includegraphics[width=.45\textwidth]{...}}      %
%   \caption{Caption.}                                                    %
% \end{figure}                                                            %
%-------------------------------------------------------------------------%

%-------------------------------------------------------------------------%
% QUOTATION MARKS                                                         %
% Double quotation marks are produced using `` and ''. The opening quote  %
% then looks like a superscript 66 and the closing quote like a           %
% superscript 99.                                                         %
%-------------------------------------------------------------------------%

%-------------------------------------------------------------------------%
% HYPHEN VS. DASH                                                         %
% The hyphen (-) is used for compound words like $n$-dimensional.         %
% (Don't write things like $n-$dimensional!)                              %
% The en-dash (--) is used for number ranges and it can stand for `and',  %
% e.g., between two names: ``Various generalizations of the               %
% Cauchy--Schwarz inequality are presented on pages 63--67.''             %
% To partition a sentence one may use the em-dash (---) with no space     %
% on either side; we prefer the en-dash with a blank on both sides.       %
% The mathematical minus sign is produced by a single - within            %
% mathematics mode.                                                       %
%-------------------------------------------------------------------------%

%-------------------------------------------------------------------------%
% REFERENCES                                                              %
% Use \label and \ref for all cross-references to equations, figures,     %
% tables, sections, subsections, etc.                                     %
%-------------------------------------------------------------------------%

%-------------------------------------------------------------------------%
% BIBLIOGRAPHY                                                            %
% References should be collected at the end of the paper and numbered in  %
% alphabetical order of the authors' names. Titles of journals should be  %
% abbreviated as in Mathematical Reviews. Please do not use small caps    %
% for the names of the authors. The preferred style is shown in the       %
% examples below.                                                         %
% Refer to the bibliography entries using \cite commands.                 %
% Using BibTeX, you simply have to add the names of your databases as     %
%   \bibliography{mydatabase}                                             %
% The file crelle.bst then ensures the correct style.                     %
%-------------------------------------------------------------------------%

\begin{thebibliography}{9}

\bibitem{article}
\textit{T.~Angel} and \textit{I.~E.~Shparlinski},
Article in a journal, J. reine angew. Math. \textbf{647} (2010), 123--156.

\bibitem{preprint}
\textit{C.~Bonnaf\'e},
Preprint, preprint 2008, \url{http://arxiv.org/abs/0805.4100}.

\bibitem{proceedings}
\textit{P. Brooksbank},
Article in a conference proceedings,
in: Finite geometries (Pingree Park 2004), Oxford University Press, Oxford (2006), 1--16.

\bibitem{phdthesis}
\textit{J.~D.~King},
Unpublished dissertation,
Ph.D. thesis, University of Cambridge, 1995.

\bibitem{book}
\textit{F. Sukochev}, \textit{S. Lord} and \textit{D. Zanin},
Book, 2nd ed., De Gruyter, Berlin 2012.

\end{thebibliography}

\end{document}
