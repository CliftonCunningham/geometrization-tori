% !TEX encoding = UTF-8 Unicode
\documentclass[11pt]{amsart}
\title[Geometrization of characters of tori]{Geometrization of characters of tori over non-archimedean local fields}
\usepackage[british]{babel}
\usepackage{datetime}
\date{\today}
\author{Clifton Cunningham}
\address{University of Calgary}
\email{cunning@math.ucalgary.ca}
\author{David Roe}
\address{Pacific Institute for the Mathematical Sciences at the University of Calgary}
\email{roed.math@gmail.com}
\usepackage[utf8]{inputenc}
\renewcommand{\baselinestretch}{1.2}
%\usepackage[notcite]{showkeys}
\usepackage{hyperref}
\usepackage{geometry}
\usepackage{amsthm}
\usepackage{amsmath}
\usepackage{amssymb}
%\usepackage[shortalphabetic]{amsrefs}
\usepackage[alphabetic]{amsrefs}
\renewcommand\MR{\relax}
%\usepackage{xypic}
\usepackage{textcomp}
\usepackage{mathrsfs}
\usepackage{yfonts}
\newcommand{\mathswab}[1]{\operatorname{\textswab{#1}}}
\usepackage[T1]{fontenc}
\usepackage{tikz}
\usetikzlibrary{shapes,arrows,calc,matrix}
\usepackage{tikz-cd}
\usepackage{manfnt}
%\usepackage{makeidx}

%\include{definitions}
\theoremstyle{plain}
      \newtheorem{theorem}{Theorem}[section]
      \newtheorem{proposition}[theorem]{Proposition}
      \newtheorem{lemma}[theorem]{Lemma}
      \newtheorem{corollary}[theorem]{Corollary}
      
      \theoremstyle{definition}
      \newtheorem{definition}[theorem]{Definition}
      
      \theoremstyle{remark}
      \newtheorem{remark}[theorem]{Remark}

%% Global TikZ settings

\tikzset{every picture/.style={>=stealth},label/.style={font=\footnotesize}}

\newcommand{\gal}[1]{{\operatorname{Gal}\hskip-1pt\left( {\bar #1}/#1 \right)}}
\newcommand{\Spec}[1]{{\operatorname{Spec}\hskip-1pt( #1 )}}

\newcommand{\FF}{{\mathbb{F}}}
\newcommand{\ZZ}{{\mathbb{Z}}}
\newcommand{\NN}{{\mathbb{N}}}
\newcommand{\CC}{{\mathbb{C}}}
\newcommand{\QQ}{{\mathbb{Q}}}
\newcommand{\RR}{{\mathbb{R}}}
\newcommand{\EE}{\mathbb{\bar Q}_\ell}
\newcommand{\OK}{\mathcal{O}_K}
\newcommand{\pK}{\mathfrak{p}_K}
\newcommand{\OL}{\mathcal{O}_L}
\newcommand{\OO}[1]{\mathcal{O}_{#1}}
\newcommand{\Zp}{\mathbb{Z}_p}
\newcommand{\Qp}{\mathbb{Q}_p}
%\newcommand{\Fp}{\mathbb{F}_p}
\newcommand{\bFq}{\bar{k}}
\newcommand{\Fq}{k}
\newcommand{\Fqm}{k_m}
%\newcommand{\bFp}{{\mathbb{\bar F}_p}}

\newcommand{\EEx}{\EE^\times}

\DeclareMathOperator{\Gal}{Gal}
\DeclareMathOperator{\W}{W}
\newcommand{\Frob}[1]{\operatorname{F}_{#1}}
\DeclareMathOperator{\Aut}{Aut}
\DeclareMathOperator{\Hom}{Hom}
\DeclareMathOperator{\ord}{ord}
\DeclareMathOperator{\coker}{coker}
\DeclareMathOperator{\Gr}{Gr}
\DeclareMathOperator{\Irrep}{Irrep}
\DeclareMathOperator{\Pic}{Pic}
\DeclareMathOperator{\id}{id}
\DeclareMathOperator{\Ext}{Ext}
\DeclareMathOperator{\Hh}{H}
\DeclareMathOperator{\Res}{Res}
\DeclareMathOperator{\Nm}{Nm}
\DeclareMathOperator{\trace}{Tr}

\newcommand{\cdef}[1]{ {#1}\index{#1} } 
\newcommand{\sheafHom}{{\mathscr{H}\hskip-4pt{\it o}\hskip-2pt{\it m}}}
\newcommand{\abs}[1]{{\vert #1 \vert}}
\newcommand{\ceq}{{\, :=\, }}
\newcommand{\tq}{{\ \vert\ }}
\newcommand{\iso}{{\ \cong\ }}
\newcommand{\obj}{{\text{obj}\, }}
\newcommand{\Gm}[1]{\mathbb{G}_{\hskip-1pt\textbf{m},#1}}
\newcommand{\GN}[1]{\mathswab{#1}}
\newcommand{\bGN}[1]{{\bar{\mathswab{#1}}}}
\newcommand{\TT}{\underline{T}}
\newcommand{\TL}{\underline{T_L}}
\newcommand{\invlim}[1]{\lim\limits_{\overleftarrow{#1}}}
\newcommand{\dirlim}[1]{\lim\limits_{\overrightarrow{#1}}}
\newcommand{\limit}[1]{\mathop{\textsc{lim}}\limits_{#1}}
\newcommand{\colimit}[1]{\mathop{\textsc{colim}}\limits_{#1}}
\newcommand{\cs}[1]{{\mathcal{#1}}}
\newcommand{\gcs}[1]{{\mathcal{\bar #1}}}
\newcommand{\dualgcs}[1]{\gcs{#1}^\dagger}
\newcommand{\dualcs}[1]{\cs{#1}^\dagger}
\newcommand{\QC}{{\mathcal{Q\hskip-0.8pt C}}}
\newcommand{\QCb}{{\QC_0}}
\newcommand{\QCf}{{\QC_f}}
\newcommand{\QCiso}[1]{\QC(#1)_{/\textit{iso}}}
\newcommand{\QCbiso}[1]{\QCb(#1)_{/\text{iso}}}
\newcommand{\QCfiso}[1]{\QCf(#1)_{/\text{iso}}}
\newcommand{\Lgroup}[1]{{\,^L\hskip-1pt{#1}}}
\newcommand{\dualgroup}[1]{{\check{#1}}}
\newcommand{\Lang}{{\operatorname{Lang}}}
\newcommand{\image}{{\operatorname{im}}}
\newcommand{\Weil}[1]{\mathcal{W}_{#1}}
\newcommand{\Loc}{{\operatorname{Loc}}}
\newcommand{\trFrob}[1]{t_{#1}}

\makeatletter
\newcommand{\labitem}[2]{%
\def\@itemlabel{\textbf{#1}}
\item
\def\@currentlabel{#1}\label{#2}}
\makeatother

\renewcommand{\bf}{\bar{f}}
\newcommand{\bg}{\bar{g}}
\newcommand{\bm}{\bar{m}}
\newcommand{\bG}{\bar{G}}
\newcommand{\bH}{\bar{H}}
\newcommand{\tight}[3]{\hspace{-#1pt}{#2}\hspace{-#3pt}}
\newcommand{\GxG}{\text{$G \tight{1}{\times}{1} G$}}
\newcommand{\bGxG}{\text{$\bar{G} \tight{1}{\times}{1} \bar{G}$}}
\newcommand{\bfxf}{\text{$\bar{f} \tight{1}{\times}{1} \bar{f}$}}
\newcommand{\GxxG}{\text{$G \tight{1}{\times}{1} G$}}
\newcommand{\LxL}{\text{$\gcs{L} \tight{0}{\boxtimes}{0} \gcs{L}$}}
\newcommand{\ExE}{\text{$\cs{E}\tight{0}{\boxtimes}{0}\cs{E}$}}
\newcommand{\bExE}{\text{$\gcs{E}\tight{0}{\boxtimes}{0}\gcs{E}$}}
\newcommand{\AxA}{\text{$A \tight{1}{\times}{1} A$}}
\newcommand{\BxB}{\text{$B \tight{1}{\times}{1} B$}}
\newcommand{\GzxGz}{\text{$G^0 \tight{1}{\times}{1} G^0$}}

\newcommand\todo[1]{\ \vspace{5mm}\par \noindent\framebox{\begin{minipage}[c]{0.95 \textwidth} \tt #1\end{minipage}} \vspace{5mm} \par}
\newcommand\Clifton[1]{\marginpar{\smaller\smaller CC: #1}}
\newcommand\David[1]{\marginpar{\smaller\smaller DR: #1}}

\begin{document}


\begin{abstract}
  We introduce and study the category of quasicharacter sheaves on smooth commutative
  group schemes $G$ locally of finite type over finite fields
  $\Fq$. Assuming that the geometric component group of $G$
  is finitely-generated, we show that the group of isomorphism classes
  of quasicharacter sheaves on $G$ is canonically isomorphic to the group
  of characters of $G(\Fq)$. We use this result to exhibit an
  isomorphism between quasicharacters of $T(K)$ and isomorphism
  classes of quasicharacter sheaves on the Greenberg transform of the Néron
  model of $T$, where $K$ is any non-archimdean local field and $T$ is
  any algebraic torus over $K$.
\end{abstract}

\maketitle

\section*{Introduction}

%\input{introduction}
%In his celebrated thesis, John Tate writes 
%\begin{quotation}
%    {\it Concerning the characters of $k^*$ [the
%    multiplicative group of a $\mathfrak{p}$-adic field], the
%    situation is different from that of $k^+$.  First of all, we are
%    interested in continuous multiplicative maps $\alpha \mapsto
%    c(\alpha)$ of $k^*$ into the complex numbers, not only in the
%    bounded ones, and shall call such a map a quasi-character,
%    reserving the word ``character'' for the conventional character of
%    absolute value $1$. Secondly, we shall find no model for the group
%    of quasi-characters, or even for the group of characters, though
%    such a model would be of the utmost importance.}
%  \cite{Tate:thesis}*{\S 2.3}
%\end{quotation}
%\noindent
What is a quasicharacter sheaf?
Let $G$ be a smooth commutative group scheme locally of finite
type over a finite field $\Fq$; $G$ need not be reductive, connected, or of
finite type over $\Fq$. Fix a prime $\ell$ that does not divide
$q$, write $m : \GxG \to G$ for the multiplication morphism,
$\bG$ for the base change of $G$ to $\bFq$ and
$\Frob{G} : \bG \to \bG$ for the Frobenius automorphism.
A quasicharacter sheaf on $G$ is a triple $\cs{L}\ceq
(\gcs{L},\mu,\phi)$ where $\gcs{L}$ is a rank-$1$ $\ell$-adic local system on $\bG$ and $\mu: m^*
\gcs{L} \to \LxL$ and $\phi : \Frob{G}^*\gcs{L} \to \gcs{L}$ are isomorphisms;
$\mu$ and $\phi$ are required to satisfy natural conditions explained below.

In Part 1 of this paper we study the category $\QC(G)$ of quasicharacter sheaves.
Since it is a rigid symmetric monoidal category
the set of isomorphism classes $\QCiso{G}$ forms a group.
Motivated by the sheaf-function dictionary for commutative, connected
algebraic groups \cite{SGA4.5}, we explore the trace of Frobenius morphism
\[
\QCiso{G} \to \Hom(G(\Fq), \EEx)
\]
and show that it is an isomorphism in Theorem \ref{thm:snake}, assuming only
that the geometric component group of $G$ is finitely generated.

We apply Theorem \ref{thm:snake} to the geometrization of quasicharacters
of non-archimedian tori in Part 2.  Let $K$ be a non-archimedian local field
with finite residue field $\Fq$; we place no restriction on the characteristic of $K$.
Write $\OK$ for the ring of integers of $K$ and $\pK$ for the maximal ideal in $\OK$.
Let $T$ be a torus defined over $K$.  The N\'eron model $\TT$ of $T$
is a smooth group scheme over $\OK$ with generic fiber $T$ and a canonical isomorphism
$\TT(\OK) \cong T(K)$.  For each positive integer $d$, let $\GN{T}_d$ be the
Greenberg transform of $\TT \times_{\Spec{\OK}} \Spec{\OK/\pK^d}$; the commutative
group scheme $\GN{T}_d$ plays the r\^{o}le of $G$.  Theorem \ref{thm:application}
now provides a geometrization of quasicharacters of $T$ in the form of a canonical isomorphism
\[
\QCiso{\GN{T}_d} 
\iso
\Hom_{< d}(T(K),\EEx)
\] 
between quasicharacter sheaves on $\GN{T}_d$ and characters of $T(K)$ with depth less than $d$.
We also describe a group scheme $\GN{T}$ simultaneously geometrizing all quasicharacters of $T(K)$.

We have not yet fully explored the applications of such a geometrization, but want to highlight
some of the potential benefits of Theorem \ref{thm:application}.  Characters on $T$ have limited
functoriality: they can be pulled back along maps to $T$, but it is less clear what to do with them
in the presence of maps from $T$ to other reductive groups.  Sheaves, on the other hand, come
equipped with a host of functors, including pushforward.  

In recent years various authors have applied Deligne-Lusztig representations to the
local Langlands correspondence \cite{}.  The next step Lusztig took in studying representations
of finite groups of Lie type was to define character sheaves for reductive groups over $\Fq$.
We hope that our quasicharacter sheaves can provide a stepping stone to a similar theory for
representations of $p$-adic groups.

The scope of the Langlands program has grown explosively in the past several decades, and
different parts of it apply to different kinds of representations.  In particular, the classic local Langlands
correspondence addresses complex (or $\ell$-adic) representations of real and $p$-adic groups, while
the geometric Langlands program aims at representations of groups in characteristic $p$.  The fact that
characteristic $p$ group schemes arise as the space underlying quasicharacter sheaves, regardless of
the characteristic of $K$, provides a potential bridge allowing the methods of the geometric Langlands
program to be applied to the classic context.

Finally, we are tantalized by the possibility of providing an alternate perspective on local class field theory
through quasicharacter sheaves.  The local Langlands correspondence for tori provides an isomorphism
\[
\Hh^1(\Weil{K}, \hat{T}_\ell) \cong \Hom(T(K), \EEx).
\]
A geometric description of the resulting isomorphism $\Hh^1(\Weil{K}, \hat{T}_\ell) \cong \QCiso{\GN{T}}$
would yield a different description of the Artin reciprocity map.

We want to situate the choice of term ``quasicharacter sheaf'' in a historical context.
Analogous sheaves on connected, commutative algebraic groups were first studied
by Deligne \cite{SGA4.5}, and more recently Drinfeld has referred to such objects
as character sheaves.  Our quasicharacter sheaves evolved from this notion, and Theorem \ref{thm:snake}
is a direct generalization of the sheaf-function dictionary to the case of commutative
group schemes which are not necessarily connected or finite type.  The other main use
of the term ``character sheaf'' traces back to Lusztig \cite{}.  Most of his energy is devoted to
handling non-commutative groups; though he does define character sheaves for some
non-connected groups he restricts his attention to reductive groups with cyclic component group.
For extensions of $\ZZ/n\ZZ$ by a torus, it is not difficult to relate his Frobenius-stable character
sheaves to our quasicharacter sheaves.

We close the introduction with a more detailed description of the structure of this paper.
In Section \ref{sec:category} we define quasicharacter sheaves and give an interpretation of
them as Weil sheaves with a Galois action.
Section \ref{sec:Frob} introduces the trace of Frobenius homomorphism,
one direction of the sheaf-function dictionary.
We define pullbacks and products of quasicharacter sheaves in Section \ref{sec:pullback}, and use
them to prove a functoriality result on the trace of Frobenius.
In Section \ref{sec:basechange} we study how quasicharacter sheaves behave under base change
and Weil restriction.
Sections \ref{sec:bounded} and \ref{sec:finite} describe bounded and finite quasicharacter sheaves,
two subcategories of $\QC(G)$.  These alternate descriptions of quasicharacter sheaves play a key
role in the proof of Proposition \ref{prop:restriction}.
In section \ref{sec:connected} we show that bounded and finite quasicharacter sheaves are the same
as standard quasicharacter sheaves when $G$ is connected, and that trace of Frobenius is an isomorphism.
We shift to \'etale group schemes $G$ in Section \ref{sec:etale} and give a more concrete interpretation
of quasicharacter sheaves in this case.  Using a notion of global sections for quasicharacter sheaves,
we give an isomorphism between $\QCiso{G}$ and $\Hh^1(\Weil{\Fq}, \Hom(G(\bFq), \EEx))$.
Section \ref{sec:etalecohom} proves a result in Galois cohomology that shows that trace of Frobenius is
an isomorphism for \'etale group schemes.
In Section \ref{sec:restriction} we show that for non-connected $G$, restriction to the identity component
induces an essentially surjective functor of quasicharacter sheaves.
Section \ref{sec:snake} closes out Part \ref{part1} with the proof of the main theorem, that trace of Frobenius
is an isomorphism for $G$ as long as the geometric component group is finitely generated.

In Section \ref{sec:GN} we introduce N\'eron models of tori and the Greenberg transform, which produce
group schemes over $\Fq$ from algebraic tori $T$ over $K$.
Section \ref{sec:bdchar} then uses Theorem \ref{thm:snake} to give a geometrization of characters of $T$
of bounded depth.
We give a reinterpretation of this result in Section \ref{sec:quasichar}, using a limit process to simultaneously
geometrize all quasicharacters of $T$.  Sections \ref{sec:bdchar} and \ref{sec:quasichar} also include
functoriality results that we hope to use in a future discussion of class field theory.
Finally, we close the paper in Section \ref{sec:transfer} by discussing a transfer principle, which allows
a geometric transfer of characters between characteristic $p$ tori to characteristic $0$ tori.

\subsection*{Acknowledgements}

We would like to thank Masoud Kamgarpour, Pramod Achar, and Hadi Salmasian
for allowing us to hijack our Research in Teams meeting at BIRS into a discussion of
quasicharacter sheaves.  Their knowledge and help have been invaluable.

We also gratefully acknowledge the financial support of PIMS and NSERC,
and the hospitality of BIRS during our weeklong stay in May 2012.
\David{Should these be abbreviated or written out in full?}

\tableofcontents

\part{Smooth commutative group schemes lft over finite fields} \label{part1}

Throughout this paper, $G$ is a commutative group scheme,
smooth and locally of finite type over a finite field $\Fq$, and $m : \GxG\to G$ is its multiplication morphism.

We fix an algebraic closure $\bFq$ of $\Fq$ and write $\bG$ for the
commutative group scheme $G \times_{\Spec{\Fq}} \Spec{\bFq}$ over $\bFq$ obtained by base change from $k$. The multiplication morphism for $\bG$ will be denoted by $\bm$. 

Let $\Frob{G}$ be the Frobenius automorphism of $\bG$.
Recall that $\Frob{G} = \id_{G}\times \Frob{\Spec{\bFq}}$ where
$\Frob{\Spec{\bFq}} = \Spec{\Frob{\Fq}}$ and $\Frob{\Fq}$  is the geometric Frobenius element in $\Gal(\bFq/\Fq)$.

For any group $A$, we will denote by $A^*$ the dual group $\Hom_\text{grp}(A, \EEx)$.


\section{Quasicharacter sheaves}\label{sec:category}

\begin{definition}\label{def:QC}
A \emph{quasicharacter sheaf on $G$} is a triple
$\cs{L}\ceq (\gcs{L},\mu,\phi)$ where:
\begin{enumerate}
\labitem{(CS.0)}{CS.0} $\gcs{L}$ is a rank-$1$ $\ell$-adic local system on $\bG$;
\labitem{(CS.1)}{CS.1} $\mu: \bm^* \gcs{L} \to \LxL$
  is an isomorphism of sheaves on $\bGxG$ such that the following diagram commutes, 
  where $m_3 \ceq m\circ (m\tight{1}{\times}{2}\id) = m\circ (\id\tight{2}{\times}{1} m)$;
%
  \[
  \begin{tikzcd}[row sep=30]
  \bm_3^*\gcs{L} \arrow{rr}{(\bm \tight{1}{\times}{2} \id)^*\mu} \arrow[swap]{d}{(\id \tight{2}{\times}{1} \bm)^*\mu}
    &&  \bm^*\gcs{L} \boxtimes \gcs{L} \dar{\mu \tight{0}{\boxtimes}{1} \id} \\
    \gcs{L} \boxtimes \bm^* \gcs{L} \arrow{rr}{\id \boxtimes \mu}
    &&  \gcs{L} \tight{0}{\boxtimes}{0} \LxL\\
  \end{tikzcd}
  \]
  
%
\labitem{(CS.2)}{CS.2} $\phi : \Frob{G}^* \gcs{L} \to \gcs{L}$ is an
  isomorphism of constructible $\ell$-adic sheaves on $\bG$ compatible with $\mu$ in the sense that the following diagram commutes.
%
  \[
  \begin{tikzcd}[row sep=20]
  \Frob{\GxxG}^* \bm^* \gcs{L} \arrow{rr}{\Frob{\GxxG}^*\mu}
    && \Frob{\GxxG}^*(\LxL)\\
    \arrow[equal]{u} \bm^*  \Frob{G}^* \gcs{L} \arrow[swap]{d}{\bm^* \phi}
    && \Frob{G}^*\gcs{L}\boxtimes \Frob{G}^*\gcs{L} \dar{F\boxtimes \phi} \arrow[equal]{u} \\
    \bm^*\gcs{L} \arrow{rr}{\mu}
    && \LxL\\
  \end{tikzcd}
  \]
\end{enumerate}
%In short, a quasicharacter sheaf on $G$  is a rank-$1$ Weil local system $(\gcs{L},\phi)$ on $G$ equipped with an isomorphism $m^* (\gcs{L},\phi) \to (\gcs{L},\phi)\boxtimes (\gcs{L},\phi)$. 
\end{definition}

Morphisms in the category of quasicharacter sheaves on $G$, denoted by $\QC(G)$, are defined in the natural way:
\begin{enumerate}
\labitem{(CS.3)}{CS.3} if $\cs{L} = (\gcs{L},\mu,\phi)$ and
  $\cs{L'} = (\gcs{L'},\mu',\phi')$ are quasicharacter sheaves on $G$ then
  $\Hom(\cs{L},\cs{L'})$ is the set of morphisms $\alpha : \gcs{L} \to \gcs{L'}$
  of constructible $\ell$-adic sheaves on $\bG$ such that the following diagrams both commute.
  \[
  \begin{tikzcd}[column sep=40]
  \Frob{G}^* \gcs{L} \rar{\Frob{G}^* \alpha} \arrow[swap]{d}{\phi} & \Frob{G}^* \gcs{L'} \dar{\phi'}
  & %\arrow[draw=none]{d}[pos=.4,description]{\text{\normalsize{and}}} 
  & \arrow[swap]{d}{\mu} m^* \gcs{L} \rar{m^* \alpha} & m^* \gcs{L'} \dar{\mu'} \\
  \gcs{L} \rar{\alpha} & \gcs{L'}
  & {} & \LxL \rar{\tight{1}{\alpha\boxtimes \alpha}{1}} & \gcs{L'} \tight{0}{\boxtimes}{0} \gcs{L'}
  \end{tikzcd}
  \]
\end{enumerate}

%\begin{remark} \label{rem:CS-action}
The rule $(\gcs{L},\mu,\phi) \mapsto (\gcs{L},\phi)$ defines a forgetful functor from quasicharacter sheaves on $G$ to ($\ell$-adic) Weil sheaves on $G$ \cite{Deligne:Weil2}*{Def.~1.1.10 (i)}.

As mentioned after \cite{Deligne:Weil2}*{Def.~1.1.10}, a Weil sheaf $(\gcs{L},\phi)$ on $G$ may be interpreted as a constructible $\ell$-adic sheaf $\gcs{L}$ on $\bG$ together with an action of the Weil group $\Weil{\Fq}$ on $\gcs{L}$ compatible with the action of $\Gal(\bFq/\Fq)$ on $\bG$. Since we will need this point of view later, we review that interpretation here. Setting 
\[
\varphi(\Frob{\Fq}^n) \ceq  \phi \circ \Frob{G}^*(\phi) \circ \cdots \circ (\Frob{G}^{n-1})^*(\phi)
\]
defines, for every $w\in \Weil{\Fq}$, an isomorphism
$\varphi(w) : w_G^* \gcs{L}\to \gcs{L}$, such that
$\varphi(uv) = \varphi(v) \circ v_G^* \varphi(u)$ for all $u,v\in \Weil{\Fq}$.
Note that $\phi = \varphi(\Frob{\Fq})$. 
If we further define $\varphi_1(w) \ceq (w_G)_*(\varphi(w)^{-1})$ then
$\varphi_1(w) : (w_G)_* \gcs{L}\to \gcs{L}$ is an isomorphism and
$\varphi_1(uv) = \varphi_1(u) \circ (u_G)_* \varphi_1(v)$ for all $u,v\in \Weil{\Fq}$.
This means the pair $(\gcs{L},\varphi_1)$ is almost an action $\Weil{\Fq}$ on $\gcs{L}$ compatible with the action of $\Gal(\bFq/\Fq)$ in the sense of \cite{SGA4.5}*{Expos\'e XIII,~1.1}. The pair $(\gcs{L},\varphi_1)$ fails to fulfill the requirements of \cite{SGA4.5}*{Expos\'e XIII,~1.1} only because $\Weil{\Fq}$ is profinite. This fact goes to the heart of the notion of a Weil sheaf and is one of the essential aspects of our definition of $\QC(G)$. 
%\end{remark}

\section{Trace of Frobenius}\label{sec:Frob}

%\begin{proposition}\label{tensor}
The category $\QC(G)$ of quasicharacter sheaves on $G$ is a rigid symmetric monoidal category \Clifton{I still haven't found a good reference for this!} under
$\cs{L} \otimes \cs{L'} \ceq (\gcs{L}\otimes\gcs{L'}, \mu\otimes\mu', \phi\otimes \phi')$
with duals given by applying the functor 
$\sheafHom(\ - \ ,\EE)$ (sheaf hom) to
conditions \ref{CS.0}, \ref{CS.1} and \ref{CS.2}.
%\end{proposition}
(The category of characters sheaves on $G$ is not abelian, so $\QC(G)$
is not a tensor category in the sense of \cite{Deligne:tensorielles}.)
%\begin{proof}\end{proof}
%
This rigid monoidal category structure for $\QC(G)$ puts a group
structure on the set $\QCiso{G}$ of isomorphism classes in
$\QC(G)$. Our main result regarding quasicharacter sheaves on $G$ (Theorem~\ref{thm:snake})
determines $\QCiso{G}$ under the hypothesis that the component group scheme for $G$ is geometrically finitely generated; see Theorem~\ref{thm:snake}. To do this, we use the trace of Frobenius.

\begin{definition}\label{def:trFrob}
Let $(\gcs{L},\phi)$ be a Weil sheaf on $G$. on constructible $\ell$-adic sheaf on $\bG$. Every $g\in G(\Fq)$
determines a point $\bg$ on $\bG$ fixed by $\Frob{G}$ and
therefore an automorphism $\phi_{\bg}$ of $\gcs{L}_{\bg}$. The \emph{trace of Frobenius} of $(\gcs{L},\phi)$ at $g\in G(\Fq)$ is 
\[
\trFrob{(\gcs{L},\phi)}(g) \ceq \trace(\phi_{\bg}, \gcs{L}_{\bg}).
\]
If $\cs{L} = (\gcs{L},\mu,\phi)$ is a quasicharacter sheaf, 
then we will abbreviate $\trFrob{(\gcs{L},\phi)}$ to $\trFrob{\cs{L}}$.
\end{definition}

The isomorphism $\mu : \bm^* \gcs{L} \to \LxL$ guarantees 
that the function $\trFrob{\cs{L}} : G(\Fq)\to \EEx$ is a group homomorphism.
It is also clear that $\trFrob{\cs{L}\otimes \cs{L'}} = \trFrob{\cs{L}} \cdot \trFrob{\cs{L'}}$
(pointwise multiplication of functions) and $\trFrob{\cs{L}} = \trFrob{\cs{L'}}$
if $\cs{L} \iso \cs{L'}$ in $\QC(G)$. In this way we obtain a group homomorphism
\[
\cdef{\trFrob{G}} : \QCiso{G} \longrightarrow G(\Fq)^*,
\qquad \text{defined by}\qquad \cs{L} \mapsto \trFrob{\cs{L}}.
\]
This isomorphism is functorial, in the following sense.

\begin{proposition}\label{prop:functorialG}
Trace of Frobenius $\QCiso{G} \to G(\Fq)$ is a natural transformation 
between the following two additive functors from the category of smooth commutative
group schemes locally of finite type over $\Fq$ to the category of abelian groups:
\begin{enumerate}
\item[$F_1$:] $G \mapsto \QCiso{G}$;
\item[$F_2$:] $G \mapsto G(\Fq)$.
\end{enumerate}
\end{proposition}

The proof of Proposition~\ref{prop:functorialG} requires the results from the next section.

\section{Pull-back and products}\label{sec:pullback}

\begin{proposition}\label{prop:pullback}
  If $f : H\to G$ is a morphism of commutative group schemes locally of finite type over $\Fq$ then
  $(\gcs{L},\mu,\phi) \mapsto (\bf^*\gcs{L},(\bfxf)^*\mu,\bf^*F)$
  defines a monoidal functor $f^* : \QC(G) \to \QC(H)$ such that
  \[
  \begin{tikzcd}[row sep=20, column sep=30]
   \QCiso{G} \rar{f^*} \arrow[swap]{d}{\trFrob{G}} & \QCiso{H} \dar{\trFrob{H}} \\
   G(\Fq)^* \rar & H(\Fq)^*
  %\Hom_\text{grp}(G(\Fq),\EEx) \rar & \Hom_\text{grp}(H(\Fq),\EEx)
  \end{tikzcd}
  \]
  is a commutative diagram of groups, where the lower homomorphism is
  dual to $f : H(\Fq)\to G(\Fq)$.  Moreover, $(f\circ g)^* = g^* \circ f^*$.
\end{proposition}

\begin{proof}
  Let $\cs{L}$ be a quasicharacter sheaf on $G$. We start by showing that
  $\bf^*\gcs{L}$ is locally constant on connected components of
  $\bH$. Let $c_j :\bH^j \hookrightarrow \bH$ be any
  connected component and let $i$ be the image of $j$ under the group
  homomorphism $\pi_0(\bG) \to \pi_0(\bH)$ obtained by
  applying the functor $\pi_0$ to $\bf$; see \cite{vdG&M}*{III, 3.28}, for example.
  \Clifton{I haven't found a better reference for this yet!}
  Write $c_i : \bG^i \hookrightarrow \bG$ for
  the inclusion of $\bG^i$ and $\bf^j : \bH^j \to \bG^i$
  for the restriction of $f$ to $\bH^j$.  Since $\cs{L}$
  is a quasicharacter sheaf, $\gcs{L}$ is locally constant on the connect
  components of $\bG$; thus, there is a finite etale covering
  $a_i : X_i \to \bG^i$ such that $a_i^* (\gcs{L}\vert_{\bG^i})$
  is constant. Let $b_j : Y_j \to \bH^j$ be the
  pull-back of $a_i$ along $\bf^j$; let $g_j : Y_j \to X_i$ be
  the map opposite $\bf^j$. Then $b_j$ is a finite etale covering
  of $\bH^j$ and
  \[
  b_j^* \left( (\bf^* \gcs{L})\vert_{\bH^j}\right)
  = (\bf\circ c_j \circ b_j)^*\gcs{L}
  = (c_i\circ a_i\circ g_j)^*\gcs{L} = g_j^* a_i^* (\gcs{L}\vert_{\bG^i}).
  \]

  Since $a_i^* (\gcs{L}\vert_{\bG^i})$ is a constant sheaf and since the pull-back
  of a constant sheaf (along $g_j$, in this case) is a constant sheaf,
  it follows that $b_j^* \left( (\bf^*\gcs{L})\vert_{\bH^j}\right)$
  is a constant sheaf. Thus, $\bf^*\gcs{L}$ is locally constant on
  $\bH^j$. Since $(\bf^*\gcs{L})_{\bg} = \gcs{L}_{f(\bg)}$ for every
  geometric point ${\bg}$ on $\bG$, and since $\gcs{L}$ has
  rank $1$ on connected components of $\bG$, it follows that
  $f^*\gcs{L}$ also has rank $1$ on connected components of $\bG$.
  This completes the proof that $\bf^*\gcs{L}$ satisfies condition \ref{CS.0}.

  To see that $(\bfxf)^* \gcs{L}$ satisfies
  condition \ref{CS.1} simply apply the functor $(\bfxf)^*$
  to \ref{CS.1} for $\cs{L}$ and use the canonical isomorphism
  $(\bfxf)^*(\LxL) \iso \bf^*\gcs{L} \tight{-3}{\boxtimes}{-3} \bf^*\gcs{L}$
  (a few times). Likewise, to see that $\bf^*\gcs{L}$ satisfies condition
  \ref{CS.2}, apply the same functor to \ref{CS.2} for $\gcs{L}$.
  Then use the fact that $f$ is defined over $\Fq$, so
  $(\bfxf)^*\Frob{\GxxG}^* \iso \Frob{\GxxG}^* (\bfxf)^*$,
  and the fact that $f$ is a morphism of group schemes, so
  $(\bfxf)^* \bm^*\iso \bm^* \bf^*$
  (isomorphisms of functors on constructible sheaves).

  To see that $(\gcs{L},\mu,\phi) \mapsto (\bf^*\gcs{L},(\bfxf)^*\mu,\bf^*F)$
  defines a functor $f^* : \QC(G) \to \QC(H)$ one applies the functors $\bf^*$ and
  $\bf^*\tight{1}{\times}{1}\bf^*$ to \ref{CS.3} and argue as
  above. It follows from the fact that sheaf hom commutes with these
  functors that $f^* : \QC(G) \to \QC(H)$ is a monoidal functor.
%
  The fact that the functor $f^* : \QC(G) \to \QC(H)$ commutes with
  the trace of Frobenius, in the sense above, follows immediately from
  the definitions; see \cite{Laumon}*{1.1.1.2}, where the ambient
  hypothesis that $X$ is of finite type over $\Fq$ can be replaced by
  the hypothesis that $X$ is locally of finite type over $\Fq$.
  
  Finally, the fact that $(f\circ g)^* = g^* \circ f^*$ follows from the analogous
  statements about the pullbacks functor on $\ell$-adic constructible sheaves.
 % \David{Is this explanation of $(f\circ g)^* = g^* \circ f^*$ sufficient?}
\end{proof}

Consider the short exact sequence in the category of smooth group
schemes, locally of finite type over $\Fq$, defining the component
group scheme for $G$:
\begin{equation}\label{eq:pi-2}
\begin{tikzcd}
0 \rar & G^0 \rar & G \rar & \pi_0(G) \rar & 0.
\end{tikzcd}
\end{equation}
Since $\pi_0(G)$ is an commutative group scheme, smooth (in fact, etale) and locally of finite type over $\Fq$, 
it follows from Proposition~\ref{prop:pullback} that \eqref{eq:pi-1} defines a sequence of functors
\[
\begin{tikzcd}
\QC(0) \rar & \QC(\pi_0(G)) \rar & \QC(G) \rar & \QC(G^0) \rar & \QC(0)
\end{tikzcd}
\]
which induces, upon passage to isomorphism classes of objects, a sequence of group homomorphisms 
\begin{equation}\label{eq:pi-1}
\begin{tikzcd}
\QCiso{0} \rar & \QCiso{\pi_0(G)} \rar & \QCiso{G} \rar & \QCiso{G^0} \rar & \QCiso{0}.
\end{tikzcd}
\end{equation}
It is easy to see that $\QCiso{0}$ is trivial. In this paper we will find $\QCiso{G^0}$ and $\QCiso{\pi_0(G)}$ under the hypothesis that $\pi_0(G)$ is geometrically finitely generated; 
the snake lemma will then allow us to show
\[
\QCiso{G} \iso G(\Fq)^*,%\Hom_\text{grp}(G(\Fq),\EE^\times),
\]
canonically.
%\section{Products} 
We end this section with a simple result on products of quasicharacter sheaves, which is needed to show that the isomorphism above is functorial.

\begin{proposition}\label{prop:product}
The rule $\boxtimes : (\cs{L}_1,\cs{L}_2)\to \cs{L}_1\boxtimes\cs{L}_2$ defines an equivalence of categories 
\[
\QC(G_1)\times(G_2) \to \QC(G_1\times G_2)
\]
and 
\[
\begin{tikzcd}[column sep=60]
\arrow{d}{\trFrob{G_1} \times \trFrob{G_2}} \QCiso{G_1}\times \QCiso{G_2} \arrow{r}{\boxtimes} & \arrow{d}{\trFrob{G_1\times G_2}} \QCiso{G_1\times G_2}\\
(G_1)(\Fq)^*\times (G_2)(\Fq)^* \arrow{r}{(\chi_1,\chi_2)\mapsto \chi_1\otimes \chi_2}  & (G_1\times G_2)(\Fq)^*
\end{tikzcd}
\]
commutes.
\end{proposition}

\begin{proof}
The only non-trivial part is essential surjectivity. Set $G \ceq G_1\times G_1$. Let $\cs{L}$ be a geochar on $G$. Define $f : G\to G\times G$ by $f(g_1,g_2) = (g_1,e_2,e_1,g_2)$. Observe that $m\circ f = \id_G$. 
Then the isomorphism $\mu : m^* \gcs{L} \to \gcs{L}\boxtimes \gcs{L}$ gives an isomorphism
\[
f^*\mu : f^* m^* \gcs{L} \to f^*(\gcs{L}\boxtimes \gcs{L}) = f^*p_1^*\gcs{L}\otimes f^* p_2^*\gcs{L}.
\]
Observe that $p_1\circ f = r_1 \circ q_1$ and $f_2 = r_2\circ f$ where $p_1 , p_2 : G\times G \to G$ are the usual things and $r_1 : G_1\times G_2 \to G_1$ and $r_2 : G_1\times G_2 \to G_2$ are the obvious things and $q_1 : G_1\to G_1\times G_2$ and  $q_2 : G_2\to G_1\times G_2$ are the usual things. Then $f^*p_1^*\gcs{L} = r_1^* q_1^* \gcs{L}$ and $f^*p_2^*\gcs{L} = r_2^* q_2^* \gcs{L}$. Now $\cs{L}_1 \ceq r_1^* \cs{L}$ is a geochar on $G_1$ and $\cs{L}_2 \ceq r_2^* \cs{L}$ is a geochar on $G_2$, by Proposition~\ref{prop:pullback}. Then
$ f^*p_1^*\gcs{L}\otimes f^* p_2^*\gcs{L} = r_1^*\gcs{L}_1\otimes r_2^* \gcs{L}_2 = \gcs{L}_1\boxtimes \gcs{L}_2$. In this way, $\mu$ determines an isomorphism
\[
f^*\mu : \gcs{L} \to  \gcs{L}_1\boxtimes \gcs{L}_2.
\]
So $\boxtimes$ is essentially surjective.
\end{proof}

\begin{proof}[Proof of Proposition~\ref{prop:functorialG}]
The first part of Proposition~\ref{prop:pullback} shows that $F_1$ is a functor, while the second part shows that Trace of Frobenius is a natural transformation $t: F_1 \to F_2$. When further combined with Proposition~\ref{prop:product}, we see that $F_1$ is an additive functor and $t: F_1 \to F_2$ is a natural transformation between additive functors.
\end{proof}

\section{Base change}\label{sec:basechange}

When using quasicharacter sheaves to study characters, it is useful to understand
how quasicharacter sheaves behave under change of fields.
Let $k'$ be a finite extension of $k$. Then $k\to k'$ induces a group homomorphism $i_{k'/k} : G(k) \hookrightarrow G(k')$ which, in turn, gives
\[
i_{k'/k}^* : G(k')^* \to G(k)^* \qquad \text{defined by}\qquad \chi \mapsto \chi\circ i_{k'/k}.
\]
We can interpret this operation on characters in terms of quasicharacter sheaves, as follows.
%

\begin{proposition} \label{prop:csbe}
Set $G_{k'} = G\times_\Spec{k} \Spec{k'}$ and let
\[
\QC(\Res_{k'/k}(G_{k'})) \xrightarrow{\iota^*} \QC(G)
\]
be the functor obtained by pull-back along the canonical closed immersion of $k$-schemes  $\iota : G \hookrightarrow \Res_{k'/k}(G_{k'})$ \cite{BLR}*{\S 7.6}. 
The following diagram commutes:
\[
\begin{tikzcd}
\QCiso{\Res_{k'/k}(G_{k'})} \arrow[two heads]{r}{\iota} \dar[swap]{\trFrob{\Res_{k'/k}(G_{k'})}} & \QCiso{G} \dar{\trFrob{G}} \\
G(k')^* \arrow[two heads]{r}{i_{k'/k}^*} & G(k)^*.
\end{tikzcd}
\]
\end{proposition}

In the opposite direction, let $\Nm_{k'/k} : G(k') \to G(k)$ be the norm map and consider the group homomorphism:
\[
\Nm_{k'/k}^* : G(k)^* \to G(k')^* \qquad \text{defined by}\qquad \chi \mapsto \chi\circ \Nm_{k'/k}.
\]
We can also interpret this operation on characters in terms of quasicharacter sheaves, as follows.

If $\cs{L} = (\gcs{L}, \mu, \phi)$ is a quasicharacter sheaf on $G$ we may define
a quasicharacter sheaf $\cs{L}' = (\gcs{L}, \mu, \phi_{k'})$ on 
$G_{k'}\ceq G\times_{\Spec{k}} \Spec{k'}$ by setting
\[
\phi_{k'} = \phi \circ \Frob{G}^*(\phi) \circ \cdots \circ (\Frob{G}^{n-1})^*(\phi).
\]
The commutativity of the diagram (CS.2) for $\phi_{k'}$ 
follows from the fact that $\Frob{G_{k'}} = \Frob{G}^n$.
Note that we may also think about the construction of $\phi_{k'}$ from $\phi$
as taking the action $\varphi$ of $\Weil{k}$ on $\gcs{L}$ 
defined in Section~\ref{sec:category} and restricting it to the subgroup $\Weil{k'}$.

\begin{proposition}\label{prop:basechange}
 The rule $\nu_{k'/k}: (\gcs{L}, \mu, \varphi) \mapsto (\gcs{L}, \mu, \varphi\vert_{\Weil{\Fq}})$ 
 defines a monoidal functor $\QC(G) \to \QC(G_{k'})$. 
 Moreover, the following diagram commutes, 
 where the top map is induced by $\QC(G) \to \QC(G_{k'})$ 
 and the bottom map is induced by the norm $G(k') \rightarrow G(k)$:
\[
\begin{tikzcd}[column sep=60]
\QCiso{G} \rar{\nu_{k'/k}} \dar{\trFrob{G}} & \QCiso{G_{k'}} \dar{\trFrob{G_{k'}}} \\
G(k)^*  \rar{\Nm_{k'/k}^*} & G(k')^*.
\end{tikzcd}
\]
\end{proposition}


Finally, we explain how to geometrize the canonical isomorphism between characters of $G'(k')$ and of $(\Res_{k'/k}G')(k)$.
The base change $(\Res_{k'/k}G')_{k'}$ of $\Res_{k'/k}G'$ to $k'$
decomposes into a product of copies of $G'$, indexed by elements of $\Gal(k'/k)$:
\[
(\Res_{k'/k}G')_{k'} \cong \prod_{\Gal(k'/k)} G'.
\]
The component corresponding to the identity element yields a natural inclusion 
\[
G' \hookrightarrow (\Res_{k'/k}G')_{k'}
\]
of $k'$-schemes.  The definition of base change yields a map of $k$-schemes
\[
(\Res_{k'/k}G')_{k'} \to \Res_{k'/k}G'.
\]
Consecutive pullback along these maps yields a functor
\[
\QC(G') \to \QC(\Res_{k'/k}G').
\]

\begin{proposition}
The following diagram commutes:
\[
\begin{tikzcd}
\QCiso{G'} \dar{\trFrob{G'}} \rar & \QCiso{\Res_{k'/k} G'} \dar{\trFrob{\Res_{k'/k} G'}} \\
 G'(k')^* \rar & (\Res_{k'/k}G')(k)^*
\end{tikzcd}
\]
\end{proposition}

\section{Bounded quasicharacter sheaves}\label{sec:bounded}

\begin{definition}
Let $\QCb(G)$ be the category of pairs $(\cs{L}_0,\mu_0)$ 
where $\cs{L}_0$ a rank-$1$ $\ell$-adic local system on $G$, 
equipped with an isomorphism $\mu_0 : m^* \cs{L}_0 \to \cs{L}_0 \boxtimes \cs{L}_0$ 
satisfying the analogue of Condition~\ref{CS.1} on $G$; 
morphisms in $\QCb(G)$ are defined as in (the second part of) Condition~\ref{CS.2}. 
This is the category of \emph{bounded quasicharacter sheaves} on $G$. 
\end{definition}

The category of bounded quasicharacter sheaves is a rigid monoidal category in the obvious way. 
%with duals given by $\cs{L}_0^\vee \ceq \sheafHom(\cs{L}_0,\EE)$.

\begin{proposition}
$\QCb(G)$ is (equivalent to) a full subcategory of $\QC(G)$.
\end{proposition}

\begin{proof}
 Let $b_G : {\bar G} \to G$ be the pull-back of $\Spec{\bFq} \to \Spec{\Fq}$ along $G\to \Spec{\Fq}$.
 Let $(\cs{L}_0,\mu_0)$ be a bounded quasicharacter sheaf on $G$. 
 Then $\cs{L}_0$ is, in particular, an $\ell$-adic constructible sheaf on $G$. 
 Then $b_G^* \cs{L}_0$ comes equipped with an isomorphism 
 $\phi : \Frob{G}^* b_G^*\cs{L}_0 \to b_G^* \cs{L}_0$; 
 moreover, the functor $\cs{L}_0 \mapsto (b_G^* \cs{L}_0,\phi)$  
 from $\ell$-adic constructible sheaves on $G$ to $\ell$-adic constructible sheaves on $G$
 is full and faithful; see \cite{SGA7.2}*{Expos\'e XIII} or even \cite{BBD}*{Prop. 5.2.1}. 
 This functor preserves local constancy, so takes local systems to local systems. 
 Set $\mu = b_{G\times G}^*\mu_0$; clearly, this satisfies Condition~\ref{CS.2} 
 with $b_G^*\cs{L}_0$ playing the role of $\gcs{L}$.
 Moreover, $\phi$ is compatible with $\mu$ in the sense of Condition~\ref{CS.3}.
 It follows that functor $(\cs{L}_0,\mu_0) \mapsto (b_G^*\cs{L}_0,b_{G\times G}^* \mu_0, \phi)$, denoted by  $B_G^*: \QCb(G) \hookrightarrow \QC(G)$ below, is full and faithful.
\end{proof}

The full subcategory $\QCb(G)$ of bounded quasicharacter sheaves is an essentially proper subcategory of $\QC(G)$, in general. Indeed, it follows from our main result regarding quasicharacter sheaves, Theorem~\ref{thm:snake}, that the restriction of $\trFrob{G} : \QCiso{G} \to G(\Fq)^*$ to the subgroup $\QCbiso{G}$ is an isomorphism onto the subgroup of $\chi \in G(\Fq)^*$ such that the image of $\chi$ is bounded in $\EEx$; see Section~\ref{sec:snake}.

\section{Finite quasicharacter sheaves}\label{sec:finite}

 %In this section we describe a full subcategory of $\QCb(G)$
 %which will illuminate some features of $\QCb(G)$ and also play a role in some arguments below.
 
To define the category of finite quasicharacter sheaves we must work a bit harder.
Consider the following category $C(G)$.
Let us say that $f : H\to G$ is a \emph{discrete isogeny} if it is a a finite, surjective etale morphism and a morphism of commutative group $\Fq$-schemes, for which the action of $\Gal(\bFq/\Fq)$ on the etale group scheme $f^{-1}{\bar e}$ is trivial.
An object in $C(G)$ is a pair $(f,\psi)$ where: 
$f : H\to G$ is a discrete isogeny 
and $\psi : \ker f\to \Aut(V)$ is a representation on a finite-dimensional $\EE$-vector space $V$;
a map $(f,\psi) \to (f',\psi')$ in $C(G)$ is a pair $(\alpha,T)$ 
where $\alpha$ is a morphism of schemes such that $f' = f\circ \alpha$
and $T : V\to V'$ is a linear transformation equivariant 
for the action of $\ker f'$ on $V'$ by $\psi'$ 
and the action of $\ker '$ on $V$ by $\psi \circ \alpha$.
\[
\begin{tikzcd}
V \arrow{d}{T} & \Aut(V) & \arrow[swap]{l}{\psi} \ker f \arrow[hook]{r} & H \arrow{r}{f} & \arrow[equal]{d} G\\
V' & \Aut(V) & \arrow[swap]{l}{\psi} \arrow[swap]{u}{\alpha\vert_{\ker f}} \ker f \arrow[hook]{r} & \arrow[swap]{u}{\alpha} H' \arrow{r}{f'} & G\\
\end{tikzcd}
\]
We say that $(\alpha,T)$ in $C(G)$ is a \emph{weak isomorphism} if $\alpha$ is surjective
and $T$ is an isomorphism of vector spaces.
Let \cdef{$C(G)[W^{-1}]$} be the localization of $C(G)$ at weak isomorphisms.

We put a tensor category structure on $C(G)$ 
and a rigid monoidal category structure on $C_1(G)$ as follows.
Let $(f,\psi)$ and $(f',\psi')$ be objects in $C(G)$. 
The category of isogenies to $G$ admits products: 
write $f\times_G f' : H\times_G H' \to G$ for the product of $f : H\to G$ and $f' : H'\to G$. 
Since $\ker(f\times_G f') \iso (\ker f)\times (\ker f')$, canonically,
the tensor product of the representations $\psi : \ker f\to \Aut(V)$ and $\psi' : \ker f'\to \Aut(V')$ 
is the representation $\psi \otimes\psi' :  \ker(f\times_G f') \to \Aut(V\otimes V')$. 
This allows us to set $(f,\psi)\otimes(f',\psi') \ceq (f\times_G f' ,\psi\otimes\psi')$. 
The same fact makes sense of $(f,\psi)\oplus(f',\psi')\ceq (f\times_G f' ,\psi\oplus\psi')$. 
The dual of $(f,\psi)$ is $(f,\psi)^\vee \ceq (f,\psi^\vee)$ 
where $\psi^\vee : \ker f \to \Aut(V^\vee)$ and $V^\vee = \Hom_\text{vec}(V,\EE)$. 
The verification that these operations make $C(G)$ a tensor category is straightforward.
Likewise, $C(G)[W^{-1}]$ is a tensor category.
Let $C_1(G)$ be the rigid monoidal full subcategory of $C(G)$ consisting of objects $(f,\psi)$ such that $\psi$ is a one-dimensional representation.

\begin{definition}
The category of \emph{finite quasicharacter sheaves} of $G$, denoted by $\QCf(G)$, is the rigid symmetric monoidal category obtained by localizing $C_1(G)$ at weak isomorphisms.
\end{definition}

\begin{proposition}\label{prop:bounded}
 $\QCf(G)$ is (equivalent to) a full subcategory of $\QCb(G)$.
\end{proposition}

\begin{proof}
 To prove the proposition we exhibit a full and faithful monoidal functor $L_G : \QCf(G) \to \QCb(G)$.
 %
 We will do this by constructing a faithful monoidal functor $L : C_1(G) \to \QCb(G)$ 
 such that $L(\alpha,T)$ is an isomorphism precisely when $(\alpha,T)$ is a weak isomorphism. 
 Factoring along $C_1(G) \to \QCb(G)$ will produce the full and faithful monoidal functor we seek.

 Now we introduce a faithful functor $L : C(G) \to \Loc(G)$ to the category of local systems on $G$.
 For any object $(f,\psi)$ in $C(G)$, let $L(f,\psi)$ be the $\psi$-isotypic component of $f_* V_H$, 
 where $V_H$ refers to $V$ as a constant sheaf on $H$. 
 We may identify the stalk $(f_* V_H)_{\bar g}$ of $f_* V_H$ at a geometric point ${\bar g}$ on $G$ 
 with the $\EE$-vector space $\Hom_\text{set}(\abs{f^{-1}({\bar g})},V)$ 
 and the stalk $L(f,\psi)_{\bar g}$ of $L(f,\psi)$ at a geometric point ${\bar g}$ on $G$ 
 with the $\EE$-vector space consisting of those functions $s: \abs{f^{-1}({\bar s})} \to V$ such that 
 $s(a\cdot {\bar h}) = \psi(a)(s({\bar h}))$ for all $a\in \ker f$ and for all ${\bar h} \in f^{-1}({\bar g})$.
 Since $L(f,\psi)$ is trivialized by $f$, it is a local system on $G$.
 For any map $(\alpha,T)$ in $C(G)$, let $L(\alpha,T) : L(f,\psi) \to L(f',\psi')$ be the homomorphism of local systems defined 
 %at geometric points $L(\alpha,T)_{\bar g} : L(f,\psi)_{\bar g} \to L(f',\psi')_{\bar g}$ 
 by $s \mapsto s' \ceq T\circ s\circ \alpha$. 
 This defines a monoidal functor $L$ from $C(G)$ to $\Loc(G)$. 
 This functor commutes with pull-back along any morphism of group schemes.
  
 Now consider the localization $C(G)[W^{-1}]$ of $C(G)$ at weak isomorphisms.
 Morphisms in $C(G)[W^{-1}]$ are simple to describe because category $C(G)$ 
 admits a calculus of left fractions
 %\footnote{See {\tt http://ncatlab.org/nlab/show/calculus+of+fractions}.} 
 for weak isomorphisms:
% every isomorphism in $C(G)$ is a weak isomorphism; 
% push-outs exist in $C(G)$ and the push-out of a weak isomorphism is a weak isomorphism; 
% and weak isomorphisms enjoy the left-cancellation property. 
% It follows that
 a morphism in $C(G)[W^{-1}]$ is the equivalence class of 
  \begin{equation}\label{eq:morGCf}
   \begin{tikzcd}
   (f,\psi) \arrow{r}{(\alpha_1,T_1)} & (f_1,\psi_1) & \arrow[swap]{l}{(\beta_1,U_1)} (f',\psi')
   \end{tikzcd}
  \end{equation}
 where $(\alpha',T') : (f,\psi) \to (f',T')$ is any map in $C(G)$ 
 and $(\beta_1,U_1) : (f_1,\psi_1) \to (f',\psi')$ is a weak isomorphism.

 Observe that $L$ takes weak isomorphisms in $C(G)$ to isomorphisms in $\Loc(G)$. 
 To see this, suppose $(\beta,U) : (f,\psi)\to (f',\psi')$ is a weak isomorphism. 
 Since $U$ is an isomorphism, we may replace $V'$ with $V$ and $U$ with $\id_V$, without loss of generality. 
 Then the equivariance condition on $U$ becomes $\psi' = \psi\circ \beta$ on the finite etale group scheme $\ker f'$. 
 Since $f$ and $f'$ are finite etale, and $f' = f\circ \beta$, and $\beta$ is surjective, $\beta$ is an etale cover; therefore, $\beta$ admits a section $\beta' : H \to H'$. This provides the inverse to $L(\beta,U) : L(f,\psi) \to L(f',\psi')$.
 
 Conversely, if $L(\alpha,T)$ is an isomorphism of local systems, 
 then $T$ must be an isomorphism of vector spaces and $\beta$ must be surjective, whence $(\alpha,T)$ is a weak isomorphism: $W$ is saturated.
  
 Since $L : C(G) \to \Loc(G)$ takes weak isomorphisms to isomorphisms,
 it factors, uniquely, through the localisation functor $C(G) \to C(G)[W^{-1}]$ 
 to define a faithful functor $L_{W} : C(G)[W^{-1}] \to \Loc(G)$. 
 Since $W$ is saturated, the functor $L_{W} : C(G)[W^{-1}] \to \Loc(G)$ is full. 
   
 By Lemma~\ref{lemma:finite-iso}, each $(f,\psi) \in \QCf(G)$ determines 
 an isomorphism $\mu(f,\psi) : m^*(f,\psi) \to (f,\psi) \boxtimes (f,\psi)$ in $\QCf(G\times G)$, canonically.
 It follows immediately that the functor 
 \[
 L_G  \ceq  L_{W}\vert_{\QCf(G)} : \QCf(G) \to \QCb(G),
 \] 
 given on objects by $(f,\psi) \mapsto (L(f,\psi), L_{W}(\mu(f,\psi)))$,
 is a full and faithfull monoidal functor, completing the proof.
\end{proof}


\begin{lemma}\label{lemma:finite-pull-back}
If $g : G'\to G$ is a morphism of group schemes then $(f,\psi) \mapsto (f_g,\psi_g)$ defines a monomial functor $g^* : C(G)[W^{-1}] \to C(G')[W'^{-1}]$ where $f_g : H\times_G G'\to G'$ is a pull-back of $f$ along $g$ and $\psi_g : \ker f_g \to \Aut(V)$ is defined by the diagram below.
\[
\begin{tikzcd}
V & \Aut(V) & \arrow[swap]{l}{\psi} \ker f \arrow[hook]{r} & H \arrow{r}{f} &  G\\
  &  & \arrow[dashed]{ul}{\psi_g} \arrow[swap]{u}{(g_f)\vert_{\ker g_f}} \ker f_g \arrow[hook]{r} & \arrow[swap]{u}{g_f} H\times_G G' \arrow{r}{f_g} & \arrow{u}{g} G'.
\end{tikzcd}
\]
\end{lemma}

\begin{proof}
The key point is that pull-back along $g$ takes isogenies to isogenies, and maps over $G$ to maps over $G'$, as pictured below.
\[
\begin{tikzcd}[row sep=10,column sep=10]
\ & V \arrow{dd}{T} && \Aut(V) && \arrow[swap]{ll}{\psi} \ker f \arrow[hook]{rr} && H \arrow{rr}{f} && \arrow[equal]{dd} G\\
 && && \arrow{ul} \ker f\times_G G' \arrow{ur} \arrow[hook]{rr} && H\times_G G' \arrow{ur} \arrow{rr} && G' \arrow{ur}{g} & \\
& V' && \Aut(V) && \arrow[swap]{ll}{\psi'} \arrow[swap]{uu}{} \ker f'  \arrow[hook]{rr} && \arrow[swap]{uu} H' \arrow{rr}{f'} && G\\
 && && \arrow{ul} \ker f'\times_G G' \arrow{uu} \arrow{ur} \arrow[hook]{rr} && \arrow{uu} H'\times_G G' \arrow{ur} \arrow{rr} && \arrow[equals]{uu} G' \arrow{ur}{g} & \\
\end{tikzcd}
\]
In this form we could treat $g^*$ as a functor $G(G')\to C(G)$, where it not for the fact that pull-back is defined only up to isomorphism. In the localized categories, this ambiguity disappears. 
\end{proof}


\begin{lemma}\label{lemma:finite-iso}
Each $(f,\psi)\in \QCf(G)$ determines a canonical isomorphism $m^*(f,\psi) \to (f,\psi)\boxtimes(f,\psi)$ in $\QCf(G\times G)$.
 \end{lemma}

\begin{proof}
As in Lemma~\ref{lemma:finite-pull-back}, we have $m^*(f,\psi) = (f_m,\psi_m)$ where $f_m$ and $\psi_m$ are pictured below, where $H_m = H\times_G(G\times G)$.
\[
\begin{tikzcd}
V & \Aut(V) & \arrow[swap]{l}{\psi} A \arrow[hook]{r} & H \arrow{r}{f} &  G\\
  &  & \arrow[dashed]{ul}{\psi_g} \arrow[swap]{u}{(m_f)\vert_{\ker m_f}} \ker f_m \arrow[hook]{r} & \arrow[swap]{u}{m_f} H_m \arrow{r}{f_m} & \arrow{u}{m} G\times G.
\end{tikzcd}
\]
On the other hand, $(f,\psi) \boxtimes (f,\psi) = (f\times f, \psi\otimes \psi)$.
Since $f\circ m_H = m \circ (f\times f)$, by the universal property of pull-back, there is a unique $\theta : H\times H \to H_m$ making the following commute.
\[
\begin{tikzcd}
\ &  & \arrow{ld}{\psi} A \arrow[hook]{r} & H \arrow{r}{f} &  G\\
V \arrow{d}{T} & \Aut(V)  & \arrow{l}{\psi_g} \arrow{u} A_m \arrow[hook]{r} & \arrow[swap]{u}{m_f} H_m \arrow{r}{f_m} & \arrow{u}{m} G\times G\\
V\otimes V & \Aut(V\otimes V) & \arrow[swap]{l}{\psi\otimes\psi} \arrow[bend left]{uu}{m_H} \arrow[swap]{u}{\theta\vert_{A\times A}} A \times A \arrow[hook]{r}{} & \arrow[bend left]{uu}{m_H} \arrow[swap]{u}{\theta} H \times H \arrow{r}{f\times f} & \arrow[equal]{u} G\times G\\
\end{tikzcd}
\]
Choose a basis $\{ v_1 \}$ for $V$; then $\{ v_1\otimes v_1 \}$ is a basis for $V\otimes V$. Define $T : V \to V\otimes V$ by $T(v_1) = v_1\otimes v_1$. Then the lower part of this diagram defines a weak isomorphism $(\theta,T): m^*(f,\psi) \to (f,\psi)\boxtimes (f,\psi)$  in $C_1(G)$. 

While $\theta$ was canonical, $T$ was not, as it depended on the choice of a basis for $V$. However, after passing from $C_1(G)$ to $\QCf(G)$, the resulting isomorphism $\mu(f,\psi): m^*(f,\psi) \to (f,\psi)\boxtimes (f,\psi)$ is canonical in $\QCf(G)$.
\end{proof}



%\Clifton{I'll address this point in the next section: "Because $\Gal(\bFq/\Fq)$ acts trivially on $\Aut({\bar f})$, the local system ${\bar L}_G(f,\psi)\ceq b_G^* L(f,\psi)$ comes equipped with a canonical isomorphism $F : \Frob{G}^* {\bar L}_G(f,\psi) \to {\bar L}_G(f,\psi)$ of constructible sheaves on ${\bar G}$."}

The full subcategory $\QCf(G) \subset \QCb(G)$ of finite quasicharacter sheaves is a proper subcategory, in general. Indeed, it follows from our main result regarding quasicharacter sheaves, Theorem~\ref{thm:snake}, that the restriction of $\trFrob{G} : \QCiso{G} \to G(\Fq)^*$ to the subgroup $\QCfiso{G}$ is an isomorphism onto the subgroup of $\chi \in G(\Fq)^*$ such that the image of $\chi$ is finite in $\EEx$; see Section~\ref{sec:snake}.
 
\section{Connected commutative algebraic groups} \label{sec:connected}

%\input{connected-1}
% !TEX encoding = UTF-8 Unicode
%In this section we review a fundamental property of character
%sheaves on geometrically connected, commutative algebraic groups over $\Fq$.

\begin{proposition}\label{prop:connected}
  If $G$ is a connected commutative algebraic group over $\Fq$ then 
  the full subcategories $\QCf(G) \hookrightarrow \QCb(G) \hookrightarrow \QC(G)$ 
  appearing in Section~\ref{sec:bounded} are equivalences 
  and
  \[
  \trFrob{G} : \QCiso{G} \to G(\Fq)^*
  \]
  is an isomorphism of groups.
\end{proposition}

\begin{proof}
 %Let $G$ be a geometrically connected algebraic group over $\Fq$.
  Observe that the forgetful functor $(\gcs{L},\mu,\phi) \mapsto (\gcs{L},\phi)$
  sends quasicharacter sheaves on $G$ to $\ell$-adic Weil sheaves on $G$ \cite{Deligne:Weil2}*{1.1.10}.
  While is it not true that all Weil sheaves on $G$ descend to local systems on $G$, 
  we will see that those that appear in the image of this forgetful functor from quasicharacter sheaves do. 

  Let $g$ be any $\Fq$-rational point on $G$ 
  and let $\bg$ be the geometric point on $G$ lying above $g$ 
  (we fixed $\Fq \hookrightarrow \bFq$ long ago). 
  Since $G$ is connected, the geometric point $\bg$ determines
  an equivalence between the category of $\ell$-adic Weil local systems on $G$ and
  $\ell$-adic representations of $\W(G,\bg)$ \cite{Deligne:Weil2}*{1.1.12}. 
  Now let $(\gcs{L},\mu,\phi)$ be a quasicharacter sheaf on $G$ 
  and let $\rho : \W(G, \bg) \to \EEx$ be the character determined by $(\gcs{L},\phi)$. 
  The $\Fq$-rational point $g$ under the geometric point $\bg$ determines a splitting
  $\Weil{\Fq}\to \W(G,\bg)$ of $\W(G,\bg)\to \Weil{\Fq}$. 
  In this way the Weil local system $(\gcs{L},F)$ determines an $\ell$-adic character $\rho_g : \Weil{\Fq} \to \EEx$.
  %\[
  %\begin{tikzcd}
  %1 \rar & \pi_1(\bG, \bg) \rar &  \arrow{d}{\rho} \W(G,\bg) \rar
  %& \ar[swap,bend right, dotted]{l}{g} \arrow[dotted]{dl}{\rho_g} \Weil{\Fq} \rar & 1\\
  %&&\EEx&&
  %\end{tikzcd}
  %\]
  This character is given by the trace of Frobenius of $\cs{L}$ at $g$, as defined in Section~\ref{sec:Frob}:
  \[
  \rho_g(\Frob{\Fq}) =  \trFrob{\cs{L}}(g).
  \]
  
  On the other hand, we have already seen that $\trFrob{\cs{L}} : G(\Fq) \to \EEx$
  is a group homomorphism. Since $G$ is an algebraic group over $\Fq$, it is a
  variety over $\Fq$, so finitely generated over $\Fq$, so $G(\Fq)$ is finite;
  it follows that $\trFrob{\cs{L}}(g) = \rho_g(\Frob{\Fq})$ is a root of unity,
  for every $g\in G(\Fq)$.  Since $\Weil{\Fq}$ is generated by
  $\Frob{\Fq}$ and $\rho_g : \Weil{\Fq} \to \EEx$ is
  a character, it follows that the image of $\rho_g$ is a finite group.
  Thus, $\rho_g$ extends to an $\ell$-adic character of $\Gal(\bFq/\Fq)$, 
  denoted by the same symbol, below.
  \[
  \begin{tikzcd}
  1 \rar & \ar[equal]{d} \pi_1(\bG, \bg) \rar & \W(G,\bg) \rar \dar & \Weil{\Fq} \arrow[bend left]{rr}{\rho_g} \rar \dar & 1 & \EEx\\
  1 \rar &  \pi_1(\bG, \bg) \rar & \pi_1(G,\bg) \rar & \Gal(\bFq/\Fq) \arrow[dotted]{rru} \rar & 1 &
  \end{tikzcd}
  \]

  We may now lift the $\ell$-adic character $\rho_g : \Gal(\bFq/\Fq) \to \EEx$
  to an $\ell$-adic character $\pi_1(G,\bg) \to \EEx$ using the canonical topological group homomorphism
  $\pi_1(G,\bg) \to \Gal(\bFq/\Fq)$ determined by the structure map for $G$. But $\bg$ also
  determines an equivalence between the category of $\ell$-adic
  representations of $\pi_1(G,\bg)$ and $\ell$-adic local systems on $G$. Let
  $\cs{L}_0$ be a local system on $G$ in the isomorphism class
  determined by this $\ell$-adic character of $\pi_1(G,\bg)$.
  Then $b_G^*\cs{L}_0 \iso \gcs{L}$.
  Since $b_G^*$ is full and faithful (again, see \cite{SGA7.2}*{Expos\'e XIII} or even \cite{BBD}*{Prop. 5.2.1}),
  $
  b_{G\times G}^* : \Hom(m^*\cs{L}_0,\cs{L}_0\boxtimes\cs{L}_0) \to \Hom({\bar m}^*\gcs{L},\gcs{L}\boxtimes\gcs{L})
  $ 
  is a bijection 
  (hom taken in the categories on constructible $\ell$-adic sheaves on 
  $G\times G$ and ${\bar G}\times {\bar G}$ respectively, 
  in which $\ell$-adic local systems sit as full subcategories). 
  Let $\mu_0 : m^*\cs{L}_0 \to \cs{L}_0\boxtimes\cs{L}_0$ be the isomorphism matching 
  $\mu : {\bar m}^*\gcs{L} \to \gcs{L}\boxtimes\gcs{L}$, 
  the latter appearing in the definition of $\cs{L}$. 
  Then, as in Section~\ref{sec:bounded}, $(\cs{L}_0,\mu_0)$ is an object in $\QCb(G)$ 
  and $\cs{L} = (\gcs{L},\mu,\phi)$ is isomorphic to $(b_G^*\cs{L}_0,b_{G\times G}^*\mu_0)$ in $\QC(G)$.
  Thus, the full and faithful functor $B_G^* : \QCb(G) \to \QC(G)$ from Section~\ref{sec:bounded}
  is also essentially surjective, hence an equivalence.
  %(when $G$ is a connected, commutative algebraic group over $\Fq$).
  
  In Section~\ref{sec:finite} we saw that $\QCf(G)$ is a full subcategory of $\QCb(G)$;
  more precisely, we exhibited a full and faithful functor $L_G : \QCf(G) \to \QCb(G)$.
  We now show that this functor is essentially surjective 
  when $G$ is a connected, commutative algebraic group over $\Fq$.
  
  Let $(\cs{L}_0,\mu_0)$ be a bounded quasicharacter sheaf on $G$. 
  Then $\cs{L}\ceq B_G^*(\cs{L}_0,\mu_0)$ is a quasicharacter sheaf on $G$.
  Recall the definition of the character $\trFrob{\cs{L}} : G(\Fq) \to \EEx$ from Section~\ref{sec:Frob}. 
  Let $f : G\to G$ be the Lang isogeny.
  Recall that $\ker f = G(\Fq) = G^{\Frob{\Fq}}$; 
  in particular, the action of $\Gal(\bFq/\Fq)$ on $\ker f$ is trivial.
  Let $V = \EE$ and let $\psi : \ker f \to \Aut(V)$ be the representation given by 
  $\psi(g)(v) = \trFrob{\cs{L}}(g) v$ for $v\in V$. 
  Then $(f,\psi)$ is a finite quasicharacter sheaf. 
  
  To show that $L_G(f,\psi) \iso (\cs{L}_0,\mu_0)$ we simply apply the functions--sheaves dictionary
  as follows ({\it cf} \cite{Laumon}*{1.1.3} or, ultimately, \cite{SGA4.5}*{Sommes trigonométriques}).
  First, recall that $L_G(f,\psi) = (L(f,\psi), \mu(f,\psi))$. 
  By  \cite{Laumon}*{1.1.3.3}, $\trFrob{(G_{k'}\hookrightarrow G)^*L(f,\psi)} = \trFrob{\cs{L}} \circ N_{k'/k}$, for each finite extension $k'/k$.
  On the other hand, $\trFrob{\cs{L}} \circ N_{k'/k} = \trFrob{(G_{k'}\hookrightarrow G)^*\cs{L}}$,  by Proposition~\ref{prop:basechange}.
  %Using notation from \cite{Laumon}*{1.1}, this means $t_{\bullet L(f,\psi)} = t_{\bullet \cs{L}_0}$.
  Using \cite{Laumon}*{Th\'eor\`eme~1.1.2}, it follows that  $L(f,\psi) \iso \cs{L}_0$ in $\Loc(G)$.
  It is now clear that $(L(f,\psi),\mu(f,\psi)) \iso (\cs{L}_0,\mu_0)$ in $\QCb(G)$.
  This completes the proof that $L_G : \QCf(G) \to \QC(G)$ is essentially surjective.
  Since we have already seen (in Section~\ref{sec:finite}) that $L_G$ is full and faithful, 
  it follows that $L_G$ is an equivalence.
  
  To finish the proof of Proposition~\ref{prop:connected}, it suffices to observe that the group homomorphism
  $
  \trFrob{G} : \QCiso{G} \to G(\Fq)^*
  $
  is surjective because each character $\chi \in G(\Fq)^*$ determines a finite quasicharacter sheaf $(\Lang,\chi)$ such that the trace of Frobenius of the quasicharacter sheaf $B_G^* (L_G(\Lang,\chi^{-1}))$ is $\chi$.
\end{proof}

As Proposition~\ref{prop:connected} shows, 
when $G$ is a connected algebraic group over $\Fq$, 
it is appropriate to replace quasicharacter sheaves on $G$ 
with the conceptually simpler category $\QCf(G)$ of finite quasicharacter sheaves on $G$. 
In this context, finite quasicharacter sheaves may also be apprehended as \emph{certain} $\ell$-adic characters of fundamental group $\pi_1(G,{\bar e})$. 
As explained in \cite{Kamgarpour}*{\S~2}, 
the relevant characters of $\pi_1(G,{\bar e})$ are precisely those 
that factor through a particular quotient of $\pi_1(G,{\bar e})$ 
denoted there by $\Pi_\text{disc}(G)$.
%\[\pi_1(G,{\bar e}) \to \Pi_\text{et}(G) \to \Pi_\text{et}(G)_{\Gal(\bFq/\Fq)} = \Pi_\text{disc}(G)\]
From this it is apparent that the category of discrete isogenies to $G$ is a Galois category, in the sense of \cite{SGA1}*{Expos\'e V, \S 4}. Moreover, since $\Pi_\text{disc}(G) \iso G(\Fq)$, as shown in \cite{Kamgarpour}*{App'x B}, this provides an alternate proof of $\QCiso{G} \to G(\Fq)^*$.

In principle, something similar is possible in the more general 
context of this paper, where $G$ is a smooth group scheme 
locally of finite type over $\Fq$, to which we now return. 
Every quasicharacter sheaf $(\gcs{L},\mu,\phi)$ on $G$ 
determines a Weil sheaf $(\gcs{L},\phi)$ on $G$. 
So, if we choose a geometric point ${\bar g}_x$ in each component $\bG^x$ of $G$, then this choice can be used to convert 
the Weil sheaf $(\gcs{L},\phi)$ into an $\ell$-adic character of $\prod_{x\in \pi_0(\bG)}\W(\bG^x, {\bar g}_x)$. 
It proved cumbersome, however, to manage various 
choices of families of base points and track the action 
of $\Gal(\bFq/\Fq)$ on this (generally) infinite product of 
Weil groups, and difficult to identify the relevant quotient of this group. Viewed in this light, Definition~\ref{def:QC} seemed comparatively simple.

\section{Commutative \'etale group schemes} \label{sec:etale}

%\input{etale-1}
%In this section we establish a fundamental property of character
% sheaves on commutative etale group schemes over $\Fq$ that are
% geometrically finitely generated.

\'Etale group schemes form a counterpoint to connected groups, since the component group of a group scheme locally of finite type over $\Fq$ is an \'etale group scheme (locally of finite type) over $\Fq$ \cite{vdG&M}*{III, \S 4}.  The functor $G \mapsto G(\bFq)$ defines an equivalence of categories
between the category of \'etale group schemes over $\Fq$ and the category of groups equipped
with an action of $\Gal(\bFq/\Fq)$, continuous for the discrete topology on the group.
Under this equivalence, a quasicharacter sheaf on $G$ is given by the data of:
\begin{enumerate}
 \labitem{(cs.0)}{cs.0} an indexed set of one-dimensional
  $\EE$-vector spaces $\gcs{L}_x$, as $x$ runs over
  $G(\bFq)$;

 \labitem{(cs.1)}{cs.1} an indexed set of isomorphisms
  $\mu_{x,y} : \gcs{L}_{x+y} \mathop{\longrightarrow}\limits^{\iso} \gcs{L}_{x} \otimes \gcs{L}_{y}$,
  for all $x,y \in G(\bFq)$, such that
  \[
   \begin{tikzcd}[row sep=40]
    \gcs{L}_{x+y+z} \arrow{rr}{\mu_{x+y,z}} \arrow[swap]{d}{\mu_{x,y+z}}
    && \gcs{L}_{x+y}\otimes \gcs{L}_{z} \dar{\mu_{x,y} \tight{0.5}{\otimes}{1} \id} \\
    \gcs{L}_{x} \otimes \gcs{L}_{y+z} \arrow{rr}{\id \otimes \mu_{y,z}}
    && \gcs{L}_{x} \otimes\gcs{L}_{y} \otimes \gcs{L}_{z}
   \end{tikzcd}
  \]
  commutes, for all $x,y,z\in G(\bFq)$; and

 \labitem{(cs.2)}{cs.2} an indexed set of isomorphisms $\phi_{x} : \gcs{L}_{\Frob{\Fq}(x)} \to \gcs{L}_x$
  such that
  \[
   \begin{tikzcd}[row sep=40]
    \gcs{L}_{\Frob{\Fq}(x)+\Frob{\Fq}(y)} \arrow[swap]{d}{\phi_{x+y}} \arrow{rr}{\mu_{\Frob{\Fq}(x),\Frob{\Fq}(y)}}
    && \gcs{L}_{\Frob{\Fq}(x)}\otimes \gcs{L}_{\Frob{\Fq}(y)} \dar{\phi_x \tight{0}{\otimes}{0} \phi_y} \\
    \gcs{L}_{x+y} \arrow{rr}{\mu_{x,y}}
    && \gcs{L}_x \otimes\gcs{L}_y
   \end{tikzcd}
  \]
  commutes, for all $x,y\in G(\bFq)$.
\end{enumerate}
Under this equivalence, a morphism $\alpha : \cs{L} \to \cs{L'}$ of quasicharacter sheaves on $G$ is given by 
\begin{enumerate}
 \labitem{(cs.3)}{cs.3} an indexed set $\alpha_x : \gcs{L}_x \to \gcs{L'}_x$
  of linear transformations (of one-dimensional $\EE$-vector spaces),
  as $x$ runs over $G(\bFq)$, such that
  \[
   \begin{tikzcd}[column sep=40]
    \arrow[swap]{d}{\phi_x} \gcs{L}_{\Frob{\Fq}(x)} \rar{\alpha_{\Frob{\Fq}(x)}} & \gcs{L'}_{\Frob{\Fq}(x)} \dar{\phi_x'}
    &\arrow[draw=none]{d}[pos=.4,description]{\text{\normalsize{and}}}& \arrow[swap]{d}{\mu} \gcs{L}_{x+y} \rar{\alpha_{x+y}} & \gcs{L'}_{x+y} \dar{\mu'_{x+y}} \\
    \gcs{L}_x \rar{\alpha_x} & \gcs{L'}_x
    & {} & \gcs{L}_x\otimes\gcs{L}_y \rar{\alpha_x\otimes \alpha_y} & \gcs{L'}_x \otimes \gcs{L'}_y
   \end{tikzcd}
  \]
  both commute.
\end{enumerate}

Let us say that \cdef{global section}
of a quasicharacter sheaf $\cs{L}$ on an \'etale group scheme $G$ as a choice 


\begin{lemma}\label{lemma:section}
 If $\cs{L}$ is a quasicharacter sheaf on a commutative, \'etale group scheme $G$ over $\Fq$ and if $G(\bFq)$ is finitely generated, then there is a global section $s = \{ s(x) \in \gcs{L}_x \tq x \in G(\bFq) \}$ satisfying
 \begin{equation}\label{section}
  \forall x,y \in G(\bFq), \qquad \mu_{x,y}(s(x+y)) = s(x)\otimes s(y).
 \end{equation}
\end{lemma}

\begin{proof}
  Since $G(\bFq)$ is a finitely-generated abelian group, it admits a decomposition
  \[
  G(\bFq) \iso \ZZ^r \oplus \bigoplus_{j=1}^t \ZZ / m_j \ZZ;
  \]
  let $x_1, \ldots, x_r, y_1, \ldots, y_t$ be a corresponding set of generators of $G(\bFq)$.
  We will define a section $s$ by choosing values on these generators and extending to all
  of $G(\bFq)$ through \eqref{section}.
  
  There is a unique isomorphism $\gcs{L}_0 \cong \EE$ under which $\mu_{0,0}$ corresponds
  to the multiplication map $\EE \otimes \EE \to \EE$.
  The condition $\mu_{0,0}(s(0)) = s(0) \otimes s(0)$ together with the non-triviality of $s$
  then forces $s(0) = 1$.
  
  The $\mu_{x,y}$ yield an isomorphism $\alpha_j : \gcs{L}_{y_j}^{\otimes m_j} \to \gcs{L}_0$.
  An initial nonzero choice of $s(y_j)$ need only be scaled by $\alpha_j(s(y_j))^{-1/m_j}$ in order to map to
  $1$ under this isomorphism.  Since compatibility under the $\alpha_j$ are the only constraints on the
  $s(x_j)$ and $s(y_j)$, global sections exist.
\end{proof}

These sections will allow us to construct the isomorphism in the following proposition.

\begin{proposition}\label{prop:etale}
  If $G$ is a commutative \'etale group scheme over $\Fq$ and
  $G(\bFq)$ is finitely generated then there is a canonical
  isomorphism
  \[
  \QCiso{G} \longrightarrow \Hh^1(\Weil{\Fq}, G(\bFq)^*),
  \qquad \text{given by}\qquad
  [\cs{L}] \mapsto [\tau_\cs{L}],
  \]
  where $\tau_\cs{L}: \Weil{\Fq}\to G(\bFq)^*$ is a cocycle such that
  \[
  \forall x\in G(\Fq), \qquad \tau_\cs{L}(\Frob{\Fq})(x) = \trFrob{\cs{L}}(x).
  \]
\end{proposition}

\begin{proof}
  Suppose $\cs{L}$ is a quasicharacter sheaf on $G$.  Using Lemma~\ref{lemma:section}, we may choose a section $s = \{ s(x) \in \gcs{L}_x \tq x \in G(\bFq) \}$ satisfying \eqref{section}.  We will construct a group isomorphism between $\QCiso{G(\bFq)}$ and
  $\Hh^1(\Weil{\Fq}, G(\bFq)^*)$ using $s$, then prove independence from the choice of section.

  Starting from $\cs{L}$, we define a cocyle $\tau_{\cs{L}} \in Z^1(\Weil{\Fq},G(\bFq)^*)$ as follows.
  For $x \in G(\bFq)$, $\cs{L}$ determines an isomorphism $\phi_x : \gcs{L}_{\Frob{\Fq}(x)} \to \gcs{L}_x$
  of one-dimensional $\EE$-vector spaces; define $\tau_\cs{L}$ by
  \begin{equation}\label{t}
   \phi_{x}(s(\Frob{\Fq}(x))) = \tau_\cs{L}(\Frob{\Fq})(x) s(x).
  \end{equation}
  Note that if $x$ is fixed by $\Frob{k}$ then $\tau_\cs{L}(\Frob{\Fq})(x)$ is just $\trFrob{\cs{L}}(x)$, so the restriction of
  $\tau_\cs{L}(\Frob{k})$ to $G(k)$ is just $\trFrob{\cs{L}}$.
  In general, conditions \ref{cs.0} and \ref{cs.2}, together with \eqref{section}, guarantee that
  the scalar $\tau_\cs{L}(\Frob{k})(x)$ is non-zero.  Moreover, condition~\ref{cs.1} forces
  $\tau_\cs{L}(\Frob{\Fq})(x+y) = \tau_\cs{L}(\Frob{\Fq})(x) \ \tau_\cs{L}(\Frob{\Fq})(y)$
  for all $x,y \in G(\bFq)$.  Now define $\tau_\cs{L} : \Weil{\Fq} \to G(\bFq)^*$ recursively by
  \[
   \tau_\cs{L}(\Frob{\Fq}^n) \ceq \tau_\cs{L}(\Frob{\Fq})\cdot \,^{\Frob{\Fq}} \tau_\cs{L}(\Frob{\Fq}^{n-1}),
  \]
  where $\Frob{\Fq}$ acts on $G(\bFq)^*$ through pre-composition and $\cdot$ refers
  to pointwise multiplication of functions.

  Although we used a global section $s$ in the definition of
  $\tau_\cs{L}$, the class of $\tau_\cs{L}$ does not depend on this
  choice. To see why, let $s_1$ and $s_2$ be two global sections of
  $\cs{L}$ and let $\tau_1$ and $\tau_2$ be the cocycles defined by $s_1$ and $s_2$.
  For each $x \in G(\bFq)$ there is a unique scalar
  $a(x) \in \EEx$ such that $s_2(x) = a(x) s_1(x)$. Then \eqref{section} and
  condition~\ref{cs.1} together imply $a(x+y) = a(x)a(y)$, so
  $a\in G(\bFq)^*$.  The coboundary $\tau_0 : \Weil{\Fq} \to G(\bFq)^*$
  defined by $\tau_0(\Frob{\Fq}) = \,^{\Frob{\Fq}} a \cdot a^{-1}$ satisfies $\tau_2 = \tau_0 \tau_1$,
  so $\tau_2$ and $\tau_1$ are cohomologous. Thus
  \begin{equation}\label{cohomologous}
    \obj \QC(G(\bFq)) \longrightarrow \Hh^1(\Weil{\Fq},G(\bFq)^*),
    \qquad \text{defined by} \qquad \cs{L} \mapsto [\tau_\cs{L}],
  \end{equation}
  is independent of the choice of global sections made above.

  We now show that $[\tau_\cs{L}]$ depends only on the isomorphism class of $\cs{L}$.
  Let $\varphi : \cs{L} \to \cs{L'}$ be an isomorphism and
  let $\tau$ and $\tau'$ be any cocycles representing the classes in the
  image of $\cs{L}$ and $\cs{L'}$ under \eqref{cohomologous}; by construction,
  $\tau$ and $\tau'$ are defined by sections $s$ and $s'$. Now, $\varphi_x(s(x))$ is a
  non-zero scalar multiple of $s'(x)$; define $\delta: G(\bFq) \to \EEx$
  by $\varphi_x(s(x)) = \delta(x) s'(x)$.
  By the second part of condition~\ref{cs.3},
  $\delta(x+y) = \delta(x)\delta(y)$, so $\delta \in G(\bFq)^*$. By the first part of
  condition~\ref{cs.3}, $\tau'(\Frob{\Fq}) = \tau(\Frob{\Fq}) \cdot (\,^{\Frob{\Fq}^{-1}}\delta \cdot \delta^{-1})$,
  so $\tau$ and $\tau'$ are cohomologous. This concludes the definition of
  \begin{equation}\label{pich1}
    \QCiso{G(\bFq} \longrightarrow \Hh^1(\Weil{\Fq},G(\bFq)^*).
  \end{equation}

  It only remains to show that \eqref{pich1} is a group
  isomorphism.  Suppose that $\cs{L}$ and $\cs{L}'$ are two quasicharacter sheaves.
  If $s$ is a global section for $\cs{L}$ and $s'$ for $\cs{L}'$ then $s \otimes s'$ is a global section for
  $\cs{L} \otimes \cs{L}'$.  The three versions of \eqref{t} for $\cs{L}, \cs{L'}$ and $\cs{L} \otimes \cs{L}'$ imply that
  $\tau_{\cs{L} \otimes \cs{L}'}(\Frob{k}) = \tau_\cs{L}(\Frob{k}) \tau_{\cs{L}'}(\Frob{k})$, so \eqref{pich1} is a group homomorphism.
\Clifton{I'm working on this.}
\end{proof}


\section{On the cohomology of the dual of an etale group scheme} \label{sec:etalecohom}
%\input{etale-2}
%Define $\rho \colon \Hh^1(\Weil{\Fq}, \Hom(G(\bFq), \EEx))
% \rightarrow \Hom(G(\Fq), \EEx)$ by $\rho(\phi)(g) =
% \phi(\Frob)(g)$.

Although this section does not mention quasicharacter sheaves, 
we will need Proposition~\ref{prop:X} to prove our main result on quasicharacter sheaves, 
Theorem~\ref{thm:snake}.

\begin{proposition}\label{prop:X}
  Let $X$ be a finitely generated abelian group equipped with an
  action of $\Gal(\bFq/\Fq)$, continuous for the discrete topology on
  $X$.  Then
  \begin{align*}
%    \Hh^1(\Fq, \Hom(X, \EEx)) &\xrightarrow{\rho} \Hom(\Hh^0(\Fq, X), \EEx) \\
    \Hh^1(\Fq, X^*) &\xrightarrow{\rho} \Hh^0(\Fq, X)^* \\
    \rho([z])(x) &= z(\Frob{\Fq})(x)
  \end{align*}
  is an isomorphism of groups.
\end{proposition}
\begin{proof}
  We first prove that $\rho$ is surjective.  Suppose that
  $f \in \Hom(X^{\Frob{\Fq}}, \EEx).$ Since $\EEx$ is divisible
  as an abelian group it is injective as a $\ZZ$-module and thus
  $\Ext^1_{\ZZ}(X/X^{\Frob{\Fq}}, \EEx) = 0$.  Applying the functor
  $\Hom(\mbox{---}, \EEx)$ to
  \[
  \begin{tikzcd}
  0 \rar & X^{\Frob{\Fq}} \rar & X \rar & X/X^{\Frob{\Fq}} \rar & 0
  \end{tikzcd}
  \]
  and considering the associated long exact sequence gives the
  existence of an $\tilde{f} \in \Hom(X, \EEx)$ restricting to
  $f$.  Define a $1$-cocycle $z \in \Hh^1(\Fq, \Hom(X,\EEx))$ by
  mapping $\Frob{\Fq}$ to $\tilde{f}$.  Then $\rho([z]) = f$.

  Now suppose that $[z] \in \Hh^1(\Fq,\Hom(X,\EEx))$ with
  $\rho([z]) = 1$; write $\tilde{f}$ for $z(\Frob{\Fq})$.  The injectivity
  of $\rho$ is equivalent to the existence of a homomorphism $\psi
  \colon X \rightarrow \EEx$ with
  $$\tilde{f}(x) = \frac{\psi(\Frob{\Fq}(x))}{\psi(x)}$$
  for $x \in X$.  In fact, we may replace the construction of $\psi$
  for all $\ZZ[\Gal(\bFq/\Fq)]$-modules $X$ with the construction of
  analogous $\psi$s for a different class of modules.

  Since the action of $\Gal(\bFq/\Fq)$ is continuous, there is an $m$
  so that $\Frob{\Fq}^m$ acts trivially on $X$; write $\Fqm$ for the degree
  $m$ extension of $\Fq$ and $\Gamma_m$ for $\Gal(\Fqm/\Fq)$.  Since
  $T^m-1$ factors as a squarefree product of cyclotomic polynomials,
  the Chinese remainder theorem expresses $\ZZ[\Gamma_m]$ as a product
  of $\ZZ[\zeta_d]$ for $d$ dividing $m$.  Each $\ZZ[\zeta_d]$ is the
  maximal order in a number field and thus $\ZZ[\Gamma_m]$ is a
  product of Dedekind domains.  Therefore $X$ decomposes into a direct
  sum of cyclic $\ZZ[\Gamma_m]$-modules \cite{Brandal}*{?}, each of the
  form $\ZZ[\Gamma_m] / I$ for some ideal $I \subset \ZZ[\Gamma_m]$.
  The ideal $I$ corresponds to a direct product of ideals $I_d$ in the
  $\ZZ[\zeta_d]$, and within each $\ZZ[\zeta_d]$ the Chinese remainder
  theorem allows a further decomposition of $\ZZ[\zeta_d] / I_d$ as a
  direct sum of $\ZZ[\zeta_d]$-modules of the form $\ZZ[\zeta_d]/P^s$.
  So if we can construct, for every cyclotomic ring $\ZZ[\zeta_d]$,
  prime ideal $P \subset \ZZ[\zeta_d]$ and positive integer $s$, a
  $\psi \colon \ZZ[\zeta_d]/P^s \rightarrow \EEx$ with
  \begin{equation} \label{eq:psi-condition}
    \tilde{f}(x) = \frac{\psi(\zeta_d \cdot x)}{\psi(x)},
  \end{equation}
  then the existence of the original $\psi$ follows because $X$ is a
  direct sum of such $\ZZ[\zeta_d]/P^s$.

  We now fix a $d$, $P$ and $s$, write $\zeta$ for $\zeta_d$ and
  choose a polynomial
  $$h(T) = a_rT^r + a_{r-1}T^{r-1} + \cdots + a_0$$
  so that $P^s = (p^s, h(\zeta_d))$.
  % Setting $\psi = 1$ will satisfy Equation \ref{eq:psi-condition}
  % when $d = 1$.
  An an abelian group, the quotient $\ZZ[\zeta] / P^s$ is generated by
  $\zeta^i$ for $0 \le i < \phi(d)$; we define
  \begin{equation} \label{eq:psi-def}
    \psi(\zeta^i) = \alpha \prod_{j=0}^{i-1} \tilde{f}(\zeta^j)
  \end{equation}
  for an $\alpha \in \EEx$ to be chosen later.  Since there are
  relations among the $\zeta^i$ we first need to check that $\psi$ is
  well-defined.

  The relations among the generators $\zeta^i$ are additively
  generated by $\zeta^ch(\zeta)$ as $c$ ranges from $0$ to
  $\phi(d)-r-1$.  If we set
  $$A_c = \sum_{i=0}^r a_i \sum_{j=0}^{i+c-1} \zeta^j$$
  then
  $$\psi(\zeta^ch(\zeta)) = \alpha^{h(1)} \tilde{f}(A_c).$$
  Now consider $(\zeta - 1)A_c$: the sums telescope and we get that
  $(\zeta-1)A_c = \zeta^ch(\zeta) - h(1).$ There are now two
  possibilities:
  \begin{enumerate}
  \item $d$ is a power of $p$; write $s = \phi(d)m + n$.  We may take
    $h(T)$ to be $p^m(T-1)^n$, in which case $\zeta^ch(\zeta) - h(1) \in P^s$,
    so $A_c$ is fixed under multiplication by $\zeta$ in
    $\ZZ[\zeta]/P^s$ and thus $\tilde{f}(A_c) = 1$.  In this case
    $\alpha^{h(1)} = 1$ regardless of $\alpha$.
  \item $d$ is not a power of $p$.  If two distinct primes divide $d$
    then $\zeta - 1$ is a unit; otherwise $\zeta - 1$ is a unit modulo
    $P^s$.  Let $\omega \in \ZZ[\zeta]$ satisfy
    $\omega(\zeta - 1) \equiv 1 \pmod{P^s}$ and set
    \begin{equation} \label{alpha-def}
      \alpha = \tilde{f}(\omega).
    \end{equation}
    Then $A_c + h(1)\omega$ is fixed under multiplication by $\zeta$
    and thus $\tilde{f}(A_c) = \tilde{f}(-h(1)\omega)$.
  \end{enumerate}
  In either case, $\psi(\zeta^ch(\zeta)) = 1$ for all $c$.  The only
  other relations imposed on the $\zeta^i$ are that they are all
  killed under multiplication by $p^s$.  Since $\tilde{f}$ is also a
  homomorphism $\ZZ[\zeta]/P^s \rightarrow \EEx$ the values of
  $\tilde{f}$ and thus of $\psi$ will be $p^s$-roots of unity.
  Therefore $\psi$ is well defined.

  We now check Equation~\eqref{eq:psi-condition} for
  $x = \zeta^{\phi(d)-1}$.  Write
  $\Phi_d(T) = T^{\phi(d)} - b_{\phi(d)-1}T^{\phi(d)-1} - \cdots - b_0$
  and set
  $$B = \sum_{i=0}^{\phi(d)-1} -\zeta^i + b_i \sum_{j=0}^{i-1} \zeta^j.$$
  We have

  \begin{align*}
    \frac{\psi(\zeta^{\phi(d)})}{\psi(\zeta^{\phi(d)-1})\tilde{f}(\zeta^{\phi(d)-1})}
    &= \frac{\psi(b_{\phi(d)-1}\zeta^{\phi(d)-1} + \cdots + b_0)}{\psi(\zeta^{\phi(d)-1})\tilde{f}(\zeta^{\phi(d)-1})} \\
    &= \alpha^{-\Phi_d(1)} \tilde{f}(B)
  \end{align*}
  As before we see that $(\zeta - 1)B = \Phi_d(1) - \Phi_d(\zeta) = \Phi_d(1)$.
  We consider the same cases as before:

  \begin{enumerate}
  \item $d$ is a power of $p$, in which case $\Phi_d(1) = p$.  Thus
    $(\zeta-1)B = p = u(\zeta-1)^{\phi(d)}$ for a unit $u \in \ZZ[\zeta]$
    and $B - u(\zeta-1)^{\phi(d)-1}$ is fixed by $\zeta$,
    so that $\tilde{f}(B) = \tilde{f}(u(\zeta - 1)^{\phi(d)-1})$.  The
    multiplicative order of $\tilde{f}(u(\zeta-1)^{\phi(d)-1})$ is at most
    $\lceil \frac{s - 1 - (\phi(d)-1)}{\phi(d)}\rceil = \lceil \frac{s}{\phi(d)} \rceil - 1$
    while the additive order of $1 \in \ZZ[\zeta]/P^s$ is $\lceil \frac{s}{\phi(d)} \rceil$.
    Thus we may choose $\alpha = \psi(1)$ with $\alpha^p = \tilde{f}(B)$.
  \item $d$ is not a power of $p$ and $\omega$ is the inverse of
    $\zeta-1$ modulo $P^s$.  Then $\tilde{f}(B) = \tilde{f}(\omega\Phi_d(1))$.
  \end{enumerate}
  In either case, $\alpha^{-\Phi_d(1)}\tilde{f}(B) = 1$ and thus
  $\psi$ satisfies Equation~\eqref{eq:psi-condition} for
  $x = \zeta^{\phi(d)-1}$.  The definition of $\psi$ in \ref{eq:psi-def}
  immediately implies that $\psi$ satisfies equation
  \ref{eq:psi-condition} for lesser powers of $\zeta$, and thus for
  all $x$ by linearity.  The existence of $\psi$ shows that $\rho$ is injective.
\end{proof}

\section{Restriction to the identity component} \label{sec:restriction}

%\input{restriction}
% !TEX encoding = UTF-8 Unicode
%\subsection*{Restriction to the identity component is surjective}

Let $G^0$ be the connected component of $G$ containing the identity.
By Proposition~\ref{prop:pullback}, pull-back along
$G^0\hookrightarrow G$ determines a monoidal functor
$\cs{L}\mapsto \cs{L}\vert_{G^0}$ from $\QC(G)$ to $\QC(G^0)$.  In
this section we show that this functor is essentially surjective. 

\begin{proposition}\label{prop:restriction}
  The restriction functor $\QC(G)\to \QC(G^0)$ is essentially surjective.
\end{proposition}

\begin{proof}
  Because $G$ is smooth over $\Fq$, the identity component, 
  $G^0$, is a connected algebraic group over $\Fq$, 
  clopen in $G$ \cite{vdG&M}*{3.17}. 
  By Proposition~\ref{prop:connected}, every
  quasicharacter sheaf on $G^0$ is isomorphic to a 
  finite quasicharacter sheaf on $G^0$, 
  so to prove the proposition it suffices to show that every 
  finite geochar on $G^0$ extends to a geochar on $G$.
  
 Let $(\pi,\psi)$ be a finite geochar on $G^0$.
 By Lemma~\ref{lemma:ext}, there is an extension of the 
 descrete isogeny $\pi : B \to G^0$ 
 (see Section~\ref{sec:finite} for the definition of a discrete isogeny) 
 to a discrete isogeny $f : H \to G^0$ 
 such that $\pi_0(f) : \pi_0(H)\to \pi_0(G)$ is an isomorphism.
 Then $(f,\psi)$ is a finite geochar on $G$.
 Then $(f,\psi)\vert_{G^0} \iso (\pi,\psi)$,
 completing the proof that $\QC(G)\to \QC(G^0)$ is essentially surjective.
\end{proof}

\begin{lemma}\label{lemma:ext}
Every discrete isogeny onto $G^0$ extends to a discrete isogeny onto $G$ such that the component group of the covering space is isomorphic to the component group of $G$.
\end{lemma}

\begin{proof}
Let $\pi: B \to G^0$ be a discrete isogeny; set $A = \ker \pi$.
  We will find a discrete isogeny $f: H\to G$
  such that that $H^0 = B$, $f^0 =\pi$ and
  $\pi_0(f) : \pi_0(H)\to \pi_0(G)$ is an isomorphism of component
  groups.
  \begin{equation}\label{extension-diagram}
  \begin{tikzcd}
  A \arrow[equal]{r} \dar & A \dar \\
  B \rar \dar[swap]{\pi} & H \rar \dar[swap]{f} & \pi_0(H) \arrow{d}[below,rotate=90]{\sim}[swap]{\pi_0(f)} \\
  G^0 \rar & G \rar & \pi_0(G)
  \end{tikzcd}
  \end{equation}
  where all rows and columns are exact and all maps are defined over
  $\Fq$.  We will do so by passing back and forth between group
  schemes over $\Fq$ and their $\bFq$-points.  
  
  Extensions of
  $G^0(\bFq)$ by $A(\bFq)$, such as $A(\bFq) \to H^0(\bFq) \to G^0(\bFq)$,
  correspond to classes in $\Ext^1_{\ZZ[\Gamma]}(G^0(\bFq), A(\bFq))$.
  Similarly, extensions of $G(\bFq)$ by $A(\bFq)$ correspond to
  classes in $\Ext^1_{\ZZ[\Gamma]}(G(\bFq), A(\bFq))$.  The map
  $G^0(\bFq) \to G(\bFq)$ induces the map
  $\Ext^1_{\ZZ[\Gamma]}(G(\bFq), A(\bFq)) \to \Ext^1_{\ZZ[\Gamma]}(G^0(\bFq), A(\bFq))$,
  fitting into the long exact sequence derived from applying
  $\Hom(\mbox{---}, A(\bFq))$ to $G^0(\bFq) \to G(\bFq) \to \pi_0(G)(\bFq)$:
  $$\Ext^1_{\ZZ[\Gamma]}(G(\bFq), A(\bFq)) \to \Ext^1_{\ZZ[\Gamma]}(G^0(\bFq), A(\bFq)) \to \Ext^2_{\ZZ[\Gamma]}(\pi_0(G)(\bFq), A(\bFq)).$$
  We now show that $\Ext^2_{\ZZ[\Gamma]}(\pi_0(G)(\bFq), A(\bFq)) = 0$.

  It suffices to show that $\pi_0(G)(\bFq)$ has a two-term resolution
  by injective $\ZZ[\Gamma]$-modules.  Since the action of $\Gamma$ on
  $\pi_0(G)(\bFq)$ is continuous, it factor through a finite quotient
  $\Gamma_m = \Gamma / H_m$.  Any product of Dedekind domains has
  cohomological dimension $1$ \cite[?]{?}, so $\pi_0(G)(\bFq)$ has a
  two term resolution by injective $\ZZ[\Gamma_m]$-modules.  Let $I$
  be one of these modules: we need to show that $I$ is injective as a
  $\ZZ[\Gamma]$-module.  Consider the universal diagram for injectivity:
  
  \[
  \begin{tikzcd}
  A \rar{a} \arrow[hook]{d} & I \\
  B
  \end{tikzcd}
  \]
  where $A$ and $B$ are arbitrary $\ZZ[\Gamma]$-modules.  Since $H_m$ acts trivially on $I$,
  $I_{H_m}A := \langle (\gamma-1)a : \gamma \in H_m, a \in A \rangle \subseteq \ker(a).$
  Since $B / I_{H_m}B \cong B \otimes_A A / I_{H_m}A$, we get
  \[
  \begin{tikzcd}
  A \arrow[two heads]{r} \arrow[hook]{d} & A/I_{H_m}A \rar \arrow[hook]{d} & I \\
  B \arrow[two heads]{r} & B/I_{H_m}B \arrow[dashed]{ur} & {}
  \end{tikzcd}
  \]
  The dotted arrow exists since $I$ is injective as a
  $\ZZ[\Gamma_m]$-module, and the outer diagram shows that $I$ is
  injective as a $\ZZ[\Gamma]$-module.

  We therefore have the existence of diagram \eqref{extension-diagram}
  at the level of $\bFq$-points.  This expresses $H(\bFq)$ as a
  disjoint union of translates of $B(\bFq)$; by transport of structure
  we may take $H$ to be a group scheme over $\bFq$.  Similarly, the
  restriction of $f$ to each component of $H$ is a morphism of
  schemes, and thus $f$ is as well.  Finally, the whole diagram
  descends to a diagram of $\Fq$-schemes since the $\bFq$-points of
  the objects come equipped with continuous $\Gamma$-actions, and the
  morphisms are $\Gamma$-equivariant.
\end{proof}

\section{Main result regarding quasicharacter sheaves}
\label{sec:snake}

We saw, in Proposition~\ref{prop:functorialG}, that the trace of Frobenius  $\trFrob{G} : \QCiso{G} \to G(\Fq)^*$ is a functorial group homomorphism. Our main result regarding quasicharacter sheaves shows that trace of Frobenius is an isomorphism in the case that we care about.

%\input{snake}
\begin{theorem}\label{thm:snake}
  If $G$ is a smooth abelian group scheme, locally of finite type over
  $\Fq$, and the abelian group $\pi_0(G)(\bFq)$ (the geometric
  component group) is finitely generated, then the trace of Frobenius
  \[
  \trFrob{G} : \QCiso{G} \to G(\Fq)^*
  \]
  is an isomorphism of groups.
\end{theorem}

\begin{proof}
  Consider the short exact sequence in the category of smooth group
  schemes, locally of finite type over $\Fq$, defining the component
  group scheme for $G$:
  \begin{equation}\label{eq:pi0}
  \begin{tikzcd}
    1 \rar & G^0 \rar & G \rar & \pi_0(G) \rar & 0.
  \end{tikzcd}
  \end{equation}
  Let
  \[
  \begin{tikzcd}[row sep=30]
    {}& \ker \trFrob{\pi_0(G)} \dar & \arrow[dashed]{d} \ker \trFrob{G} & \ker \trFrob{G^0} \dar & \\
    0 \rar & \QCiso{\pi_0(G)} \rar \dar{\trFrob{\pi_0(G)}}
    & \QCiso{G} \rar \dar{\trFrob{G}} & \QCiso{G^0} \rar \dar{\trFrob{G^0}} & 0\\
    0 \rar & \pi_0(G)(\Fq)^* \rar \dar
    & \arrow[dashed]{d} G(\Fq)^* \rar & G^0(\Fq)^* \rar \dar & 0\\
    & \coker \trFrob{\pi_0(G)} & \coker \trFrob{G} &  \coker \trFrob{G^0} &
  \end{tikzcd}
  \]
  be the commutative diagram of abelian groups obtained by applying
  Proposition~\ref{prop:pullback} to \eqref{eq:pi0}.  By
  Proposition~\ref{prop:connected}, $\ker \trFrob{G^0} =0$ and $\coker \trFrob{G^0}=0$.
  By Propositions~\ref{prop:etale} and \ref{prop:X}, $\ker \trFrob{\pi_0(G)}=0$
  and $\coker \trFrob{\pi_0(G)}=0$.

  The sequence of character groups is left-exact: by
  Proposition~\ref{prop:pullback}, it is dual to the exact sequence of
  abelian groups
  \begin{equation}\label{eq:pi1}
  \begin{tikzcd}
    1 \rar & G^0(\Fq) \rar & G(\Fq) \rar & \pi_0(G)(\Fq) \rar & 0;
  \end{tikzcd}
  \end{equation}
  since $\Hom(\ - \ ,\EEx)$ is left-exact, the sequence of
  character groups is left exact.

  The sequence of groups of isomorphism classes of quasicharacter sheaves is right-exact: it is exact at
  $\QCiso{G^0}$ by Proposition~\ref{prop:restriction}; and
  it is exact at $\QCiso{G}$ because the left-square
  commutes (by Proposition~\ref{prop:pullback}), the bottom sequence
  is exact at $G(\Fq)^*$ and $\coker \trFrob{\pi_0(G)} =0$
  (by Propositions~\ref{prop:etale} and \ref{prop:X}) and
  \[
  \image(\QCiso{\pi_0(G)} \to \QCiso{G}) \subseteq \ker(\QCiso{G} \to \QCiso{G^0}),
  \]
  by the last part of Proposition~\ref{prop:pullback}. Specifically,
  suppose $\cs{L}$ is a quasicharacter sheaf on $G$ that is trivial on
  $G^0$. Then $\trFrob{\cs{L}}\vert_{G^0(\Fq)} = \trFrob{\cs{L}^0} =0$,
  by Proposition~\ref{prop:pullback} again. Since the sequence of
  character groups is exact at $G(\Fq)^*$, there is
  some $\chi \in \pi_0(G)(\Fq)^*$ such that the
  pull-back of $\chi$ along $G(\Fq)\to \pi_0(G)(\Fq)$ is
  $\trFrob{\cs{L}}$. By Propositions~\ref{prop:etale} and \ref{prop:X},
  there is a a quasicharacter sheaf $\cs{L}_\chi$ on $\pi_0(G)$, unique up
  to isomorphism, such that $\trFrob{\cs{L}_\chi} = \chi$. By
  Proposition~\ref{prop:pullback} again, $\cs{L}$ is isomorphic to the
  image of $\cs{L}_\chi$ under the monoidal functor $\QC(\pi_0(G)) \to \QC(G)$.

  It now follows immediately from the snake lemma that $\trFrob{G}$ is an
  isomorphism of groups.
\end{proof}

With Theorem~\ref{thm:snake} on hand, 
we can see why neither bounded quasicharacter sheaves 
nor finite quasicharacter sheaves are up to the task of 
geometrizing characters of $G(\Fq)$ in the generality we need.  
%
Consider the case when $G$ is the discrete etale group scheme $\ZZ$. 
Let $\chi : G(\Fq) \to \EEx$ be the character determined by $\chi(1) = \ell$ 
and let $\cs{L}$ be a quasicharacter sheaf in the isomorphism class 
corresponding to the character $\chi : G(\Fq) \to \EEx$ under Theorem~\ref{thm:snake}. 
Then $\trFrob{\cs{L}}(1) = \ell$. Suppose $\cs{L}$ was a bounded quasicharacter sheaf. 
Then the restriction of $\cs{L}$ to the connected component $1\in G$ 
would be a rank-1 $\ell$-adic local system on $1$, 
thus an $\ell$-adic character $\rho_1 : \pi_1(1,{\bar 1}) \to \EEx$ with $\rho_1(\Frob{\Fq}) = \ell$. But $\pi_1(1,{\bar 1}) = \Gal(\bFq/\Fq) \iso {\hat \ZZ}$ is compact, 
while $\rho_1(\pi_1(1,{\bar 1}))$ is not compact in $\EEx$ (it contains $\ell^\ZZ$). 
So the Weil sheaf $\gcs{L}$ does not descend to a local system on $G$.
It follows that the quasicharacter sheaf $\cs{L}$ is not a bounded quasicharacter sheaf.
%
We can play the same game with the character $\chi : G(\Fq) \to \EEx$ 
defined by $\chi(1) = 1+\ell$. In this case, the corresponding quasicharacter sheaf $\cs{L}$ is bounded but not finite, since $1+\ell$ is not a root of unity. 
%
These simple examples illustrate why the categories $\QCb(G)$ and $\QCf(G)$ are inadequate to geometrize characters of $G(\Fq)$.


\part{Application to characters of algebraic tori over local fields}

Let $K$ be a non-archimedean local field with residue field $\Fq$.
We place no restriction on the characteristic of $K$. 
We write $\OK$ for the ring of integers of $K$ and $\pK$ for the maximal ideal in $\OK$. 
For any positive integer $d$, set $A_d\ceq \OK/\pK^d$.
Let $T$ be an algebraic torus over $K$.
We write $\Hom_{< d}(T(K),\EEx)$ for the group of characters of $T(K)$ depth less than $d$ 
and $\Hom_{<\infty}(T(K),\EEx)$ for the group of characters of $T(K)$ of arbitrary finite depth.

\section{Greenberg of Neron}
\label{sec:GN}

A lft-N\'eron model for $T$, denoted here by $\TT$, 
is a separated, smooth, commutative group scheme locally of finite type over $\OK$ with generic fibre $T$, that enjoys the N\'eron mapping property: if
$S$ is a smooth scheme over $\OK$ then every morphism $S_\eta \to T$
over $K$ extends uniquely to a morphism $S \to \TT$. Equivalently, the
N\'eron mapping property states that the function
$\Hom_{\OK} (S,\TT) \to \Hom_K(S_\eta,T)$, given by restriction to
generic fibres, is a bijection; 
in particular, taking $S = \Spec{\OK}$, this implies $\TT(\OK) = T(K)$.

\begin{definition}\label{def:GN}
For each positive integer $d$, set $\TT_{d} \ceq \TT \times_\Spec{\OK} \Spec{A_d}$. Observe that $A_d$ is an Artin ring and $\TT_{d}$ is smooth and locally of finite type over $A_d$.
%
Let $\GN{T}_{d} = \Gr(\TT_d)$ be the group scheme over $\Fq$ produced by applying
the Greenberg functor to $\TT_{d}$; this is the \emph{Greenberg transform of $\TT_d$}.
\end{definition}

See \cite{Greenberg:2} and \cite[Ch. 9, \S 6]{BLR} and the first few sections of \cite{Stasinski} for background on the Greenberg functor. The fundamental properties of $\GN{T}_d$ that we need are:
 \begin{enumerate}
   \labitem{(GN.0)}{dFq} $\GN{T}_d(\Fq) = \TT(A_d)$, canonically.
  \labitem{(GN.1)}{GNd} $\GN{T}_d$ is a smooth, commutative group scheme, locally of finite type over $\Fq$, and the component group of $\GN{T}_d$ is geometrically finitely generated.
  \end{enumerate}
In fact, $\pi_0(\GN{T}_d) = \pi_0(\TT) \times_{\Spec{\OK}} \Spec{\Fq}$, and the latter is finitely generated by \cite{Xarles}. 

\section{Application to quasicharacters of bounded depth} \label{sec:bdchar}

\begin{theorem}\label{thm:application}
Let $d$ be a positive integer.
\begin{enumerate}
\item 
For every algebraic torus $T$ over $K$, the trace of Frobenius 
  \[
\trFrob{T,d}:  \QCiso{\GN{T}_d} \to \Hom_{< d}(T(K),\EEx)
  \]
is an isomomorphism. 
\item The isomorphism $\trFrob{T,d}$ is functorial in the following sense: $T \mapsto \trFrob{T,d}$ is a natural transformation
from the additive functor
\[T \mapsto \QCiso{\GN{T}_d}\]
to the additive functor
\[
T \mapsto \Hom_{<d}(T(K),\EEx).
\]
\end{enumerate}
\end{theorem}

\begin{proof}
  By \ref{GNd}, we may apply Theorem~\ref{thm:snake} to
  $\GN{T}_d$ which provides a canonical isomorphism
  $\QCiso{\GN{T}_d} \iso \GN{T}_d(\Fq)^*$.
  By \ref{dFq}, $\GN{T}_d(\Fq) = \TT(A_d)$.
  Then we recall that the filtration of $T(K)$ used to
  define depth has the property that $\TT(A_d) \cong T(K) / T(K)_d$
  \cite[Prop 5.2]{Yu}, so characters of $\GN{T}_d(\Fq)$ 
  are in bijection with characters of $T(K)$ vanishing on $T(K)_d$. 
  This shows that $\trFrob{T,d}$ is an isomorphism.
  
  Next we show that this isomorphism is functorial.
The first part of Proposition~\ref{prop:pullback} shows that $F_1$ is a functor, while the second part shows that Trace of Frobenius is a natural transformation $t: F_1 \to F_2$. When further combined with Lemma~\ref{prop:product}, we see that $F_1$ is an additive functor and $t: F_1 \to F_2$ is a natural transformation between additive functors.
\end{proof}


\section{The geometrization of quasicharacters} \label{sec:quasichar} 

\begin{definition}
The \emph{Greenberg transform of the N\'eron model of $T$} is the group pro-scheme over $\Fq$ given by
\begin{equation}\label{projlim}
  \cdef{\GN{T}} \ceq \varprojlim_{d\in \ZZ_{>0}} \GN{T}_{d}.
\end{equation}
Let$\QC(\GN{T})$ be the ind-category $\varinjlim_{d\in \ZZ_{>0}} \QC(\GN{T}_d)$;
we refer to objects of $\QC(\GN{T})$ as \emph{quasicharacter sheaves for $T$}.
\end{definition}

\begin{remark}
The Greenberg transform of the N\'eron model of $T$ is a commutative group scheme over $\Fq$. 
We will not use that fact in this paper. 
\end{remark}

\begin{corollary}\label{cor:application}
\begin{enumerate}
\item 
For every algebraic torus $T$ over $K$, the trace of Frobenius 
  \[
\trFrob{T}:  \QCiso{\GN{T}} \to \Hom_{\text{qc}}(T(K),\EEx)
  \]
is an isomomorphism. 
\item The isomorphism $\trFrob{T}$ is functorial in the following sense: $T \mapsto \trFrob{T}$ is a natural transformation
from the additive functor
\[T \mapsto \QCiso{\GN{T}}\]
to the additive functor
\[
T \mapsto \Hom_{\text{qc}}(T(K),\EEx).
\]
\end{enumerate}
\end{corollary}


\section{Transfer of quasicharacter sheaves} \label{sec:transfer}

The techniques in this paper apply to all non-archimdean local fields $K$ with residue field $k$, and to all algebraic tori $T$ over $K$, without placing any restrictions on the characteristic of $K$ or on the ramification of $T$. Because of this, it is natural to ask if we can compare quasicharacter sheaves on $T$ and on $T'$ under conditions when there is a natural comparison between characters of $T(K)$ and of $T'(K')$. This comparison of has been made precise in \cite{CY}, building on ideas of \cite{Deligne:limites}. 

We recall the notion of $N$-congruent tori from \cite{CY}*{\S 2}.  
Suppose $T$ and $T'$ are tori over non-archimedian local fields $K$ and $K'$, 
splitting over $L$ and $L'$ respectively.
Then $T$ and $T'$ are said to be \emph{$N$-congruent} if there are isomorphisms
 \begin{align*}
  \alpha : \OL/\pi_K^N \OL &\to \OO{L'}/\pi_{K'}^N \OO{L'} \\
  \beta : \Gal(L/K) &\to \Gal(L'/K') \\
  \phi : X^*(T) &\to X^*(T')
 \end{align*}
 satisfying the following conditions:
 \begin{enumerate}
  \item $\alpha$ induces an isomorphism $\OK/\pi_K^N \OK \to \OO{K'}/\pi_{K'}^N \OO{K'}$,
  \item $\alpha$ is $\Gal(L/K)$-equivariant relative to $\beta$: $\alpha(\gamma x) = \beta(\gamma) \alpha(x)$
  for $\gamma \in \Gal(L/K)$ and $x \in \OL/\pi_K^N \OL$,
  \item $\phi$ is $\Gal(L/K)$-equivariant relative to $\beta$.
 \end{enumerate}
It follows from this definition that if $T$ and $T'$ are $N$-congruent then $\alpha$, $\beta$ and $\phi$ determine an isomorphism
\begin{equation}\label{transfer}
  \Hom_{<N}(T(K), \EEx) \iso \Hom_{<N}(T'(K'),\EEx).
\end{equation}
Note that if $T$ and $T'$ are $N$-congruent, then they are $N'$ congruent for every $N' \leq N$.
In this section we ask if the isomorphism \eqref{transfer} comes from an equivalence of categories of quasicharacter sheaves on $\GN{T}_d$ and those on $\GN{T'}_d$.

One of the main results of \cite{CY} gives an isomorphism of group schemes between
$\TT_d$ and $\TT'_d$ assuming that $T$ and $T'$ are sufficiently congruent.
They define a quantity $h$ as the smallest integer so that $\pi^h$ lies in the
Jacobian ideal associated to a natural embedding of $T$ into an induced torus \cite{CY}*{\S 8.1}; let $h'$ be this quantity for $T'$.
Then it follows from \cite{CY}*{Thm. 8.5} that if $T$ and $T'$ are $N$-congruent and $h=h'$ and $N > 3h$, then there is a canonical isomorphism
 \[
  \TT_{N-3h} \to \TT'_{N-3h}
 \]
 determined by $\alpha, \beta$ and $\phi$.
As an immediate consequence of this theorem and Theorem \ref{thm:application} we have the following

\begin{proposition}
 With notation above, suppose that $T$ and $T'$ are $N$-congruent and $h=h'$ and $N > 3h$.  Set $d = N-3h$.
 Then there is a canonical equivalence of categories
 \[
  \QC(\GN{T}'_{d}) \to \QC(\GN{T}_{d})
 \]
 determined by $\alpha, \beta$ and $\phi$ inducing the isomorphism
 \[
  \Hom_{<d}(T'(K'), \EEx) \to \Hom_{<d}(T(K), \EEx).
 \]
\end{proposition}

\section{Toward geometric reciprocity}

As indicated in the Introduction, we are tantalized 
by the idea that quasicharacter sheaves for $T$ 
may play a key role in a geometric, categorial version 
of the local Langlands correspondence for algebraic tori 
over non-archimedean local fields. 
As with the local Langlands correspondence for algebraic tori, 
it will be possible to reduce the essential points of the argument to the case $T= \Gm{K}$, provided that the category of quasicharacter sheaves behaves properly with regard to Weil restriction; see \cite{Yu:tori}*{Thm 7.5 (2)}. 
The next proposition shows that the category of quasicharacter sheaves for 
algebraic tori does indeed behave properly with regard to Weil restriction.

\begin{proposition}
Let $K'/K$ be a finite Galois extension; 
let $k'/k$ be the corresponding finite Galois extension of residue fields.
Let $T$ be an algebraic torus over $K$ and 
set $T' = T \times_\Spec{K} \Spec{K'}$.  
The natural inclusion $T \hookrightarrow \Res_{K'/K} T'$ 
induces a map of $k$-schemes
\[
\GN{T} \hookrightarrow \Res_{k'/k} \GN{T}'
\]
and thus a functor on quasicharacter sheaves through pullback.
\[
\QC(\Res_{k'/k} \GN{T}') \to \QC(\GN{T}).
\]
Through trace of Frobenius, this functor induces the homomorphism
\[
\begin{tikzcd}[column sep=60]
\Hom_{<\infty}(T(K'), \EEx) \arrow{r}{\chi \mapsto \chi\vert_{T(K)}} &\Hom_{<\infty}(T(K), \EEx).
\end{tikzcd}
\] 
\end{proposition}

\begin{proof}
\Clifton{The proof will require the results in the base change section.}
There is a canonical isomorphism
\[
\Res_{k'/k} \Gr(\TL) \cong \Gr(\underline{\Res_{L/K}(T_L)}).
\]
\end{proof}


This paper has shown how quasicharacter sheaves 
on the Greenberg transform of the Neron model of $T$ 
determine admissible characters of $T(K)$, and vice versa, functorially. When coupled with the reciprocity map $\text{rec}_T $ for $T$, 
which enjoys the same functorial properties, 
it follows that quasicharacter sheaves for $T$ 
determine Langlands parameters for $T$, and vice versa, functorially.
\[
\begin{tikzcd}
\ & \arrow[swap]{dl}{\trFrob{G}} \QCiso{\GN{T}} \arrow[dashed]{dr}  & \\
\Hom_{<\infty}(T(K),\EEx) \arrow{rr}{\text{rec}_T} && H^1(K,\hat{T}_\ell) \\
\end{tikzcd}
\]
We suspect that it is possible to extract Langlands parameters 
from quasicharacter sheaves directly and that the reciprocity map 
can then be be viewed as pairing characters and parameters coming
 from the same quasicharacter sheaf. 
 Results in this paper reduce the problem to the case $T=\Gm{K}$. 
 In that case, our suspicions are all but confirmed 
 by the class field theory of Serre-Hasewinkel as revisited in \cite{Suzuki&Yoshida:refinement}. 
 What remains to be done is a careful comparison of 
 quasicharacter sheaves for $\Gm{K}$ and certain local systems on
  the fpqc-site of the pro-quasi-algebraic varieties appearing in \cite{Suzuki&Yoshida:refinement}. 

%\printindex

\begin{bibdiv}
\begin{biblist}

\bib{BBD}{article}{
   author={Be{\u\i}linson, A. A.},
   author={Bernstein, J.},
   author={Deligne, P.},
   title={Faisceaux pervers},
%   language={French},
   conference={
      title={Analyse et topologie sur les espaces singuliers (I)},
      address={Luminy},
      date={1981},
   },
   book={
      series={Ast\'erisque},
      volume={100},
      publisher={Soc. Math. France},
      place={Paris},
   },
   date={1982},
%   pages={5--171},
%   review={\MR{751966 (86g:32015)}},
}

\bib{BLR}{book}{
   author={Bosch, Siegfried},
   author={L{\"u}tkebohmert, Werner},
   author={Raynaud, Michel},
   title={N\'eron models},
   series={Ergebnisse der Mathematik und ihrer Grenzgebiete (3) [Results in
   Mathematics and Related Areas (3)]},
   volume={21},
   publisher={Springer-Verlag},
   place={Berlin},
   date={1990},
%   pages={x+325},
%   isbn={3-540-50587-3},
%   review={\MR{1045822 (91i:14034)}},
}

\bib{Brandal}{book}{
 author={Brandal, Willy},
 title={Commutative Rings whose Finitely Generated Modules Decompose},
 publisher={Springer-Verlag},
 date={1979},
 series={Lecture Notes in Mathematics},
 number={723}
}

\bib{Chai}{article}{
   author={Chai, C.L.},
   title={N\'eron models for semiabelian varieties: congruence and change of base field.},
   journal={Asian J. Math.},
   volume={4},
   number={4},
   date={2000},
   pages={715--736}}
   
\bib{CY}{article}{
   author={Chai, C.L.},
   author={Yu, J.-K.},
   title={Congruences of N\'eron models for tori and the Artin conductor},
   journal={Ann. of Math.},
   volume={154},
   number={2},
   date={2001},
   pages={347-382}}

\bib{Deligne:tensorielles}{article}{
   author={Deligne, P.},
   title={Cat\'egories tensorielles},
%   language={French, with English summary},
%   note={à Yu. I. Manin, en témoignage d'admiration},
   journal={Mosc. Math. J.},
   volume={2},
   date={2002},
   number={2},
   pages={227--248},
%   issn={1609-3321},
%   review={\MR{1944506 (2003k:18010)}},
}

\bib{Deligne:Weil2}{article}{
   author={Deligne, Pierre},
   title={La conjecture de Weil. II},
%   language={French},
   journal={Inst. Hautes \'Etudes Sci. Publ. Math.},
   number={52},
   date={1980},
   pages={137--252},
%   issn={0073-8301},
%   review={\MR{601520 (83c:14017)}},
}

\bib{Deligne:limites}{article}{
   author={Deligne, P.},
   title={Les corps locaux de caract\'eristique $p$, limites de corps locaux
   de caract\'eristique $0$},
%   language={French},
   conference={
      title={Representations of reductive groups over a local field},
   },
   book={
      series={Travaux en Cours},
      publisher={Hermann},
      place={Paris},
   },
   date={1984},
   pages={119--157},
%   review={\MR{771673 (86g:11068)}},
}

\bib{SGA4.5}{book}{
   author={Deligne, P.},
   title={Cohomologie \'etale},
   series={Lecture Notes in Mathematics, Vol. 569},
   note={S\'eminaire de G\'eom\'etrie Alg\'ebrique du Bois-Marie 1963-64 SGA
   4$\frac{1}{2}$; 
   Avec la collaboration de J. F. Boutot, A. Grothendieck, L. Illusie et J.
   L. Verdier},
   publisher={Springer-Verlag},
   place={Berlin},
%date={4h},
   date={1977},
%   pages={iv+312pp},
%   review={\MR{0463174 (57 \#3132)}},
}

\bib{SGA7.2}{book}{
   title={Groupes de monodromie en g\'eom\'etrie alg\'ebrique. II},
%   language={French},
 author={Deligne, P.},
 author={Katz, N.},
   series={Lecture Notes in Mathematics, Vol. 340},
   note={S\'eminaire de G\'eom\'etrie Alg\'ebrique du Bois-Marie 1967--1969
   (SGA 7 II);
   Dirig\'e par P. Deligne et N. Katz},
   publisher={Springer-Verlag},
   place={Berlin},
   date={1973},
%   pages={x+438},
%   review={\MR{0354657 (50 \#7135)}},
}

\bib{vdG&M}{book}{
author={Geer, van der},
author={Moonen},
title={Abelian Varieties},
note={http://staff.science.uva.nl/$\sim$bmoonen/boek/BookAV.html}
}

\bib{Greenberg:2}{article}{
   author={Greenberg, Marvin J.},
   title={Schemata over local rings. II},
   journal={Ann. of Math. (2)},
   volume={78},
   date={1963},
   pages={256--266},
%   issn={0003-486X},
%   review={\MR{0156855 (28 \#98)}},
}

\bib{SGA1}{collection}{
   title={Rev\^etements \'etales et groupe fondamental (SGA 1)},
%   language={French},
   author={Grothendieck, A.},
   series={Documents Math\'ematiques (Paris), 3},
   note={S\'eminaire de g\'eom\'etrie alg\'ebrique du Bois Marie 1960--61.
   Un s\'eminaire dirig\'e par A. Grothendieck
Augment\'e de deux expos\'es de Mme M. Raynaud
\'Edition recompos\'e et annot\'e du volume 224 des Lecture Notes in Mathematics publi\'e en 1971 par Springer-Verlag},
   publisher={Soci\'et\'e Math\'ematique de France},
   place={Paris},
   date={2003},
   pages={xviii+327},
%  isbn={2-85629-141-4},
%   review={\MR{2017446 (2004g:14017)}},
}



\bib{Kamgarpour}{article}{
   author={Kamgarpour, M.},
   title={Stacky abelianization of algebraic groups},
   journal={Transform. Groups},
   volume={14},
   date={2009},
   number={4},
   pages={825--846},
%   issn={1083-4362},
%   review={\MR{2577200 (2011b:20136)}},
%   doi={10.1007/s00031-009-9067-8},
}

\bib{Laumon}{article}{
   author={Laumon, G.},
   title={Transformation de Fourier, constantes d'\'equations fonctionnelles
   et conjecture de Weil},
%   language={French},
   journal={Inst. Hautes \'Etudes Sci. Publ. Math.},
   number={65},
   date={1987},
   pages={131--210},
%   issn={0073-8301},
%   review={\MR{908218 (88g:14019)}},
}

\bib{Serre:isogenies}{article}{
   author={Serre, Jean-Pierre},
   title={Corps locaux et isog\'enies},
%   language={French},
   conference={
      title={S\'eminaire Bourbaki, Vol.\ 5},
   },
   book={
      publisher={Soc. Math. France},
      place={Paris},
   },
   date={1995},
   pages={Exp.\ No.\ 185, 239--247},
%   review={\MR{1603470}},
}


\bib{Stasinski}{article}{
   author={Stasinski, Alexander},
   title={Reductive group schemes, the Greenberg functor, and associated
   algebraic groups},
   journal={J. Pure Appl. Algebra},
   volume={216},
   date={2012},
   number={5},
   pages={1092--1101},
%   issn={0022-4049},
%   review={\MR{2875329}},
%   doi={10.1016/j.jpaa.2011.10.027},
}

\bib{Suzuki&Yoshida:refinement}{article}{
   author={Suzuki, Takashi},
   author={Yoshida, Manabu},
   title={A refinement of the local class field theory of Serre and
   Hazewinkel},
   conference={
      title={Algebraic number theory and related topics 2010},
   },
   book={
      series={RIMS K\^oky\^uroku Bessatsu, B32},
      publisher={Res. Inst. Math. Sci. (RIMS), Kyoto},
   },
   date={2012},
   pages={163--191},
%   review={\MR{2986923}},
}

\bib{Tate:thesis}{article}{
   author={Tate, J. T.},
   title={Fourier analysis in number fields, and Hecke's zeta-functions},
   conference={
      title={Algebraic Number Theory (Proc. Instructional Conf., Brighton,
      1965)},
   },
   book={
      publisher={Thompson, Washington, D.C.},
   },
   date={1967},
   pages={305--347},
   review={\MR{0217026 (36 \#121)}},
}



\bib{Xarles}{article}{
   author={Xarles, Xavier},
   title={The scheme of connected components of the N\'eron model of an algebraic torus},
   journal={J. Reine Angew. Math.},
   volume={437},
   date={1993},
   pages={167--179},
%   issn={0075-4102},
%   review={\MR{1212256 (94d:14044)}},
%   doi={10.1515/crll.1993.437.167},
}


\bib{Yu}{article}{
   author={Yu, Jiu-Kang},
   title={Smooth models associated to concave functions in Bruhat-Tits theory}
}

\bib{Yu:tori}{article}{
   author={Yu, Jiu-Kang},
   title={On the local Langlands correspondence for tori},
   conference={
      title={Ottawa lectures on admissible representations of reductive
      $p$-adic groups},
   },
   book={
      series={Fields Inst. Monogr.},
      volume={26},
      publisher={Amer. Math. Soc.},
      place={Providence, RI},
      editor={Cunningham, C.},
      editor={Nevins, M.},
   },
   date={2009},
   pages={177--183},
   review={\MR{2508725 (2009m:11201)}},
}

\end{biblist}
\end{bibdiv}

\end{document}