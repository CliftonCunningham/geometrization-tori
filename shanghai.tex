% !TEX encoding = UTF-8 Unicode
\documentclass[11pt]{amsart}
\title[Short groovy title goes here]{Long groovy title goes here}
\usepackage[british]{babel}
\usepackage{datetime}
\date{\today}
\author{Clifton Cunningham}
\address{University of Calgary}
\email{cunning@math.ucalgary.ca}
\author{David Roe}
\address{Pacific Institute for the Mathematical Sciences at the University of Calgary}
\email{roed.math@gmail.com}
\usepackage[utf8]{inputenc}
\renewcommand{\baselinestretch}{1.2}
\usepackage[notcite,color]{showkeys}
\usepackage{hyperref}
\usepackage{geometry}
\usepackage{amsthm}
\usepackage{amsmath}
\usepackage{amssymb}
%\usepackage[shortalphabetic]{amsrefs}
\usepackage[alphabetic]{amsrefs}
\renewcommand\MR{\relax}
\usepackage{xypic}
\usepackage{textcomp}
\usepackage{mathrsfs}
\usepackage{yfonts}
\usepackage[T1]{fontenc}
\usepackage{tikz}
\usetikzlibrary{shapes,arrows}
\usepackage{manfnt}

%\include{definitions}
\theoremstyle{plain}
      \newtheorem{theorem}{Theorem}[section]
      \newtheorem{proposition}[theorem]{Proposition}
      \newtheorem{lemma}[theorem]{Lemma}
      \newtheorem{corollary}[theorem]{Corollary}
      
      \theoremstyle{definition}
      \newtheorem{definition}[theorem]{Definition}
      
      \theoremstyle{remark}
      \newtheorem{remark}[theorem]{Remark}

%%%% Some Named Categories

%%%% Functors

\newcommand{\gal}[1]{{\operatorname{Gal}\hskip-1pt\left( {\bar #1}/#1 \right)}}
\newcommand{\Spec}[1]{{\operatorname{Spec}\hskip-1pt( #1 )}}

\newcommand{\FF}{{\mathbb{F}}}
\newcommand{\ZZ}{{\mathbb{Z}}}
\newcommand{\NN}{{\mathbb{N}}}
\newcommand{\CC}{{\mathbb{C}}}
\newcommand{\QQ}{{\mathbb{Q}}}
\newcommand{\RR}{{\mathbb{R}}}
\newcommand{\EE}{\mathbb{\bar Q}_\ell}
\newcommand{\OK}{\mathcal{O}_K}
\newcommand{\pK}{\mathfrak{p}_K}
\newcommand{\Zp}{\mathbb{Z}_p}
\newcommand{\Qp}{\mathbb{Q}_p}
%\newcommand{\Fp}{\mathbb{F}_p}
\newcommand{\bFq}{\bar{k}}
\newcommand{\Fq}{k}
\newcommand{\Fqm}{k_m}
%\newcommand{\bFp}{{\mathbb{\bar F}_p}}

\DeclareMathOperator{\Gal}{Gal}
\DeclareMathOperator{\Weil}{W}
\newcommand{\Frob}{{\operatorname{Fr}}}
\DeclareMathOperator{\Aut}{Aut}
\DeclareMathOperator{\Hom}{Hom}
\DeclareMathOperator{\ord}{ord}
\DeclareMathOperator{\coker}{coker}
\DeclareMathOperator{\Gr}{Gr}
\DeclareMathOperator{\Irrep}{Irrep}
\DeclareMathOperator{\Pic}{Pic}
\DeclareMathOperator{\id}{id}
\DeclareMathOperator{\Ext}{Ext}
\DeclareMathOperator{\Hh}{H}

\newcommand{\cdef}[1]{{{\color{cyan}#1}\index{#1}}} 
\newcommand{\sheafHom}{{\mathscr{H}\hskip-4pt{\it o}\hskip-2pt{\it m}}}
\newcommand{\abs}[1]{{\vert #1 \vert}}
\newcommand{\ceq}{{\, :=\, }}
\newcommand{\tq}{{\ \vert\ }}
\newcommand{\iso}{{\ \cong\ }}
\newcommand{\obj}{{\text{obj}\, }}
\newcommand{\Gm}[1]{\mathbb{G}_{\hskip-2pt\textbf{m},#1}}
\newcommand{\GN}[1]{\textswab{#1}}
\newcommand{\bGN}[1]{{\bar{\textswab{#1}}}}
\newcommand{\TT}{\underline{T}}
\newcommand{\invlim}[1]{\lim\limits_{\overleftarrow{#1}}}
\newcommand{\dirlim}[1]{\lim\limits_{\overrightarrow{#1}}}
\newcommand{\limit}[1]{\mathop{\textsc{lim}}\limits_{#1}}
\newcommand{\colimit}[1]{\mathop{\textsc{colim}}\limits_{#1}}
\newcommand{\cs}[1]{{\mathcal{#1}}}
\newcommand{\gcs}[1]{{\mathcal{\bar #1}}}
\newcommand{\dualgcs}[1]{\gcs{#1}^\dagger}
\newcommand{\dualcs}[1]{\cs{#1}^\dagger}
\newcommand{\CS}{{\mathcal{C\hskip-1.8pt S}}}
\newcommand{\CSiso}[1]{\CS(#1)_{/\textit{iso}}}
\newcommand{\CSbiso}[1]{\CS_0(#1)_{/\text{iso}}}
\newcommand{\CE}{{\mathcal{C\hskip-1.8pt E}}}
\newcommand{\Lgroup}[1]{{\,^L\hskip-1pt{#1}}}
\newcommand{\dualgroup}[1]{{\check{#1}}}
\newcommand{\Lang}{{\operatorname{Lang}}}
\newcommand{\image}{{\operatorname{im}}}

\makeatletter
\newcommand{\labitem}[2]{%
\def\@itemlabel{\textbf{#1}}
\item
\def\@currentlabel{#1}\label{#2}}
\makeatother

\newcommand{\tighttimes}{\hskip-1pt\times\hskip-1pt}
\newcommand{\tightertimes}{\hskip-3pt\times\hskip-3pt}
\newcommand\todo[1]{\ \vspace{5mm}\par \noindent\framebox{\begin{minipage}[c]{0.95 \textwidth} \tt #1\end{minipage}} \vspace{5mm} \par}
\newcommand\Clifton[1]{\marginpar{\smaller\smaller CC: #1}}
\newcommand\David[1]{\marginpar{\smaller\smaller DR: #1}}

\begin{document}


\begin{abstract}
  We study the category of character sheaves on smooth commutative
  group schemes $G$ locally of finite type over finite fields
  $\mathbb{F}_q$. Assuming that the geometric component group of $G$
  is finitely-generated, we show that the group of isomorphism classes
  of character sheaves on $G$ is canonically isomorphic to the group
  of characters of $G(\mathbb{F}_q)$. We use this result to exhibit an
  isomorphism between admissible characters of $T(K)$ and isomorphism
  classes of character sheaves on the Greenberg transform of the Néron
  model of $T$, where $K$ is any non-Archimdean local field and $T$ is
  any algebraic torus over $K$.
\end{abstract}

\maketitle

\section{Introduction}

%\input{introduction}
In his celebrated thesis, John Tate writes 
\begin{quotation}
    {\it Concerning the characters of $k^*$ [the
    multiplicative group of a $\mathfrak{p}$-adic field], the
    situation is different from that of $k^+$.  First of all, we are
    interested in continuous multiplicative maps $\alpha \mapsto
    c(\alpha)$ of $k^*$ into the complex numbers, not only in the
    bounded ones, and shall call such a map a quasi-character,
    reserving the word ``character'' for the conventional character of
    absolute value $1$. Secondly, we shall find no model for the group
    of quasi-characters, or even for the group of characters, though
    such a model would be of the utmost importance.}
  \cite{Tate:thesis}*{\S 2.3}
\end{quotation}
\noindent
In this paper we show that the monidal category of character sheaves
on the Greenberg transform of the locally finite type Neron model of
$T$ provides an elegant, canonical model for the group of
(quasi-)characters of $T(K)$, where $K$ is any non-Archimedean local
field and $T$ is any algebraic torus over $K$.

Since the character sheaves that appear in this paper are not quite an
instance of those defined by George Luzstig, who coined the term, we
begin by explaining what we mean by `character sheaves' in this
paper. Let $G$ be a smooth commutative group scheme locally of finite
type over $\Fq$ \David{I think our notation is clearer if we use $k$
  rather than $\Fq$}; $G$ need not be reductive, connected, nor of
finite type over $\Fq$. Fix a prime $\ell$ that does not divide
$q$. By a {character sheaf} on $G$ we mean a triple $\cs{L}\ceq
(\gcs{L},\mu,F)$ where $\gcs{L}$ is a constructible $\ell$-adic sheaf
$\gcs{L}$ on the \'etale site of ${\bar G} \ceq G \times_{\Spec\Fq}
\Spec{\bFq}$, locally constant of rank-$1$ on every connected
component of ${\bar G}$, equipped with an isomorphism $\mu: m^*
\gcs{L} \to \gcs{L}\boxtimes \gcs{L}$, where $m : G\times G \to G$ is
the group operation for $G$, and an isomorphism $F : \Frob_{G}^*
\gcs{L} \to \gcs{L}$, where $\Frob_{G}$ is the Frobenius automorphism
on ${\bar G}$, satisfying natural compatibility conditions with $\mu$;
these compatibility conditions are explained below.

Character sheaves on $G$ form a category, denoted by $\CS(G)$. In
fact, $\CS(G)$ is a rigid symmetric monoidal category, so isomorphism
classes of character sheaves on $G$ form a group, denoted by
$\CSiso{G}$; this is the Picard group of the category of
character sheaves on $G$. Our application of character sheaves to
characters of algebraic tori over non-Archimedean local fields rests
on the Picard group of character sheaves on $G$, for certain smooth
group schemes $G$.

We now say a few words about the commutative group schemes locally of
finite type over $\Fq$ that appear in our geometrization of admissible
characters of algebraic tori over local fields. Let $K$ be a
non-Archimedean local field with residual field $\Fq$; we place no
restrictions on the characteristic of $K$ -- in particular, it may be
$0$ or not. Write $\OK$ for ring of integers of $K$ and $\pK$ for the
maximal ideal in $\OK$.  Let $T$ be an algebraic torus defined over
$K$. Let $\TT$ be a N\'eron model for $T$; recall that this is a
smooth group scheme locally of finite type over $\OK$. For each
non-negative integer $d$, let $\GN{T}_d$ be the Greenberg transform of
$\TT \otimes \OK/\pK^d$; the commutative group scheme $\GN{T}_d$
plays the r\^{o}le of $G$, above.
 
The main result of this paper is a canonical isomorphism
\[
\CSiso{\textswab{T}_d} 
\iso
\Hom_{\leq d}(T(K),\EE^\times)
\] 
between the Picard group of the category of character sheaves on
$\textswab{T}_d$ and the group of characters $T(K) \to \EE^\times$ of
depth at most $d$.  Since these isomorphisms are compatible with
respect to $d$, we also .....  \Clifton{I'll come back to working on
  this when we're closer to being finished. I'd be just as happy if
  you took this cup from my lips, though.}  In particular, by
specializing to the case $T= \mathbb{G}_{\hskip-1pt\text{m},K}$ and by
choosing an isomorphism $\EE \approx \CC$ (unnatural though it may
be), we arrive at a model for the group of quasi-characters of
$K^\times$: the monoidal category of character sheaves on the
Greenberg transform of the Neron model of
$\mathbb{G}_{\hskip-2pt\text{m},K}$.


Structure of the paper:

Thanks to:

\tableofcontents

\section{The Picard group of character sheaves}\label{section:category}

%\input{category}
%\subsection*{Smooth commutative group schemes}

Throughout this paper, $G$ stands for a commutative group scheme,
smooth and locally of finite type over a finite field $\Fq$; we write
$m : G\tightertimes G\to G$ for the multiplication morphism. We fix an
algebraic closure $\bFq$ of $\Fq$ and write ${\bar G} $ for the
commutative group scheme, smooth and locally of finite type over
$\bFq$ obtained by base change; the multiplication morphism for
${\bar G}$ will be denoted by ${\bar m}$. Let $\Frob_{G}$
(resp. $\Frob_{G\tighttimes G}$) be the Frobenius automorphism of
${\bar G}$ (resp. ${\bar G}\tightertimes {\bar G}$).
\Clifton{I'll insert a reference to SGA where Frobenius is defined for
  a large class of schemes over finite fields.}

%\subsection*{The category of character sheaves}

The category of character sheaves on $G$, denoted by \cdef{$\CS(G)$},
is defined as follows.  A \cdef{character sheaf} on $G$ is a triple
$\cs{L}\ceq (\gcs{L},\mu,F)$ where:
\begin{enumerate}
\labitem{(CS.0)}{CS.0} $\gcs{L}$ is a constructible $\ell$-adic
  sheaf on (the \'etale site of) ${\bar G}$, locally constant of rank
  $1$ on connected components of ${\bar G}$;

\labitem{(CS.1)}{CS.1} $\mu: {\bar m}^* \gcs{L} \to \gcs{L}\boxtimes \gcs{L}$
  is an isomorphism of sheaves on ${\bar G}\tightertimes {\bar G}$ such that

  \[
  \xymatrix{
    {\bar m}_3^*\gcs{L} \ar[rr]^{({\bar m}\times \id)^*\mu} \ar[d]_{(\id\tightertimes {\bar m})^*\mu}
    &&  {\bar m}^*\gcs{L}\boxtimes \gcs{L} \ar[d]^{\mu \boxtimes \id} \\
    \gcs{L} \boxtimes {\bar m}^* \gcs{L} \ar[rr]
    &&  \gcs{L} \boxtimes\gcs{L} \boxtimes \gcs{L}\\
  }
  \]
  commutes, where $m_3 \ceq m\circ (m\tightertimes\id) = m\circ (\id\tighttimes m)$;

\labitem{(CS.2)}{CS.2} $F : \Frob_{G}^* \gcs{L} \to \gcs{L}$ is an
  isomorphism of sheaves on ${\bar G}$ compatible with $\mu$ in the sense that

  \[
  \xymatrix{
    \Frob_{G\tighttimes G}^* {\bar m}^* \gcs{L} \ar[rr]^{\Frob_{G\tighttimes G}^*\mu}
    && \Frob_{G\tighttimes G}^*(\gcs{L}\boxtimes \gcs{L})\\
    \ar@{=}[u] {\bar m}^*  \Frob_{G}^* \gcs{L} \ar[d]_{{\bar m}^* F}
    && \ar[d]^{F\boxtimes F} \ar@{=}[u] \Frob_{G}^*\gcs{L}\boxtimes \Frob_{G}^*\gcs{L}\\
    {\bar m}^*\gcs{L} \ar[rr]^{\mu}
    && \gcs{L} \boxtimes\gcs{L}\\
  }
  \]
  commutes.
\end{enumerate}
Morphisms in $\CS(G)$ are defined in the natural way:
\begin{enumerate}
\labitem{(CS.3)}{CS.3} if $\cs{L} = (\gcs{L},\mu,F)$ and
  $\cs{L'} = (\gcs{L'},\mu',F')$ are character sheaves on $G$ then
  $\Hom(\cs{L},\cs{L'})$ is the set of morphisms $\varphi : \gcs{L} \to \gcs{L'}$
  of constructible $\ell$-adic sheaves on ${\bar G}$ such then
  \[
  \xymatrix{
  \ar[d]_{F} \Frob_G^* \gcs{L} \ar[r]^{\Frob_G^* \varphi} & \ar[d]^{F'} \Frob_G^* \gcs{L'}
  & \text{and}& \ar[d]_{\mu} m^* \gcs{L} \ar[r]^{m^* \varphi} & \ar[d]^{\mu'} m^* \gcs{L'} \\
  \gcs{L} \ar[r]^{\varphi} & \gcs{L'}
  && \gcs{L}\boxtimes\gcs{L} \ar[r]^{\varphi\boxtimes \varphi} & \gcs{L'} \boxtimes \gcs{L'}\\
  }
  \]
  both commute.
\end{enumerate}

%\subsection*{The Picard group of character sheaves}

%\begin{proposition}\label{tensor}
The category $\CS(G)$ of character sheaves on $G$ is a rigid monoidal
category [insert reference] under
$\cs{L} \otimes \cs{L'} \ceq (\gcs{L}\otimes\gcs{L'}, \mu\otimes\mu', F\otimes F')$
with duals formed by applying the functor $\sheafHom(\ - \ ,(\EE)_{\bar G})$ to
conditions \ref{CS.0}, \ref{CS.1} and \ref{CS.2}.
%\end{proposition}
(The category of characters sheaves on $G$ is not abelian, so $\CS(G)$
is not a tensor category in the sense of \cite{Deligne:tensorielles}.)
%\begin{proof}\end{proof}
%
This rigid monoidal category structure for $\CS(G)$ puts a group
structure on the set $\CSiso{G}$ of isomorphism classes in
$\CS(G)$; we refer to this as the \cdef{Picard group} of character
sheaves on $G$; {\it cf.} {\tt http://ncatlab.org/nlab/show/Picard+group}.
\Clifton{Replace ncatlab with other references.}

%\subsection*{Trace of Frobenius}\label{sec:Frob}

Let $\cs{L}$ be a character sheaf on $G$. Every $g\in G(\Fq)$
determines a point ${\bar g}$ on ${\bar G}$ fixed by $\Frob_G$ and
therefore an automorphism $F_{\bar g}$ of $\gcs{L}_{\bar g}$. Note
that $\Aut_{\EE}(\gcs{L}_{\bar g}) \iso \EE^\times$, canonically (this
follows from second condition appearing in the definition of character
sheaves). The isomorphism $\mu : {\bar m}^* \gcs{L} \to \gcs{L}\boxtimes\gcs{L}$
guarantees that the function $t^\Frob_\cs{L} : G(\Fq)\to \EE^\times$,
defined by $g \mapsto F_{\bar g}$, is a group homomorphism;
we refer to $t^\Frob_\cs{L}$ as the \cdef{trace of Frobenius of $\cs{L}$}.
\Clifton{There is no actual trace involved, since $\gcs{L}$ is a rank
  $1$ local system. More generally, trace is needed.}

It is clear that $t^\Frob_{\cs{L}\otimes \cs{L'}} = t^\Frob_\cs{L} \cdot t^\Frob_\cs{L'}$
(pointwise multiplication of functions) and that $t^\Frob_\cs{L} = t^\Frob_\cs{L'}$
if $\cs{L} \iso \cs{L'}$ in $\CS(G)$. In this way we obtain a group homomorphism
\[
\cdef{t_{G}} : \CSiso{G} \longrightarrow \Hom_\text{grp}(G(\Fq),\EE^\times),
\qquad \text{defined by}\qquad \cs{L} \mapsto t^\Frob_\cs{L},
\]
from the Picard group of character sheaves on $G$ to the character group of $G(\Fq)$.

Our main result on the geometrization of admissible characters of
$p$-adic tori [insert internal reference here] follows from a general
fact established later in this paper: if the geometric component group
of $G$ is finitely-generated then the trace of Frobenius defines an
isomorphism from the Picard group of character sheaves $G$ to the
group of characters of $G(\Fq)$ (here, as throughout the paper, $G$ is
a commutative group scheme, smooth and locally of finite type over $\Fq$).

%\subsection*{Pull-back of character sheaves}\label{subsection:pullback}

\begin{proposition}\label{prop:pullback}
  If $f : H\to G$ is a morphism of commutative group schemes locally of finite type over $\Fq$ then
  $(\gcs{L},\mu,F) \mapsto ({\bar f}^*\gcs{L},({\bar f}\tightertimes{\bar f})^*\mu,{\bar f}^*F)$
  defines a monoidal functor $f^* : \CS(G) \to \CS(H)$ such that
  \[
  \xymatrix{
  \CSiso{G} \ar[r]^{f^*} \ar[d]^{t_{G}} &  \ar[d]^{t_{H}} \CSiso{H}\\
  \Hom_\text{grp}(G(\Fq),\EE^\times) \ar[r] & \Hom_\text{grp}(H(\Fq),\EE^\times)
  }
  \]
  is a commutative diagram of groups, where the lower homomorphism is
  dual to $f : H(\Fq)\to G(\Fq)$.  Moreover, $(f\circ g)^* = g^* \circ f^*$.
\end{proposition}

\begin{proof}
  Let $\cs{L}$ be a character sheaf on $G$. We start by showing that
  ${\bar f}^*\gcs{L}$ is locally constant on connected components of
  ${\bar H}$. Let $c_j :{\bar H}^j \hookrightarrow {\bar H}$ be any
  connected component and let $i$ be the image of $j$ under the group
  homomorphism $\pi_0({\bar G}) \to \pi_0({\bar H})$ obtained by
  applying the functor $\pi_0$ to ${\bar f}$; see \cite{vdG&M}*{III, 3.28}, for example.
  \Clifton{Replace with a better reference, eventually.}
  Write $c_i : \bar{G}^i \hookrightarrow \bar{G}$ for
  the inclusion of $\bar{G}^i$ and ${\bar f}^j : {\bar H}^j \to {\bar G}^i$
  for the restriction of $f$ to $\bar{H}^j$.  Since $\cs{L}$
  is a character sheaf, $\gcs{L}$ is locally constant on the connect
  components of ${\bar G}$; thus, there is a finite etale covering
  $a_i : X_i \to {\bar G}^i$ such that $a_i^* (\gcs{L}\vert_{\bar G^i})$
  is constant. Let $b_j : Y_j \to {\bar H}^j$ be the
  pull-back of $a_i$ along ${\bar f}^j$; let $g_j : Y_j \to X_i$ be
  the map opposite ${\bar f}^j$. Then $b_j$ is a finite etale covering
  of ${\bar H}^j$ and
  \[
  b_j^* \left( ({\bar f}^* \gcs{L})\vert_{{\bar H}^j}\right)
  = ({\bar f}\circ c_j \circ b_j)^*\gcs{L}
  = (c_i\circ a_i\circ g_j)^*\gcs{L} = g_j^* a_i^* (\gcs{L}\vert_{\bar G^i}).
  \]

  Since $a_i^* (\gcs{L}\vert_{\bar G^i})$ is a constant sheaf and since the pull-back
  of a constant sheaf (along $g_j$, in this case) is a constant sheaf,
  it follows that $b_j^* \left( ({\bar f}^*\gcs{L})\vert_{{\bar H}^j}\right)$
  is a constant sheaf. Thus, ${\bar f}^*\gcs{L}$ is locally constant on
  ${\bar H}^j$. Since $({\bar f}^*\gcs{L})_{\bar g} = \gcs{L}_{f({\bar g})}$ for every
  geometric point ${\bar g}$ on ${\bar G}$, and since $\gcs{L}$ has
  rank $1$ on connected components of ${\bar G}$, it follows that
  $f^*\gcs{L}$ also has rank $1$ on connected components of ${\bar G}$.
  This completes the proof that ${\bar f}^*\gcs{L}$ satisfies condition \ref{CS.0}.

  To see that $({\bar f}\tightertimes {\bar f})^* \gcs{L}$ satisfies
  condition \ref{CS.1} simply apply the functor $({\bar f}\tightertimes {\bar f})^*$
  to \ref{CS.1} for $\cs{L}$ and use the canonical isomorphism
  $({\bar f}\tightertimes {\bar f})^*(\gcs{L} \boxtimes\gcs{L}) \iso {\bar f}^*\gcs{L} \boxtimes {\bar f}^*\gcs{L}$
  (a few times). Likewise, to see that ${\bar f}^*\gcs{L}$ satisfies condition
  \ref{CS.2}, apply the same functor to \ref{CS.2} for $\gcs{L}$.
  Then use the fact that $f$ is defined over $\Fq$, so
  $({\bar f}\tightertimes {\bar f})^*\Frob_{G\tighttimes G}^* \iso \Frob_{G\tighttimes G} ({\bar f}\tightertimes {\bar f})^*$,
  and the fact that $f$ is a morphism of group schemes, so
  $({\bar f}\tightertimes {\bar f})^* {\bar m}^*\iso {\bar m}^* {\bar f}^*$
  (isomorphisms of functors on constructible sheaves).

  To see that $(\gcs{L},\mu,F) \mapsto ({\bar f}^*\gcs{L},({\bar f}\tightertimes {\bar f})^*\mu,{\bar f}^*F)$
  defines a functor $f^* : \CS(G) \to \CS(H)$ one applies the functors ${\bar f}^*$ and
  ${\bar f}^*\tightertimes{\bar f}^*$ to \ref{CS.3} and argue as
  above. It follows from the fact that sheaf hom commutes with these
  functors that $f^* : \CS(G) \to \CS(H)$ is a monoidal functor.

  The fact that the functor $f^* : \CS(G) \to \CS(H)$ commutes with
  the trace of Frobenius, in the sense above, follows immediately from
  the definitions; see \cite{Laumon}*{1.1.1.2}, where the ambient
  hypothesis that $X$ is of finite type over $\Fq$ can be replaced by
  the hypothesis that $X$ is locally of finite type over $\Fq$.
  \Clifton{I have not proved $(f\circ g)^* = g^* \circ f^*$, yet.}
\end{proof}


\section{Connected abelian algebraic groups}

%\input{connected-1}
% !TEX encoding = UTF-8 Unicode
%In this section we review a fundamental property of character
%sheaves on geometrically connected, commutative algebraic groups over $\Fq$.

\begin{proposition}\label{prop:connected-1}
  If $G$ is a geometrically connected commutative algebraic group over $\Fq$ then
  \[
  t_{G} : \CSiso{G} \to \Hom_\text{grp}(G(\Fq),\EE^\times),
  \]
  is an isomorphism of groups.
\end{proposition}

\begin{proof}
  We will prove this proposition by reducing it to the functions--sheaves dictionary
  \cite{Laumon}*{1.1}. The main point of this proof will therefore be to show that,
  {\it since $G$ is geometrically connected}, every character sheaf on $G$ descends to
  a local system on $G$ satisfying the analogue of \ref{CS.1} on $G$.

  First, observe that the forgetful functor $(\gcs{L},\mu,F) \mapsto (\gcs{L},F)$
  sends character sheaves on $G$ to Weil local systems on
  $G$ \cite{Deligne:Weil2}*{1.1.10} (by a Weil local system on $G$ we
  mean a Weil sheaf on $G$ which is a local system on ${\bar G}$).
  While is it not true that all Weil local systems on $G$
  descend to sheaves on $G$, those that appear in the image of the
  forgetful functor from character sheaves do.

  To see why this is so, consider the category \cdef{$\CS_0(G)$} of
  pairs $(\cs{L}_0,\mu_0)$ where $\cs{L}_0$ is a local system on $G$
  and $\mu_0: m^*\cs{L}_0 \to \cs{L}_0 \boxtimes \cs{L}_0$ satisfies
  condition \ref{CS.1} adapted to sheaves on $G$; maps in $\CS_0(G)$
  are defined by the second part of condition \ref{CS.2}, {\it mutatis mutandis}.
  \David{Should we move this definition since it will be important elsewhere?}
  Let $b : {\bar G} \to G$ be base change
  (pull-back of $G\to \Spec{\Fq}$ along $\Spec{\bFq} \to \Spec{\Fq}$).
  The proof of Proposition~\ref{prop:pullback} adapts
  (simplifies, in fact) to show that $b^* : \CS_0(G) \to \CS(G)$ is a
  monoidal functor. This functor is fully faithful; see
  \cite{SGA7.2}*{Expos\'e XIII} and \cite{BBD}*{Prop. 5.2.1}.  We will
  show that $b^*$ is essentially surjective.

  Let $g$ be any $\Fq$-rational point on $G$ (such as the identity for
  $G$) and let ${\bar g}$ be the geometric point on $G$ lying above
  $g$ (we fixed $\Fq \hookrightarrow \bFq$ long ago). Since $G$ is
  geometrically connected, the geometric point ${\bar g}$ determines
  an equivalence between the category of Weil local systems on $G$ and
  $\ell$-adic representations of $\Weil(G,{\bar g})$
  \cite{Deligne:Weil2}*{1.1.12}. Now let $(\gcs{L},\mu,F)$ be a
  character sheaf on $G$. Then the isomorphism class of the Weil local
  system $(\gcs{L},F)$ corresponds to an $\ell$-adic character of the
  Weil group $\Weil(G, {\bar g})$. The $\Fq$-rational point $g$ under
  the geometric point ${\bar g}$ determines a splitting
  $\Weil(\bFq/\Fq)\to \Weil(G,{\bar g})$ of the exact sequence
  \[
  \xymatrix{
    1 \ar[r] & \pi_1({\bar G}, {\bar g}) \ar[r] &  \Weil(G,{\bar g}) \ar[r]
    & \ar@/_1pc/@{.>}[l]_{g} \Weil(\bFq/\Fq) \ar[r] & 1
    }
  \]
  In this way the Weil local system $(\gcs{L},F)$ determines an $\ell$-adic character
  \[
  t_{(\gcs{L},F)}(g) : \Weil(\bFq/\Fq) \to \EE^\times \qquad \text{defined by}\qquad \Frob \mapsto t^\Frob_{(\gcs{L},F)}(g)
  \]
  of the Weil group $\Weil(\bFq/\Fq)$; the image of $\Frob$ under this
  character is $t^\Frob_\cs{L}(g)$, the trace of Frobenius of $\cs{L}$ at $g$.

  On the other hand, we have already seen that $t^\Frob_\cs{L} : G(\Fq) \to \EE^\times$
  is a group homomorphism. Since $G$ is an algebraic group over $\Fq$, it is a
  variety over $\Fq$, so finitely generated over $\Fq$, so $G(\Fq)$ is finite;
  it follows that $t^\Frob_\cs{L}(g) = t^\Frob_{(\gcs{L},F)}(g)$ is a root of unity,
  for every $g\in G(\Fq)$.  Since $\Weil(\bFq/\Fq)$ is generated by
  $\Frob$ and $t_{(\gcs{L},F)}(g) : \Weil(\bFq/\Fq) \to \EE^\times$ is
  a character, it follows that the image of $t_{(\gcs{L},F)}(g)$ is a
  finite subset of $\mu(\EE)$. Thus, $t_{(\gcs{L},F)}(g)$ extends to
  an $\ell$-adic character of $\Gal(\bFq/\Fq)$, which will be denoted
  by the same symbol.
  \[
  \xymatrix{
    1 \ar[r] & \ar@{=}[d] \pi_1({\bar G}, {\bar g}) \ar[r] & \ar[d] \Weil(G,{\bar g}) \ar[r] & \ar[d]  \Weil(\bFq/\Fq) \ar[r] & 1\\
    1 \ar[r] &  \pi_1({\bar G}, {\bar g}) \ar[r] & \pi_1(G,{\bar g}) \ar[r] & \Gal(\bFq/\Fq) \ar[r] & 1\\
    }
  \]

  We may now lift the $\ell$-adic character $t_{(\gcs{L},F)}(g) : \Gal(\bFq/\Fq) \to \EE^\times$
  to an $\ell$-adic character of $\pi_1(G,{\bar g})$ using the canonical topological group map
  $\pi_1(G,{\bar g}) \to \Gal(\bFq/\Fq)$, above. But ${\bar g}$ also
  determines an equivalence between the category of $\ell$-adic
  representations of $\pi_1(G,{\bar g})$ and local systems on $G$. Let
  $\cs{L}_0$ be a local system on $G$ in the isomorphism class
  determined by this $\ell$-adic character of $\pi_1(G,{\bar g})$.
  Then $b^*\cs{L}_0 \iso \gcs{L}$, so $b^*$ is essentially surjective.

  We have now shown that $b^* : \CS_0(G) \to \CS(G)$ is an equivalence
  of rigid monoidal categories. It follows immediately that $b^*$
  determines an isomorphism
  \[
  b^* : \CSbiso{G} \mathop{\longrightarrow}^{\iso} \CSiso{G}
  \]
  of the Picard groups determined by these rigid monoidal categories.
  \Clifton{\textdbend This last sentence glosses over some subtle points;
    I might need to address them in the `two perspectives' section.}

  To finish the proof observe that the functions--sheaves dictionary,
  as it pertains to characters, shows that
  \[
  t_{G} : \CSbiso{G} \to \Hom_\text{grp}(G(\Fq),\EE^\times)
  \]
  is an isomorphism of groups; see \cite{Laumon}*{1.1.3} or \cite{SGA4.5}*{Sommes trigonométriques}.
\end{proof}

\section{Commutative etale group schemes}

%\input{etale-1}
%In this section we establish a fundamental property of character
% sheaves on commutative etale group schemes over $\Fq$ that are
% geometrically finitely generated.

\begin{proposition}\label{prop:etale}
  If $G$ is a commutative \'etale group scheme over $\Fq$ and
  $G(\bFq)$ is finitely generated then there is a canonical
  isomorphism
  \[
  \CSiso{G} \longrightarrow H^1(\Weil(\bFq/\Fq), \Hom_\text{grp}(G(\bFq),\EE^\times)),
  \qquad \text{given by}\qquad
  [\cs{L}] \mapsto [t_\cs{L}],
  \]
  where $t_\cs{L}: \Weil(\bFq/\Fq)\to \Hom_\text{grp}(G(\bFq),\EE^\times)$ is a cocycle such that
  \[
  \forall g\in G(\Fq), \qquad t_\cs{L}(\Frob)(g) = t^\Frob_\cs{L}(g).
  \]
\end{proposition}

\begin{proof}
  To simplify notation, we set $W_\Fq \ceq \Weil(\bFq/\Fq)$ and
  $G(\bFq)^* \ceq \Hom_\text{grp}(G(\bFq),\EE^\times)$.  Recall the
  equivalence between etale group schemes over $\Fq$ and the category
  of groups equipped with a continuous action of $\Gal(\bFq/\Fq)$ (for
  the discrete topology on the group), given by $G \mapsto G(\bFq)$.
  Under this equivalence any character sheaf $\cs{L}=(\gcs{L},\mu,F)$
  on $G$ corresponds to:
  \begin{enumerate}
  \labitem{(cs.0)}{cs.0} an indexed set of one-dimensional
    $\EE$-vector spaces $\gcs{L}_x \ceq x^*\gcs{L}$, as $x$ runs over
    $G(\bFq)$;

  \labitem{(cs.1)}{cs.1} an indexed set of isomorphisms
    $\mu_{x,y} \ceq (x \times y)^*\mu : \gcs{L}_{x+y} \mathop{\longrightarrow}\limits^{\iso} \gcs{L}_{x} \otimes \gcs{L}_{y}$,
    for all $x,y \in G(\bFq)$, such that
    \[
    \xymatrix{
      \gcs{L}_{x+y+z} \ar[rr]^{\mu_{x+y,z}} \ar[d]_{\mu_{x,y+z}}
      && \gcs{L}_{x+y}\otimes \gcs{L}_{z} \ar[d]^{\mu_{x,y} \otimes \id} \\
      \gcs{L}_{x} \otimes \gcs{L}_{y+z} \ar[rr]^{\id \otimes \mu_{y,z}}
      && \gcs{L}_{x} \otimes\gcs{L}_{y} \otimes \gcs{L}_{z}\\
      }
    \]
    commutes, for all $x,y,z\in G(\bFq)$; and

  \labitem{(cs.2)}{cs.2} an indexed set of isomorphisms $F_{x} \ceq x^*F: \gcs{L}_{\Frob(x)} \to \gcs{L}_x$
    \Clifton{\textdbend There's hidden subtlety here too: maybe I
      should explain which Frobenius this is and how it acts. }
    such that
    \[
    \xymatrix{
      \gcs{L}_{\Frob(x)+\Frob(y)} \ar[d]_{F_{x+y}} \ar[rr]^{\mu_{\Frob(x),\Frob(y)}}
      &&\ar[d]^{F_x\otimes F_y}  \gcs{L}_{\Frob(x)}\otimes \gcs{L}_{\Frob(y)}\\
      \gcs{L}_{x+y} \ar[rr]^{\mu_{x,y}}
      && \gcs{L}_x \otimes\gcs{L}_y\\
      }
    \]
    commutes, for all $x,y\in G(\bFq)$.
  \end{enumerate}
  Under this equivalence, a map $\varphi : \cs{L} \to \cs{L'}$ of
  character sheaves on $G$ corresponds to:
  \begin{enumerate}
  \labitem{(cs.3)}{cs.3} an indexed set $\varphi_x \ceq x^*\varphi: \gcs{L}_x \to \gcs{L'}_x$
    of linear transformations (of one-dimensional $\EE$-vector spaces),
    as $x$ runs over $G(\bFq)$, such that
    \[
    \xymatrix{
      \ar[d]_{F_x} \gcs{L}_{\Frob(x)} \ar[r]^{\varphi_{\Frob(x)}} & \ar[d]^{F_x'} \gcs{L'}_{\Frob(x)}
      &\text{and}& \ar[d]_{\mu} \gcs{L}_{x+y} \ar[r]^{\varphi_{x+y}} & \ar[d]^{\mu'_{x+y}} \gcs{L'}_{x+y} \\
      \gcs{L}_x \ar[r]^{\varphi_x} & \gcs{L'}_x
      && \gcs{L}_x\otimes\gcs{L}_y \ar[r]^{\varphi_x\otimes \varphi_y} & \gcs{L'}_x \otimes \gcs{L'}_y\\
      }
    \]
    both commute.
  \end{enumerate}
  It is clear that the category defined by conditions \ref{cs.0},
  \ref{cs.1}, \ref{cs.2} and \ref{cs.3}, denoted by
  \cdef{$\CS(G(\bFq))$}, is equivalent to $\CS(G)$. We will prove the
  proposition by exhibiting an isomorphism between
  $\CSiso{G(\bFq)}$ and $H^1(W_{\Fq}, G(\bFq)^*)$.

  To produce the isomorphism between $\CSiso{G(\bFq)}$ and
  $H^1(W_{\Fq}, G(\bFq)^*)$ we must pick a global section of the line
  bundle $(\gcs{L},\mu)$, for each character sheaf $\cs{L}$ on $G$. We
  will show that the group isomorphism does not depend on these
  choices.

  First, however, we show that such global sections exist.  Since
  $G(\bFq)$ is finitely-generated abelian, it admits a decomposition
  $G(\bFq) \iso G(\bFq)_\text{free} \oplus G(\bFq)_\text{tors}$ such
  that $G(\bFq)_\text{tors}$ is finite.  Let $x_1, \ldots , x_r$ be a
  basis for the free part $G(\bFq)_\text{free}$ of $G(\bFq)$ and let
  $y_1, \ldots , y_t$ be a generators for the torsion part of
  $G(\bFq)$ corresponding to the decomposition
  $G(\bFq)_\text{tors} \iso \oplus_{j=1}^t \ZZ/m_j\ZZ$, with
  $m_j \vert m_{j+1}$. For each such $x_i$ pick $s(x_i)\in \gcs{L}_{x_i}$;
  for each such $y_j$ pick $s(y_j) \in \gcs{L}_{y_j}$; also set
  $s({\bar e}) \ceq 0 \in \gcs{L}_{\bar e}$. ....
  \Clifton{I'll finish this later.}
  In this way we define a section
  $s \ceq s_\cs{L} : G(\bFq) \to \coprod_{z\in G(\bFq)} \gcs{L}_z$ of
  projection $\coprod_{z\in G(\bFq)} \gcs{L}_z \to G(\bFq)$ such that

  \begin{equation}\label{section}
    \forall x,y \in G(\bFq), \qquad \mu_{x,y}(s(x+y)) = s(x)\otimes s(y).
  \end{equation}
  Under the equivalence at the beginning of this section, $s = s_\cs{L}$
  corresponds to a global section of the line bundle $(\gcs{L},\mu)$.

  We use the section $s_\cs{L}$ to define a cocycle $z_\cs{L} \in Z^1(W_\Fq,G(\bFq)^*)$
  as follows. Recall that $\cs{L}$ also determines a family of linear transformations
  $F_x : \gcs{L}_{\Frob(x)} \to \gcs{L}_x$ of one-dimensional $\EE$-vector
  spaces, so $F_x(s(\Frob(x)))$ is a scalar multiply of $s(x)$, for
  each $x\in G(\bFq)$; let $t_\cs{L}(\Frob)(x)$ be that scalar, so
  \begin{equation}\label{t}
    \forall x\in G(\bFq),\qquad F_{x}(s(\Frob(x)) = t_\cs{L}(\Frob)(x) s(x).
  \end{equation}
  Conditions \ref{cs.0} and \ref{cs.2}, together with \eqref{section}, guarantee that
  the scalar $z_\cs{L}(\Frob)(x)$ is non-zero, while condition~\ref{cs.1} forces
  $t_\cs{L}(\Frob)(x+y) = t_\cs{L}(\Frob)(x) \ t_\cs{L}(\Frob)(y)$,
  for all $x,y \in G(\bFq)$.
  % \Clifton{The alternative to this approach is to change the
  %   definition of character sheaves to incorporate the notion of
  %   Galois sheaves. That might not be so bad, but would add another
  %   layer of complication, which we wouldn't use profitably, expect
  %   here.}

  Now $t_\cs{L} : W_\Fq \to G(\bFq)^*$ is defined, recursively, by
  $t_\cs{L}(\Frob^n) \ceq t_\cs{L}(\Frob)\cdot \,^\Frob t_\cs{L}(\Frob^{n-1})$,
  where $\Frob$ acts on $G(\bFq)^*$ through pre-composition and $\cdot$ refers
  to pointwise multiplication of functions. This gives
  \[
  \obj \CS(G(\bFq)) \longrightarrow Z^1(W_\Fq,G(\bFq)^*),
  \qquad \text{defined by} \qquad \cs{L} \mapsto t_\cs{L}.
  \]

  Although we used a global section $s_\cs{L}$ in the definition of
  $t_\cs{L}$, the class of $t_\cs{L}$ does not depend on this
  choice. To see why, let $s_1$ and $s_2$ be two global sections of
  the line bundle $(\gcs{L},\mu)$, in the sense above and let $t_1$
  and $t_2$ be the cocycles defined by $s_1$ and $s_2$, respectively,
  as above. Since, for each $x\in G(\bFq)$, $\gcs{L}_x$ is a
  one-dimensional $\EE$-vector space, there is a unique scalar
  $a(x) \in \EE$ such that $s_2(x) = a(x) s_1(x)$; moreover, $a(x)$ is
  non-zero, since $s_2(x)$ is non-zero. Then \eqref{section} and
  condition~\ref{cs.1} together imply $a(x+y) = a(x)a(y)$, so
  $a\in G(\bFq)^*$. Now consider the coboundary $t_0 : W_\Fq \to G(\bFq)^*$
  defined by $t_0(\Frob) = \,^\Frob a \cdot a^{-1}$. Then $t_2 = t_0 t_1$,
  so $t_2$ and $t_1$ are cohomologous. In this way we see that
  \begin{equation}\label{cohomologous}
    \obj \CS(G(\bFq)) \longrightarrow H^1(W_\Fq,G(\bFq)^*),
    \qquad \text{defined by} \qquad \cs{L} \mapsto [t_\cs{L}],
  \end{equation}
  is independent of the choice of global sections made above.

  We now show that the function $\obj \CS(G(\bFq)) \longrightarrow H^1(W_\Fq,G(\bFq)^*)$
  factors through $\obj \CS(G(\bFq)) \to \CSiso{G(\bFq)}$.
  To see this, let $\varphi : \cs{L} \to \cs{L'}$ be an isomorphism and
  let $t$ and $t'$ be any cocycles representing the classes in the
  image of $\cs{L}$ and $\cs{L'}$ under \eqref{cohomologous}; by construction,
  $t$ and $t'$ are defined by sections $s$ and $s'$. Now, $\varphi_x(s(x))$ is a
  non-zero scalar multiple of $s'(x)$; define $\phi: G(\bFq) \to \EE^\times$
  by $\varphi_x(s(x)) = \phi(x) s'(x)$. By the second part of condition~\ref{cs.3},
  $\phi(x+y) = \phi(x)\phi(y)$, so $\phi \in G(\bFq)^*$. By the first part of
  condition~\ref{cs.3}, $t'(\Frob) = t(\Frob) \cdot (\,^{\Frob^{-1}}\phi \cdot \phi^{-1})$,
  so $t$ and $t'$ are cohomologous. This concludes the definition of
  \begin{equation}\label{pich1}
    \CSiso{G(\bFq} \longrightarrow H^1(W_\Fq,G(\bFq)^*),
    \qquad \text{defined by} \qquad [\cs{L}] \mapsto [t_\cs{L}].
  \end{equation}
  Now it follows from \eqref{t} that the image of the isomorphism
  class of $\cs{L}$ is indeed represented by a cocycle $t_\cs{L}$ such
  that $t_\cs{L}(\Frob) = t^\Frob_\cs{L}$ on $G(\Fq)$, which is to
  say, when evaluated at $\Frob$, the cocycle $t_\cs{L}$ is an
  extension of the trace of Frobenius of $\cs{L}$.

  It only remains to show that \eqref{pich1} is a group
  isomorphism. This is straightforward.
  \Clifton{Straightforward, yes, but it still needs to be written down here.}
\end{proof}

%\input{etale-2}
%Define $\rho \colon \Hh^1(W_{\Fq}, \Hom(G(\bFq), \EE^\times))
% \rightarrow \Hom(G(\Fq), \EE^\times)$ by $\rho(\phi)(g) =
% \phi(\Frob)(g)$.

\begin{proposition}\label{prop:X}
  Let $X$ be a finitely generated abelian group equipped with an
  action of $\Gal(\bFq/\Fq)$, continuous for the discrete topology on
  $X$.  Then
  \begin{align*}
    \Hh^1(\Fq, \Hom(X, \EE^\times)) &\xrightarrow{\rho} \Hom(\Hh^0(\Fq, X), \EE^\times) \\
    \rho([z])(x) &= z(\Frob)(x)
  \end{align*}
  is an isomorphism of groups.
\end{proposition}
\begin{proof}
  We first prove that $\rho$ is surjective.  Suppose that
  $f \in \Hom(X^\Frob, \EE^\times).$ Since $\EE^\times$ is divisible
  as an abelian group it is injective as a $\ZZ$-module and thus
  $\Ext^1_{\ZZ}(X/X^\Frob, \EE^\times) = 0$.  Applying the functor
  $\Hom(\mbox{---}, \EE^\times)$ to
  $$0 \rightarrow X^\Frob \rightarrow X \rightarrow X/X^\Frob \rightarrow 0$$
  and considering the associated long exact sequence gives the
  existence of an $\tilde{f} \in \Hom(X, \EE^\times)$ restricting to
  $f$.  Define a $1$-cocycle $z \in \Hh^1(\Fq, \Hom(X,\EE^\times))$ by
  mapping $\Frob$ to $\tilde{f}$.  Then $\rho([z]) = f$.

  Now suppose that $[z] \in \Hh^1(\Fq,\Hom(X,\EE^\times))$ with
  $\rho([z]) = 1$; write $\tilde{f}$ for $z(\Frob)$.  The injectivity
  of $\rho$ is equivalent to the existence of a homomorphism $\psi
  \colon X \rightarrow \EE^\times$ with
  $$\tilde{f}(x) = \frac{\psi(\Frob(x))}{\psi(x)}$$
  for $x \in X$.  In fact, we may replace the construction of $\psi$
  for all $\ZZ[\Gal(\bFq/\Fq)]$-modules $X$ with the construction of
  analogous $\psi$s for a different class of modules.

  Since the action of $\Gal(\bFq/\Fq)$ is continuous, there is an $m$
  so that $\Frob^m$ acts trivially on $X$; write $\Fqm$ for the degree
  $m$ extension of $\Fq$ and $\Gamma_m$ for $\Gal(\Fqm/\Fq)$.  Since
  $T^m-1$ factors as a squarefree product of cyclotomic polynomials,
  the Chinese remainder theorem expresses $\ZZ[\Gamma_m]$ as a product
  of $\ZZ[\zeta_d]$ for $d$ dividing $m$.  Each $\ZZ[\zeta_d]$ is the
  maximal order in a number field and thus $\ZZ[\Gamma_m]$ is a
  product of Dedekind domains.  Therefore $X$ decomposes into a direct
  sum of cyclic $\ZZ[\Gamma_m]$-modules \cite[?]{Brandal}, each of the
  form $\ZZ[\Gamma_m] / I$ for some ideal $I \subset \ZZ[\Gamma_m]$.
  The ideal $I$ corresponds to a direct product of ideals $I_d$ in the
  $\ZZ[\zeta_d]$, and within each $\ZZ[\zeta_d]$ the Chinese remainder
  theorem allows a further decomposition of $\ZZ[\zeta_d] / I_d$ as a
  direct sum of $\ZZ[\zeta_d]$-modules of the form $\ZZ[\zeta_d]/P^s$.
  So if we can construct, for every cyclotomic ring $\ZZ[\zeta_d]$,
  prime ideal $P \subset \ZZ[\zeta_d]$ and positive integer $s$, a
  $\psi \colon \ZZ[\zeta_d]/P^s \rightarrow \EE^\times$ with
  \begin{equation} \label{eq:psi-condition}
    \tilde{f}(x) = \frac{\psi(\zeta_d \cdot x)}{\psi(x)},
  \end{equation}
  then the existence of the original $\psi$ follows because $X$ is a
  direct sum of such $\ZZ[\zeta_d]/P^s$.

  We now fix a $d$, $P$ and $s$, write $\zeta$ for $\zeta_d$ and
  choose a polynomial
  $$h(T) = a_rT^r + a_{r-1}T^{r-1} + \cdots + a_0$$
  so that $P^s = (p^s, h(\zeta_d))$.
  % Setting $\psi = 1$ will satisfy equation \ref{eq:psi-condition}
  % when $d = 1$.
  An an abelian group, the quotient $\ZZ[\zeta] / P^s$ is generated by
  $\zeta^i$ for $0 \le i < \phi(d)$; we define
  \begin{equation} \label{eq:psi-def}
    \psi(\zeta^i) = \alpha \prod_{j=0}^{i-1} \tilde{f}(\zeta^j)
  \end{equation}
  for an $\alpha \in \EE^\times$ to be chosen later.  Since there are
  relations among the $\zeta^i$ we first need to check that $\psi$ is
  well-defined.

  The relations among the generators $\zeta^i$ are additively
  generated by $\zeta^ch(\zeta)$ as $c$ ranges from $0$ to
  $\phi(d)-r-1$.  If we set
  $$A_c = \sum_{i=0}^r a_i \sum_{j=0}^{i+c-1} \zeta^j$$
  then
  $$\psi(\zeta^ch(\zeta)) = \alpha^{h(1)} \tilde{f}(A_c).$$
  Now consider $(\zeta - 1)A_c$: the sums telescope and we get that
  $(\zeta-1)A_c = \zeta^ch(\zeta) - h(1).$ There are now two
  possibilities:
  \begin{enumerate}
  \item $d$ is a power of $p$; write $s = \phi(d)m + n$.  We may take
    $h(T)$ to be $p^m(T-1)^n$, in which case $\zeta^ch(\zeta) - h(1) \in P^s$,
    so $A_c$ is fixed under multiplication by $\zeta$ in
    $\ZZ[\zeta]/P^s$ and thus $\tilde{f}(A_c) = 1$.  In this case
    $\alpha^{h(1)} = 1$ regardless of $\alpha$.
  \item $d$ is not a power of $p$.  If two distinct primes divide $d$
    then $\zeta - 1$ is a unit; otherwise $\zeta - 1$ is a unit modulo
    $P^s$.  Let $\omega \in \ZZ[\zeta]$ satisfy
    $\omega(\zeta - 1) \equiv 1 \pmod{P^s}$ and set
    \begin{equation} \label{alpha-def}
      \alpha = \tilde{f}(\omega).
    \end{equation}
    Then $A_c + h(1)\omega$ is fixed under multiplication by $\zeta$
    and thus $\tilde{f}(A_c) = \tilde{f}(-h(1)\omega)$.
  \end{enumerate}
  In either case, $\psi(\zeta^ch(\zeta)) = 1$ for all $c$.  The only
  other relations imposed on the $\zeta^i$ are that they are all
  killed under multiplication by $p^s$.  Since $\tilde{f}$ is also a
  homomorphism $\ZZ[\zeta]/P^s \rightarrow \EE^\times$ the values of
  $\tilde{f}$ and thus of $\psi$ will be $p^s$-roots of unity.
  Therefore $\psi$ is well defined.

  We now check equation \ref{eq:psi-condition} for
  $x = \zeta^{\phi(d)-1}$.  Write
  $\Phi_d(T) = T^{\phi(d)} - b_{\phi(d)-1}T^{\phi(d)-1} - \cdots - b_0$
  and set
  $$B = \sum_{i=0}^{\phi(d)-1} -\zeta^i + b_i \sum_{j=0}^{i-1} \zeta^j.$$
  We have

  \begin{align*}
    \frac{\psi(\zeta^{\phi(d)})}{\psi(\zeta^{\phi(d)-1})\tilde{f}(\zeta^{\phi(d)-1})}
    &= \frac{\psi(b_{\phi(d)-1}\zeta^{\phi(d)-1} + \cdots + b_0)}{\psi(\zeta^{\phi(d)-1})\tilde{f}(\zeta^{\phi(d)-1})} \\
    &= \alpha^{-\Phi_d(1)} \tilde{f}(B)
  \end{align*}
  As before we see that $(\zeta - 1)B = \Phi_d(1) - \Phi_d(\zeta) = \Phi_d(1)$.
  We consider the same cases as before:

  \begin{enumerate}
  \item $d$ is a power of $p$, in which case $\Phi_d(1) = p$.  Thus
    $(\zeta-1)B = p = u(\zeta-1)^{\phi(d)}$ for a unit $u \in \ZZ[\zeta]$
    and $B - u(\zeta-1)^{\phi(d)-1}$ is fixed by $\zeta$,
    so that $\tilde{f}(B) = \tilde{f}(u(\zeta - 1)^{\phi(d)-1})$.  The
    multiplicative order of $\tilde{f}(u(\zeta-1)^{\phi(d)-1})$ is at most
    $\lceil \frac{s - 1 - (\phi(d)-1)}{\phi(d)}\rceil = \lceil \frac{s}{\phi(d)} \rceil - 1$
    while the additive order of $1 \in \ZZ[\zeta]/P^s$ is $\lceil \frac{s}{\phi(d)} \rceil$.
    Thus we may choose $\alpha = \psi(1)$ with $\alpha^p = \tilde{f}(B)$.
  \item $d$ is not a power of $p$ and $\omega$ is the inverse of
    $\zeta-1$ modulo $P^s$.  Then $\tilde{f}(B) = \tilde{f}(\omega\Phi_d(1))$.
  \end{enumerate}
  In either case, $\alpha^{-\Phi_d(1)}\tilde{f}(B) = 1$ and thus
  $\psi$ satisfies equation \ref{eq:psi-condition} for
  $x = \zeta^{\phi(d)-1}$.  The definition of $\psi$ in \ref{eq:psi-def}
  immediately implies that $\psi$ satisfies equation
  \ref{eq:psi-condition} for lesser powers of $\zeta$, and thus for
  all $x$ by linearity.  The existence of $\psi$ shows that $\rho$ is injective.
\end{proof}

\section{Restriction to the identity component}

%\input{restriction}
% !TEX encoding = UTF-8 Unicode
%\subsection*{Restriction to the identity component is surjective}

Let $G^0$ be the connected component of $G$ containing the identity.
By Proposition~\ref{prop:pullback}, pull-back along
$G^0\hookrightarrow G$ determines a monoidal functor
$\cs{L}\mapsto \cs{L}\vert_{G^0}$ from $\CS(G)$ to $\CS(G^0)$.  In
this section we show that this functor is essentially surjective. To
do this we employ a perspective on characters sheaves which is
different from that in Section~\ref{section:category} but which works
very well for geometrically connected abelian algebraic groups; this
perspective fails to extend properly to the more general context of
Section~\ref{section:category}.

\begin{proposition}\label{prop:restriction}
  The restriction functor $\CS(G)\to \CS(G^0)$ is essentially surjective.
\end{proposition}

\begin{proof}
  Because $G$ is smooth over $\Fq$, the identity component $G^0$ is a
  geometrically connected algebraic group over $\Fq$, clopen in $G$
  \cite{vdG&M}*{3.17}. By Proposition~\ref{prop:connected-1}, every
  character sheaf $\cs{E} = (\gcs{E},\mu^0,F^0)$ on $G^0$ descends to
  a line bundle on $G^0$.

  Let us review how line bundles on $G^0$ determine character sheaves
  on $G^0$.  Up to isomorphism, line bundles on $G^0$ may be
  apprehended in terms of pairs $(\pi,\psi)$,
  \David{Aren't there line bundles which are not character sheaves,
    for which $B$ is not an algebraic group?}
  where $\pi : B\to G^0$ is a finite etale morphism of algebraic
  groups and a character $\psi : A\to \EE^\times$, where
  $\Gal(\bFq/\Fq)$ acts trivially on $A \ceq \ker {\bar \pi}$, as
  summarized by the following diagram.
  \begin{equation}\label{E}
    \xymatrix{
      \EE^\times & 1\ar[r] & \ar@/_1pc/[ll]_{\psi} A  \ar[r] & B \ar[r]^{\pi} & G^0 \ar[r] & 1\\
      }
  \end{equation}
  (In fact, we may always take $B = G^0$ and $\pi = \Lang_{G^0}$, the
  Lang isogeny for $G^0$.)  Indeed, $\cs{E}$ is precisely the
  component of $\pi_! (\EE)_{B}$ on which $A$ acts through $\psi$.  To
  see that $\cs{E}$ is a line bundle on $G^0$, start by pulling back
  $\pi: B \to G^0$ along $m_0 : G^0\times G^0 \to G^0$.
  \begin{equation}
    \xymatrix{
      \EE^\times & 1\ar[r] & \ar@/_1pc/[ll]_{\psi} A  \ar[r] & B \ar[r]^{\pi} & G^0 \ar[r] & 1\\
      & 1\ar[r] & \ar@{.>}@/^1pc/[llu] \ar[u] A\times A  \ar[r]
      & \ar[u]^{m_{B}} B\times B \ar[r]^{\pi\tighttimes \pi}  & \ar[u]^{m_0} G^0\times G^0 \ar[r] & 1\\
      }
  \end{equation}
  As pictured above, the kernel of this pull-back is $A\times A$;
  indeed, $A\times A\to B\times B$ is the pull-back of $A\to B$ along
  $m_{B}$. Use this Cartesian square to define a character (the dotted
  arrow, above) of $A\times A$; that is, define $A\times A\to \EE^\times$
  by the composition of $A\times A \to A$ and $\psi : A\to \EE^\times$.
  Note that this character is $\psi \otimes \psi : A\times A\to \EE^\times$.
  \Clifton{I haven't explained how maps between diagrams define sheaf
    homomorphisms between local systems.}
  Since the lower sequence is precisely the sequence defining
  $\cs{E}\boxtimes\cs{E}$, this shows that, by construction, $\cs{E}$
  comes equipped with a canonical isomorphism $m_0^*\cs{E} \iso \cs{E}\boxtimes\cs{E}$.
  Now the base change $\cs{E}$ comes equipped with isomorphisms
  $F^0 : \Frob_{G^0}^* \gcs{E} \to \gcs{E}$ and
  $\mu^0 : {\bar m}_0^* \gcs{E} \to \gcs{E}\boxtimes \gcs{E}$
  satisfying conditions \ref{CS.1} and \ref{CS.2}. Then $(\gcs{E},\mu^0,F^0)$
  is a character sheaf of $G^0$.

  The next step is to show that $\pi: B \to G^0$ extends to a finite
  etale morphism of group schemes $f: H\to G$, smooth and locally of
  finite type over $\Fq$, in such a way that $H^0 = B$, $f^0 =\pi$ and
  $\pi_0(f) : \pi_0(H)\to \pi_0(G)$ is an isomorphism of component
  groups.
  \begin{equation} \label{extension-diagram}
    \xymatrix{
      & A \ar[d] \ar@{=}[r] & A \ar[d] &  \\
      & B \ar[r] \ar[d]_{\pi} & H \ar[r] \ar[d]_{f} & \pi_0(H) \ar[d]_{\pi_0(f)}^{\iso} \\
      & G^0 \ar[r] & G \ar[r] & \pi_0(G).
    }
  \end{equation}
  where all rows and columns are exact and all maps are defined over
  $\Fq$.  We will do so by passing back and forth between group
  schemes over $\Fq$ and their $\bFq$-points.  Extensions of
  $G^0(\bFq)$ by $A(\bFq)$, such as $A(\bFq) \to H^0(\bFq) \to G^0(\bFq)$,
  correspond to classes in $\Ext^1_{\ZZ[\Gamma]}(G^0(\bFq), A(\bFq))$.
  Similarly, extensions of $G(\bFq)$ by $A(\bFq)$ correspond to
  classes in $\Ext^1_{\ZZ[\Gamma]}(G(\bFq), A(\bFq))$.  The map
  $G^0(\bFq) \to G(\bFq)$ induces a map
  $\Ext^1_{\ZZ[\Gamma]}(G(\bFq), A(\bFq)) \to \Ext^1_{\ZZ[\Gamma]}(G^0(\bFq), A(\bFq))$,
  fitting into the long exact sequence derived from applying
  $\Hom(\mbox{---}, A(\bFq))$ to $G^0(\bFq) \to G(\bFq) \to \pi_0(G)(\bFq)$:
  $$\Ext^1_{\ZZ[\Gamma]}(G(\bFq), A(\bFq)) \to \Ext^1_{\ZZ[\Gamma]}(G^0(\bFq), A(\bFq)) \to \Ext^2_{\ZZ[\Gamma]}(\pi_0(G)(\bFq), A(\bFq)).$$
  We now show that $\Ext^2_{\ZZ[\Gamma]}(\pi_0(G)(\bFq), A(\bFq)) = 0$.

  It suffices to show that $\pi_0(G)(\bFq)$ has a two-term resolution
  by injective $\ZZ[\Gamma]$-modules.  Since the action of $\Gamma$ on
  $\pi_0(G)(\bFq)$ is continuous, it factor through a finite quotient
  $\Gamma_m = \Gamma / H_m$.  Any product of Dedekind domains has
  cohomological dimension $1$ \cite[?]{?}, so $\pi_0(G)(\bFq)$ has a
  two term resolution by injective $\ZZ[\Gamma_m]$-modules.  Let $I$
  be one of these modules: we need to show that $I$ is injective as a
  $\ZZ[\Gamma]$-module.  Consider the universal diagram for injectivity:
  \[
  \xymatrix{
    & B \\
    I & A \ar@{>->}[u] \ar[l]_{a}
    }
  \]

  where $A$ and $B$ are arbitrary $\ZZ[\Gamma]$-modules.  Since $H_m$ acts trivially on $I$,
  $I_{H_m}A := \langle (\gamma-1)a : \gamma \in H_m, a \in A \rangle \subseteq \ker(a).$
  Since $B / I_{H_m}B \cong B \otimes_A A / I_{H_m}A$, we get
  \[
  \xymatrix{
    & B/I_{H_m}B \ar@{..>}[dl] & B \ar@{->>}[l] \\
    I & A/I_{H_m}A \ar@{>->}[u] \ar[l] & A \ar@{>->}[u] \ar@{->>}[l]
    }
  \]
  The dotted arrow exists since $I$ is injective as a
  $\ZZ[\Gamma_m]$-module, and the outer diagram shows that $I$ is
  injective as a $\ZZ[\Gamma]$-module.

  We therefore have the existence of diagram \eqref{extension-diagram}
  at the level of $\bFq$-points.  This expresses $H(\bFq)$ as a
  disjoint union of translates of $B(\bFq)$; by transport of structure
  we may take $H$ to be a group scheme over $\bFq$.  Similarly, the
  restriction of $f$ to each component of $H$ is a morphism of
  schemes, and thus $f$ is as well.  Finally, the whole diagram
  descends to a diagram of $\Fq$-schemes since the $\bFq$-points of
  the objects come equipped with continuous $\Gamma$-actions, and the
  morphisms are $\Gamma$-equivariant.

  Now we show that $f : H\to G$, as above, together with a character
  $\psi : \ker {\bar f} \to \EE^\times$, defines a character sheaf on
  $G$. We will also show that the restriction of this character sheaf
  on $G$ to $G^0$ is precisely the character sheaf on $G^0$ defined by
  the pair $(f^0,\psi)$ as above.

  Set $X \ceq \pi_0(G)\times_{\Spec{\Fq}} \Spec{\bFq}$. We will
  identify $X$ with $\pi_0(H)\times_{\Spec{\Fq}} \Spec{\bFq}$ using
  the base change of $\pi_0(f): \pi_0(H)\to \pi_0(G)$. For each $x\in X$,
  write ${\bar f}^x : {\bar H}^x \to {\bar G}^x$ for the
  restriction of ${\bar f}$ to ${\bar H}^x$, as a cover of
  ${\bar G}^x$. For each $x,y\in X$, let
  ${\bar m}_{x,y} : {\bar H}^x \times {\bar H}^{y} \to {\bar H}^{x+y}$
  be the map obtained from ${\bar m}$ by restriction. Define
  ${\bar m}_{x} : A\to \Aut({\bar f}^x)$ by
  ${\bar m}_{x}(a): g \mapsto {\bar m}_{0,x}(a,g)$; then
  ${\bar m}_{x} : A\to \Aut({\bar f}^x)$ is an isomorphism.
  \Clifton{I need to show that ${\bar m}_{x} : A\to \Aut({\bar f}^x)$ is an isomorphism.}

  For each $x\in X$ let $\gcs{L}^x$ be the rank-$1$ local system on
  ${\bar G}^x$ determined by the finite etale cover
  ${\bar f}^x : {\bar H}^x \to {\bar G}^x$ and the character
  $\psi_x \ceq \psi \circ {\bar m}_x^{-1}$.  Since
  ${\bar G} = \coprod_{x\in X} {\bar G}^x$ (disjoint union), this
  determines a constructible sheaf $\gcs{L}$ on ${\bar G}$, locally
  constant of rank-$1$ on each connected component of ${\bar G}$.
  Taking the case $x=0$ we observe that $\gcs{L}^0 =\gcs{E}$,
  the latter defined above. To finish the proof of the proposition we
  equip $\gcs{L}$ with isomorphisms
  $\mu : {\bar m}^* \gcs{L} \to \gcs{L} \boxtimes\gcs{L}$ and
  $F : \Frob_{G}^* \gcs{L} \to \gcs{L}$ satisfying conditions \ref{CS.1}
  and \ref{CS.2}.

  Observe that ${\bar m}: {\bar G}\times {\bar G}\to {\bar G}$ is
  completely determined by the family of maps
  ${\bar m}_{x,y} : {\bar G}^x \times {\bar G}^{y} \to {\bar G}^{x+y}$.
  Accordingly, to define $\mu$ for $\gcs{L}$ it is sufficient to define isomorphisms
  $\mu_{x,y} : {\bar m}_{x,y}^*\gcs{L}^{x+y} \iso \gcs{L}^{x}\boxtimes \gcs{L}^{y}$
  for all $x,y\in X$. To do that, note first that the local system
  $\gcs{L}^{x}\boxtimes \gcs{L}^{y}$ is defined by the following diagram.
  \[
  \xymatrix{
    \EE^\times && \ar[ll]_{\hskip-20pt \psi_x\otimes \psi_y} \Aut({\bar f}^x)\times \Aut({\bar f}^y)
    & \ar@(ul,dl) & {\bar H}^x\times {\bar H}^y \ar[rr]^{{\bar f}^x\times {\bar f}^y} && {\bar G}^x\times {\bar G}^y\\
    }
  \]
  On the other hand, the diagram for the local system ${\bar m}_{x,y}^* \gcs{L}^{x+y}$
  is found by pulling-back the diagram for $\gcs{L}^{x+y}$ along
  ${\bar m}_{x,y} : {\bar H}^x\times{\bar H}^y \to {\bar H}^{x+y}$;
  the result is the middle horizontal diagram below, where we use the
  fact that the covering group of the middle arrow is $\Aut({\bar f}^{x+y})$, canonically.
  \[
  \xymatrix{
    & \ar[dl]_{\psi_{x+y}} \Aut({\bar f}^{x+y})
    & \ar@(ul,dl) & {\bar H}^{x+y} \ar[rr]^{{\bar f}^{x+y}} && {\bar G}^{x+y}\\
    \EE^\times & \ar@{.>}[l] \Aut({\bar f}^{x+y})  \ar@{=}[u] &  \ar@{.>}@(ul,dl)
    & \ar[u] {\bar H}^{x+y} \times_{{\bar G}^x\times{\bar G}^y} {\bar G}^x\times{\bar G}^y \ar[rr]
    && \ar[u]_{{\bar m}_{x,y}} {\bar G}^x\times {\bar G}^y\\
    & \ar[lu]_{\psi_x\otimes \psi_y} \ar@{.>}[u] \Aut({\bar f}^x)\times \Aut({\bar f}^y)
    & \ar@(ul,dl) & \ar[u] {\bar H}^x\times {\bar H}^y \ar[rr]^{{\bar f}^x\times {\bar f}^y}
    && \ar@{=}[u] {\bar G}^x\times {\bar G}^y\\
    }
  \]
  Now, the homomorphism from the covering group of
  ${\bar f}^x\times{\bar f}^y$ to the covering group of
  ${\bar f}^{x+y}$ induced by the composition
  $ {\bar H}^x\times {\bar H}^y \to {\bar H}^{x+y} \times_{{\bar G}^x\times{\bar G}^y} {\bar G}^x\times{\bar G}^y \to {\bar H}^{x+y}$
  is precisely $({\bar m}_x(a),{\bar m}_y(b)) \mapsto {\bar m}_{x+y}(ab)$.
  Since this map commutes with the characters $\psi_x\otimes\psi_y$
  and $\psi_{x+y}$ (so, the far-left triangle commutes in the diagram above)
  it follows that, for each $x,y\in X$ the diagram above defines a canonical isomorphism
  $\mu_{x,y} : {\bar m}_{x,y}^* \gcs{L}^{x+y} \to \gcs{L}^x \boxtimes\gcs{L}^y$.
  This defines $\mu: {\bar m}^* \gcs{L} \iso \gcs{L} \boxtimes\gcs{L}$.
  It is straightforward (but laborious) to show that the pair
  $(\gcs{L},\mu)$ satisfies condition \ref{CS.1}.

  Now we turn to the definition of $F : \Frob_{G}^* \gcs{L} \to \gcs{L}$.
  As above, we define this isomorphism through a family of
  isomorphisms on the connected components of ${\bar G}$. The
  restriction of $\Frob_{G}$ to ${\bar G}^x$ defines an isomorphism
  $\Frob_x : {\bar G}^x \to {\bar G}^{\Frob(x)}$.
  %\[
  %\xymatrix{
  %\ar[d]_{\Frob_x} {\bar G}^x \ar[r] & \ar[d]_{\Frob_{G}} {\bar G}\\
  %{\bar G}^{\Frob(x)} \ar[r] & {\bar G}\\
  %}
  %\]
  Pull back the finite etale covering in the diagram for
  $\gcs{L}^{\Frob(x)}$ along $\Frob_x$ and compare the result with the
  finite etale covering in the diagram for $\gcs{L}^{x}$.

  \[
  \xymatrix{
    && \ar[dll]_{\psi_{\Frob(x)}} \Aut({\bar f}^{\Frob(x)}) & \ar@(ul,dl)
    & {\bar H}^{\Frob(x)} \ar[rr]^{{\bar f}^{\Frob(x)}} && {\bar G}^{\Frob(x)}\\
    \EE^\times & & \ar@{.>}[ll] \Aut({\bar f}^{\Frob(x)})  \ar@{=}[u] & \ar@{.>}@(ul,dl)
    & \ar[u] {\bar H}^{\Frob(x)} \times_{{\bar G}^{\Frob(x)}} {\bar G}^x \ar[rr] && \ar[u]_{\Frob_{x}} {\bar G}^x\\
    & &  \ar[ull]^{\psi_x}  \ar@{.>}[u] \Aut({\bar f}^{x}) & \ar@(ul,dl)
    & \ar[u] {\bar H}^x \ar[rr] && \ar@{=} [u] {\bar G}^x\\
    }
  \]
  The covering group of ${\bar H}^{\Frob(x)} \times_{{\bar G}^{\Frob(x)}} {\bar G}^x$
  is isomorphic to $\Aut({\bar f}^{\Frob(x)})$, canonically, so
  $\Frob_x^* \gcs{L}^{\Frob(x)}$ is defined by the middle (horizontal)
  diagram, above.  We define the sheaf homomorphism
  $F_x : \Frob_x^* \gcs{L}^{\Frob(x)} \to \gcs{L}^x$ by the canonical morphism
  ${\bar H}^x \to {\bar H}^{\Frob(x)} \times_{{\bar G}^{\Frob(x)}} {\bar G}^x \to $;
  to see that this defines a sheaf homomorphism it is sufficient to
  check that the character $\psi_x$ commutes with the family of isomorphisms
  $F_x : (\Frob_{G}^* \gcs{L})\vert_{{\bar G}^x} \iso \Frob_x^* \gcs{L}^{\Frob{x}} \to \gcs{L}^x$
  now defines the isomorphism $F : \Frob_{G}^* \gcs{L} \to \gcs{L}$.
  \Clifton{Add some detail here.}
  This isomorphism is compatible with $\mu$ in the sense of condition~\ref{CS.2}.
\end{proof}

\begin{remark}
  We have seen that if $f : H\to G$ is a finite etale cover of $G$ by
  a smooth group scheme and if all geometric automorphisms of $f$ are
  fixed by Frobenius, then every character of $\ker f$ determines a
  character sheaf on $G$. However, it is not true that every character
  sheaf on $G$, in the sense of Section~\ref{sec:category} arises not
  all character sheaves on $G$ arise in this manner.
\end{remark}
%\input{centralextensions}

\section{Main result regarding character sheaves}

%\input{snake}
\begin{theorem}\label{thm:snake}
  If $G$ is a smooth abelian group scheme, locally of finite type over
  $\Fq$, and the abelian group $\pi_0(G)(\bFq)$ (the geometric
  component group) is finitely generated, then the trace of Frobenius
  homomorphism
  \[
  t_{G} : \CSiso{G} \to \Hom(G(\Fq),\EE^\times),
  \]
  from the Picard group of the category of character sheaves on $G$ to
  the group of characters of $G(\Fq)$, is an isomorphism.
\end{theorem}

\begin{proof}
  Consider the short exact sequence in the category of smooth group
  schemes, locally of finite type over $\Fq$, defining the component
  group scheme for $G$:
  \begin{equation}\label{eq:pi0}
    \xymatrix{
      1 \ar[r] & G^0 \ar[r] & G \ar[r] & \pi_0(G) \ar[r] & 0.
      }
  \end{equation}
  Let
  \[
  \xymatrix{
    & \ar[d] \ker t_{\pi_0(G)} & \ar@{.>}[d] \ker t_{G} & \ar[d] \ker t_{G^0} & \\
    0\ar[r] & \ar[d]^{t_{\pi_0(G)}} \CSiso{\pi_0(G)} \ar[r]
    & \ar[d]^{t_{G}} \CSiso{G} \ar[r] & \ar[d]^{t_{G^0}} \CSiso{G^0} \ar[r] & 0\\
    0 \ar[r] & \ar[d] \Hom(\pi_0(G)(\Fq),\EE^\times) \ar[r]
    & \ar@{.>}[d] \Hom(G(\Fq),\EE^\times) \ar[r] & \ar[d] \Hom(G^0(\Fq),\EE^\times) \ar[r] & 0\\
    & \coker t_{\pi_0(G)} & \coker t_{G} &  \coker t_{G^0} & \\
    }
  \]
  be the commutative diagram of abelian groups obtained by applying
  Proposition~\ref{prop:pullback} to \eqref{eq:pi0}.  By
  Proposition~\ref{prop:connected-1}, $\ker t_{G^0} =0$ and $\coker t_{G^0}=0$.
  By Propositions~\ref{prop:etale} and \ref{prop:X}, $\ker t_{\pi_0(G)}=0$
  and $\coker t_{\pi_0(G)}=0$.

  The sequence of character groups is left-exact: by
  Proposition~\ref{prop:pullback}, it is dual to the exact sequence of
  abelian groups
  \begin{equation}\label{eq:pi0}
    \xymatrix{
      1 \ar[r] & G^0(\Fq) \ar[r] & G(\Fq) \ar[r] & \pi_0(G)(\Fq) \ar[r] & 0;
    }
  \end{equation}
  since $\Hom(\ - \ ,\EE^\times)$ is left-exact, the sequence of
  character groups is left exact.

  The sequence of Picard groups is right-exact: it is exact at
  $\CSiso{G^0}$ by Proposition~\ref{prop:restriction}; and
  it is exact at $\CSiso{G}$ because the left-square
  commutes (by Proposition~\ref{prop:pullback}), the bottom sequence
  is exact at $\Hom(G(\Fq),\EE^\times)$ and $\coker t_{\pi_0(G)} =0$
  (by Propositions~\ref{prop:etale} and \ref{prop:X}) and
  \[
  \image(\CSiso{\pi_0(G)} \to \CSiso{G}) \subseteq \ker(\CSiso{G} \to \CSiso{G^0}),
  \]
  by the last part of Proposition~\ref{prop:pullback}. Specifically,
  suppose $\cs{L}$ is a character sheaf on $G$ that is trivial on
  $G^0$. Then $t^\Frob_\cs{L}\vert_{G^0(\Fq)} = t^\Frob_{\cs{L}^0} =0$,
  by Proposition~\ref{prop:pullback} again. Since the sequence of
  character groups is exact at $\Hom(G(\Fq),\EE^\times)$, there is
  some $\chi \in \Hom(\pi_0(G)(\Fq),\EE^\times)$ such that the
  pull-back of $\chi$ along $G(\Fq)\to \pi_0(G)(\Fq)$ is
  $t^\Frob_\cs{L}$. By Propositions~\ref{prop:etale} and \ref{prop:X},
  there is a a character sheaf $\cs{L}_\chi$ on $\pi_0(G)$, unique up
  to isomorphism, such that $t^\Frob_{\cs{L}_\chi} = \chi$. By
  Proposition~\ref{prop:pullback} again, $\cs{L}$ is isomorphic to the
  image of $\cs{L}_\chi$ under the monoidal functor $\CS(\pi_0(G)) \to \CS(G)$.

  It now follows immediately from the snake lemma that $t_{G}$ is an
  isomorphism of groups.
\end{proof}

\section{Application to characters of bounded depth}

%\input{GdeN}
Let $p$ be a prime number. Let $K$ be a local field with residue field
$\Fq$, where $p$ divides $q$. Let $\OK$ be the ring of integers of $K$
and let $\pK$ be the maximal ideal in $\OK$.
%
In this section we introduce a group scheme $\GN{T}$ over $\Fq$
obtained by applying the Greenberg functor to the N\'eron lft-model
for $T$ and establish some basic facts about $\GN{T}$

%\subsection{The N\'eron locally finite type model for $T$}

The lft-N\'eron model for $T$, denoted here by $\TT$, is a separated,
smooth, commutative group scheme locally of finite type over $\OK$
with generic fibre $T$, that enjoys the N\'eron mapping property: if
$X$ is a smooth scheme over $\OK$ then every morphism $X_\eta \to T$
over $K$ extends uniquely to a morphism $X \to \TT$. Equivalently, the
N\'eron mapping property states that the function
$\Hom_{\OK} (X,\TT) \to \Hom_K(X_\eta,T)$, given by restriction to
generic fibres, is a bijection; in particular, this implies $\TT(\OK) = T(K)$.

For each $d\in \ZZ_{>0}$, let $\TT_{d}$ be the smooth group scheme over
$\OK/\pK^d$ obtained from $\TT$ by base change along $\OK \to
\OK/\pK^d$. Observe that $\OK/\pK^d$ is an Artin ring and
$\TT_{d}$ is locally of finite type over $\OK/\pK^d$.

Let $\GN{T}_{d}$ be the group scheme over $\Fq$ produced by applying
the Greenberg functor to $\TT_{d}$. (See \cite{Greenberg:2} and
\cite[Ch. 9, \S 6]{BLR} for background on the Greenberg functor.)
%Then $\GN{T}_d(\Fq) = \TT(\OK/\pK^d)$.

\begin{proposition}\label{prop:GN}
  Fundamental properties of the group $\Fq$-scheme $\GN{T}_d$, for all $d\in \NN$:
  \begin{enumerate}
  \labitem{(GNd.0)}{GN0} $\GN{T}_d^0$, the identity component of
    $\GN{T}_d$, is a commutative algebraic group over $\Fq$;
  \labitem{(GNd.1)}{GNd} $\GN{T}_d$ is smooth and locally of finite type over $\Fq$;
  \labitem{(GNd.2)}{pi0} $\pi_0(\GN{T}_d) = \pi_0(\TT) \times_{\Spec{\OK}} \Spec{\Fq}$;
  \labitem{(GNd.3)}{fg} the component group of $\GN{T}_d$ is geometrically finitely generated;
  \labitem{(GNd.4)}{dFq} $\GN{T}_d(\Fq) = \TT(\OK/\pK^d)$, canonically;
\end{enumerate}
\end{proposition}

\begin{proof}
\Clifton{Proofs still missing, but all straightforward. I'll cite as much as I can, here.}
\end{proof}

%\input{application}
\begin{theorem}\label{thm:application}
  Let $T$ be an algebraic torus over a non-Archimdean local field $K$.
  Let $\GN{T}_d$ and $\GN{T}$ be as above.  For each $d \in \ZZ_{>0}$, the
  Picard group of the category of character sheaves on $\GN{T}_d$ is
  canonically isomorphic to the group of characters of $T(K)$ of depth
  less than $d$:
  \[
  \CSiso{\GN{T}_d} \iso \Hom_{< d}(T(K),\EE^\times).
  \]
\end{theorem}

\begin{proof}
  By \ref{GNd} and \ref{fg}, we may apply Theorem~\ref{thm:snake} to
  $\GN{T}_d$ which provides a canonical isomorphism
  \[
  \CSiso{\GN{T}_d} \iso \Hom(\GN{T}_d(\Fq),\EE^\times).
  \]
  By \ref{dFq}, $\GN{T}_d(\Fq) \iso \TT(\OK/\pK^d)$ canonically.
  To finish the proof we recall that the filtration of $T(K)$ used to
  define depth has the property that $\TT(\OK/\pK^d) \cong T(K) / T(K)_d$
  \cite[Prop 5.2]{Yu} for $d \in \ZZ_{>0}$.  Thus characters of $\GN{T}_d(\Fq)$ are
  in bijection with characters of $T(K)$ vanishing on $T(K)_d$.
\end{proof}

\section{Application to admissible characters}

%\input{pro}
To conclude the paper we offer an interpretation of
Theorem~\ref{thm:application} that makes no mention of depth.

\Clifton{Explain what happens as $d\in \NN$ varies.}
Suppose $d\leq d'$.

We may now assemble the group $\Fq$-schemes $\TT_{d}$ into one
glorious group pro-scheme over $\Fq$:
\begin{equation}\label{projlim}
  \GN{T} \ceq \varprojlim_{d\in \NN} \GN{T}_{d}.
\end{equation}
This is the \cdef{Greenberg transform of the lft-N\'eron model of $T$}.

\Clifton{Prove that $\GN{T}$ is a scheme? Strictly speaking, we don't
  care about \ref{scheme}.}
\begin{proposition}
  Fundamental properties of $\GN{T}$:
  \begin{enumerate}
  \labitem{(GN.0)}{scheme} $\GN{T}$ is a commutative group scheme over $\Fq$;
  \labitem{(GN.1)}{T(K)} $\GN{T}(\Fq) = T(K)$, canonically.
  \end{enumerate}
\end{proposition}


\begin{corollary}
  \[
  \CSiso{\GN{T}} \iso \Hom_\text{ad}(T(K),\EE^\times)
  \]
\end{corollary}



\begin{bibdiv}
\begin{biblist}

\bib{BBD}{article}{
   author={Be{\u\i}linson, A. A.},
   author={Bernstein, J.},
   author={Deligne, P.},
   title={Faisceaux pervers},
%   language={French},
   conference={
      title={Analyse et topologie sur les espaces singuliers (I)},
      address={Luminy},
      date={1981},
   },
   book={
      series={Ast\'erisque},
      volume={100},
      publisher={Soc. Math. France},
      place={Paris},
   },
   date={1982},
%   pages={5--171},
%   review={\MR{751966 (86g:32015)}},
}

\bib{BLR}{book}{
   author={Bosch, Siegfried},
   author={L{\"u}tkebohmert, Werner},
   author={Raynaud, Michel},
   title={N\'eron models},
   series={Ergebnisse der Mathematik und ihrer Grenzgebiete (3) [Results in
   Mathematics and Related Areas (3)]},
   volume={21},
   publisher={Springer-Verlag},
   place={Berlin},
   date={1990},
%   pages={x+325},
%   isbn={3-540-50587-3},
%   review={\MR{1045822 (91i:14034)}},
}

\bib{Brandal}{book}{
 author={Brandal, Willy},
 title={Commutative Rings whose Finitely Generated Modules Decompose},
 publisher={Springer-Verlag},
 date={1979},
 series={Lecture Notes in Mathematics}
 number={723}
}

\bib{Deligne:tensorielles}{article}{
   author={Deligne, P.},
   title={Cat\'egories tensorielles},
%   language={French, with English summary},
%   note={à Yu. I. Manin, en témoignage d'admiration},
   journal={Mosc. Math. J.},
   volume={2},
   date={2002},
   number={2},
   pages={227--248},
%   issn={1609-3321},
%   review={\MR{1944506 (2003k:18010)}},
}

\bib{Deligne:Weil2}{article}{
   author={Deligne, Pierre},
   title={La conjecture de Weil. II},
%   language={French},
   journal={Inst. Hautes \'Etudes Sci. Publ. Math.},
   number={52},
   date={1980},
   pages={137--252},
%   issn={0073-8301},
%   review={\MR{601520 (83c:14017)}},
}

\bib{Greenberg:2}{article}{
   author={Greenberg, Marvin J.},
   title={Schemata over local rings. II},
   journal={Ann. of Math. (2)},
   volume={78},
   date={1963},
   pages={256--266},
%   issn={0003-486X},
%   review={\MR{0156855 (28 \#98)}},
}

\bib{Laumon}{article}{
   author={Laumon, G.},
   title={Transformation de Fourier, constantes d'\'equations fonctionnelles
   et conjecture de Weil},
%   language={French},
   journal={Inst. Hautes \'Etudes Sci. Publ. Math.},
   number={65},
   date={1987},
   pages={131--210},
%   issn={0073-8301},
%   review={\MR{908218 (88g:14019)}},
}

\bib{SGA4.5}{book}{
   author={Deligne, P.},
   title={Cohomologie \'etale},
   series={Lecture Notes in Mathematics, Vol. 569},
   note={S\'eminaire de G\'eom\'etrie Alg\'ebrique du Bois-Marie 1963-64 SGA
   4$\frac{1}{2}$; 
   Avec la collaboration de J. F. Boutot, A. Grothendieck, L. Illusie et J.
   L. Verdier},
   publisher={Springer-Verlag},
   place={Berlin},
%date={4h},
   date={1977},
   pages={iv+312pp},
%   review={\MR{0463174 (57 \#3132)}},
}

\bib{SGA7.2}{book}{
   title={Groupes de monodromie en g\'eom\'etrie alg\'ebrique. II},
%   language={French},
 author={Deligne, P.},
 author={Katz, N.},
   series={Lecture Notes in Mathematics, Vol. 340},
   note={S\'eminaire de G\'eom\'etrie Alg\'ebrique du Bois-Marie 1967--1969
   (SGA 7 II);
   Dirig\'e par P. Deligne et N. Katz},
   publisher={Springer-Verlag},
   place={Berlin},
   date={1973},
   pages={x+438},
%   review={\MR{0354657 (50 \#7135)}},
}

\bib{Tate:thesis}{article}{
   author={Tate, J. T.},
   title={Fourier analysis in number fields, and Hecke's zeta-functions},
   conference={
      title={Algebraic Number Theory (Proc. Instructional Conf., Brighton,
      1965)},
   },
   book={
      publisher={Thompson, Washington, D.C.},
   },
   date={1967},
   pages={305--347},
   review={\MR{0217026 (36 \#121)}},
}

\bib{vdG&M}{book}{
author={van der Geer},
author={Moonen},
title={Abelian Varieties},
note={http://staff.science.uva.nl/$\sim$bmoonen/boek/BookAV.html}
}


\bib{Xarles}{article}{
   author={Xarles, Xavier},
   title={The scheme of connected components of the N\'eron model of an algebraic torus},
   journal={J. Reine Angew. Math.},
   volume={437},
   date={1993},
   pages={167--179},
%   issn={0075-4102},
%   review={\MR{1212256 (94d:14044)}},
%   doi={10.1515/crll.1993.437.167},
}


\bib{Yu}{article}{
   author={Yu, Jiu-Kang},
   title={Smooth models associated to concave functions in Bruhat-Tits theory}
}

\end{biblist}
\end{bibdiv}

\end{document}
