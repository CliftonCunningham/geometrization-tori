% !TEX encoding = UTF-8 Unicode
\documentclass{amsart}
\pdfoutput=1

%%%%%%%%%%%%%%%%%% HEADING %%%%%%%%%%%%%%%%%%

\title[Lost in Translation: The function--sheaf dictionary]{Lost in Translation: \\ The function--sheaf dictionary for smooth commutative group schemes over finite fields}

\date{\today}
\author{Clifton Cunningham}
\email{cunning@math.ucalgary.ca}
\address{University of Calgary}
\author{David Roe}
\email{roed.math@gmail.com}
\address{Pacific Institute for the Mathematical Sciences, University of Calgary}
\subjclass[2010]{14F05 (primary), 14L15 (secondary)}
\keywords{sheaf--function dictionary, geometrization, quasicharacter sheaves, commutative group schemes}

%%%%%%%%%%%%%%%%% PACKAGES %%%%%%%%%%%%%%%%%
% Encodings
\usepackage[utf8]{inputenc}
\usepackage[T1]{fontenc}
% AMS packages
\usepackage{amsmath, amsthm, amssymb, todonotes}% showkeys}
\usepackage[alphabetic]{amsrefs}
% Fonts
\usepackage{mathrsfs, yfonts}
\usepackage{geometry}
% TikZ
\usepackage{tikz}
\usetikzlibrary{shapes,arrows,calc,matrix}
\usepackage{tikz-cd}
% Hyperrefs
\usepackage{hyperref}

%%%%%%%%%%%%%%% THEOREM STYLES %%%%%%%%%%%%%%%
\theoremstyle{plain}
      \newtheorem{theorem}{Theorem}[section]
      \newtheorem{proposition}[theorem]{Proposition}
      \newtheorem{lemma}[theorem]{Lemma}
      \newtheorem{corollary}[theorem]{Corollary}

      \theoremstyle{definition}
      \newtheorem{definition}[theorem]{Definition}

      \theoremstyle{remark}
      \newtheorem{remark}[theorem]{Remark}

%%%%%%%%%%%%%%%% TIKZ SETTINGS %%%%%%%%%%%%%%%%
\tikzset{every picture/.style={>=stealth},label/.style={font=\footnotesize}}

%%%%%%%%%%%%%%% RINGS AND GROUPS %%%%%%%%%%%%%%%
\newcommand{\FF}{{\mathbb{F}}}
\newcommand{\ZZ}{{\mathbb{Z}}}
\newcommand{\NN}{{\mathbb{N}}}
\newcommand{\CC}{{\mathbb{C}}}
\newcommand{\QQ}{{\mathbb{Q}}}
\newcommand{\RR}{{\mathbb{R}}}
\newcommand{\EE}{\mathbb{\bar Q}_\ell}
\newcommand{\Zl}{\mathbb{\bar Z}_\ell}
\newcommand{\OK}{\mathcal{O}_K}
\newcommand{\pK}{\mathfrak{p}_K}
\newcommand{\OKp}{\mathcal{O}_{K'}}
\newcommand{\pKp}{\mathfrak{p}_{K'}}
\newcommand{\OL}{\mathcal{O}_L}
\newcommand{\OO}[1]{\mathcal{O}_{#1}}
\newcommand{\Zp}{\mathbb{Z}_p}
\newcommand{\Qp}{\mathbb{Q}_p}
\newcommand{\bFq}{\bar{k}}
\newcommand{\Fq}{k}
\newcommand{\Fqm}{k_m}
\newcommand{\WW}{\mathbb{W}}
\newcommand{\EEx}{\EE^\times}
\newcommand{\Zlx}{\mathbb{\bar Z}_\ell^\times}
\newcommand{\Weil}[1]{\mathcal{W}_{#1}}

%%%%%%%%%%%%%%% ALGEBRAIC GROUPS %%%%%%%%%%%%%%%
\newcommand{\mathswab}[1]{\operatorname{\textswab{#1}}}
\newcommand{\Gm}[1]{\mathbb{G}_{\hskip-1pt\textbf{m},#1}}
\newcommand{\GN}[1]{\mathswab{#1}}
\newcommand{\bGN}[1]{{\bar{\mathswab{#1}}}}
\newcommand{\TT}{\underline{T}}
\newcommand{\TL}{\underline{T_L}}
\newcommand{\TTp}{\underline{T}'}

%%%%%%%%%%%%%%% NAMED OPERATORS %%%%%%%%%%%%%%%
\DeclareMathOperator{\Gal}{Gal}
\DeclareMathOperator{\W}{W}
\newcommand{\Frob}[1]{\operatorname{F}_{#1}}
\DeclareMathOperator{\Aut}{Aut}
\DeclareMathOperator{\Hom}{Hom}
\DeclareMathOperator{\ord}{ord}
\DeclareMathOperator{\coker}{coker}
\DeclareMathOperator{\Gr}{Gr}
\DeclareMathOperator{\Irrep}{Irrep}
\DeclareMathOperator{\Pic}{Pic}
\DeclareMathOperator{\id}{id}
\DeclareMathOperator{\Ext}{Ext}
\DeclareMathOperator{\Hh}{H}
\DeclareMathOperator{\Res}{Res}
\DeclareMathOperator{\Nm}{Nm}
\DeclareMathOperator{\trace}{Tr}
\DeclareMathOperator{\obj}{obj}
\DeclareMathOperator{\mor}{mor}
\DeclareMathOperator{\Lang}{Lang}
\DeclareMathOperator{\image}{im}
\DeclareMathOperator{\Loc}{Loc}
\newcommand{\gal}[1]{{\operatorname{Gal}\hskip-1pt\left( {\bar #1}/#1 \right)}}
\newcommand{\Spec}[1]{{\operatorname{Spec}\hskip-1pt( #1 )}}

%%%%%%%%%%%% MISCELLANEOUS OPERATORS %%%%%%%%%%%%
\newcommand{\sheafHom}{{\mathscr{H}\hskip-4pt{\it o}\hskip-2pt{\it m}}}
\newcommand{\abs}[1]{{\vert #1 \vert}}
\newcommand{\ceq}{{\, :=\, }}
\newcommand{\tq}{{\ \vert\ }}
\newcommand{\iso}{{\ \cong\ }}
%% Limits
\newcommand{\invlim}[1]{\lim\limits_{\overleftarrow{#1}}}
\newcommand{\dirlim}[1]{\lim\limits_{\overrightarrow{#1}}}
\newcommand{\limit}[1]{\mathop{\textsc{lim}}\limits_{#1}}
\newcommand{\colimit}[1]{\mathop{\textsc{colim}}\limits_{#1}}
%% Fonts for quasicharacter sheaves
\newcommand{\qcs}[1]{{\mathcal{#1}}}
\newcommand{\gqcs}[1]{{\mathcal{\bar #1}}}
\newcommand{\dualgcs}[1]{\gqcs{#1}^\dagger}
\newcommand{\dualcs}[1]{\qcs{#1}^\dagger}
%% Categories
\newcommand{\QC}{{\mathcal{Q\hskip-0.8pt C}}}
\newcommand{\QCb}{{\QC_0}}
\newcommand{\QCf}{{\QC_f}}
\newcommand{\QCiso}[1]{\QC(#1)_{/\text{iso}}}
\newcommand{\QCbiso}[1]{\QCb(#1)_{/\text{iso}}}
\newcommand{\QCfiso}[1]{\QCf(#1)_{/\text{iso}}}
\newcommand{\trFrob}[1]{t_{#1}}
\newcommand{\TrFrob}[1]{T_{#1}}
%% amsmath replacements for \atop
\newcommand{\latop}[2]{\genfrac{}{}{0pt}{0}{#1}{#2}}
\newcommand{\partop}[2]{\genfrac{(}{)}{0pt}{0}{#1}{#2}}
%% Labeled items
\makeatletter
\newcommand{\labitem}[2]{
\def\@itemlabel{\textbf{#1}}
\item
\def\@currentlabel{#1}\label{#2}}
\makeatother
%% Shorthand for bars
\renewcommand{\bf}{\bar{f}}
\newcommand{\bg}{\bar{g}}
\newcommand{\bm}{\bar{m}}
\newcommand{\bG}{\bar{G}}
\newcommand{\bH}{\bar{H}}
%% Spacing control
\newcommand{\tight}[3]{\hspace{-#1pt}{#2}\hspace{-#3pt}}
\newcommand{\GxG}{\text{$G \tight{1}{\times}{1} G$}}
\newcommand{\bGxG}{\text{$\bar{G} \tight{1}{\times}{1} \bar{G}$}}
\newcommand{\bfxf}{\text{$\bar{f} \tight{1}{\times}{1} \bar{f}$}}
\newcommand{\GxxG}{\text{$G \tight{1}{\times}{1} G$}}
\newcommand{\LxL}{\text{$\gqcs{L} \tight{0}{\boxtimes}{0} \gqcs{L}$}}
\newcommand{\ExE}{\text{$\qcs{E}\tight{0}{\boxtimes}{0}\qcs{E}$}}
\newcommand{\bExE}{\text{$\gqcs{E}\tight{0}{\boxtimes}{0}\gqcs{E}$}}
\newcommand{\AxA}{\text{$A \tight{1}{\times}{1} A$}}
\newcommand{\BxB}{\text{$B \tight{1}{\times}{1} B$}}
\newcommand{\GzxGz}{\text{$G^0 \tight{1}{\times}{1} G^0$}}
%% Margin notes
\newcommand\Clifton[1]{\marginpar{\smaller\smaller CC: #1}}
\newcommand\David[1]{\marginpar{\smaller\smaller DR: #1}}
%% Hyphenation override
\hyphenation{quasi-character}

%%%%%%%%%%%% BEGIN DOCUMENT %%%%%%%%%%%%
\begin{document}

\begin{abstract}
  We extend Deligne's function--sheaf dictionary from
  connected commutative algebraic groups over finite fields to smooth
  commutative group schemes over finite fields.
  To do so, we introduce the monoidal category of quasicharacter sheaves on
  smooth commutative group schemes $G$ over $\mathbb{F}_q$
  and show that the Trace of Frobenius is a functorial surjection
  from the group of isomorphism classes of quasicharacter sheaves on $G$
  onto the character group of $G(\mathbb{F}_q)$.
  If $G$ is connected, the kernel of this surjection is trivial.
  When $G$ is etale, however, the kernel is generally non-trivial,
  leading to interesting notion of invisible quasicharacter sheaves.
  We find a classification of invisible quasicharacter sheaves
  on etale commutative group schemes over $\mathbb{F}_q$
  in terms of the cohomology of $G(\mathbb{F}_q)$.
   \end{abstract}

\maketitle

\section*{Introduction}


\subsection*{Overview}
Deligne's function--sheaf dictionary translates characters of connected, commutative algebraic groups $G$ over finite fields $\Fq$ into certain local systems on G, from which the character of $G(\Fq)$ can be recovered using the action of Frobenius on $G$.
The local systems which appear in this dictionary are easily characterised and often referred to as \emph{character sheaves} on $G$.
The result is a perfect dictionary between characters of $G(\Fq)$ and character sheaves on G, with which the formidable artillery of algebraic geometry can be brought to bear on the study of characters of $G(\Fq)$.
Although this dictionary extends to the case when $G$ is not connected, and even to all smooth commutative group schemes over $\Fq$, the result is not perfect.
In this generality, the correct class of sheaves on G are called \emph{quasicharacter sheaves}; when $G$ is connected and of finite type, quasicharacter sheaves on $G$ specialize to character sheaves on $G$.
While every character of $G(\Fq)$ may be converted, canonically, into a quasicharacter sheaf on G from which the character of $G(\Fq)$ may be recovered using the trace of Frobenius on G, generally there exist non-trivial quasicharacter sheaves on G for which the trace of Frobenius is trivial; these are called \emph{invisible quasicharacter sheaves} on $G$.
Invisible quasicharacter sheaves on G are precisely the quasicharacter sheaves on $G$ that are lost in translation under the function-sheaf dictionary.
Assuming only that $G$ is a smooth commutative group scheme over $\Fq$ with finitely generated component group, this paper gives a simple cohomological description of invisible quasicharacter sheaves on $G$. 
%The most interesting case is that of etale commutative group schemes over $\Fq$.

\subsection*{Main result}

In order to describe the main result of this paper with more precision, let us briefly review how the function--sheaf dictionary for connected commutative algebraic groups over finite fields \cite{deligne:SGA4.5}*{sommes trig.} translates characters into local systems.
Let $\Fq$ be a finite field and let $G$ be a connected commutative algebraic group over $\Fq$; fix an algebraic closure $\bFq$ of $\Fq$
and let $\bG$ be the base change of $G$ to $\bFq$.
Fix a prime $\ell$ invertible in $\Fq$ and let $\chi : G(\Fq) \to \EEx$ be any character.
Using little more than the Lang map for $G$, $\chi$ determines an $\ell$-adic local system $\qcs{L}$ on $G$ from which the character $\chi$ can be recovered using the Frobenius automorphism of $\bG$.
While this recovery process applies to all $\ell$-adic local systems $\qcs{L}$ on $G$, those that produce characters of $G(\Fq)$ are distinguished by a simple property: there exists an isomorphism
\begin{equation}\label{introbox}
m^* \qcs{L} \iso \qcs{L} \boxtimes \qcs{L}
\end{equation}
 of $\ell$-adic local systems on $G$, where $m : G\times G \to G$ is the multiplication map for $G$. The function--sheaf dictionary for $G$ identifies a monoidal category of $\ell$-adic local systems on $G$ such that the group of isomorphism classes in this category is canonically identified with the $\ell$-adic character group of $G(\Fq)$, thus providing the geometrization and categorification of characters of $G(\Fq)$.

Although hidden in the discussion above, the connectness of $G$ played a crucial role in the geometrization and categorification of characters of $G(\Fq)$.  So it is natural to ask how the function--sheaf dictionary generalizes to not-necessarily connected commutative algebraic groups over finite fields. In this paper we answer that question by exhibiting a function--sheaf dictionary, in the sense above, for any smooth commutative group scheme $G$ over a finite field $\Fq$.
To do this we must pass from $\ell$-adic local systems on $G$ to Weil local systems on $G$
and also keep track of the isomorphisms \eqref{introbox} (mere existence of such an isomorphism is not enough), leading to the notion of {\it quasicharacter sheaves on $G$}.
%So, what is a quasicharacter sheaf?

A quasicharacter sheaf on $G$ is a triple $\qcs{L}\ceq
(\gqcs{L},\mu,\phi)$ where $\gqcs{L}$ is an $\ell$-adic local system on $\bG$ of rank~one and $\mu: m^*
\gqcs{L} \to \LxL$ and $\phi : \Frob{G}^*\gqcs{L} \to \gqcs{L}$ are isomorphisms satisfying natural compatibilty conditions; here $\Frob{G}$ is the Frobenius automorphism of ${\bar G}$.

The main result of this paper is the following statement (see Theorem~\ref{thm:snake}):
the category $\QC(G)$ of quasicharacter sheaves on $G$
is a rigid monoidal category and Trace of Frobenius provides a functorial surjection
\begin{equation}
\QCiso{G} \to \Hom(G(\Fq), \EEx)
\end{equation}
with a canonical section,
and the kernel of the Trace of Frobenius is classified by $\Hh^1(\Weil{},\Hh^2(G(\bFq),\EEx))$.
In particular, in contrast to the case of connected commutative algebraic groups, the Trace of Frobenius may not injective.
We refer to quasicharacter sheaves in the kernel of Trace of Frobenius as \emph{invisible} quasicharacter sheaves.
The canonical section of the Trace of Frobenius allows one to convert any quasicharacter of $G(\Fq)$
into a quasicharacter sheaf on $G$.
In this way, quasicharacter sheaves geometrize characters of $G(\Fq)$.


\subsection*{Illustrative examples}

Three examples will illustrate some of the issues that arise when extending the function--sheaf dictionary to etale, commutative group schemes over $\Fq$.
First, consider the etale group scheme $\ZZ$ over $\Fq$.
While the $\ell$-adic character $\ZZ \to \EEx$ defined by
$1 \mapsto -1$, has finite order, the character defined by
$1 \mapsto 1+\ell$ does not, although its image is bounded;
on the other hand, the character $\ZZ \to \EEx$ defined by
$1 \mapsto \ell$ has unbounded image.
As we show in the paper, $\ell$-adic local systems on $G$ can only be used to geometrize characters of finite order; quasicharacter sheaves, however, provide the geometrization of all characters of $G(\Fq)$.
Second, consider the etale group scheme $\ZZ \times \ZZ$.
\todo{Build an example of an invisible quasicharacter sheaf here.}
Finally, consider the etale group scheme $\ZZ \times \ZZ$ over $\Fq$ with Galois action determined by $\sigma(n,m) = (m,n)$ when the restriction of $\sigma$ to $\mathbb{F}_{q^2}$ is non-trivial.
Although there are infinitly-many geometric points on this group, it has exactly one $\Fq$-rational point.
In this case, the group carries only one isomorphism class of quasicharacter sheaves -- in particular, it has no invisible quasicharacter sheaves -- so the dictionary is perfect, albeit trivial, in this case.

\subsection*{Motivation and Application}

Our interest in the function--sheaf dictionary for smooth commutative group schemes over $\Fq$, especially etale commutative group schemes, comes from an application to $p$-adic representation theory.
In \cite{cunningham-roe:13a} we consider quasicharacter sheaves on the
Greenberg transform of the N\'eron model of an algebraic torus over an arbitrary local field.
The ability to work with non-connected group schemes,
of which the component group of the special fibre of the Neron model of a torus is an important example,
plays a crucial role in this application,
since these group schemes are not connected in general.
This process of creating a category from the group of quasicharacters of such a torus
informs our choice of the term \emph{quasicharacter sheaf} here.

\subsection*{Comparison with other character sheaves}

We also want to situate our terminology in a historical context.
Some people refer to local systems satisfying \eqref{introbox} on a connected, commutative algebraic group as character sheaves;
see for example, \cite{kamgarpour:09a}.
As explained above, quasicharacter sheaves evolved from this notion,
with an eye toward quasicharacters of $p$-adic groups.
However, the main use of the term character sheaf is of course due to Lusztig.
It is applied to certain perverse sheaves on connected reductive algebraic groups over algebraically closed fields in
\cite{lusztig:85a}*{Def.~2.10} and to certain perverse sheaves on reductive groups
over algebraically closed fields with finite cyclic component groups in the series of papers
beginning with \cite{lusztig:disconnected1}.
When commutative, such groups are extensions of $\ZZ/n\ZZ$ by a torus,
in which case it is not difficult to relate Frobenius-stable character sheaves to our quasicharacter sheaves. 
In particular, all quasicharacter sheaves on such groups are visible.

\subsection*{Structure of the paper}

%We close the introduction with a more detailed description of the structure of this paper.
After setting some notation in Section~\ref{sec:QCschemes}, in Section~\ref{sec:category} we define quasicharacter sheaves and give an interpretation of
them as Weil sheaves equipped with an isomorphism of the form  \eqref{introbox}.
Section~\ref{sec:Frob} introduces Trace of Frobenius,
one direction of the function--sheaf dictionary.
We define pullbacks and products of quasicharacter sheaves in Section~\ref{sec:pullback}, and use
them to prove a functoriality result on Trace of Frobenius.
In Section~\ref{sec:basechange} we study how quasicharacter sheaves behave under base change
and Weil restriction.
Sections~\ref{sec:bounded} and \ref{sec:finite} describe bounded quasicharacter sheaves and finite quasicharacter sheaves, respectively,
forming two important subcategories of $\QC(G)$.
%These subcategories play a key role in the proof of Proposition~\ref{prop:restriction}.
In Section \ref{sec:connected} we show that all quasicharacter sheaves on $G$ are finite quasicharacter sheaves when $G$ is connected, and that Trace of Frobenius is an isomorphism in this case.
We shift to \'etale group schemes $G$ in Section~\ref{sec:etale} and give a more concrete interpretation
of quasicharacter sheaves in this case.  Using a notion of global sections for quasicharacter sheaves, in this section
we also give isomorphisms between $\QCiso{G}$, $\Hh^1(\Weil{}, \Hom(G(\bFq), \EEx))$ and $\Hom(G(\Fq), \EEx)$, where $\Weil{}$ is the Weil group for $\Fq$.
In Section~\ref{sec:restriction} we show that for non-connected $G$, restriction to the identity component
induces an essentially surjective functor of quasicharacter sheaves.
Section~\ref{sec:snake} presents the main theorem of this paper: Trace of Frobenius
is a functorial isomorphism for $\QC$-schemes.

%\begin{acknowledgements}
\subsection*{Acknowledgements}
We thank Masoud Kamgarpour, Pramod Achar, and Hadi Salmasian
for allowing us to hijack much of our Research in Teams meeting at BIRS into a discussion of
quasicharacter sheaves; their knowledge and help have been invaluable.
We also thank Takashi Suzuki, who made many helpful observations and suggestions.
%
Finally, we gratefully acknowledge the financial support of the Pacific Institute for the Mathematical Sciences
and the National Science and Engineering Research Council,
as well the hospitality of the Banff International Research Station during our weeklong stay in May 2012.

%\end{acknowledgements}

\tableofcontents

\section{The category of quasicharacter sheaves}


\subsection{Notations}\label{sec:QCschemes}

Throughout this paper, $G$ is a smooth, commutative group scheme
over a finite field $\Fq$, and $m : \GxG\to G$ is its multiplication morphism.  
%For the sake of brevity, we define a \emph{$\QC$-scheme} to be such a $G$ with the additional property that the geometric component group of $G$ is finitely generated.  This last condition is only needed in the proof of Lemma \ref{lemma:section}.  A morphism of $\QC$-schemes is a morphism of group schemes over $\Fq$.

We fix an algebraic closure $\bFq$ of $\Fq$ and write $\bG$ for the
smooth commutative group scheme $G \times_{\Spec{\Fq}} \Spec{\bFq}$ over $\bFq$
obtained by base change from $k$. The multiplication morphism for $\bG$ will be denoted by $\bm$.

Let $\Frob{}$ denote the geometric Frobenius element in $\Gal(\bFq/\Fq)$ as
well as the corresponding automorphism of $\Spec{\bFq}$. The Weil group
$\Weil{}\subset \Gal(\bFq/\Fq)$ is the subgroup generated by $\Frob{}$.
Let $\Frob{G} \ceq \id_{G} \times \Frob{}$ be the Frobenius automorphism of $\bG$.
%When no other fields play a role we will write $\Frob{}$ and $\Weil{}$ instead.

We fix a prime $\ell$, invertible in $\Fq$.
We will work with constructible $\ell$-adic sheaves \citelist{\cite{deligne:80a}*{\S 1.1} \cite{SGA5}*{Expos\'es V, VI}}
on schemes locally of finite type over $\Fq$, employing the standard formalism.
We also make extensive use of the external tensor product of $\ell$-adic sheaves,
defined as follows: if $\mathcal{F}$ and $\mathcal{G}$ are constructible $\ell$-adic
sheaves on schemes $X$ and $Y$ and $p_X : X\times Y\to X$ and $p_Y : X\times Y \to Y$
are the projections, then $\mathcal{F}\boxtimes \mathcal{G} \ceq p_X^* \mathcal{F} \otimes p_Y^*\mathcal{G}$.

For any group $A$, we will denote by $A^*$ the dual group $\Hom_\text{grp}(A, \EEx)$.

\subsection{Main definition}\label{sec:category}

\begin{definition}\label{def:QC}
A \emph{quasicharacter sheaf on $G$} is a triple
$\qcs{L}\ceq (\gqcs{L},\mu,\phi)$ where:
\begin{enumerate}
\labitem{(QC.1)}{QC.1} $\gqcs{L}$ is a rank~one $\ell$-adic local system on $\bG$, by which we mean
a constructible $\ell$-adic sheaf on $\bG$, locally constant on each connected
component of $\bG$, whose stalks are one-dimensional $\EE$-vector spaces;
\labitem{(QC.2)}{QC.2} $\mu: \bm^* \gqcs{L} \to \LxL$ is an isomorphism of
sheaves on $\bGxG$ such that the following diagram commutes,
  where $m_3 \ceq m\circ (m\tight{1}{\times}{2}\id) = m\circ (\id\tight{2}{\times}{1} m)$;
  \[
  \begin{tikzcd}[row sep=30]
  \bm_3^*\gqcs{L} \arrow{rr}{(\bm \tight{1}{\times}{2} \id)^*\mu} \arrow[swap]{d}{(\id \tight{2}{\times}{1} \bm)^*\mu}
    &&  \bm^*\gqcs{L} \boxtimes \gqcs{L} \dar{\mu \tight{0}{\boxtimes}{1} \id} \\
    \gqcs{L} \boxtimes \bm^* \gqcs{L} \arrow{rr}{\id \boxtimes \mu}
    &&  \gqcs{L} \tight{0}{\boxtimes}{0} \LxL
  \end{tikzcd}
  \]
\labitem{(QC.3)}{QC.3} $\phi : \Frob{G}^* \gqcs{L} \to \gqcs{L}$ is an
  isomorphism of constructible $\ell$-adic sheaves on $\bG$ compatible with
  $\mu$ in the sense that the following diagram commutes.
  \[
  \begin{tikzcd}[row sep=20]
  \Frob{\GxxG}^* \bm^* \gqcs{L} \arrow{rr}{\Frob{\GxxG}^*\mu}
    && \Frob{\GxxG}^*(\LxL)\\
    \arrow[equal]{u} \bm^*  \Frob{G}^* \gqcs{L} \arrow[swap]{d}{\bm^* \phi}
    && \Frob{G}^*\gqcs{L}\boxtimes \Frob{G}^*\gqcs{L} \dar{\phi\boxtimes \phi} \arrow[equal]{u} \\
    \bm^*\gqcs{L} \arrow{rr}{\mu}
    && \LxL
  \end{tikzcd}
  \]
\end{enumerate}
\end{definition}

Morphisms in the category of quasicharacter sheaves on $G$, denoted by $\QC(G)$, are defined in the natural way:
\begin{enumerate}
\labitem{(QC.4)}{QC.4} if $\qcs{L} = (\gqcs{L},\mu,\phi)$ and
  $\qcs{L'} = (\gqcs{L'},\mu',\phi')$ are quasicharacter sheaves on $G$ then
  a morphism $\qcs{L} \to \qcs{L}'$ is a map $\alpha : \gqcs{L} \to \gqcs{L'}$
  of constructible $\ell$-adic sheaves on $\bG$ such that the following diagrams both commute.
  \[
  \begin{tikzcd}[column sep=40]
  \Frob{G}^* \gqcs{L} \rar{\Frob{G}^* \alpha} \arrow[swap]{d}{\phi} & \Frob{G}^* \gqcs{L'} \dar{\phi'}
  & & \arrow[swap]{d}{\mu} m^* \gqcs{L} \rar{m^* \alpha} & m^* \gqcs{L'} \dar{\mu'} \\
  \gqcs{L} \rar{\alpha} & \gqcs{L'}
  & {} & \LxL \rar{\tight{1}{\alpha\boxtimes \alpha}{1}} & \gqcs{L'} \tight{0}{\boxtimes}{0} \gqcs{L'}
  \end{tikzcd}
  \]
\end{enumerate}

The rule $(\gqcs{L},\mu,\phi) \mapsto (\gqcs{L},\phi)$ defines a forgetful functor from quasicharacter
sheaves on $G$ to ($\ell$-adic) Weil sheaves on $G$ \cite{deligne:80a}*{Def.~1.1.10 (i)}.

A Weil sheaf $(\gqcs{L},\phi)$ on $G$
may be interpreted as a constructible $\ell$-adic sheaf $\gqcs{L}$ on $\bG$ together with
an action of the Weil group $\Weil{}$ on $\gqcs{L}$ compatible with the action of
$\Gal(\bFq/\Fq)$ on $\bG$. We now briefly review this point of view for later use in
Section~\ref{sec:basechange}; see also \cite{deligne:80a}*{Def.~1.1.10}.  If $w \ceq \Frob{}^n$ we write $w_G$ for $\Frob{G}^n$.
For each such $w$ we define an isomorphism $\varphi(w) : w_G^* \gqcs{L}\to \gqcs{L}$ by
$
\varphi(w) \ceq  \phi \circ \Frob{G}^*(\phi) \circ \cdots \circ (\Frob{G}^{n-1})^*(\phi)
$;
these isomorphisms satisfy $\varphi(uv) = \varphi(v) \circ v_G^* \varphi(u)$ for $u,v\in \Weil{}$, and
we have $\phi = \varphi(\Frob{})$.
If we further define $\varphi_1(w) \ceq (w_G)_*(\varphi(w)^{-1})$ then
$\varphi_1(w) : (w_G)_* \gqcs{L}\to \gqcs{L}$ is an isomorphism and
$\varphi_1(uv) = \varphi_1(u) \circ (u_G)_* \varphi_1(v)$ for $u,v\in \Weil{}$.
Thus the pair $(\gqcs{L},\varphi_1)$ almost satisfies the criteria for an action of $\Weil{}$
on $\gqcs{L}$ compatible with the action of $\Gal(\bFq/\Fq)$ in the sense of
\cite{deligne:SGA4.5}*{Expos\'e XIII, 1.1}, failing only because $\Weil{}$ is not profinite.
As we will see in Corollary~\ref{cor:bounded-and-finite}, our use of $\Weil{}$ rather than $\Gal(\bFq/\Fq)$
allows quasicharacter sheaves to correspond to unbounded characters.

The category $\QC(G)$ of quasicharacter sheaves on $G$ is a rigid monoidal category
\cite{etingof:09a}*{\S1.10} under the tensor product
$\qcs{L} \otimes \qcs{L'} \ceq (\gqcs{L}\otimes\gqcs{L'}, \mu\otimes\mu', \phi\otimes \phi')$,
with duals given by applying the sheaf hom functor
$\sheafHom(\ - \ ,\EE)$.
The category of quasicharacter sheaves on $G$ is not abelian, so $\QC(G)$
is not a tensor category in the sense of \cite{deligne:02a}*{0.1}.
This rigid monoidal category structure for $\QC(G)$ gives the set $\QCiso{G}$
of isomorphism classes in $\QC(G)$ the structure of a group.

\subsection{Trace of Frobenius}\label{sec:Frob}

We now begin developing the tools to relate $\QCiso{G}$ to $G(\Fq)^*$.
%
Let $(\gqcs{L},\phi)$ be a Weil sheaf on $G$. Every $g\in G(\Fq)$
determines a point $\bg$ on $\bG$ fixed by $\Frob{G}$ and
therefore an automorphism $\phi_{\bg}$ of $\gqcs{L}_{\bg}$. Set $\trFrob{(\gqcs{L},\phi)}(g) \ceq \trace(\phi_{\bg}, \gqcs{L}_{\bg})$,
and note that if $(\gqcs{L},\phi) \iso (\gqcs{L'},\phi')$ then $\trFrob{(\gqcs{L},\phi)} = \trFrob{(\gqcs{L'},\phi')}$.
 If $\qcs{L} = (\gqcs{L},\mu,\phi)$ is a quasicharacter sheaf
then we will abbreviate $\trFrob{(\gqcs{L},\phi)}$ to $\trFrob{\qcs{L}}$.

\begin{definition}\label{def:trFrob}
The function
\begin{align*}
\TrFrob{G}: \QCiso{G} &\to G(\Fq)^* \\
\qcs{L} &\mapsto t_{\qcs{L}}
\end{align*}
is called \emph{Trace of Frobenius}.
\end{definition}

The isomorphism $\mu : \bm^* \gqcs{L} \to \LxL$ together with
\ref{QC.3} guarantee
that the function $\trFrob{\qcs{L}} : G(\Fq)\to \EEx$ is a group homomorphism.  If we write
$\trFrob{\qcs{L}} \cdot \trFrob{\qcs{L'}}$ for pointwise multiplication, then
 $\trFrob{\qcs{L}\otimes \qcs{L'}} = \trFrob{\qcs{L}} \cdot \trFrob{\qcs{L'}}$.
 Therefore $\TrFrob{G}$ is a homomorphism, which is functorial in the following sense:

\begin{proposition}\label{prop:functorialG}
Trace of Frobenius $\TrFrob{G} : \QCiso{G} \to G(\Fq)^*$ is a natural transformation
between the two additive functors
\begin{align*}
F_1 : G &\mapsto \QCiso{G} \\
F_2 : G &\mapsto G(\Fq)^*
\end{align*}
from the category of smooth commutative group schemes over $\Fq$ to the category of abelian groups.
\end{proposition}

The proof of Proposition~\ref{prop:functorialG} will be given at the end of Section~\ref{sec:pullback}.


\subsection{Invisible quasicharacter sheaves}\label{sec:invisible}

We will see that the Trace of Frobenius may not provide complete
information about isomorphism classes of quasicharacter sheaves on $G$ when $G$ is not connected.
Our main tool for understanding this phenomenon is the function
\begin{equation}\label{2cocycle}
S_{G}: \QCiso{G} \to H^0(\Weil{},H^2(G(\bFq),\EEx))
\end{equation}
defined here.
Let $\qcs{L} = (\gqcs{L},\mu,\phi)$ be a quasicharacter sheaf on $G$.
For each geometric point $x\in G(\bFq)$, choose a basis $\{ v_x \}$ for $\gqcs{L}_x$.
This choice determines a function $g : G(\bFq)\times G(\bFq) \to \EEx$ defined by
\[
\mu_{x,y}(v_{x+y}) = g(x,y) v_x \otimes v_y.
\]
Chasing through the diagram in Condition~\ref{QC.2} we see that
\[
g(x+y,z) g(x,y) = g(x,y+z) g(y,z),
\]
for all $x,y,z\in G(\bFq)$, so $g \in Z^2(G(\bFq),\EEx)$.
It is easy to see that the class of $g$ is independent of the choice made above;
let $\sigma_\mathcal{L}) \in H^2(G(\bFq),\EEx)$ the class of $g\in Z^2(G(\bFq),\EEx)$.
It is also easy to see that $S_G(\mathcal{L}) = S_G(\mathcal{L'})$ when $\qcs{L} \iso \qcs{L'}$. This defines
\[
S_G: \QCiso{G} \to H^2(G(\bFq),\EEx).
\]
We claim that $S_G(\qcs{L})$ is also fixed by the action of $\Weil{}$ on $H^2(G(\bFq),\EEx)$
inherited from the action of $\Weil{}$ on $G(\bFq)$.
To see this, we return to the choice $\{ v_x \in \gqcs{L}_x^\times \tq x\in G(\bFq) \}$
and define $f : G(\bFq) \to \EEx$ by
\[
\phi_x(v_{\Frob{G}(x)}) = f(x) v_x.
\]
Chasing through the diagram in Condition~\ref{QC.3} we see that
\begin{equation}\label{nohom}
g(\Frob{G}(x),\Frob{G}(y)) f(x) f(y)  = g(x,y) f(x+y)
\end{equation}
for all $x,y \in G(\bFq)$; thus, the class of $\,^{\Frob{}}g$ coincides with the class of $g$.
This concludes the definition of
$S_{G}: \QCiso{G} \to H^0(\Weil{},H^2(G(\bFq),\EEx))$.
It is easy to see that the function $S_{G}: \QCiso{G} \to H^0(\Weil{},H^2(G(\bFq),\EEx))$ is a group homomorphism:
\[
g_{\qcs{L}\otimes \qcs{L'}} = g_{\qcs{L}} \, g_\qcs{L'}
\]
for every $\qcs{L}, \qcs{L'} \in \QC(G)$.\todo{I still need to write down the details, for ourselves.}

\begin{definition}\label{def:invisible}
Let $\qcs{L}$ be a quasicharacter sheaf on $G$. We say $\qcs{L}$ is \emph{visible} $g_\qcs{L} \in H^0(\Weil{},H^2(G(\bFq),\EEx))$ is trivial; otherwise, we say $\qcs{L}$ is \emph{invisible}.
\end{definition}

If $G$ is a connected and of finite type, then $\Hh^2(G(\bFq),\EEx)$ is trivial\todo{Reference?}
so all quasicharacter sheaves on connected commutative algebraic groups $G$ are visible;
 by contrast, if $G$ is etale then $\Hh^1(\Weil{},\Hh^2(G(\bFq),\EEx))$ may be non-trivial so $G$ may admit invisible quasicharacters sheaves.
We will return to that case in Section~\ref{sec:etale}.

While neither $T_{G}$ nor $S_{G}$ are injective, their direct sum is.

\begin{proposition}
Let
\[
R_{G} : \QCiso{G} \to  \Hh^0(\Weil{},\Hh^2(G(\bFq),\EEx) \oplus \Hh^1(\Weil{}, \Hh^1(G(\bFq),\EEx))
\]
be the map formed from  . . .
Then $R_{G}$ is injective.
\end{proposition}

\subsection{Pull-back and products}\label{sec:pullback}

As a step toward the proof of Proposition~\ref{prop:functorialG} we show that every
morphism  $H \to G$ of smooth commutative groups schemes over $\Fq$ induces a monoidal functor from quasicharacter
sheaves on $G$ to quasicharacters on $H$. This simple but important result will also
play a role in the proof of many other results in this paper.

\begin{proposition}\label{prop:pullback}
  If $f : H\to G$ is a morphism of smooth commutative group schemes over $\Fq$ then
  \begin{align*}
  f^* : \QC(G) &\to \QC(H) \\
  (\gqcs{L},\mu,\phi) &\mapsto (\bf^*\gqcs{L},(\bfxf)^*\mu,\bf^*F)
  \end{align*}
  defines a monoidal functor dual to $f \colon H(\Fq) \to G(\Fq)$:
  \[
  \begin{tikzcd}[row sep=20, column sep=30]
   \QCiso{G} \rar{f^*} \arrow[swap]{d}{\TrFrob{G}} & \QCiso{H} \dar{\TrFrob{H}} \\
   G(\Fq)^* \rar & H(\Fq)^*
  \end{tikzcd}
  \]
  is a commutative diagram of groups.  Moreover, $(f\circ g)^* = g^* \circ f^*$.
\end{proposition}

\begin{proof}
  Let $\qcs{L}$ be a quasicharacter sheaf on $G$. We start by showing that
  $\bf^*\gqcs{L}$ is a local system of rank~one. Let $c_j :\bH^j \hookrightarrow \bH$ be any
  connected component and let $d_i : \bG^i \hookrightarrow \bG$ be the image of $j$ under the group
  homomorphism $\pi_0(\bf) : \pi_0(\bH) \to \pi_0(\bG)$.  Write $\bf^j : \bH^j \to \bG^i$
  for the restriction of $f$ to $\bH^j$.  Since $\gqcs{L}$ is locally constant on the connected
  components of $\bG$, there is a finite \'etale covering
  $a_i : X_i \to \bG^i$ such that $a_i^* (\gqcs{L}\vert_{\bG^i})$
  is constant.  Define $Y_j$ in the following diagram through pullback:
  \[
  \begin{tikzcd}[row sep=20, column sep=30]
   Y_j \dar[dashed]{g_j} \rar[dashed]{b_j} & \bH^j \dar{\bf^j} \rar[hook]{c_j} & \bH \dar{\bf} \\
   X_i \rar{a_i} & \bG^i \rar[hook]{d_i} & \bG
  \end{tikzcd}
  \]
  Then $b_j$ is a finite \'etale covering
  of $\bH^j$ and
  \[
  b_j^* \left( (\bf^* \gqcs{L})\vert_{\bH^j}\right)
  = (\bf\circ c_j \circ b_j)^*\gqcs{L}
  = (d_i\circ a_i\circ g_j)^*\gqcs{L} = g_j^* a_i^* (\gqcs{L}\vert_{\bG^i}).
  \]
  Since $a_i^* (\gqcs{L}\vert_{\bG^i})$ is a constant sheaf, its pullback
  $b_j^* \left( (\bf^*\gqcs{L})\vert_{\bH^j}\right)$ along $g_j$ is also constant.
  Thus $\bf^*\gqcs{L}$ is locally constant on
  $\bH^j$. To complete the proof that $\bf^*\gqcs{L}$ satisfies
  \ref{qc.4}
  simply observe that $\bf^*\gqcs{L}$ has rank~one since
  $(\bf^*\gqcs{L})_{\bg} = \gqcs{L}_{f(\bg)}$, for every geometric point ${\bg}$ on $\bG$.


  To see that $(\bfxf)^* \mu$ satisfies \ref{QC.2},
  apply the functor $(\bfxf)^*$
  to \ref{QC.2} for $\qcs{L}$ and use the canonical isomorphism
  $(\bfxf)^*(\LxL) \iso \bf^*\gqcs{L} \tight{-3}{\boxtimes}{-3} \bf^*\gqcs{L}$.
  To show that $f^*\qcs{L}$ satisfies
  \ref{QC.3}, apply the same functor to \ref{QC.3} for $\qcs{L}$.
  Since $f$ is a morphism of group schemes defined over $\Fq$
  it provides isomorphisms $(\bfxf)^*\Frob{\GxxG}^* \iso \Frob{\GxxG}^* (\bfxf)^*$
  and $(\bfxf)^* \bm^*\iso \bm^* \bf^*$ between functors of constructible sheaves.

  Applying $\bf^*$ and $\bf^* \tight{1}{\times}{1}\bf^*$ to \ref{QC.4} defines the action
  of $f^*$ on morphisms of quasicharacter sheaves; arguing as above shows that $f^*$ is
  a functor from $\QC(G)$ to $\QC(H)$.  Since tensor products commute with pullback in schemes,
  $f^* : \QC(G) \to \QC(H)$ is a monoidal functor.
  The diagram relating $f^* : \QC(G) \to \QC(H)$, $f^* : G(k)^* \to H(k)^*$ and Trace of Frobenius
  commutes by \cite{laumon:87a}*{1.1.1.2}, where the ambient
  hypothesis that $X$ is of finite type over $\Fq$ can be replaced by
  the hypothesis that $X$ is locally of finite type over $\Fq$.

  Finally, the fact that $(f\circ g)^* = g^* \circ f^*$ follows from the analogous
  statements about the pullback functor on $\ell$-adic constructible sheaves.
\end{proof}

We end this section with a simple result on products of quasicharacter sheaves,
which is needed to show that the isomorphism above is functorial.

\begin{proposition}\label{prop:product}
If $G_1$ and $G_2$ are smooth commutative group schemes over $\Fq$ then the rule
$\boxtimes : (\qcs{L}_1,\qcs{L}_2)\to \qcs{L}_1\boxtimes\qcs{L}_2$ defines an equivalence of categories
\[
\QC(G_1)\times \QC(G_2) \to \QC(G_1\times G_2)
\]
such that
\[
\begin{tikzcd}[column sep=60]
\arrow{d}{\TrFrob{G_1} \times \TrFrob{G_2}} \QCiso{G_1}\times \QCiso{G_2} \arrow{r}{\boxtimes}
& \arrow{d}{\TrFrob{G_1\times G_2}} \QCiso{G_1\times G_2}\\
(G_1)(\Fq)^*\times (G_2)(\Fq)^* \arrow{r}{(\chi_1,\chi_2)\mapsto \chi_1\otimes \chi_2}  & (G_1\times G_2)(\Fq)^*
\end{tikzcd}
\]
commutes.
\end{proposition}

\begin{proof}
The only non-trivial part of the proof is to show that $\boxtimes : \QC(G_1)\times \QC(G_2) \to \QC(G_1\times G_2)$
is essentially surjective, so we will only address that point here.

Set $G \ceq G_1\times G_2$
and write $e_1$ and $e_2$ for the identity elements of $G_1$ and $G_2$.
Define $f : G\to G\times G$ by $f(g_1,g_2) \ceq (g_1,e_2,e_1,g_2)$.
Observe that $m\circ f = \id_G$.
Let $p_1$, $p_2$ be the projection morphisms pictured below:
\[
\begin{tikzcd}
G & \arrow[swap]{l}{p_1} G\times G \arrow{r}{p_2} & G.
\end{tikzcd}
\]
Let $r_1$ and $r_2$ be the projection morphisms pictured below,
with sections $q_1$ and $q_2$, also morphisms of group schemes:
\[
\begin{tikzcd}
G_1  \arrow[swap, bend right]{r}{q_1} &
\arrow[swap, bend right]{l}{r_1} G_1\times G_2 \arrow[bend left]{r}{r_2} &
\arrow[bend left]{l}{q_2} G_2.
\end{tikzcd}
\]
Observe that $p_1\circ f = q_1 \circ r_1$ and $p_2 \circ f = q_2\circ r_2$.

Now, let $\qcs{L} \ceq (\gqcs{L},\mu,\phi)$ be a quasicharacter sheaf on $G$.
Set $\qcs{L}_1 \ceq q_1^* \qcs{L}$ and $\qcs{L}_2 \ceq q_2^* \qcs{L}$;
by Proposition~\ref{prop:pullback}, $\qcs{L}_1$ a quasicharacter sheaf on $G_1$
and $\qcs{L}_2$ is a quasicharacter sheaf on $G_2$.
We will obtain an isomorphism $\qcs{L} \iso  \qcs{L}_1\boxtimes \qcs{L}_2$.
Applying the functor $f^*$ to the isomorphism $\mu$ yields
\begin{equation}\label{eq:fm}
f^*\mu : f^* m^* \gqcs{L} \to f^*(\gqcs{L}\boxtimes \gqcs{L}) .
\end{equation}
We have already observed that $m\circ f = \id_G$, so $f^* m^* \gqcs{L} = \gqcs{L}$.
On the other hand, $f^*(\gqcs{L}\boxtimes \gqcs{L}) = f^*p_1^*\gqcs{L}\otimes f^* p_2^*\gqcs{L}$
by definition.  Since $f^*p_1^*\gqcs{L} = r_1^* q_1^* \gqcs{L}$ and $f^*p_2^*\gqcs{L} = r_2^* q_2^* \gqcs{L}$
we get that $f^*(\gqcs{L}\boxtimes \gqcs{L}) = \gqcs{L}_1\boxtimes \gqcs{L}_2$.
It follows that \eqref{eq:fm} gives an isomorphism $\gqcs{L} \to  \gqcs{L}_1\boxtimes \gqcs{L}_2$.
It is a matter of routine to show that this morphism satisfies
\ref{qc.3} as it applies here,
from which it follows that we have exhibited an isomorphism
$\qcs{L} \to \qcs{L}_1\boxtimes \qcs{L}_2$ of quasicharacters sheaves on $G\times G$.
This completes the proof that $\boxtimes$ is essentially surjective.
\end{proof}

\begin{proof}[Proof of Proposition~\ref{prop:functorialG}]
The first part of Proposition~\ref{prop:pullback} shows that $F_1$ is a functor,
while the second part shows that Trace of Frobenius is a natural transformation
$T: F_1 \to F_2$. When further combined with Proposition~\ref{prop:product},
we see that $F_1$ is an additive functor and $T: F_1 \to F_2$ is a natural
transformation between additive functors,
concluding the proof of Proposition~\ref{prop:functorialG}.
\end{proof}

\subsection{Base change}\label{sec:basechange}

When using quasicharacter sheaves to study characters, it is useful to understand
how quasicharacter sheaves behave under change of fields.
Let $k'$ be a finite extension of $k$. Then $k \hookrightarrow k'$ induces a group homomorphism
$i_{k'/k} : G(k) \hookrightarrow G(k')$ and thus a homomorphism
\begin{align*}
i_{k'/k}^* : G(k')^* &\to G(k)^* \\
\chi &\mapsto \chi\circ i_{k'/k}.
\end{align*}
We can interpret this operation on characters in terms of quasicharacter sheaves:

\begin{proposition} \label{prop:csbe}
Set $G_{k'} \ceq G\times_\Spec{k} \Spec{k'}$ and let
\[
\QC(\Res_{k'/k}(G_{k'})) \xrightarrow{\iota^*} \QC(G)
\]
be the functor obtained by pullback along the canonical closed immersion of $k$-schemes
$\iota : G \hookrightarrow \Res_{k'/k}(G_{k'})$
\cite{bosch-lutkebohmert-reynaud:NeronModels}*{\S 7.6}.
The following diagram commutes:
\[
\begin{tikzcd}
\QCiso{\Res_{k'/k}(G_{k'})} \arrow[two heads]{r}{\iota^*} \dar[swap]{\TrFrob{\Res_{k'/k}(G_{k'})}} & \QCiso{G} \dar{\TrFrob{G}} \\
G(k')^* \arrow[two heads]{r}{i_{k'/k}^*} & G(k)^*.
\end{tikzcd}
\]
\end{proposition}
\begin{proof}
This result follows immediately from Proposition~\ref{prop:pullback} together with the identifications
\[
\Res_{k'/k}(G_{k'})(k) \cong G_{k'}(k') \cong G(k')
\]
from the definitions of Weil restriction and base change.
\end{proof}

In the opposite direction, let $\Nm : G(k') \to G(k)$ be the norm map and consider the group homomorphism:
\begin{align*}
\Nm^* : G(k)^* &\to G(k')^* \\
\chi &\mapsto \chi\circ \Nm.
\end{align*}
We can also interpret this operation on characters in terms of quasicharacter sheaves.

If $\qcs{L} \ceq (\gqcs{L}, \mu, \phi)$ is a quasicharacter sheaf on $G$, we define
a quasicharacter sheaf $\qcs{L}' \ceq (\gqcs{L}, \mu, \phi_{k'})$ on the base change
$G_{k'}$ of $G$ to $k'$ by setting
\[
\phi_{k'} \ceq \phi \circ \Frob{G}^*(\phi) \circ \cdots \circ (\Frob{G}^{n-1})^*(\phi).
\]
The commutativity of the diagram (QC.3) for $\phi_{k'}$
follows from the fact that $\Frob{G_{k'}} = \Frob{G}^n$.
Note that we may also think about the construction of $\phi_{k'}$ from $\phi$
as taking the action $\varphi$ of $\Weil{k}$ on $\gqcs{L}$
defined in Section~\ref{sec:category} and restricting it to the subgroup $\Weil{k'}$.

\begin{proposition}\label{prop:basechange}
 The rule $\nu_{k'/k}: (\gqcs{L}, \mu, \phi) \mapsto (\gqcs{L}, \mu, \phi_{k'})$
 defines a monoidal functor $\QC(G) \to \QC(G_{k'})$.
 Moreover, the following diagram commutes:
\[
\begin{tikzcd}[column sep=60]
\QCiso{G} \rar{\nu_{k'/k}} \dar{\TrFrob{G}} & \QCiso{G_{k'}} \dar{\TrFrob{G_{k'}}} \\
G(k)^*  \rar{\Nm^*} & G(k')^*.
\end{tikzcd}
\]
\end{proposition}
\begin{proof}
Let $\qcs{L} \ceq (\gqcs{L}, \mu, \phi) \in \QC(G)$ and write $F$ for $\Frob{G}$.  For any $x \in G(k')$,
we may compute the value of $t_{G_{k'}}(\nu_{k'/k}\qcs{L})(x)= t_{\nu_{k'/k}\qcs{L}}(x)$ as the trace of $\phi_{k'}$ on $\gqcs{L}_x$,
and the value of $\Nm^*(\TrFrob{G}(\qcs{L}))(x)$ as the trace of $\phi$ on $\gqcs{L}_{\Nm(x)}$.
Applying \ref{QC.3} to the stalk of $\gqcs{L}^{\boxtimes n}$ at the point $(x, \Frob{}(x), \ldots, \Frob{}^{n-1}(x))$ yields a diagram
\[
\begin{tikzcd}
\gqcs{L}_{\Nm(x)} \rar \dar{\phi_{\Nm(x)}} & \gqcs{L}_{F(x)} \otimes \gqcs{L}_{F^2(x)} \otimes \cdots \otimes \gqcs{L}_x
\dar{\phi_x \otimes (F^*\phi)_x \otimes \cdots \otimes ((F^{n-1})^*\phi)_x} \\
\gqcs{L}_{\Nm(x)} \rar & \gqcs{L}_x \otimes \gqcs{L}_{F(x)} \otimes \cdots \otimes \gqcs{L}_{F^{n-1}(x)}.
\end{tikzcd}
\]
Choose a basis vector $v$ for $\gqcs{L}_{\Nm(x)}$ and write the image of $v$ under the
bottom map as $v_0 \otimes v_1 \otimes \cdots \otimes v_{n-1}$
for $v_i \in \gqcs{L}_{\Frob{}^i(x)}$.  By \ref{QC.2}, $v$ maps to
$v_1 \otimes v_2 \otimes \cdots \otimes v_0$ along the top of the diagram.
Let $\alpha_i \in \EEx$ represent $((F^i)^*\phi)_x$ with respect to these bases and let $\alpha$ be
the trace of $\phi_{\Nm(x)}$.  We may now equate the trace $\alpha$ of $\phi$ on $\gqcs{L}_{\Nm(x)}$
with the product $\alpha_0 \cdots \alpha_{n-1}$, which is the trace of $\phi_{k'}$ on $\gqcs{L}_x$.
\end{proof}

Finally, let $G'$ be a smooth commutative group scheme over $k'$;
we explain how to geometrize the canonical isomorphism between characters of $G'(k')$ and of $(\Res_{k'/k}G')(k)$.
We may decompose the base change $(\Res_{k'/k}G')_{k'}$ of $\Res_{k'/k}G'$ to $k'$
into a product of copies of $G'$, indexed by elements of $\Gal(k'/k)$:
\[
(\Res_{k'/k}G')_{k'} \cong \prod_{\Gal(k'/k)} G'.
\]
Since products and coproducts agree for group schemes we have a natural inclusion of $k'$-schemes
\[
G' \hookrightarrow (\Res_{k'/k}G')_{k'}
\]
mapping $G'$ into the summand corresponding to $1 \in \Gal(k'/k)$.  Composing $\nu_{k'/k}$
from Proposition~\ref{prop:basechange} with pullback along this map yields a functor
\[
\rho : \QC(\Res_{k'/k}G') \to \QC(G').
\]

\begin{proposition}
The following diagram of isomorphisms commutes, where the bottom map is the identity:
\[
\begin{tikzcd}
\QCiso{\Res_{k'/k} G'} \dar{\TrFrob{\Res_{k'/k} G'}} \rar{\rho} & \QCiso{G'} \dar{\TrFrob{G'}}\\
G'(k')^* \rar & G'(k')^*
\end{tikzcd}
\]
\end{proposition}
\begin{proof}
By Proposition~\ref{prop:pullback} the pullback part of the definition of $\rho$ corresponds to the map
\[
(\Res_{k'/k}G')(k') \to G'(k')^*
\]
induced by $g \mapsto (g, 1, \ldots, 1)$.  Since the action of $\Gal(k'/k)$ on
\[
(\Res_{k'/k}G')_{k'} \cong \prod_{\Gal(k'/k)} G'
\]
is given by permuting coordinates, composition with the norm map yields the identity on $G'(k')$.
\end{proof}

\section{Two special subcategories of quasicharacter sheaves}%\label{sec:bounded}

\subsection{Bounded quasicharacter sheaves}\label{sec:bounded}

In this section we consider a class of quasicharacter sheaves on $G$ obtained by
replacing the Weil sheaf $(\gqcs{L}, \phi)$ on $\bG$ in the definition of a quasicharacter
sheaf with an $\ell$-adic local system on $G$ itself.

\begin{definition}
Let $\QCb(G)$ be the category of pairs $(\qcs{L}_0,\mu_0)$
where $\qcs{L}_0$ an $\ell$-adic local system on $G$ of rank~one,
equipped with an isomorphism $\mu_0 : m^* \qcs{L}_0 \to \qcs{L}_0 \boxtimes \qcs{L}_0$
satisfying the analogue of \ref{QC.1} on $G$;
morphisms in $\QCb(G)$ are defined as in the second part of
\ref{QC.3}.
This is the category of \emph{bounded quasicharacter sheaves} on $G$.
\end{definition}

The category of bounded quasicharacter sheaves is a rigid monoidal category in the obvious way.

\begin{proposition}\label{prop:BG}
Extension of scalars defines a full and faithful functor
$B_G : \QCb(G) \hookrightarrow \QC(G)$.
\end{proposition}

\begin{proof}
 Let $b_G : {\bar G} \to G$ be the pullback of $\Spec{\bFq} \to \Spec{\Fq}$ along $G\to \Spec{\Fq}$.
 Let $(\qcs{L}_0,\mu_0)$ be a bounded quasicharacter sheaf on $G$.
 Then $\qcs{L}_0$ is an $\ell$-adic constructible sheaf on $G$ and
 $b_G^* \qcs{L}_0$ comes equipped with an isomorphism
 $\phi : \Frob{G}^* b_G^*\qcs{L}_0 \to b_G^* \qcs{L}_0$.
 Moreover, the functor $\qcs{L}_0 \mapsto (b_G^* \qcs{L}_0,\phi)$
 from $\ell$-adic constructible sheaves on $G$ to $\ell$-adic constructible sheaves on $G$
 is full and faithful \citelist{\cite{deligne-katz:SGA7.2}*{Expos\'e XIII} \cite{beilinson-bernstein-deligne:81a}*{Prop. 5.2.1}}.
 This functor preserves local constancy, so takes local systems to local systems.
 Set $\mu \ceq b_{G\times G}^*\mu_0$; clearly, this satisfies \ref{QC.2}
 with $b_G^*\qcs{L}_0$ playing the role of $\gqcs{L}$.
 Moreover, $\phi$ is compatible with $\mu$ in the sense of \ref{QC.3}.
 This construction defines the functor $B_G : (\qcs{L}_0,\mu_0) \mapsto (b_G^*\qcs{L}_0,\mu, \phi)$
 and also shows that it is full and faithful.
\end{proof}

In general, $\QCb(G)$ is an essentially
proper subcategory of $\QC(G)$. We will identify $\QCb(G)$ with its image under $B_G$
and say that a quasicharacter sheaf $\qcs{L} \in \QC(G)$ is \emph{bounded}
if it is isomorphic to some $B_G(\qcs{L}_0, \mu_0)$.
We will see in Corollary~\ref{cor:bounded-and-finite} that the bounded quasicharacter sheaves
are precisely those whose Trace of Frobenius have bounded image.

\subsection{Finite quasicharacter sheaves}\label{sec:finite}

In this section we define \emph{finite quasicharacter sheaves}, a further refinement
of bounded quasicharacter sheaves that play a role in Section \ref{sec:restriction}.

The notion of a finite quasicharacter sheaf is informed
by a certain perspective on local systems on connected groups.
Recall that every geometric point $\bg$ on a connected scheme $G$
determines an equivalence between the categories of $\ell$-adic local systems on $G$
and $\ell$-adic representations of the \'etale fundamental group $\pi_1(G, \bg)$
\citelist{\cite{SGA5}*{VI,Prop. 1.2.5} \cite{deligne:80a}*{1.1.1}}.
 Since $\pi_1(G, \bg)$ is a projective limit of automorphism groups of finite \'etale covers,
 a general local system corresponds to a projective limit of characters of such groups.
 But some local systems can be described in terms of a single cover $f: H \to G$
 and a representation $\psi: \ker f \to \Aut(V)$;
 two such descriptions yield isomorphic local systems if and only if they are related by a weak
 isomorphism in $C(G)$.  While the description of local systems in terms of $\pi_1(G, \bg)$ does
 not generalize well to non-connected $G$, a perspective using finite \'etale covers does.

The definition of finite quasicharacter sheaves will require several steps,
beginning with the localization of a category introduced here.
Let us say that a morphism $f : H\to G$ of smooth commutative group schemes over $\Fq$ is a \emph{discrete isogeny}
if it is finite, surjective and \'etale and the action of $\Gal(\bFq/\Fq)$ on the \'etale group
scheme $\ker f \ceq f^{-1}({\bar e)}$ is trivial.
Consider the category $C(G)$ whose objects consist of pairs $(f,\psi)$, where
\begin{itemize}
\item $f : H\to G$ is a discrete isogeny of smooth commutative group schemes over $\Fq$,
\item $\psi : \ker f\to \Aut(V)$ is a representation on a finite-dimensional $\EE$-vector space $V$.
\end{itemize}
A morphism $(f,\psi) \to (f',\psi')$ in $C(G)$ is a pair $(g,T)$, where
\begin{itemize}
\item $g : H' \to H$ is a morphism of schemes such that $f' = f\circ g$,
\item $T : V\to V'$ is a linear transformation, equivariant for the action of
$\ker f'$ on $V'$ by $\psi'$ and the action of $\ker f'$ on $V$ by $\psi \circ \alpha$.
\[
\begin{tikzcd}
V \arrow{d}{T} & \Aut(V) & \arrow[swap]{l}{\psi} \ker f \arrow[hook]{r} & H \arrow{r}{f} & \arrow[equal]{d} G\\
V' & \Aut(V') & \arrow[swap]{l}{\psi'} \arrow[swap]{u}{g\vert_{\ker f'}} \ker f' \arrow[hook]{r} & \arrow[swap]{u}{g} H' \arrow{r}{f'} & G
\end{tikzcd}
\]
\end{itemize}

We put a tensor category structure on $C(G)$ as follows.
Let $(f,\psi)$ and $(f',\psi')$ be objects in $C(G)$.
Write $f\times_G f' : H\times_G H' \to G$ for the product of $f$ and $f'$
in the category of isogenies to $G$. We define an external tensor product
$\psi \otimes\psi' :  \ker(f\times_G f') \to \Aut(V\otimes V')$ and a external sum
$\psi \oplus\psi' :  \ker(f\times_G f') \to \Aut(V\oplus V')$
using the canonical isomorphism $\ker(f\times_G f') \iso (\ker f)\times (\ker f')$.
We may thus define $(f,\psi)\otimes(f',\psi') \ceq (f\times_G f' ,\psi\otimes\psi')$ and
$(f,\psi)\oplus(f',\psi')\ceq (f\times_G f' ,\psi\oplus\psi')$. To define duals in $C(G)$,
let $V^\vee \ceq \Hom_\text{vec}(V,\EE)$ and $\psi^\vee : \ker f \to \Aut(V^\vee)$ be
the dual representation.  We may then set $(f,\psi)^\vee \ceq (f,\psi^\vee)$.
These operations give $C(G)$ the structure of a tensor category.

Now we localize $C(G)$.
Let us say that $(g,T)$ in $C(G)$ is a \emph{weak isomorphism} if $g$ is surjective
and $T$ is an isomorphism of vector spaces.  Recall from
\cite{kashiwara-schapira:CatgoriesSheaves}*{Def 7.1.5} that a \emph{left multiplicative system}
is a family of morphisms $W$ that
\begin{itemize}
\item contains all isomorphisms,
\item is closed under composition,
\item is closed under pullback by arbitrary morphisms in $C(G)$,
\item and if the cokernel of a morphism $\phi \in \mor C(G)$ is in
$W$ then so is the kernel of $\phi$.
\end{itemize}

We further say that $W$ is \emph{saturated} if the following property holds for
morphisms $f, g$ and $h$ that can be composed appropriately:
whenever $g \circ f$ and $h \circ g$ are in $W$ then so is $h$
\cite{kashiwara-schapira:CatgoriesSheaves}*{Def 7.1.19}.

\begin{lemma}
The family of weak isomorphisms in $C(G)$ is a saturated, left multiplicative system.
\end{lemma}
\begin{proof}
Any isomorphism $(g, T)$ in $C(G)$ has both $g$ and $T$ isomorphisms, and thus is a weak isomorphism.
Clearly, the composition of two weak isomorphisms is a weak isomorphism.  The remaining properties
follow from the fact that in a weak isomorphism $(g, T)$, $g$ is an \'etale cover, together with standard
facts on \'etale covers and pullbacks.
\end{proof}
Let $W$ be the family of weak isomorphisms and $C(G)[W^{-1}]$ be the localization of $C(G)$ at $W$.
A morphism in $C(G)[W^{-1}]$ is an equivalence class of diagrams
\[
\begin{tikzcd}[column sep=30]
   (f,\psi) \arrow{r}{(g,T)} & (f',\psi') & \arrow[swap]{l}{(h,U)} (f_1,\psi_1)
   \end{tikzcd}
\]
 where $(g,T) : (f,\psi) \to (f',\psi')$ is any map in $C(G)$
 and $(h,U) :  (f',\psi') \to (f_1,\psi_1)$ is a weak isomorphism in $C(G)$ \cite{kashiwara-schapira:CatgoriesSheaves}*{Thm 7.1.16}.

\begin{definition}
Let $C_1(G)$ be the full, rigid monoidal subcategory of $C(G)$ consisting of objects
$(f,\psi)$ such that $\psi$ is a one-dimensional representation.
The category of \emph{finite quasicharacter sheaves} of $G$, denoted by $\QCf(G)$,
is the rigid monoidal category obtained by localizing $C_1(G)$ at weak isomorphisms.
\end{definition}

Having defined finite quasicharacter sheaves,
we justify our terminology by showing $\QCf(G)$ is a full subcategory of $\QCb(G)$.
To prepare for that result, we need two lemmas.

\begin{lemma}\label{lemma:finite-pullback}
Let $g : J \to G$ be a morphism of $\QC$-schemes.
Let $f_g : H\times_G J\to J$
be the pullback of $f$ along $g$ and define $\psi_g : \ker f_g \to \Aut(V)$ as in the diagram below:
\[
\begin{tikzcd}
V & \Aut(V) & \arrow[swap]{l}{\psi} \ker f \arrow[hook]{r} & H \arrow{r}{f} &  G\\
  &  & \arrow[dashed]{ul}{\psi_g} \arrow[swap]{u}{(g_f)\vert_{\ker f_g}} \ker f_g \arrow[hook]{r}
  & \arrow[swap]{u}{g_f} H\times_G J \arrow{r}{f_g} & \arrow{u}{g} J.
\end{tikzcd}
\]
Then $(f,\psi) \mapsto (f_g,\psi_g)$
defines a monomial functor $g^* : C(G)[W^{-1}] \to C(J)[W^{-1}]$.
\end{lemma}

\begin{proof}
The key point is that pullback along $g$ takes discrete isogenies to discrete isogenies,
and maps over $G$ to maps over $J$, as pictured below.
\[
\begin{tikzcd}[row sep={{{{40,between origins}}}},column sep={{{{50,between origins}}}}]
%\ V \arrow{dd}{T} &
\Aut(V) && \arrow[swap]{ll}{\psi} \ker f \arrow[hook]{rr} \arrow[leftarrow]{dd} && H \arrow{rr}{f} \arrow[leftarrow]{dd} && \arrow[equal]{dd} G\\
% &
& \arrow{ul} \ker f\times_G J \arrow{ur} \arrow[hook, crossing over]{rr} && H\times_G J \arrow{ur} \arrow[crossing over]{rr} && J \arrow[swap]{ur}{g} & \\
%V' &
 \Aut(V') && \arrow[swap,pos=0.25]{ll}{\psi'} \ker f'  \arrow[hook]{rr}
&& H' \arrow[pos=0.25]{rr}{f'} && G\\
% &
 & \arrow{ul} \ker f'\times_G J \arrow[crossing over]{uu} \arrow{ur} \arrow[hook]{rr}
 && \arrow[crossing over]{uu} H'\times_G J \arrow{ur} \arrow{rr} && \arrow[equals, crossing over]{uu} J \arrow[swap]{ur}{g} &
\end{tikzcd}
\]
In this form we could treat $g^*$ as a functor $C(J)\to C(G)$,
were it not for the fact that pullback is defined only up to isomorphism.
In the localized categories, this ambiguity disappears.
\end{proof}

\begin{lemma}\label{lemma:finite-iso}
Each finite quasicharacter sheaf $(f,\psi)$ in $\QCf(G)$ determines a
canonical isomorphism $\mu(f,\psi) : m^*(f,\psi) \to (f,\psi)\boxtimes(f,\psi)$ in $\QCf(G\times G)$.
 \end{lemma}

\begin{proof}
We will prove the proposition by finding a weak isomorphism
$m^*(f,\psi) \to (f,\psi)\boxtimes (f,\psi)$ in $C_1(G\times G)$
and showing that, after passing from $C_1(G\times G)$ to
$\QCf(G\times G)$, the resulting isomorphism,
denoted by $\mu(f,\psi)$, is canonically determined by $(f,\psi)$.

Set $H_m \ceq H\times_G(G\times G)$,
write $f_m : H_m \to G \times G$ for the pullback of $f$ to $H_m$,
and let $A$ and $A_m$ be the kernels of $f$ and $f_m$ respectively.  Consider the diagram of Lemma~\ref{lemma:finite-pullback}:
\[
\begin{tikzcd}[column sep=40]
V & \Aut(V) & \arrow[swap]{l}{\psi} A \arrow[hook]{r} & H \arrow{r}{f} &  G\\
  &  & \arrow[dashed]{ul}{\psi_m} \arrow[swap]{u}{(m_f)\vert_{A_m}} A_m \arrow[hook]{r}
  & \arrow[swap]{u}{m_f} H_m \arrow{r}{f_m} & \arrow{u}{m} G\times G.
\end{tikzcd}
\]
We have $(f,\psi) \boxtimes (f,\psi) = (f\times f, \psi\otimes \psi)$.
Since $f\circ m_H = m \circ (f\times f)$, the universal property of pullback gives
a unique $\theta : H\times H \to H_m$ making the following commute.
\[
\begin{tikzcd}[row sep=20,column sep=8]
\ & V \arrow{dd}{T} && \Aut(V) && A \arrow[hook]{rr} \arrow[swap]{ll}{\psi} &&  H \arrow{rr}{f} && G\\
 && && \arrow{ul}{\psi_m} A_m  \arrow[hook]{rr} \arrow{ur} && H_m \arrow[pos=0.3]{ur}{m_f} \arrow[swap, pos=0.8]{rr}{f_m}
 && G \times G \arrow[equal]{dr} \arrow[pos=0.42]{ur}{m} & \\
& V\otimes V && \Aut(V\otimes V) && \arrow[swap]{ll}{\psi \otimes \psi} \arrow[dashed,swap,pos=0.3]{ul}{\theta\vert_{A\times A}} A\times A \arrow[hook]{rr}
&& \arrow[dashed,swap]{ul}{\theta} \arrow[crossing over,swap,pos=0.7]{uu}{m_H} H\times H \arrow{rr}{f\times f} && G\times G
\end{tikzcd}
\]
Choose a basis $\{ v \}$ for $V$; then $\{ v\otimes v \}$ is a basis for $V\otimes V$.
Let $T : V \to V\otimes V$ be the linear transformation taking $v$ to $v\otimes v$. Then the lower part of this
diagram defines a weak isomorphism $(\theta,T): m^*(f,\psi) \to (f,\psi)\boxtimes (f,\psi)$  in $C_1(G\times G)$.

While $\theta$ was canonical, $T$ was not, as it depended on the choice of a basis for $V$.
However, after passing from $C_1(G\times G)$ to $\QCf(G\times G)$, the resulting isomorphism in $\QCf(G\times G)$
$\mu(f,\psi): m^*(f,\psi) \to (f,\psi)\boxtimes (f,\psi)$ is independent of the choice of basis for $V$.
 \end{proof}

\begin{proposition}\label{prop:bounded}
 $\QCf(G)$ is equivalent to a full subcategory of $\QCb(G)$.
\end{proposition}

\begin{proof}
 To prove the proposition we exhibit a full and faithful monoidal functor $L_G : \QCf(G) \to \QCb(G)$.

 If $V$ is a finite-dimensional $\EE$-vector space then we write $V_H$ for the constant sheaf $V$ on $H$.
If $(f,\psi)$ is a finite quasicharacter on $G$ where $\psi$ is a representation of $\ker f$ on $V$,
let $L(f,\psi)$ be the local system on $G$ given by the $\psi$-isotypic component of the local system $f_* V_H$ on $G$.
 We may identify the stalk $(f_* V_H)_{\bar g}$ of $f_* V_H$ at a geometric point ${\bar g}$ on $G$
 with the $\EE$-vector space $\Hom_\text{set}(\abs{f^{-1}({\bar g})},V)$
 and the stalk $L(f,\psi)_{\bar g}$ of $L(f,\psi)$ at a geometric point ${\bar g}$ on $G$
 with the $\EE$-vector space consisting of those functions $s: \abs{f^{-1}({\bar g})} \to V$ such that
 $s(a\cdot {\bar h}) = \psi(a)(s({\bar h}))$ for all $a\in \ker f$ and for all ${\bar h} \in f^{-1}({\bar g})$.

 For any map $(g,T)$ in $C(G)$, let $L(g,T) : L(f,\psi) \to L(f',\psi')$ be the homomorphism of local systems defined
 by $s \mapsto s' \ceq T\circ s\circ g$.
 This construction defines a full monoidal functor $L$ from $C(G)$ to the category $\Loc(G)$ of local systems on $G$ that
 commutes with pullback along any morphism of $\QC$-schemes,
 as defined using Lemma~\ref{lemma:finite-pullback}.
 Moreover, since $L(f,T)$ is an isomorphism of local systems
 if and only if $(f,T)$ is a weak isomorphism,
 the full functor $L : C(G) \to \Loc(G)$ factors through the localization functor $C(G) \to C(G)[W^{-1}]$
 to define a full and faithful functor $L_{W} : C(G)[W^{-1}] \to \Loc(G)$;
 see \cite{kashiwara-schapira:CatgoriesSheaves}*{Exercise 7.5}.

 By Lemma~\ref{lemma:finite-iso}, each $(f,\psi) \in \QCf(G)$ canonically determines
 an isomorphism
 \[
 \mu(f,\psi) : m^*(f,\psi) \to (f,\psi) \boxtimes (f,\psi)
 \]
 in $\QCf(G\times G)$.
 It follows immediately that the functor
 \[
 L_G  \ceq  L_{W}\vert_{\QCf(G)} : \QCf(G) \to \QCb(G),
 \]
 given on objects by $(f,\psi) \mapsto (L(f,\psi), L_{W}(\mu(f,\psi)))$,
 is a full and faithfull monoidal functor, completing the proof.
\end{proof}

On general $\QC$-schemes $G$, there are bounded quasicharacter sheaves which are not finite.  Identifying
$\QCf(G)$ with its image under $B_G \circ L_G$, as defined in
Proposition~\ref{prop:BG} and in the proof of Proposition~\ref{prop:bounded},
we will see in Corollary~\ref{cor:bounded-and-finite} that
a quasicharacter sheaf is finite if and only if its Trace of Frobenius has finite image.

\section{The case of connected commutative algebraic groups} \label{sec:connected}

When $G$ is connected, we may apply the classic function--sheaf dictionary.  In order to do so,
however, we must relate our notion of quasicharacter sheaf to the sheaf side of the standard dictionary.
Choose any $\Fq$-rational point of $G$ and let $\bg$ be the geometric point on $G$ lying above $g$.
Recall that the \emph{Weil group} of $G$, which we will denote by $\W(G,\bg)$, is a subgroup of the \'etale
fundamental group defined by the following diagram:
\[
 \begin{tikzcd}
 1 \rar & \ar[equal]{d} \pi_1(\bG, \bg) \rar & \W(G,\bg) \rar \dar[hook] & \Weil{} \rar \dar[hook] & 1 \\
 1 \rar &  \pi_1(\bG, \bg) \rar & \pi_1(G,\bg) \rar & \Gal(\bFq/\Fq) \rar & 1.
 \end{tikzcd}
\]
The $\Fq$-rational point $g$ under the geometric point $\bg$ determines a splitting
$\Weil{}\to \W(G,\bg)$ of $\W(G,\bg)\to \Weil{}$.

For connected groups, the subcategories introduced in the previous two sections agree with $\QC(G)$:

\begin{proposition}\label{prop:connected}
  If $G$ is a connected commutative algebraic group over $\Fq$ then
  the full subcategories $\QCf(G) \hookrightarrow \QCb(G) \hookrightarrow \QC(G)$
  appearing in Sections~\ref{sec:bounded} and \ref{sec:finite} are equivalences.
\end{proposition}

\begin{proof}
  Observe that the forgetful functor $(\gqcs{L},\mu,\phi) \mapsto (\gqcs{L},\phi)$
  sends quasicharacter sheaves on $G$ to $\ell$-adic Weil sheaves on $G$ \cite{deligne:80a}*{1.1.10}.
  While is it not true that all Weil sheaves on $G$ descend to local systems on $G$,
  we will see that those that appear in the image of this forgetful functor from quasicharacter sheaves do.

  Since $G$ is connected, the geometric point $\bg$ determines
  an equivalence between the category of $\ell$-adic Weil local systems on $G$ and
  $\ell$-adic representations of $\W(G,\bg)$ \cite{deligne:80a}*{1.1.12}.
  Now let $(\gqcs{L},\mu,\phi)$ be a quasicharacter sheaf on $G$
  and let $\rho : \W(G, \bg) \to \EEx$ be the character determined by $(\gqcs{L},\phi)$.
  Composing with the splitting $\Weil{} \to \W(G,\bg)$ yields an $\ell$-adic character
  $\rho_g : \Weil{} \to \EEx$, which is the same as the Trace of Frobenius defined in Section~\ref{sec:Frob}:
  \[
  \rho_g(\Frob{}) =  \trFrob{\qcs{L}}(g).
  \]

  On the other hand, we have already seen that $\trFrob{\qcs{L}} : G(\Fq) \to \EEx$
  is a group homomorphism. Since $G$ is an algebraic group over $\Fq$, it is a
  variety over $\Fq$ and thus $G(\Fq)$ is finite.
  Therefore $\trFrob{\qcs{L}}(g) = \rho_g(\Frob{})$ is a root of unity
  for every $g\in G(\Fq)$.  Since $\Weil{}$ is generated by
  $\Frob{}$ and $\rho_g : \Weil{} \to \EEx$ is
  a character, it follows that the image of $\rho_g$ is a finite group.
  Thus, $\rho_g$ extends to an $\ell$-adic character of $\Gal(\bFq/\Fq)$,
  which we will also denote $\rho_g$.

  We may now lift the $\ell$-adic character $\rho_g : \Gal(\bFq/\Fq) \to \EEx$
  to an $\ell$-adic character $\pi_1(G,\bg) \to \EEx$ using the canonical topological group homomorphism
  $\pi_1(G,\bg) \to \Gal(\bFq/\Fq)$. But $\bg$ also
  determines an equivalence between the category of $\ell$-adic
  representations of $\pi_1(G,\bg)$ and $\ell$-adic local systems on $G$. Let
  $\qcs{L}_0$ be a local system on $G$ in the isomorphism class
  determined by this $\ell$-adic character of $\pi_1(G,\bg)$.
  Then $b_G^*\qcs{L}_0 \iso \gqcs{L}$.
  Since $b_G^*$ is full and faithful
  \citelist{\cite{deligne-katz:SGA7.2}*{Expos\'e XIII} \cite{beilinson-bernstein-deligne:81a}*{Prop. 5.2.1}},
  $
  b_{G\times G}^* : \Hom(m^*\qcs{L}_0,\qcs{L}_0\boxtimes\qcs{L}_0) \to \Hom({\bar m}^*\gqcs{L},\gqcs{L}\boxtimes\gqcs{L})
  $
  is a bijection
  (hom taken in the categories on constructible $\ell$-adic sheaves on
  $G\times G$ and ${\bar G}\times {\bar G}$ respectively,
  in which $\ell$-adic local systems sit as full subcategories).
  Let $\mu_0 : m^*\qcs{L}_0 \to \qcs{L}_0\boxtimes\qcs{L}_0$ be the isomorphism matching
  $\mu : {\bar m}^*\gqcs{L} \to \gqcs{L}\boxtimes\gqcs{L}$,
  the latter appearing in the definition of $\qcs{L}$.
  Then, as in Section~\ref{sec:bounded}, $(\qcs{L}_0,\mu_0)$ is an object in $\QCb(G)$
  and $\qcs{L} \ceq (\gqcs{L},\mu,\phi)$ is isomorphic to $(b_G^*\qcs{L}_0,b_{G\times G}^*\mu_0)$ in $\QC(G)$.
  Thus, the full and faithful functor $B_G : \QCb(G) \to \QC(G)$ from Section~\ref{sec:bounded}
  is also essentially surjective, hence an equivalence.

  In Section~\ref{sec:finite} we saw that $\QCf(G)$ is a full subcategory of $\QCb(G)$;
  more precisely, we exhibited a full and faithful functor $L_G : \QCf(G) \to \QCb(G)$.
  We now show that this functor is essentially surjective
  when $G$ is a connected, commutative algebraic group over $\Fq$.

  Let $(\qcs{L}_0,\mu_0)$ be a bounded quasicharacter sheaf on $G$.
  Then $\qcs{L}\ceq B_G(\qcs{L}_0,\mu_0)$ is a quasicharacter sheaf on $G$.
  Recall the definition of the character $\trFrob{\qcs{L}} : G(\Fq) \to \EEx$ from Section~\ref{sec:Frob}.
  Let $f : G\to G$ be the Lang isogeny.
  Recall that $\ker f = G(\Fq) = G^{\Frob{}}$;
  in particular, the action of $\Gal(\bFq/\Fq)$ on $\ker f$ is trivial.
  Let $V \ceq \EE$ and let $\psi : \ker f \to \Aut(V)$ be the representation given by
  $\psi(g)(v) \ceq \trFrob{\qcs{L}}(g) v$ for $v\in V$.
  Then $(f,\psi)$ is a finite quasicharacter sheaf.

  To show that $L_G(f,\psi) \iso (\qcs{L}_0,\mu_0)$ we apply the function--sheaf dictionary
  \citelist{\cite{laumon:87a}*{1.1.3} \cite{deligne:SGA4.5}*{sommes trig.}}  as follows.
  First, recall that $L_G(f,\psi) \ceq (L(f,\psi), \mu(f,\psi))$.
  Let $k'/k$ be any degree-$n$ extension;
  let $\Nm : G(k')\to G(k)$ be the norm map
  and let $b : G_{k'} \to G$ be the base change map.
  By \cite{laumon:87a}*{1.1.3.3},
  $\trFrob{b^*L(f,\psi)} = \trFrob{\qcs{L}} \circ \Nm$;
  on the other hand, $\trFrob{\qcs{L}} \circ \Nm = \trFrob{b^*\qcs{L}}$,
   by Proposition~\ref{prop:basechange}.
  Using \cite{laumon:87a}*{Th\'eor\`eme 1.1.2}, it follows that  $L(f,\psi) \iso \qcs{L}_0$ in $\Loc(G)$.
  It is now clear that $(L(f,\psi),\mu(f,\psi)) \iso (\qcs{L}_0,\mu_0)$ in $\QCb(G)$.
  This completes the proof that $L_G $ is essentially surjective.
  Since we have already seen (in Section~\ref{sec:finite}) that $L_G$ is full and faithful,
  it follows that $L_G$ is an equivalence.
\end{proof}

\begin{corollary}
 If $G$ is a connected, commutative algebraic group over $\Fq$ then
 \[
  \TrFrob{G} : \QCiso{G} \to G(\Fq)^*
 \]
 is an isomorphism of groups.
\end{corollary}
\begin{proof}
 It suffices to observe that the group homomorphism
 $\TrFrob{G} : \QCiso{G} \to G(\Fq)^*$
 is surjective because each character $\chi \in G(\Fq)^*$ determines a finite quasicharacter
 sheaf $(\Lang,\chi^{-1})$ such that Trace of Frobenius of the quasicharacter sheaf
 $B_G (L_G(\Lang,\chi^{-1}))$ is $\chi$.
\end{proof}

As Proposition~\ref{prop:connected} shows,
when $G$ is a connected algebraic group over $\Fq$,
it is appropriate to replace quasicharacter sheaves on $G$
with the conceptually simpler category $\QCf(G)$ of finite quasicharacter sheaves on $G$.
In this context, finite quasicharacter sheaves may also be apprehended as
\emph{certain} $\ell$-adic characters of fundamental group $\pi_1(G,{\bar e})$.
As explained in \cite{kamgarpour:09a}*{\S2},
the relevant characters of $\pi_1(G,{\bar e})$ are precisely those
that factor through a particular quotient of $\pi_1(G,{\bar e})$
denoted there by $\Pi_\text{disc}(G)$.
Using this fact, one can show that the category of discrete isogenies to $G$ is a Galois category,
in the sense of \cite{grothendieck:SGA1}*{Expos\'e V, \S 4}.
Moreover, since $\Pi_\text{disc}(G) \iso G(\Fq)$ \cite{kamgarpour:09a}*{App'x B},
the fundamental functor attached to this Galois category can be used to produce an equivalence
between the category of finite quasicharacter sheaves on $G$ and
the category of one-dimensional representations of $G(\Fq)^*$, thus
providing an alternate proof that $\QCiso{G} \iso G(\Fq)^*$
when $G$ is a connected algebraic group over $\Fq$.

In principle, something similar is possible in the more general
context of this paper, where $G$ is a $\QC$-scheme, to which we now return.
In this generality, however, neither of the full subcategories $\QCf(G) \subset \QCb(G) \subset \QC(G)$ are equivalences;
see the end of Section~\ref{sec:snake}.
Nevertheless, every quasicharacter sheaf $(\gqcs{L},\mu,\phi)$ on $G$
determines a Weil sheaf $(\gqcs{L},\phi)$ on $G$
so if we choose a geometric point ${\bar g}_x$ in each component $\bG^x$ of $G$
then this choice can be used to convert
the Weil sheaf $(\gqcs{L},\phi)$ into an $\ell$-adic character of $\prod_{x\in \pi_0(\bG)}\W(\bG^x, {\bar g}_x)$.
However, it seemed cumbersome to manage various
choices of families of base points and track the action
of $\Gal(\bFq/\Fq)$ on this (generally) infinite product of
Weil groups and also difficult to identify the relevant quotient of this group.
Viewed in this light, Definition~\ref{def:QC} is comparatively simple.

\section{The case of \'etale commutative group schemes} \label{sec:etale}

\subsection{Quasicharacter sheaves, revisited}

\'Etale group schemes form a counterpoint to connected groups,
since the component group of a smooth commutative group scheme
is an \'etale group scheme \cite{vdGeer-Moonen:AbelianVarieties}*{III, \S 4}.
However, the function--sheaf dictionary is substantially differentl in the two cases.
To see why, we begin by specializing the definition of quasicharacter sheaves to the case
of etale commutative group schemes,
using the equivalence $G \mapsto G(\bFq)$
from the category of \'etale group schemes over $\Fq$ to the category of groups equipped
with an action of $\Gal(\bFq/\Fq)$, continuous for the discrete topology on the group.
Under this equivalence, a quasicharacter sheaf on an etale commutative group scheme $G$ over $\Fq$ is given by the data of:
\begin{enumerate}
 \labitem{(qc.1)}{qc.1} an indexed set of one-dimensional
  $\EE$-vector spaces $\gqcs{L}_x$, as $x$ runs over
  $G(\bFq)$;

 \labitem{(qc.2)}{qc.2} an indexed set of isomorphisms
  $\mu_{x,y} : \gqcs{L}_{x+y} \xrightarrow{\iso} \gqcs{L}_{x} \otimes\gqcs{L}_{y}$,
  for all $x,y \in G(\bFq)$, such that
  \[
   \begin{tikzcd}[row sep=40]
    \gqcs{L}_{x+y+z} \arrow{rr}{\mu_{x+y,z}} \arrow[swap]{d}{\mu_{x,y+z}}
    && \gqcs{L}_{x+y}\otimes\gqcs{L}_{z} \dar{\mu_{x,y} \tight{0.5}{\otimes}{1} \id} \\
    \gqcs{L}_{x} \otimes\gqcs{L}_{y+z} \arrow{rr}{\id \otimes\mu_{y,z}}
    && \gqcs{L}_{x} \otimes\gqcs{L}_{y} \otimes\gqcs{L}_{z}
   \end{tikzcd}
  \]
  commutes, for all $x,y,z\in G(\bFq)$; and
 \labitem{(qc.3)}{qc.3} an indexed set of isomorphisms $\phi_{x} : \gqcs{L}_{\Frob{}(x)} \to \gqcs{L}_x$
  such that
  \[
   \begin{tikzcd}[row sep=40]
    \gqcs{L}_{\Frob{}(x)+\Frob{}(y)} \arrow[swap]{d}{\phi_{x+y}} \arrow{rr}{\mu_{\Frob{}(x),\Frob{}(y)}}
    && \gqcs{L}_{\Frob{}(x)}\otimes\gqcs{L}_{\Frob{}(y)} \dar{\phi_x \tight{0}{\otimes}{0} \phi_y} \\
    \gqcs{L}_{x+y} \arrow{rr}{\mu_{x,y}}
    && \gqcs{L}_x \otimes\gqcs{L}_y
   \end{tikzcd}
  \]
  commutes, for all $x,y\in G(\bFq)$.
\end{enumerate}
Under this equivalence, a morphism $\alpha : \qcs{L} \to \qcs{L'}$ of quasicharacter sheaves on $G$ is given by
\begin{enumerate}
 \labitem{(qc.4)}{qc.4} an indexed set $\alpha_x : \gqcs{L}_x \to \gqcs{L'}_x$
  of linear transformations such that
  \[
   \begin{tikzcd}[column sep=40]
    \arrow[swap]{d}{\phi_x} \gqcs{L}_{\Frob{}(x)} \rar{\alpha_{\Frob{}(x)}} & \gqcs{L'}_{\Frob{}(x)} \dar{\phi_x'}
    &\arrow[draw=none]{d}[pos=.4,description]{\text{\normalsize{and}}}
    & \arrow[swap]{d}{\mu} \gqcs{L}_{x+y} \rar{\alpha_{x+y}} & \gqcs{L'}_{x+y} \dar{\mu'_{x+y}} \\
    \gqcs{L}_x \rar{\alpha_x} & \gqcs{L'}_x
    & {} & \gqcs{L}_x\otimes\gqcs{L}_y \rar{\alpha_x\otimes\alpha_y} & \gqcs{L'}_x \otimes\gqcs{L'}_y
   \end{tikzcd}
  \]
  both commute for all $x, y \in G(\bFq)$.
\end{enumerate}

\subsection{Group Cohomology}

\begin{lemma}\label{lem:etale-2cocycle}
    If $G$ is an \'etale commutative group scheme over $\Fq$
    then the group homomorphism
    \[
    S_{G} : \QCiso{G} \to \Hh^0(\Weil{}, \Hh^2(G(\bFq),\EEx)),
    \]
    defined in Section~\ref{sec:invisible}, is surjective.\todo{Kernel? Canonical section?}
\end{lemma}

\begin{proof}
Pick $[g]\in \Hh^0(\Weil{}, \Hh^2(G(\bFq),\EEx))$. Then $g\in Z^2(G(\bFq),\EEx)$ is fixed by Frobenius, so $\,^{\Frob{}}g = g h$ for some $h\in B^2(G(\bFq),\EEx)$. Accordingly, $h(x,y) = \frac{f(x+y)}{f(x) f(y)}$ for some $f : G(\bFq) \to \EEx$, so
\begin{equation}\label{gf}
g(\Frob{}(x),\Frob{}(y)) = g(x,y) \frac{f(x+y)}{f(x) f(y)}
\end{equation}
for all $x,y\in G(\bFq)$.
Set $\gqcs{L}_x = \EE$ for every $x\in G(\bFq)$ and define $\mu_{x,y} : \gqcs{L}_{x+y} \to \gqcs{L}_x \otimes \gqcs{L}_y$ by $\mu_{x,y}(1) = g(x,y) 1 \otimes 1$.
Since $g\in Z^2(G(\bFq),\EEx)$, the indexed set of isomorphisms $\{ \mu_{x,y} \tq x,y \in G(\bFq) \}$ satisfies \ref{qc.2}.
Define $\phi_x : \gqcs{L}_{\Frob{}(x)}\to \gqcs{L}_x$ by $\phi_x(1) = f(x)$.
Then set of isomorphisms $\{ \phi_{x} \tq x \in G(\bFq) \}$ satisfies \ref{qc.3} by \eqref{gf}. In this way, the cocycle $g\in Z^0(\Weil{}, Z^2(G(\bFq),\EEx))$ defines a quasicharacter sheaf $\qcs{L}$. It is clear that $\qcs{L}$ maps to $[g] \in \Hh^0(\Weil{}, \Hh^2(G(\bFq),\EEx))$ under \eqref{2cocycle}.
\end{proof}

%In order to relate $\QCiso{G}$ to $G(k)^*$, we need a simple result relating duals, invariants and coinvariants.

\begin{lemma} \label{lem:dual-inv}
Let $X$ be an abelian group equipped with an action of $\Weil{}$.
 Then
\begin{align*}
 (X^*)_{\Frob{}} &\to (X^{\Frob{}})^* \\
 [f] &\mapsto f|_{X^{\Frob{}}}
\end{align*}
is an isomorphism.
\end{lemma}

\begin{proof}
We can describe $X^{\Frob{}}$ as the kernel of the map $X \xrightarrow{\Frob{}-1} X$;
let $Y = (\Frob{}-1)X$ be the augmentation ideal.  Dualizing the sequence
\[
 0 \to X^{\Frob{}} \to X \to Y \to 0
\]
yields
\[
 0 \to Y^* \to X^* \to (X^{\Frob{}})^* \to \Ext^1_\ZZ(Y, \EEx).
\]
Since $\Ext^1_\ZZ(-,\EEx)$ vanishes, we get a natural isomorphism from the cokernel of $Y^* \xrightarrow{\Frob{}-1} X^*$ to $(X^{\Frob{}})^*$.
\end{proof}

\begin{lemma}\label{lem:etale-trace}
If $G$ is an \'etale commutative group scheme over $\Fq$
then for every the $\chi \in G(\Fq)^*$ there is a visible quasicharacter sheaf $\qcs{L}$ on $G$ such that $\trFrob{\qcs{L}} = \chi$.
In particular, Trace of Frobenius
\[
\TrFrob{G} : \QCiso{G} \to G(\Fq)^*,
\]
is surjective, even when restricted to visible quasicharacter sheaves.\todo{Kernel? Canonical section?}
\end{lemma}

\begin{proof}
Pick $\chi \in G(\Fq)^*$.
By Lemma~\ref{lem:dual-inv}, there is some $f\in G(\bFq)^*$ such that $f\vert_{G(\Fq)^*} = \chi$.
Set $\gqcs{L}_x = \EE$ for every $x\in G(\bFq)$.
Define $\mu_{x,y} : \gqcs{L}_{x+y} \to \gqcs{L}_x\otimes \gqcs{L}_y$ by $\mu_{x,y}(1) = 1 \otimes 1$.
Define $\phi_{x} : \gqcs{L}_{\Frob{}(x)} \to \gqcs{L}_x$ by $\phi_{x}(1) = f(x)$.
Since $f : G(\bFq) \to \EEx$ is a group homomorphism,
condition \eqref{nohom} is satisfied with $g =1$, the trivial cocycle in $Z^2(G(\bFq),\EEx)$.
It is now clear that we have defined a \emph{visible} quasicharacter sheaf $\qcs{L}$ with $\trFrob{\qcs{L}} = \chi$.
\end{proof}


\begin{proposition} \label{prop:etale-iso}
	If $G$ is an \'etale commutative group scheme over $\Fq$
	and if $G(\bFq)$ is finitely generated,
  	then
  	\[
  	\QCiso{G} \iso \Hh^2(G(\bFq)\rtimes \Weil{}, \EEx).
  	\]
  	Moreover, this isomorphism is functorial
  	in the sense of Proposition~\ref{prop:functorialG}.
\end{proposition}

\begin{proof}
By Lemma~\ref{lem:dual-inv}, $G(\Fq)^* \iso \Hh^1(\Weil{},\Hh^1(G(\bFq),\EEx))$. Composing this isomorphism with the Trace of Frobenius gives
\[
T_{G}: \QCiso{G}\to  \Hh^1(\Weil{},\Hh^1(G(\bFq),\EEx)),
\]
which is surjective, Lemma~\ref{lem:etale-trace}.
Combining this with
\[
S_{G} : \QCiso{G} \to \Hh^0(\Weil{}, \Hh^2(G(\bFq),\EEx))
\]
gives
\[
\QCiso{G} \mathop{\longrightarrow}\limits^{T_{G}\oplus S_{G}} \Hh^1(\Weil{},\Hh^1(G(\bFq),\EEx))\oplus \Hh^0(\Weil{}, \Hh^2(G(\bFq),\EEx)).
\]
The Hochschild-Serre spectral sequence (for finitely generated abelian groups!) provides
\[
\Hh^2(G(\bFq)\rtimes \Weil{},\EEx)  \iso \Hh^1(\Weil{}, \Hh^1(G(\bFq),\EEx)) \oplus \Hh^0(\Weil{}, \Hh^2(G(\bFq),\EEx)).
\]
Composing these last two maps defines
\begin{equation}
\QCiso{G}\to \Hh^2(G(\bFq)\rtimes \Weil{},\EEx).
\end{equation}
To see that this homomorphism is an isomorphism,\todo{Clifton, you are here.}
\end{proof}

\section{The main result}

\subsection{The component group sequence} \label{sec:restriction}

Consider the short exact sequence in the category of $\QC$-schemes
defining the component group scheme for $G$:
\begin{equation}\label{eq:pi0}
\begin{tikzcd}
0 \rar & G^0 \arrow{r}{\iota_0} & G \arrow{r}{\pi_0} & \pi_0(G) \rar & 0.
\end{tikzcd}
\end{equation}
Since $\pi_0(G)$ is also a $\QC$-scheme,
Proposition~\ref{prop:pullback} implies that \eqref{eq:pi0} defines the sequence of functors
\begin{equation}\label{eq:pi1}
\begin{tikzcd}
\QC(0) \rar & \QC(\pi_0(G)) \arrow{r}{\pi_0^*} & \QC(G) \arrow{r}{\iota_0^*} & \QC(G^0) \rar & \QC(0)
\end{tikzcd}
\end{equation}
and therefore, after passing to isomorphism classes, the sequence of abelian groups
\begin{equation}\label{eq:pi2}
\begin{tikzcd}
%0 \rar &
\QCiso{\pi_0(G)} \arrow{r}{\pi_0^*} & \QCiso{G} \arrow{r}{\iota_0^*} & \QCiso{G^0} \rar & 0.
\end{tikzcd}
\end{equation}
 Note that we found the groups $\QCiso{G^0}$
and $\QCiso{\pi_0(G)}$
in Sections~\ref{sec:connected} and \ref{sec:etale}.
In this section we show that \eqref{eq:pi2} is exact.

Recall from Section~\ref{sec:finite} that
a morphism $f : H\to G$ of group $\Fq$-schemes is a discrete isogeny
if it is a finite, surjective \'etale morphism and
the action of $\Gal(\bFq/\Fq)$ on the \'etale group scheme $\ker f$ is trivial.

\begin{lemma}\label{lemma:ext}
Every discrete isogeny onto $G^0$ extends to a discrete
isogeny onto $G$ inducing an isomorphism on component groups.
\end{lemma}

\begin{proof}
Let $\pi: B \to G^0$ be a discrete isogeny and set $A \ceq \ker \pi$.
  We will find a discrete isogeny $f: H\to G$
  such that that $H^0 = B$, $f^0 =\pi$ and
  $\pi_0(f) : \pi_0(H)\to \pi_0(G)$ is an isomorphism of component
  groups.  Namely, we will fit $\pi$ into the following diagram,
  \begin{equation}\label{extension-diagram}
  \begin{tikzcd}
  A \arrow[equal]{r} \dar & A \dar \\
  B \rar \dar[swap]{\pi} & H \rar \dar[swap]{f} & \pi_0(H) \arrow{d}[below,rotate=90]{\sim}[swap]{\pi_0(f)} \\
  G^0 \rar & G \rar & \pi_0(G),
  \end{tikzcd}
  \end{equation}
  where all rows and columns are exact and all maps are defined over
  $\Fq$.  We will do so by passing back and forth between group
  schemes over $\Fq$ and their $\bFq$-points.

  Extensions of $G^0(\bFq)$ by $A(\bFq)$, such as $B(\bFq)$,
  correspond to classes in $\Ext^1_{\ZZ[\Weil{}]}(G^0(\bFq), A(\bFq))$.
  Similarly, extensions of $G(\bFq)$ by $A(\bFq)$ correspond to
  classes in $\Ext^1_{\ZZ[\Weil{}]}(G(\bFq), A(\bFq))$.  The map
  $G^0(\bFq) \to G(\bFq)$ induces the map
  \[
  \Ext^1_{\ZZ[\Weil{}]}(G(\bFq), A(\bFq)) \to \Ext^1_{\ZZ[\Weil{}]}(G^0(\bFq), A(\bFq))
  \]
  fitting into the long exact sequence derived from applying
  the functor $\Hom(\mbox{---}, A(\bFq))$ to $G^0(\bFq) \to G(\bFq) \to \pi_0(G)(\bFq)$:
  \[
  \Ext^1_{\ZZ[\Weil{}]}(G(\bFq), A(\bFq)) \to \Ext^1_{\ZZ[\Weil{}]}(G^0(\bFq), A(\bFq)) \to \Ext^2_{\ZZ[\Weil{}]}(\pi_0(G)(\bFq), A(\bFq)).
  \]
  Since $\Weil{} \cong \ZZ$ has cohomological dimension $1$ \cite{brown:CohomologyGrps}*{Ex. 4.3},
  $\Ext^2_{\ZZ[\Weil{}]}(\pi_0(G)(\bFq), A(\bFq))$ vanishes \cite{cartan-eilenberg:HomologicalAlgebra}*{Thm. 2.6}.

  We therefore have the existence of diagram \eqref{extension-diagram}
  at the level of $\bFq$-points.  This expresses $H(\bFq)$ as a
  disjoint union of translates of $B(\bFq)$; by transport of structure
  we may take $H$ to be a group scheme over $\bFq$.  Similarly, the
  restriction of $f$ to each component of $H$ is a morphism of
  schemes, and thus $f$ is as well.  Finally, the whole diagram
  descends to a diagram of $\Fq$-schemes since the $\bFq$-points of
  the objects come equipped with continuous $\Gamma$-actions, and the
  morphisms are $\Gamma$-equivariant.
\end{proof}

In order to use the results of Section~\ref{sec:connected}, we need to know that
identity component of $G$ is actually an algebraic group.

\begin{lemma} \label{lem:G0alg-grp}
If $G$ is a commutative smooth group scheme over $\Fq$ then $G^0$ is a connected algebraic group.
\end{lemma}
\begin{proof}
 Since $G$ is a smooth group scheme over $\Fq$, its
 identity component $G^0$ of $G$ is a connected smooth,
 group scheme of finite type over $\Fq$, reduced over some finite extension of $\Fq$
 \cite{vdGeer-Moonen:AbelianVarieties}*{3.17}.
 Since $\Fq$ is a finite field and hence perfect, $G^0$ is actually reduced over $\Fq$
 \cite{EGAIV2}*{Prop 6.4.1}.  Since every group scheme over a field is separated
 \cite{vdGeer-Moonen:AbelianVarieties}*{3.12},
 it follows that $G^0$ is a connected algebraic group.
\end{proof}

\begin{proposition}\label{prop:restriction}
For every quasicharacter sheaf $\qcs{L}^0$ on $G^0$ there is some finite quasicharacter sheaf
$\qcs{L}$ on $G$ such that $\qcs{L}\vert_{G^0} \iso \qcs{L}^0$.
As a consequence, the restriction functor $\iota_0^* : \QC(G)\to \QC(G^0)$ is essentially surjective.
\end{proposition}

\begin{proof}
  By Lemma~\ref{lem:G0alg-grp} and Proposition~\ref{prop:connected}, every
  quasicharacter sheaf on $G^0$ is isomorphic to a
  finite quasicharacter sheaf on $G^0$,
  so to prove the proposition it suffices to show that every
  finite quasicharacter sheaf on $G^0$ extends to a finite quasicharacter sheaf on $G$.

 Let $(\pi,\psi)$ be a finite quasicharacter sheaf on $G^0$.
 By Lemma~\ref{lemma:ext}, there is an extension of the
 discrete isogeny $\pi : B \to G^0$ to a discrete isogeny $f : H \to G$
 such that $\pi_0(f) : \pi_0(H)\to \pi_0(G)$ is an isomorphism.
 Then $(f,\psi)$ is a finite quasicharacter sheaf on $G$ and
 $(f,\psi)\vert_{G^0} \iso (\pi,\psi)$.
\end{proof}

\begin{proposition}\label{prop:middleexact}
 The sequence
 \[
  \begin{tikzcd}
%  0 \rar &
 \QCiso{\pi_0(G)} \arrow{r}{\pi_0^*} & \QCiso{G} \arrow{r}{\iota_0^*} & \QCiso{G^0} \rar & 0.
  \end{tikzcd}
 \]
 is exact.
\end{proposition}

\begin{proof}
We already know the sequence is exact at $\QCiso{G^0}$.
Here we show exactness at $\QCiso{G}$.
First note that $\iota_0^* \circ \pi_0^*$ is trivial by Proposition~\ref{prop:pullback}.
So it suffices to show that if $\qcs{L} = (\gqcs{L},\mu,\phi)$ is a quasicharacter sheaf on $G$
with $\qcs{L}\vert_{G^0} = (\EE)_{G^0}$ then $\qcs{L}$ is in the essential image of $\pi_0^*$.

For each $x\in \pi_0(\bG)$, set $\bG^x \ceq \pi_0^{-1}(x)$.
Let $g, g'$ be geometric points in the same
geometric connected component $\bG^x$.
Set $a = g^{-1}g'$ and note that $a$ is a geometric point in $\bG^0$.
Let $\mu_{g,a} : \gqcs{L}_{ga} \to \gqcs{L}_g \otimes \gqcs{L}_a$
be the isomorphism of vector spaces obtained by restriction of
$\mu : m^*\gqcs{L} \to \gqcs{L} \boxtimes \gqcs{L}$ to the
geometric point $(g,a)$ on $\bG^x \times \bG^0$.
Since $\qcs{L}\vert_{G^0} = (\EE)_{G^0}$,
the stalk of $\gqcs{L}$ at $a$ is $\EE$.
In this way the pair of geometric points $g, g' \in \bG^x$
determines an isomorphism $\varphi_{g,g'} \ceq \mu_{g,a}^{-1}$
from $\gqcs{L}_{g}$ to $\gqcs{L}_{g'}$.

The isomorphisms $\varphi_{g,g'}: \gqcs{L}_{g} \to \gqcs{L}_{g'}$ are canonical
in the following sense: if $g,g'\in \bG^x$ and $h,h'\in \bG^y$
then it follows from \ref{QC.2} and \ref{QC.3}
that
 \begin{equation}\label{eq:qc}
  \begin{tikzcd}[column sep=55]
   \gqcs{L}_{gh} \arrow{r}{\varphi_{gh,g'h'}} \arrow[swap]{d}{\mu_{g,h}}
  & \gqcs{L}_{g'h'} \arrow{d}{\mu_{g',h'}}
  &\arrow[draw=none]{d}[pos=.4,description]{\text{\normalsize{and}}}
  &  \gqcs{L}_{\Frob{}(g)} \arrow{r}{\varphi_{\Frob{}(g),\Frob{}(g')}} \arrow[swap]{d}{\phi_{g}} & \gqcs{L}_{\Frob{}(g')} \arrow{d}{\phi_{g'}} \\
  \gqcs{L}_{g} \otimes \gqcs{L}_{h} \arrow{r}{\varphi_{g,g'}\otimes \varphi_{h,h'}}
  & \gqcs{L}_{g'} \otimes \gqcs{L}_{h'}
  & {}
  & \gqcs{L}_{g} \arrow{r}{\varphi_{g,g'}} & \gqcs{L}_{g'}
  \end{tikzcd}
 \end{equation}
both commute.

For each $x\in \pi_0(\bG)$, pick $g(x)\in \bG^x$
and set $\gqcs{E}_x \ceq \gqcs{L}_{g(x)}$.
Let $\phi_x : \gqcs{E}_{\Frob{}(x)} \to \gqcs{E}_x$
be the isomorphism of $\EE$-vector spaces obtained by composing
$\varphi_{g(\Frob{}(x)),\Frob{}(g(x))} : \gqcs{L}_{g(\Frob{}(x))} \to \gqcs{L}_{\Frob{}(g(x))}$
with $\phi_{g(x)} : \gqcs{L}_{\Frob{}(g(x))} \to \gqcs{L}_{g(x)}$.
For each pair $x,y\in \pi_0(\bG)$
let $\mu_{x,y} : \gqcs{E}_{x+y}\to \gqcs{E}_x\otimes \gqcs{E}_y$
be the isomorphism of $\EE$-vector spaces obtained by composing
$\varphi_{g(x+y),g(x)g(y)} : \gqcs{L}_{g(x+y)} \to \gqcs{L}_{g(x)g(y)}$
with $\mu_{g(x),g(y)} : \gqcs{L}_{g(x)g(y)} \to \gqcs{L}_{g(x)}\otimes \gqcs{L}_{g(y)}$.
Using \eqref{eq:qc}, it follows that \ref{qc.1}, \ref{qc.2} and \ref{qc.3} are satisfied for
$\qcs{E} \ceq (\gqcs{E}_x, \mu_{x,y}, \phi_x)$, thus defining a quasicharacter sheaf on $\pi_0(G)$.

The pullback $\pi_0^*(\qcs{E})$ of $\qcs{E}$ along $\pi_0 : G \to \pi_0(G)$ is constant
on geometric connected components, with stalks given by
$(\pi_0^* \qcs{E})_g = \qcs{E}_{x}$ for all $g\in \bG^x$.  Thus both $\pi_0^*\qcs{E}$ and $\qcs{L}$
are constant on geometric connected components of $G$.
The choices above define isomorphisms
$\gqcs{L}\vert_{\bG^x} \to  (\gqcs{E}_{x})_{\bG^x}$ for each $x\in \pi_0(\bG)$.
The resulting isomorphism $\gqcs{L} \to \pi_0^* \gqcs{E}$ satisfies \ref{QC.4},
thus defining an isomorphism $\qcs{L} \to \pi_0^* \qcs{E}$ in $\QC(G)$.
\end{proof}

To close this section, we note that the injectivity of $\pi_0^*$ will be proven in Corollary~\ref{cor:exact}.

\subsection{The function--sheaf dictionary for smooth, commutative group schemes}
\label{sec:snake}

We saw in Proposition~\ref{prop:functorialG} that Trace of Frobenius
$\TrFrob{G} : \QCiso{G} \to G(\Fq)^*$ is a functorial group homomorphism.
In fact, Trace of Frobenius is an isomorphism when $G$ is a $\QC$-scheme.

\begin{theorem}\label{thm:snake}
  If $G$ is a smooth, commutative group scheme over
  $\Fq$ then Trace of Frobenius
  \[
  \TrFrob{G} : \QCiso{G} \to G(\Fq)^*
  \]
  is a functorial surjection with kernel $\Hh^0(\Weil{}, \Hh^2(\pi_0({\bar G}), \EE))$ and a canonical section.
\end{theorem}

\begin{proof}
  Recall the short exact sequence \eqref{eq:pi0} in the category of $\QC$-schemes
  defining the component group scheme for $G$:
  \[
  \begin{tikzcd}
    1 \rar & G^0 \rar & G \rar & \pi_0(G) \rar & 0.
  \end{tikzcd}
  \]
  Let
  \[
  \begin{tikzcd}[row sep=30]
    {}& \ker \TrFrob{\pi_0(G)} \dar & \arrow[dashed]{d} \ker \TrFrob{G} & \ker \TrFrob{G^0} \dar & \\
    0 \rar & \QCiso{\pi_0(G)} \rar \dar{\TrFrob{\pi_0(G)}}
    & \QCiso{G} \rar \dar{\TrFrob{G}} & \QCiso{G^0} \rar \dar{\TrFrob{G^0}} & 0\\
    0 \rar & \pi_0(G)(\Fq)^* \rar \dar
    & \arrow[dashed]{d} G(\Fq)^* \rar & G^0(\Fq)^* \rar \dar & 0\\
    & \coker \TrFrob{\pi_0(G)} & \coker \TrFrob{G} &  \coker \TrFrob{G^0} &
  \end{tikzcd}
  \]
  be the commutative diagram of abelian groups obtained by applying
  Proposition~\ref{prop:pullback} to \eqref{eq:pi0}.

  The bottom row is dual to the sequence of
  abelian groups
  \begin{equation}\label{eq:pi0k}
  \begin{tikzcd}
    1 \rar & G^0(\Fq) \rar & G(\Fq) \rar & \pi_0(G)(\Fq) \rar & 0,
  \end{tikzcd}
  \end{equation}
  which is exact by Lang's theorem on connected algebraic groups over finite fields \cite{lang:56a}.
  Since $\EEx$ is divisible, $\Hom(\ - \ ,\EEx)$ is exact and thus the dual sequence of
  character groups is also exact.

  In order to apply the snake lemma, we need to show that the top row is exact at $\QCiso{G}$ and $\QCiso{G^0}$:
  the surjectivity of $\QCiso{G} \to \QCiso{G^0}$ follows from Proposition~\ref{prop:restriction}
  while exactness at $\QCiso{G}$ is given by Proposition~\ref{prop:middleexact}.
  Now Lemma~\ref{lem:G0alg-grp} and Proposition~\ref{prop:connected}
  imply that $\ker \TrFrob{G^0} =0$ and $\coker \TrFrob{G^0}=0$.
  Similarly, Proposition~\ref{prop:etale-iso} shows that $\ker \TrFrob{\pi_0(G)}=0$
  and $\coker \TrFrob{\pi_0(G)}=0$.
  We may thus apply the snake lemma and conclude that
  $\ker \TrFrob{G}$ and $\coker \TrFrob{G}$ both vanish.
\end{proof}


\begin{corollary}\label{cor:exact}
Suppose that
\[
0 \to A \to B \to C \to 0
\]
is an exact sequence of $\QC$-schemes and that $\Hh^1(\Fq, A) = 0$.
Then pullback yields an exact sequence
\[
0 \to \QCiso{C} \to \QCiso{B} \to \QCiso{A} \to 0
\]
of abelian groups.
\end{corollary}
\begin{proof}
Since $\Hh^1(k, A) = 0$, the sequence $0 \to A(\Fq) \to B(\Fq) \to C(\Fq) \to 0$ is exact;
applying $\Hom(-, \EEx)$ yields another exact sequence since $\EEx$ is divisible.  The result now follows
from Proposition~\ref{prop:pullback} and Theorem~\ref{thm:snake}.
\end{proof}

\begin{corollary} \label{cor:bounded-and-finite}
Suppose $G$ is a $\QC$-scheme and $\qcs{L}$ is a quasicharacter sheaf on $G$.  Then
\begin{enumerate}
\item $\qcs{L}$ is bounded if and only if $\trFrob{\qcs{L}}$ has bounded image,
\item $\qcs{L}$ is finite if and only if $\trFrob{\qcs{L}}$ has finite image.
\end{enumerate}
\end{corollary}
\begin{proof}
We first reduce to the case that $G$ is an \'etale $\QC$-scheme.  On the connected component
$G^0$, all quasicharacter sheaves are bounded and finite by Proposition~\ref{prop:connected}.
On the other hand, $G^0(\Fq)$ is finite by Lemma~\ref{lem:G0alg-grp} and thus Trace of Frobenius
has finite and therefore bounded image.  If $\chi \in G(\Fq)^*$ then there is some finite-image character $\chi_0$
with the same restriction to $G^0(\Fq)$ since $G^0(\Fq)$ is lies inside the torsion part of
the finitely generated abelian group $G(\Fq)$.  Therefore $\chi$ will be bounded (have finite image)
if and only if $\chi \cdot \chi_0^{-1}$ is bounded (has finite image).  But $\chi \cdot \chi_0^{-1}$ descends
to a character of $\pi_0(G)$.  A similar argument relates boundedness/finiteness of quasicharacter
sheaves on $G$ to those on $\pi_0(G)$, using the fact that the preimage constructed in the proof of
Proposition~\ref{prop:restriction} is a finite quasicharacter sheaf on $G$.
Using these equivalences, it is enough to prove the Corollary
for \'etale $\QC$-schemes.

In the case that $G$ is \'etale, one can easily see whether a quasicharacter sheaf is bounded or finite.
For $\qcs{L} \in \QC(G)$, recall that a global section is a particular choice of bases for the vector
spaces $\qcs{L}_x$.  The isomorphisms $\phi_x : \qcs{L}_{\Frob{}(x)} \to \qcs{L}_x$ are then specified
by a choice of scalars $b_x$.  By \cite{beilinson-bernstein-deligne:81a}*{Rem. 5.2.1},
$\qcs{L}$ will be bounded if and only if $b_x \in \Zlx$ for all $x$.  Moreover, $\qcs{L}$ will be
finite if and only if $b_x$ is a root of unity for all $x$.
Let $z$ be a cocycle representing the image of $\qcs{L}$ under the isomorphism
$\QCiso{G} \cong \Hh^1(\Weil{}, G(\bFq)^*)$.  Then $z(\Frob{})(x) = b_x$, and the
map $\Hh^1(\Weil{}, \Hom(G(\bFq), \Zlx)) \to \Hom(G(\Fq), \Zlx)$ is an isomorphism
since $\Zlx$ is divisible.  So bounded characters of $G(\Fq)$ correspond to quasicharacter sheaves
$\qcs{L}$ that descend to Weil sheaves on $G$.  The conditions on the values $b_x$ then
translate into analogous conditions on the image of Trace of Frobenius.
\end{proof}

With Corollary~\ref{cor:bounded-and-finite} in hand
we can see why neither finite quasicharacter sheaves nor bounded quasicharacter sheaves
are up to the task of geometrizing all characters of $G(\Fq)$ for general $\QC$-schemes $G$ since, in general, neither of the full subcategories
$\QCf(G) \subset \QCb(G) \subset \QC(G)$ are equivalences.
For example, consider the case when $G$ is the discrete \'etale group scheme $\ZZ$.
If $\chi : \ZZ \to \EEx$ is the quasicharacter of $G(\Fq)$ determined by $\chi(1) = \ell$
and if $\qcs{L}$ is a quasicharacter sheaf on $G$ in the isomorphism class
corresponding to $\chi$ under Theorem~\ref{thm:snake},
then $\qcs{L}$ is not a bounded quasicharacter sheaf since the image of $\chi$ is not
bounded.
Likewise, if $\chi : \ZZ \to \EEx$ is the quasicharacter of $G(\Fq)$ determined by $\chi(1) = 1+\ell$
and if $\qcs{L}$ is a quasicharacter sheaf on $G$ in the isomorphism class
corresponding to $\chi$ under Theorem~\ref{thm:snake},
then $\qcs{L}$ is a bounded quasicharacter sheaf but not a finite quasicharacter sheaf
since the image of $\chi$ is bounded but not finite.
In this latter case the stalk of $\qcs{L}$ at the component $1$ is defined
by a pro-\'etale cover rather than an \'etale cover.

\subsection{How to make invisible quasicharacter sheaves disappear}

Two approaches: subcategory and quotient category\todo{Under construction}

\bibliographystyle{amsalpha}
\bibliography{Biblio}

\end{document}
%sagemathcloud={"zoom_width":120}