%%------------------------------------------------------%%
%% This file should be the main file of your article.   %%

\documentclass[10pt]{amsart}
\usepackage[a4paper]{geometry} % see geometry.pdf n how to lay out the page. There's lots.
%\geometry{a4paper} % or letter or a5paper or ... etc
% \geometry{landscape} % rotated page geometry

% See the ``Article customise'' template for come common customisations

\title{From the function-sheaf dictionary to quasicharacters of $p$-adic tori}

\author{Clifton Cunningham}
\author{David Roe}

\address{Department of Mathematics and Statistics, University of Calgary, 2500 University Drive Northwest, Alberta, Calgary, Canada, {T2N~1N4}.}
\email{cunning@math.ucalgary.ca}
%\address{Pacific Institute for the Mathematical Sciences, University of Calgary, 2500 University Drive, {T2N~1N4}, Calgary, Canada, {\tt roed.math@gmail.com}}
\address{Department of Mathematics, University of British Columbia, 1984 Mathematics Road, Vancouver, British Columbia, Canada, {V6T~1Z2}.}
\email{roed.math@gmail.com}

\subjclass[2010]{14F05 (primary), 14L15 (secondary), 22E50 (tertiary)}

\keywords{function-sheaf dictionary, trace of Frobenius, geometrization, categorification, character sheaves,smooth commutative group schemes, finite fields, local fields, algebraic tori, Neron models, Greenberg functor, quasicharacters, quasicharacter sheaves.}



%%%%%%%%%%%%%%%%% PACKAGES %%%%%%%%%%%%%%%%%
%\usepackage{todonotes}
% crellerefs is a custom version of amsrefs complying with the Crelle style guidelines
%\usepackage{crellerefs}

\usepackage{amssymb,amsrefs}

\BibSpec{lecture}{%
  +{} {\PrintAuthors}				{author}
  +{,}{ }						{title}
  +{}{ \PrintDateB}				{date}
  +{.}{ Lecture delivered at \textit}	{conference}
  +{,}{ }						{address}
  +{.}{ }						{transition}
}

% We make volume bold rather than vol. #
\BibSpec{book}{%
    +{}  {\PrintPrimary}				{transition}
    +{,} { }							{title}
    +{.} { }							{part}
    +{:} { }							{subtitle}
    +{,} { \PrintEdition}					{edition}
    +{}  { \PrintEditorsB}				{editor}
    +{,} { \PrintTranslatorsC}			{translator}
    +{,} { \PrintContributions}			{contribution}
    +{,} { }							{series}
    +{} { \textbf}						{volume}
    +{,} { }							{publisher}
    +{,} { } 							{organization}
    +{,} { }							{address}
    +{,} { \PrintDateB}					{date}
    +{,} { }							{status}
    +{}  { \parenthesize}				{language}
    +{}  { \PrintTranslation}				{translation}
    +{;} { \PrintReprint}				{reprint}
    +{.} { }							{note}
    +{.} {}							{transition}
    +{}  {\SentenceSpace \PrintReviews}	{review}
}

% Fonts
\usepackage{mathrsfs}
% Enumitem
\usepackage{enumitem}
% TikZ
\usepackage{tikz}
\usetikzlibrary{shapes,arrows,calc,matrix}
\usepackage{tikz-cd}
% Hyperrefs
\usepackage{hyperref}

%%%%%%%%%%%%%%% THEOREM STYLES %%%%%%%%%%%%%%%
\theoremstyle{plain}
      \newtheorem{theorem}{Theorem}[section]
      \newtheorem*{theorem*}{Theorem}
      \newtheorem{proposition}[theorem]{Proposition}
      \newtheorem{lemma}[theorem]{Lemma}
      \newtheorem{corollary}[theorem]{Corollary}

      \theoremstyle{definition}
      \newtheorem{definition}[theorem]{Definition}

      \theoremstyle{remark}
      \newtheorem{remark}[theorem]{Remark}

%%%%%%%%%%%%%%%% TIKZ SETTINGS %%%%%%%%%%%%%%%%
\tikzset{every picture/.style={>=stealth},label/.style={font=\footnotesize}}

%%%%%%%%%%%%%%% RINGS AND GROUPS %%%%%%%%%%%%%%%
\newcommand{\FF}{{\mathbb{F}}}
\newcommand{\ZZ}{{\mathbb{Z}}}
\newcommand{\NN}{{\mathbb{N}}}
\newcommand{\CC}{{\mathbb{C}}}
\newcommand{\QQ}{{\mathbb{Q}}}
\newcommand{\RR}{{\mathbb{R}}}
\newcommand{\EE}{\mathbb{\bar Q}_\ell}
\newcommand{\OK}{\mathcal{O}_K}
\newcommand{\OL}{\mathcal{O}_L}
\newcommand{\OO}[1]{\mathcal{O}_{#1}}
\newcommand{\bFq}{\bar{k}}
\newcommand{\Fq}{k}
\newcommand{\Fqm}{k_m}
\newcommand{\EEx}{\EE^\times}
\newcommand{\Weil}[1]{\mathcal{W}_{#1}}
\newcommand{\m}{{\mathfrak{m}}}
%%%%%%%%%%%%%%% ALGEBRAIC GROUPS %%%%%%%%%%%%%%%
\newcommand{\Gm}[1]{\mathbb{G}_{\hskip-1pt\textbf{m},#1}}
\DeclareMathOperator{\GL}{GL}
\newcommand{\comp}{\Pi} % Component group
%%%%%%%%%%%%%%% NAMED OPERATORS %%%%%%%%%%%%%%%
\DeclareMathOperator{\Gal}{Gal}
\newcommand{\Frob}[1]{\operatorname{F}_{#1}}
\DeclareMathOperator{\Aut}{Aut}
\DeclareMathOperator{\Hom}{Hom}
\DeclareMathOperator{\ord}{ord}
\DeclareMathOperator{\coker}{coker}
\DeclareMathOperator{\Gr}{Gr}
\DeclareMathOperator{\Irrep}{Irrep}
\DeclareMathOperator{\id}{id}
\DeclareMathOperator{\Ext}{Ext}
\DeclareMathOperator{\Hh}{H}
\DeclareMathOperator{\Res}{Res}
\DeclareMathOperator{\Nm}{Nm}
\DeclareMathOperator{\trace}{Tr}
\DeclareMathOperator{\obj}{obj}
\DeclareMathOperator{\mor}{mor}
\DeclareMathOperator{\Lang}{Lang}
\DeclareMathOperator{\image}{im}
\DeclareMathOperator{\Loc}{Loc}
\DeclareMathOperator{\Tot}{Tot}
\newcommand{\gal}[1]{{\operatorname{Gal}\hskip-1pt\left( {\bar #1}/#1 \right)}}
\newcommand{\Spec}[1]{{\operatorname{Spec}\hskip-1pt( #1 )}}
%%%%%%%%%%%% MISCELLANEOUS OPERATORS %%%%%%%%%%%%
\newcommand{\sheafHom}{{\mathscr{H}\hskip-4pt{\it o}\hskip-2pt{\it m}}}
\newcommand{\abs}[1]{{\vert #1 \vert}}
\newcommand{\ceq}{{\, :=\, }}
\newcommand{\tq}{{\ \vert\ }}
\newcommand{\iso}{{\ \cong\ }}
\newcommand{\trFrob}[1]{t_{#1}}
\newcommand{\TrFrob}[1]{T_{#1}}
\newcommand{\lTrFrob}[1]{\TrFrob{#1}^2}
%% Limits
\newcommand{\invlim}[1]{\lim\limits_{\overleftarrow{#1}}}
\newcommand{\dirlim}[1]{\lim\limits_{\overrightarrow{#1}}}
\newcommand{\limit}[1]{\mathop{\textsc{lim}}\limits_{#1}}
\newcommand{\colimit}[1]{\mathop{\textsc{colim}}\limits_{#1}}
%% Fonts for quasicharacter sheaves
\newcommand{\cs}[1]{{\mathcal{#1}}}
\newcommand{\gcs}[1]{{\mathcal{\bar #1}}}
\newcommand{\dualgcs}[1]{\gcs{#1}^\dagger}
\newcommand{\dualcs}[1]{\cs{#1}^\dagger}
%% Categories
\newcommand{\CS}{{\mathcal{C\hskip-0.8pt S}}}
\newcommand{\lCS}{{\CS_\text{flb}}}
\newcommand{\lCSiso}[1]{\lCS(#1)_{/\text{iso}}}
\newcommand{\bCS}{{\CS_0}}
\newcommand{\CSiso}[1]{\CS(#1)_{/\text{iso}}}
\newcommand{\bCSiso}[1]{\bCS(#1)_{/\text{iso}}}
\newcommand{\catname}[1]{\normalfont{\textsf{#1}}}
\newcommand{\Sch}[1]{{\catname{Sch}_{/#1}}}
\newcommand{\QCS}{{\mathcal{QC\hskip-0.8pt S}}}
\newcommand{\QCSiso}[1]{\QCS(#1)_{/\text{iso}}}
%% Labeled items
\makeatletter
\newcommand{\labitem}[2]{
\def\@itemlabel{\textbf{#1}}
\item
\def\@currentlabel{#1}\label{#2}}
\makeatother
%% Shorthand for bars
\renewcommand{\bf}{\bar{f}}
\newcommand{\bg}{\bar{g}}
\newcommand{\bm}{\bar{m}}
\newcommand{\bG}{\bar{G}}
\newcommand{\bH}{\bar{H}}
\newcommand{\brho}{{\bar\rho}}
%% Spacing control
\newcommand{\tight}[3]{\hspace{-#1pt}{#2}\hspace{-#3pt}}
\newcommand{\GxG}{\text{$G \tight{1}{\times}{1} G$}}
\newcommand{\bGxG}{\text{$\bar{G} \tight{1}{\times}{1} \bar{G}$}}
\newcommand{\bfxf}{\text{$\bar{f} \tight{1}{\times}{1} \bar{f}$}}
\newcommand{\GxxG}{\text{$G \tight{1}{\times}{1} G$}}
\newcommand{\LxL}{\text{$\gcs{L} \tight{0}{\boxtimes}{0} \gcs{L}$}}
\newcommand{\ExE}{\text{$\cs{E}\tight{0}{\boxtimes}{0}\cs{E}$}}
\newcommand{\bExE}{\text{$\gcs{E}\tight{0}{\boxtimes}{0}\gcs{E}$}}
\newcommand{\AxA}{\text{$A \tight{1}{\times}{1} A$}}
\newcommand{\BxB}{\text{$B \tight{1}{\times}{1} B$}}
\newcommand{\GzxGz}{\text{$G^0 \tight{1}{\times}{1} G^0$}}
%% Hyphenation override
\hyphenation{quasi-character}

% Eliminate a font warning
%\SetSymbolFont{stmry}{bold}{U}{stmry}{m}{n}

% MSC footnote
\makeatletter
\newcommand\MSC[2][2010]{%
   \let\thefootnotetwo=\thefootnote%
   \renewcommand{\thefootnote}{}%
   \long\def\@makefntext##1{%
     \parindent\CrelleIndent
	 \hb@xt@0em{}##1}%
   \footnotetext{\textup{#2} \textit{Mathematics Subject Classification.}\enspace  #1}
   \let\thefootnote=\thefootnotetwo%
   \crellefntext%
}
\makeatother

%%%%%%%%%%%% BEGIN DOCUMENT %%%%%%%%%%%

\begin{document}


\begin{abstract}
We consider the category of character sheaves on smooth commutative group schemes $G$ over finite fields $k$ and expand the scope of the function-sheaf dictionary from connected commutative algebraic groups to this setting.
We show that the group of isomorphism classes of character sheaves is an extension of the group of characters of $G(k)$ by a cohomology group determined by the component group scheme of $G$.
As an application, we find a category of sheaves for which isomorphism classes correspond to quasi\-characters of algebraic tori, abelian varieties and unipotent wound groups over non-Archimedean local fields. 
In this geometrization of quasicharacters, our sheaves are local systems on
Greenberg transforms of locally finite type N\'eron models.
\end{abstract}

\maketitle

\section*{Introduction}

%This paper extends Deligne's function-sheaf dictionary from connected commutative algebraic groups to smooth commutative group schemes over finite fields and then provides an application to the geometrization and categorification of quasicharacters of algebraic tori, abelian varieties and unipotent wound groups over non-Archimedean local fields.

As Deligne explained in \cite{deligne:SGA4.5}*{Sommes trig.}, if $G$ is a connected commutative algebraic group over a finite field $k$, then the trace of Frobenius provides a bijection between the group $G(\Fq)^*$ of $\ell$-adic characters of $G(\Fq)$ and isomorphism classes of rank-one $\ell$-adic local systems $\mathcal{E}$ on $G$ for which 
\begin{equation}\label{introbox}
m^* \cs{E} \iso \cs{E} \boxtimes \cs{E},
\end{equation}
where $m : G\times G\to G$ is the multiplication map.
%
If one wishes to make a category from this class of local systems, one is led to consider morphisms $\cs{E} \to \cs{E}'$ of sheaves which are compatible with particular choices of \eqref{introbox} for $\cs{E}$ and $\cs{E'}$. 
{\it A priori}, the composition $\cs{E} \to \cs{E}' \to \cs{E}''$ of two such morphisms need not be compatible with the choices of \eqref{introbox} for $\cs{E}$ and $\cs{E}''$.
 However, for connected $G$ the isomorphism \eqref{introbox} is unique, if it exists, and there is no impediment to making the dictionary categorical.

If $G$ is a commutative algebraic group over $\Fq$ which is not connected, however, then the isomorphism \eqref{introbox} need not be unique. In order to track the choice of isomorphism, consider the category $\bCS(G)$ of pairs $(\cs{E},\mu_\cs{E})$ where $\cs{E}$ is a rank-one local system on $G$ and $\mu_\cs{E} : m^*\cs{E} \to \cs{E}\boxtimes\cs{E}$ is a chosen isomorphism of local systems on $G\times G$. 
In this case, the trace of Frobenius provides an epimorphism from isomorphism classes of objects in $\bCS(G)$ to characters of $G(\Fq)$, but the epimomorphism need not be injective; consequently,
every character of $G(\Fq)$ may be geometrized as a pair $(\cs{E},\mu_\cs{E})$ but perhaps not uniquely.
Indeed, it follows from a special case of the main result of this paper that the kernel of the trace of Frobenius $T_{G} : \bCS(G) \to G(\Fq)^*$ trivial if and only if the \'etale group scheme of connected components of $G$ is finite cyclic.
Nevertheless, the defect in the dictionary for characters of commutative algebraic groups over finite fields may be addressed with the following observation: if $(\cs{E},\mu_\cs{E})$ and $(\cs{E}',\mu_\cs{E}')$ determine the same character of $G(\Fq)$ then $\cs{E}\iso \cs{E}'$ as local systems on $G$.


In order to extend the dictionary from commutative algebraic groups over finite fields to smooth commutative group schemes $G$ over $\Fq$,
we replace the local system $\cs{E}$ on $G$ with a Weil local system while retaining the extra structure $\mu_\cs{E}$. 
In this way we are led to the category $\CS(G)$ of {\it character sheaves} on $G$ (\S\ref{ssec:category}):
objects in $\CS(G)$ are triples $(\gcs{L}, \mu,\phi)$, where $\gcs{L}$ is a rank-one local system on ${\bar G} := G\times_{\Spec{\Fq}} \Spec{\bFq}$
and $\phi : \Frob{G}^* \gcs{L}\to \gcs{L}$ and $\mu : {\bar m}^* \gcs{L} \iso \gcs{L} \boxtimes \gcs{L}$ are isomorphisms of sheaves satisfying certain compatibility conditions;
morphisms in $\CS(G)$ are then morphisms of Weil sheaves which are compatible with the extra structure.
This paper establishes the basic properties of $\CS(G)$ by finding the relation between character sheaves on $G$ and characters of $G(\Fq)$.

We begin our study of $\CS(G)$ by returning to the case when $G$ is a connected commutative algebraic group over $\Fq$, revisiting Deligne's function-sheaf dictionary (\S\ref{sec:category}).
We consider character sheaves that arise via base change to $\bFq$ from local systems on $G$ (\S\ref{ssec:descentG}) and
those that appear in a pushforward from a constant sheaf along a discrete isogeny $H \to G$ (\S\ref{ssec:discrete_isogenies}).  
While these constructions make sense even for non-connected $G$, in the connected case we show that every character sheaf can be described in both of these ways (\S\ref{ssec:connected}).  
We use this fact to prove that $\TrFrob{G}$ is an isomorphism for connected commutative algebraic groups $G$
and to determine the automorphism groups of character sheaves on such $G$. 
%These results are all well known.

Next, we consider character sheaves on \'etale commutative group schemes $G$ over $\Fq$ (\S\ref{sec:etale}).
\'Etale group schemes form a counterpoint to connected algebraic groups, since the component group of any smooth group scheme is an \'etale group scheme.
%etale group schemes are algebraic groups only when they are finite.
Our key tools for understanding the trace of Frobenius in the \'etale case 
are a reinterpretation of $\CS(G)$ in terms of stalks (\S\ref{ssec:stalks}) and 
%a relationship with 
 the Hochschild-Serre spectral sequence (\S\ref{ssec:E}) for $\Weil{} \ltimes \bG$, where $\Weil{} \subset \Gal(\Fq/\Fq)$ is the Weil group for $\Fq$.
We define (\S\ref{ssec:S}) an isomorphism $S_G$ from $\CSiso{G}$ to the second cohomology of the total space of the spectral sequence
\[
E_2^{p,q} \ceq \Hh^p(\Weil{}, \Hh^q(\bG, \EEx)) \Rightarrow \Hh^{p+q}(\Weil{} \ltimes \bG, \EEx).
\]
Paired with the short exact sequence
\[
  0 \to \Hh^0(\Weil{},\Hh^2({\bar G},\EEx)) \to \Hh^2(E^\bullet_G) \to \Hh^1(\Weil{},\Hh^1({\bar G},\EEx)) \to 0
\]
arising from the spectral sequence, the isomorphism $S_G$ allows us to show (\S\ref{ssec:SandT}) that 
the group homomorphism 
\[
\TrFrob{G}: \CSiso{G} \to G(\Fq)^*
\]
provided by the trace of Frobenius is surjective with kernel $\Hh^2(\bG,\EEx)^{\Weil{}}$.
The necessity of using Weil local systems on $G$ in the definition of $\CS(G)$ already appears here:  if one were to use local systems on $G$ instead, the group homomorphism $\TrFrob{G}$ would not then be surjective (\S \ref{ssec:bS}). 
One can show, using the K\"unneth formula, that $\Hh^2(\bG,\EEx)$ is non-trivial unless $\bG$ is finite cyclic, so the kernel of $\TrFrob{G}$ is non-trivial in general.
%The fact that $\TrFrob{G}$ may not be an isomorphism should not be too troubling: if quasicharacter sheaves $(\gcs{L}, \mu,\phi)$ and $(\gcs{L}', \mu',\phi')$ determine the same character of $G(\Fq)$ then $(\gcs{L},\phi)$ and $(\gcs{L}',\phi')$ are isomorphic as Weil sheaves.


%The appearance of $\Hh^2(\bG,\EEx)^{\Gal(\bFq/\Fq)}$ in these two cases is interesting: on the one hand, when $G$ is a connected commutative algebraic group over $\Fq$, isomorphism classes of quasicharacter sheaves on $G$ are measured by $\Hh^2(\bG,\EEx)^{\Gal(\bFq/\Fq)}$ and the trace of Frobenius gives an isomorphism $\TrFrob{G}: \CSiso{G} \to \Hom(G(\Fq),\EEx)$; on the other hand, if $G$ an etale commutative group scheme over $\Fq$, then $\Hh^2(\bG,\EEx)^{\Gal(\bFq/\Fq)}$ measures the kernel of $\TrFrob{G}: \CSiso{G} \to \Hom(G(\Fq),\EEx)$. 
%This observation makes plain the fact that  . . .
%Indeed, quasicharacter sheaves on \'etale commutative group schemes that are trivialized by discrete isogenies are invisible to the trace of Frobenius.

Having understood $\CS(G)$ in two extreme cases --
for connected commutative algebraic groups and for \'etale commutative group schemes -- we return to the case of smooth commutative group schemes (\S\ref{sec:main}) using the component group sequence
\[
0 \to G^0 \to G \to \pi_0(G) \to 0.
\]
Using pullbacks of character sheaves we obtain a diagram
\[
  \begin{tikzcd}[row sep=20, column sep=20]
    0 \rar & \CSiso{\pi_0(G)} \rar \dar{\TrFrob{\pi_0(G)}}
    & \CSiso{G} \rar \dar{\TrFrob{G}} & \CSiso{G^0} \rar \dar{\TrFrob{G^0}} & 0\\
    0 \rar & \pi_0(G)(\Fq)^* \rar & G(\Fq)^* \rar & G^0(\Fq)^* \rar & 0.
  \end{tikzcd}
\]
We show that the rows of this diagram are exact (\S\S\ref{ssec:restriction},\ref{ssec:component}), so we may apply the snake lemma to prove the main theorem of the paper,
\begin{theorem*}[{Thm. \ref{thm:snake}}]
If $G$ is a smooth commutative group scheme over $\Fq$ then the trace of Frobenius gives a short exact sequence
\[
\begin{tikzcd}
0 \arrow{r} & \Hh^2(\pi_0(\bG),\EEx)^{\Weil{}} \arrow{r} & \CSiso{G} \arrow{r}{\TrFrob{G}} & G(\Fq)^* \arrow{r} & 0.
\end{tikzcd}
\]
\end{theorem*}
\noindent
Thus, unless the component group scheme $\pi_0(\bG)$ is finite cyclic, category $\CS(G)$ contains non-trivial character sheaves lying in the kernel of the trace of Frobenius and therefore lost in translation to characters of $G(\Fq)$.
%These {\it invisible character sheaves} are parametrized by $\Hh^2(\pi_0(\bG),\EEx)^{\Weil{}}$.


We also illuminate the nature of the category $\CS(G)$ by showing that every morphism in this category is either an isomorphism or trivial, and by showing
\begin{theorem*}[{Thm. \ref{thm:autornaught}}]
If $G$ is a smooth commutative group scheme over $\Fq$ then
\[
\Aut(\cs{L}) \iso  \Hh^1(\pi_0(\bG), \EEx)^{\Weil{}}
\]
for all quasicharacter sheaves $\cs{L}$ on $G$.
\end{theorem*}

\medskip
\noindent\textbf{Application to quasicharacters of $p$-adic tori and abelian varieties.\ }
As indicated above, our interest in the function-sheaf dictionary for smooth commutative group schemes
over finite fields comes from an application to $p$-adic representation theory,
specifically to quasicharacters (\S\ref{ssec:quasicharacters}) of $p$-adic tori.
However, we found that our method of passing from $p$-adic tori to group schemes over $\Fq$ applies more generally to
any local field $K$ with finite residue field $\Fq$ and any commutative algebraic group $X_K$ that admits a N\'eron model $X$.
This class of groups includes abelian varieties and unipotent $K$-wound groups in addition to the tori we initially considered.

In this paper we show that quasicharacters of $X(K)$ are geometrized and categorified by quasicharacter sheaves on the
Greenberg transform $\Gr_R(X)$ of the N\'eron model $X$.
Although not locally of finite type, $\Gr_R(X)$ is a commutative group scheme over $\Fq$ and also a projective limit of smooth commutative group schemes $\Gr^R_n(X)$.
This structure allows us to adapt our work on character sheaves on smooth group schemes over finite fields to construct (\S\ref{ssec:CS_on_GN}) a category $\QCS(X)$
of quasicharacter sheaves for $X$, which are certain sheaves on $\Gr_R(X)\times_{\Spec{\Fq}} \Spec{\bFq}$, with extra structure.
The ability to generalize the function-sheaf dictionary to non-connected group schemes plays a crucial role in this application.

Having defined quasicharacter sheaves for groups over $K$ and $\Fq$, we consider how
these categories are related as $K$ and $\Fq$ vary.  We describe (\S\ref{ssec:basechange}) functors between categories
of quasicharacter sheaves that model restriction and norm homomorphisms of character groups $G(k')^* \to G(k)^*$ and $G(k)^* \to G(k')^*$,
and describe how quasicharacter sheaves behave under Weil restriction (\S\ref{ssec:wrK}).  We also give (\S\ref{ssec:transfer})
a categorical version of a result of Chai and Yu \cite{chai-yu:01a},
relating quasicharacter sheaves for tori over different local fields, even local fields
with different characteristic.

Specializing to the case that $X = T$ is the N\'eron model of an algebraic torus over $K$ (\S\ref{ssec:CS_tori}), 
we give a canonical short exact sequence 
\[
0 \to \Hh^2(X_*(T_K)_{\mathcal{I}_K},\EEx)^{\Weil{}} \to \QCSiso{T_K} \to \Hom_\text{}(T(K),\EEx) \to 0,
\]
where $X_*(T_K)_{\mathcal{I}_K}$ is the group of coinvariants of the cocharacter lattice $X_*(T_K)$ of the algebraic torus $T_K$ by the action of the inertia group $\mathcal{I}_K$ of $K$, and where $\Hom_\text{}(T(K),\EEx)$ denotes the group of quasicharacters of $T(K)$.
We further show that automorphism groups in $\CS(T)$ are given, for every quasicharacter sheaf $\cs{F}$ for $T$, by
\[
\Aut(\cs{F}) \iso (\check{T}_\ell)^{\Weil{K}},
\]
where $\Weil{K}$ is the Weil group for $K$ and $\check{T}_\ell$ is the $\ell$-adic dual torus to $T$.
%The appearance of the dual torus here is quite striking. 

By any measure, there are more quasicharacter sheaves for $T$ than quasicharacters of $T(K)$.  
In this regard, we are reminded of the work of David Vogan \cite{vogan:93a}, in which he finds a geometrization of complete Langlands parameters for $p$-adic groups 
and, in the process, is led to study the representations of all the pure rational forms of the $p$-adic group, simultaneously.
A similar phenomenon appears in recent work by Joseph Bernstein in which his geometric Ansatz leads to the study of certain sheaves on the stacky classifying space of the $p$-adic group, resulting in a category which appears to be tied to the representations of all the pure rational forms of the $p$-adic group \cite{bernstein:vogan_conference}.
Indeed, Bernstein has suggested to us that our category of quasicharacters for $T$ may be tied to quasicharacters of all the pure rational forms of $T$; however, we do not pursue that suggestion in this paper.

\medskip
\noindent\textbf{Relation to other work.\ }
The process of creating a category from the group of quasicharacters of a $p$-adic  torus informs our choice of the term quasicharacter sheaf in this paper.
We also want to situate our terminology in a historical context.
It is not uncommon to refer to local systems satisfying \eqref{introbox} on a connected, commutative algebraic group as character sheaves;
see for example, \cite{kamgarpour:09a}*{Intro}.
As we have indicated, quasicharacter sheaves evolved from this notion,
with an eye toward quasicharacters of $p$-adic groups.

The main use of the term character sheaf is of course due to Lusztig.
It is applied to certain perverse sheaves on connected reductive algebraic groups over algebraically closed fields in
\cite{lusztig:85a}*{Def.~2.10} and to certain perverse sheaves on reductive groups
over algebraically closed fields with finite cyclic component groups in the series of papers
beginning with \cite{lusztig:disconnected1}.
When commutative, such groups are extensions of finite cyclic groups by algebraic tori,
in which case it is not difficult to relate Frobenius-stable character sheaves to our character sheaves.
The new features that we have found pertaining to Weil sheaves and the non-triviality of $\Hh^2(\pi_0({\bar G}),\EEx)^{\Weil{}}$ do not arise in that context.

Also mention: \cite{suzuki-yoshida:12a},\cite{Suzuki:Neron} and \cite{Cunningham-Kamgarpour}.

\medskip
\noindent\textbf{Acknowledgements.\ } {Pramod Achar, Masoud Kamgarpour and Hadi Salmasian
allowed us to hijack much of a Research in Teams meeting at the Banff International Research Station into a discussion of
character sheaves.  We thank them for their kindness, knowledge and invaluable help.
We thank Takashi Suzuki for some very helpful observations and suggestions, especially related to our use of the Hochschild-Serre spectral sequence.
We thank Alessandra Bertrapelle and Cristian Gonz\'ales-Avil\'es for disabusing us of a misapprehension concerning the Greenberg realization functor and drawing our attention to their result on Weil restriction and the Greenberg transform.
We thank Joseph Bernstein for suggesting a connection between our category of quasicharacter sheaves for $p$-adic tori and quasicharacters of its pure rational forms.}

{Finally, we gratefully acknowledge the financial support of the Pacific Institute for the Mathematical Sciences and the National Science and Engineering Research Council (Canada), as well the hospitality of the Banff International Research Station during a Research in Teams program.}

\tableofcontents

\section{Character Sheaves on commutative group schemes and recollections for connected algebraic groups} \label{sec:category}

 
\subsection{Notations}\label{ssec:notation}

Throughout this paper, $G$ is a smooth commutative group scheme
over a finite field $\Fq$ and $m : G \times G\to G$ is its multiplication morphism.

We will make use of the short exact sequence of smooth group schemes defining the component group scheme for $G$:
\[
\begin{tikzcd}
0 \rar & G^0 \arrow{r}{\iota_0} & G \arrow{r}{\pi_0} & \pi_0(G) \rar & 0.
\end{tikzcd}
\]
Then $G^0$ is a connected algebraic group and $\pi_0(G)$ is an \'etale commutative group scheme.
In contrast to the case of algebraic varieties, the component group scheme $\pi_0(G)$ for $G$ need not be finite.
%Unless noted otherwise, we place absolutely no conditions on the \'etale group scheme $\pi_0(G)$.

It follows from the smoothness of $G$ that the structure morphism $G \to \Spec{\Fq}$ is locally of finite type.
If the structure morphism $G \to \Spec{\Fq}$ is \'etale, then $G$ is an etale group scheme; this does not imply that $\pi_0(G)$ is finite.
If the structure morphism $G \to \Spec{\Fq}$ is of finite type, then $G$ is an algebraic group; in this case, $\pi_0(G)$ is finite. 
Indeed, a smooth commutative group scheme $G$ is an algebraic group if and only if the etale group scheme $\pi_0(G)$ is finite.



 
We fix an algebraic closure $\bFq$ of $\Fq$ and write $\bG$ for the
smooth commutative group scheme $G \times_{\Spec{\Fq}} \Spec{\bFq}$ over $\bFq$
obtained by base change from $k$. The multiplication morphism for $\bG$ will be denoted by $\bm$.

Let $\Frob{}$ denote the geometric Frobenius element in $\Gal(\bFq/\Fq)$ as
well as the corresponding automorphism of $\Spec{\bFq}$. The Weil group
$\Weil{}\subset \Gal(\bFq/\Fq)$ is the subgroup generated by $\Frob{}$.
Let $\Frob{G} \ceq \id_{G} \times \Frob{}$ be the Frobenius automorphism of $\bG = G \times_{\Spec{\Fq}} \Spec{\bFq}$.

We fix a prime $\ell$, invertible in $\Fq$.
We will work with constructible $\ell$-adic sheaves \citelist{\cite{deligne:80a}*{\S 1.1} \cite{SGA5}*{Expos\'es V, VI}}
on schemes locally of finite type over $\Fq$, employing the standard formalism.
We also make extensive use of the external tensor product of $\ell$-adic sheaves,
defined as follows: if $\mathcal{F}$ and $\mathcal{G}$ are constructible $\ell$-adic
sheaves on schemes $X$ and $Y$ and $p_X : X\times Y\to X$ and $p_Y : X\times Y \to Y$
are the projections, then $\mathcal{F}\boxtimes \mathcal{G} \ceq p_X^* \mathcal{F} \otimes p_Y^*\mathcal{G}$.

For any commutative group $A$, we will write $A^*$ for the dual group $\Hom(A, \EEx)$.

\subsection{Character sheaves on commutative group schemes over finite fields}\label{ssec:category}

\begin{definition}\label{def:CS}
A \emph{character sheaf on $G$} is a triple
$\cs{L}\ceq (\gcs{L},\mu,\phi)$ where:
\begin{enumerate}
\labitem{(CS.1)}{CS.1} $\gcs{L}$ is a rank-one $\ell$-adic local system on $\bG$, by which we mean a constructible $\ell$-adic sheaf on $\bG$, {\it lisse} on each connected component of $\bG$, whose stalks are one-dimensional $\EE$-vector spaces;
\labitem{(CS.2)}{CS.2} $\mu: \bm^* \gcs{L} \to \LxL$ is an isomorphism of
sheaves on $\bGxG$ such that the following diagram commutes,
  where $m_3 \ceq m\circ (m\tight{1}{\times}{2}\id) = m\circ (\id\tight{2}{\times}{1} m)$;
  \[
  \begin{tikzcd}[row sep=30]
  \bm_3^*\gcs{L} \arrow{rr}{(\bm \tight{1}{\times}{2} \id)^*\mu} \arrow[swap]{d}{(\id \tight{2}{\times}{1} \bm)^*\mu}
    &&  \bm^*\gcs{L} \boxtimes \gcs{L} \dar{\mu \tight{0}{\boxtimes}{1} \id} \\
    \gcs{L} \boxtimes \bm^* \gcs{L} \arrow{rr}{\id \boxtimes \mu}
    &&  \gcs{L} \tight{0}{\boxtimes}{0} \LxL
  \end{tikzcd}
  \]
\labitem{(CS.3)}{CS.3} $\phi : \Frob{G}^* \gcs{L} \to \gcs{L}$ is an
  isomorphism of constructible $\ell$-adic sheaves on $\bG$ compatible with
  $\mu$ in the sense that the following diagram commutes.
  \[
  \begin{tikzcd}[row sep=20]
  \Frob{\GxxG}^* \bm^* \gcs{L} \arrow{rr}{\Frob{\GxxG}^*\mu}
    && \Frob{\GxxG}^*(\LxL)\\
    \arrow[equal]{u} \bm^*  \Frob{G}^* \gcs{L} \arrow[swap]{d}{\bm^* \phi}
    && \Frob{G}^*\gcs{L}\boxtimes \Frob{G}^*\gcs{L} \dar{\phi\boxtimes \phi} \arrow[equal]{u} \\
    \bm^*\gcs{L} \arrow{rr}{\mu}
    && \LxL
  \end{tikzcd}
  \]
\end{enumerate}
\end{definition}

Morphisms in the category of character sheaves on $G$, denoted by $\CS(G)$, are defined in the natural way:
\begin{enumerate}
\labitem{(CS.4)}{CS.4} if $\cs{L} = (\gcs{L},\mu,\phi)$ and
  $\cs{L'} = (\gcs{L'},\mu',\phi')$ are character sheaves on $G$ then
  a morphism $\rho : \cs{L} \to \cs{L}'$ is a map $\brho : \gcs{L} \to \gcs{L'}$
  of constructible $\ell$-adic sheaves on $\bG$ such that the following diagrams both commute.
  \[
  \begin{tikzcd}[column sep=40]
  \Frob{G}^* \gcs{L} \rar{\Frob{G}^* \brho} \arrow[swap]{d}{\phi} & \Frob{G}^* \gcs{L'} \dar{\phi'}
  & & \arrow[swap]{d}{\mu} \bm^* \gcs{L} \rar{\bm^* \brho} & \bm^* \gcs{L'} \dar{\mu'} \\
  \gcs{L} \rar{\brho} & \gcs{L'}
  & {} & \LxL \rar{\tight{1}{\rho\boxtimes \rho}{1}} & \gcs{L'} \tight{0}{\boxtimes}{0} \gcs{L'}
  \end{tikzcd}
  \]
\end{enumerate}

Category $\CS(G)$ is a rigid monoidal category
\cite{etingof:09a}*{\S1.10} under the tensor product
$\cs{L} \otimes \cs{L'}$ defined by $(\gcs{L}\otimes\gcs{L'}, \mu\otimes\mu', \phi\otimes \phi')$,
with duals given by applying the sheaf hom functor
$\sheafHom(\ - \ ,\EE)$.
This rigid monoidal category structure for $\CS(G)$ gives the set $\CSiso{G}$
of isomorphism classes in $\CS(G)$ the structure of a group.

\begin{remark}
The category of character sheaves on $G$ is not abelian since it is not closed under direct sums; 
thus $\CS(G)$ is not a tensor category in the sense of \cite{deligne:02a}*{0.1}.  
We suspect that requiring that $\mu$ be injective rather than
an isomorphism and dropping the condition that the stalks be one-dimensional would yield an abelian category.
\end{remark}

We will describe the group $\CSiso{G}$ in Theorem~\ref{thm:snake}
and the sets $\Hom(\cs{L},\cs{L}')$ in Theorem~\ref{thm:autornaught}; in this way we provide a complete description of the category $\CS(G)$.
%
In the meantime, we close this subsection with an elementary observation about $\Hom(\cs{L},\cs{L}')$.
We call a morphism of sheaves \emph{trivial} if it is zero on every stalk.

\begin{lemma}\label{lem:autornaught}
Let $G$ be a smooth commutative group scheme over $\Fq$.
If $\cs{L}$ and $\cs{L}'$ are character sheaves on $G$, then
every $\rho\in \Hom(\cs{L},\cs{L}')$ is either trivial or an isomorphism. 
\end{lemma}

\begin{proof}
Suppose $\rho \in \Hom(\cs{L},\cs{L}')$.
We prove the lemma by considering the linear transformations $\brho_{\bar g} : \gcs{L}_{\bar g} \to \gcs{L}_{\bar g}$ at the stalks above geometric points ${\bar g}$ on $G$ and showing that, either each $\brho_{\bar g}$ trivial or each $\brho_{\bar g}$ is an isomorphism.
(This idea is expanded upon in Section~\ref{ssec:stalks}.)
Let ${\bar e}$ be the geometric point above the identity $e$ for $G$ determined by our choice of algebraic closure $\bFq$ of $\Fq$.
If $\brho_{\bar e} = 0$ then the second diagram in \ref{CS.4} implies that $\brho_{\bar g} = 0$ for all ${\bar g}$, in which case $\rho$ is trivial.
On the other hand, if $\brho_{\bar e}$ is nontrivial then the second diagram in \ref{CS.4}  implies that $\brho_{\bar g}$ is nontrivial for all ${\bar g}$ and thus an isomorphism, since the stalk of character sheaves are one-dimensional; in this case $\rho$ is an isomorphism.
\end{proof}


\subsection{Trace of Frobenius}\label{ssec:Frob}

In this section we introduce two tools which will help us understand isomorphism classes of objects in $\CS(G)$: the map $\CSiso{G} \to G(k)^*$ given by trace of Frobenius and the pullback functor $\CS(G) \to \CS(H)$ associated to a morphism $H \to G$.

Let $(\gcs{L},\phi)$ be a Weil sheaf on $G$. Every $g\in G(\Fq)$
determines a geometric point $\bg$ fixed by $\Frob{G}$. 
Together with the canonical isomorphism $(\Frob{G}^*\gcs{L})_{\bg} \iso  \gcs{L}_{\Frob{G}(\bg)}$, the automorphism $\phi$ determines an automorphism $\phi_{\bg}$ of the $\EE$-vector space $\gcs{L}_{\bg}$.
Let $\trace(\phi_{\bg};\gcs{L}_{\bg})$ be the trace of $\phi_{\bg} \in \Aut_{\EE}(\gcs{L}_{\bg})$ and let $\trFrob{(\gcs{L},\phi)} : G(\Fq)\to \EE$ be the function defined by 
\begin{equation}\label{trWeil}
\trFrob{(\gcs{L},\phi)}(g) \ceq \trace(\phi_{\bg};\gcs{L}_{\bg}),
\end{equation}
commonly called the {\em trace of Frobenius of $(\gcs{L},\phi)$}.
Note that if $(\gcs{L},\phi) \iso (\gcs{L'},\phi')$ as Weil sheaves, 
then $\trFrob{(\gcs{L},\phi)} = \trFrob{(\gcs{L'},\phi')}$ as functions on $G(\Fq)$.

Now suppose $\cs{L} = (\gcs{L},\mu,\phi)$ is a quasicharacter sheaf on $G$.
Then the isomorphism $\bm^* \gcs{L} \iso \gcs{L} \boxtimes\gcs{L}$ and the diagram of
\ref{CS.3} guarantee
that the function $\trFrob{(\gcs{L},\phi)} : G(\Fq)\to \EEx$ is a group homomorphism, which we will also denote by $\trFrob{\cs{L}}$.  Moreover,
this homomorphism depends only on the isomorphism class of $\cs{L}$, so we obtain a map
\begin{align*}
\TrFrob{G} : \CSiso{G} &\to G(\Fq)^*, \\
\cs{L} &\mapsto \trFrob{\cs{L}}.
\end{align*}
Since tensor products on the stalks of $\cs{L}$ induce pointwise multiplication on the trace of Frobenius, $\TrFrob{G}$ is a group homomorphism.  We devote the remainder of this section to
studying the behavior of the trace of Frobenius under morphisms of group schemes.  These simple but important
properties of pullbacks and external products will be used in the proofs of many results in this paper.

\begin{lemma}\label{lem:pullback}
  If $f : H\to G$ is a morphism of smooth commutative group schemes over $\Fq$, then
  \begin{align*}
  f^* : \CS(G) &\to \CS(H) \\
  (\gcs{L},\mu,\phi) &\mapsto (\bf^*\gcs{L},(\bfxf)^*\mu,\bf^*F)
  \end{align*}
  defines a monoidal functor dual to $f \colon H(\Fq) \to G(\Fq)$ in the sense that
  \[
  \begin{tikzcd}[row sep=20, column sep=30]
   \CSiso{G} \rar{f^*} \arrow[swap]{d}{\TrFrob{G}} & \CSiso{H} \dar{\TrFrob{H}} \\
   G(\Fq)^* \rar & H(\Fq)^*
  \end{tikzcd}
  \]
  is a commutative diagram of groups.  Moreover, $(f\circ g)^* = g^* \circ f^*$.
\end{lemma}
\begin{proof}
  Let $\cs{L}$ be a character sheaf on $G$. 
  Pullback by $\bf$ takes rank-one local systems to rank-one local systems.
  To see that $(\bfxf)^* \mu$ satisfies \ref{CS.2},
  apply the functor $(\bfxf)^*$
  to \ref{CS.2} for $\cs{L}$ and use the canonical isomorphism
  $(\bfxf)^*(\LxL) \iso \bf^*\gcs{L} \tight{-3}{\boxtimes}{-3} \bf^*\gcs{L}$.
  To show that $f^*\cs{L}$ satisfies
  \ref{CS.3}, apply the same functor to \ref{CS.3} for $\cs{L}$.
  Since $f$ is a morphism of group schemes defined over $\Fq$
  it provides isomorphisms $(\bfxf)^*\Frob{\GxxG}^* \iso \Frob{\GxxG}^* (\bfxf)^*$
  and $(\bfxf)^* \bm^*\iso \bm^* \bf^*$ between functors of constructible sheaves.

  Applying $\bf^*$ and $\bf^* \tight{1}{\times}{1}\bf^*$ to \ref{CS.4} defines the action
  of $f^*$ on morphisms of character sheaves; arguing as above shows that $f^*$ is
  a functor from $\CS(G)$ to $\CS(H)$.  Since tensor products commute with pullback in schemes,
  $f^* : \CS(G) \to \CS(H)$ is a monoidal functor.
  The diagram relating $f^* : \CS(G) \to \CS(H)$, $f^* : G(k)^* \to H(k)^*$ and trace of Frobenius
  commutes by \cite{laumon:87a}*{1.1.1.2}, where the ambient
 finite type hypothesis can be replaced by locally of finite type.

  Finally, the fact that $(f\circ g)^* = g^* \circ f^*$ follows from the analogous
  statements about the pullback functor on $\ell$-adic constructible sheaves.
\end{proof}

If $G_1$ and $G_2$ are smooth commutative group schemes over $\Fq$ then characters of $(G_1 \times G_2)(\Fq)$ all take the form $\chi_1\otimes \chi_2$ for characters $\chi_1$ of $G_1(\Fq)$ and $\chi_2$ of $G_2(\Fq)$. 
The next lemma shows that character sheaves on $G$ enjoy an analogous property.

\begin{lemma}\label{lem:product}
If $G_1$ and $G_2$ are smooth commutative group schemes over $\Fq$ then the following diagram commutes.
\[
\begin{tikzcd}[column sep=60]
\arrow{d}{\TrFrob{G_1} \times \TrFrob{G_2}} \CSiso{G_1}\times \CSiso{G_2} \arrow{r}{(\cs{L}_1,\cs{L}_2)\mapsto \cs{L}_1\boxtimes \cs{L}_2}
& \arrow{d}{\TrFrob{G_1\times G_2}} \CSiso{G_1\times G_2}\\
(G_1)(\Fq)^*\times (G_2)(\Fq)^* \arrow{r}{(\chi_1,\chi_2)\mapsto \chi_1\otimes \chi_2}  & (G_1\times G_2)(\Fq)^*
\end{tikzcd}
\]
Moreover, every character sheaf on $G_1\times G_2$ is isomorphic to $\cs{L}_1\boxtimes\cs{L}_2$ for some character sheaves $\cs{L}_1$ on $G_1$ and $\cs{L}_2$ on $G_2$.
\end{lemma}
\begin{proof}
The only non-trivial part is the last claim, so we will only address that point here.
%
Set $G \ceq G_1\times G_2$
and write $e_1$ and $e_2$ for the identity elements of $G_1$ and $G_2$.
Define $f : G\to G\times G$ by $f(g_1,g_2) \ceq (g_1,e_2,e_1,g_2)$.
Observe that $m\circ f = \id_G$.
Let $p_1$, $p_2$ be the projection morphisms pictured below:
\[
\begin{tikzcd}
G & \arrow[swap]{l}{p_1} G\times G \arrow{r}{p_2} & G.
\end{tikzcd}
\]
Let $r_1$ and $r_2$ be the projection morphisms pictured below,
with sections $q_1$ and $q_2$, also morphisms of group schemes:
\[
\begin{tikzcd}
G_1  \arrow[swap, bend right]{r}{q_1} &
\arrow[swap, bend right]{l}{r_1} G_1\times G_2 \arrow[bend left]{r}{r_2} &
\arrow[bend left]{l}{q_2} G_2.
\end{tikzcd}
\]
Observe that $p_1\circ f = q_1 \circ r_1$ and $p_2 \circ f = q_2\circ r_2$.
%
Now, let $\cs{L} \ceq (\gcs{L},\mu,\phi)$ be a character sheaf on $G$
and set $\cs{L}_1 \ceq q_1^* \cs{L}$ and $\cs{L}_2 \ceq q_2^* \cs{L}$.
By Lemma~\ref{lem:pullback}, $\cs{L}_1$ is a character sheaf on $G_1$
and $\cs{L}_2$ is a character sheaf on $G_2$.
We will obtain an isomorphism $\cs{L} \iso  \cs{L}_1\boxtimes \cs{L}_2$.

Applying the functor $f^*$ to the isomorphism $\mu$ yields
\begin{equation}\label{eq:fm}
f^*\mu : f^* m^* \gcs{L} \to f^*(\gcs{L}\boxtimes \gcs{L}) .
\end{equation}
We have already seen that $m\circ f = \id_G$, so $f^* m^* \gcs{L} = \gcs{L}$.  
Since $f^*p_1^*\gcs{L} = r_1^* q_1^* \gcs{L} = r_1^* \gcs{L}_1$ and $f^*p_2^*\gcs{L} = r_2^* q_2^* \gcs{L} = r_2^*\gcs{L}_2$,
we have 
\[
f^*(\gcs{L}\boxtimes \gcs{L})  = f^*p_1^*\gcs{L}\otimes f^* p_2^*\gcs{L} = \gcs{L}_1\boxtimes \gcs{L}_2.
\]
It follows that \eqref{eq:fm} gives an isomorphism $\gcs{L} \to  \gcs{L}_1\boxtimes \gcs{L}_2$.
It is routine to show that this morphism satisfies
\ref{CS.4}, as it applies here,
from which it follows that we have exhibited an isomorphism
$\cs{L} \to \cs{L}_1\boxtimes \cs{L}_2$ of characters sheaves on $G\times G$.
\end{proof}

Using these results on pullbacks and products, we may prove a naturality property of $\TrFrob{G}$.

\begin{proposition}\label{prop:functorialG}
The homomorphism $\TrFrob{G} : \CSiso{G} \to G(\Fq)^*$ defines a natural transformation
between the two contravariant additive functors
\begin{align*}
F_1 : G &\mapsto \CSiso{G} \\
F_2 : G &\mapsto G(\Fq)^*
\end{align*}
from the category of smooth commutative group schemes over $\Fq$ to the category of commutative groups.
\end{proposition}

\begin{proof}
The first part of Lemma~\ref{lem:pullback} shows that $F_1$ is a functor,
while the second part shows that Trace of Frobenius is a natural transformation
$T: F_1 \to F_2$. When further combined with Lemma~\ref{lem:product},
we see that $F_1$ is an additive functor and $T: F_1 \to F_2$ is a natural
transformation between additive functors,
concluding the proof of Proposition~\ref{prop:functorialG}.
\end{proof}

%\subsection{character sheaves on connected commutative algebraic groups} \label{sec:disc-isog}

\subsection{Descent}\label{ssec:descentG}

Before turning our attention to character sheaves on connected commutative algebraic groups in Section~\ref{ssec:connected}, we treat two special classes of character sheaves on arbitrary smooth commutative group schemes $G$: those that descend from local systems on $\bG$ to local systems on $G$ (\S \ref{ssec:descentG}); and those that are defined by discrete isogenies onto $G$ (\S \ref{ssec:discrete_isogenies}). The latter will play a role in Sections~\ref{ssec:connected} and \ref{ssec:restriction}.

In this section we consider a category of sheaves on $G$ obtained by
replacing the Weil sheaf $(\gcs{L}, \phi)$ on $\bG$ in the definition of a character sheaf with an $\ell$-adic local system on $G$ itself.


\begin{definition}
Let $\bCS(G)$ be the category of pairs $(\cs{E},\mu_\cs{E})$
where $\cs{E}$ is an $\ell$-adic local system on $G$ of rank-one,
equipped with an isomorphism $\mu_\cs{E} : m^* \cs{E} \to \cs{E} \boxtimes \cs{E}$
satisfying the analogue of \ref{CS.2} on $G$;
morphisms in $\bCS(G)$ are defined as in the second part of
\ref{CS.4}. 
\end{definition}

We put a rigid monoidal structure on $\bCS(G)$ in the same way as for $\CS(G)$.

\begin{proposition}\label{prop:BG}
Extension of scalars defines a full and faithful functor
\[
B_G : \bCS(G) \to \CS(G).
\]
\end{proposition}

\begin{proof}
 Suppose $(\cs{E},\mu_\cs{E})$ in an object of $\bCS(G)$.
 Let $b_G : {\bar G} \to G$ be the pullback of $\Spec{\bFq} \to \Spec{\Fq}$ along $G\to \Spec{\Fq}$.
 Set $\gcs{L} = b_G^* \cs{E}$. 
 The functor $b_G^*$ takes local systems on $G$ to local systems on $\bG$.
 The local system $\gcs{L}$ comes equipped with an isomorphism
 $\phi: \Frob{G}^* \gcs{L} \to \gcs{L}$.
 The resulting functor from local systems on $G$ to Weil local systems on $\bG$, given on objects by $\cs{E} \mapsto (\gcs{L},\phi)$, 
 is full and faithful; see \cite{deligne-katz:SGA7.2}*{Expos\'e XIII} and \cite{beilinson-bernstein-deligne:81a}*{Prop. 5.1.2}.
 The isomorphism $\mu \ceq b_{G\times G}^*\mu_\cs{E}$ satisfies \ref{CS.2}
 for $\gcs{L}$ and $\phi$ is compatible with $\mu$ in the sense of \ref{CS.3}.
 This construction defines the functor $B_G : \bCS(G) \to \CS(G)$ given on objects by $(\cs{E},\mu_\cs{E}) \mapsto (\gcs{L},\mu, \phi)$, as defined here. 
 Because morphisms in $\bCS(G)$ and $\CS(G)$ are morphisms of local systems on $G$ and $\bG$, respectively, satisfying condition~\ref{CS.4}, this functor is also full and faithful.
\end{proof}

We will say that a character sheaf $\cs{L} \in \CS(G)$ \emph{descends to $G$} if it is isomorphic to some $B_G(\cs{E}, \mu_\cs{E})$.

\begin{remark}\label{rem:descent}
In fact, it is not difficult to recognize character sheaves that descend to $G$: they are exactly those character sheaves $\cs{L} = (\gcs{L},\mu,\phi)$ for which the action of $W$ on $\gcs{L}$ given by $\phi$ extends to a continuous action of $\Gal(\bFq/\Fq)$ on $\gcs{L}$; see\cite{deligne-katz:SGA7.2}*{Expos\'e XIII, Rappel 1.1.3} for example. 
\end{remark}

\subsection{Discrete isogenies}\label{ssec:discrete_isogenies}

A finite, \'etale, surjective morphism $H\to G$ of smooth group schemes over $\Fq$ for which the action of $\Gal(\bFq/\Fq)$ on the kernel is trivial is called a {\it discrete isogeny}.

\begin{proposition}\label{prop:finite}
Let $f: H \to G$ be a discrete isogeny and let $A$ be the kernel of $f$.
Let $V$ be a $1$-dimensional representation of $A$ 
equipped with an isomorphism $V\to V\otimes V$.
Let $\psi : A \to \EEx$ be the character of $V$.
Then $(f_! V_H)_\psi$ (the $\psi$-isotypic component of $f_!V_H$) is an object of $\bCS(G)$.
\end{proposition}

\begin{proof}
Let $f$, $A$, $V$ and $\psi$ be as above and set $\cs{E} = (f_! V_H)_\psi$.
Since $A$ is abelian, $\cs{E}$ is an $\ell$-adic local system on $G$ of rank one.
We must show that $\cs{E}$ comes equipped with an isomorphism $\mu_\cs{E} : m^* \cs{E} \to \cs{E}\boxtimes\cs{E}$.
To do this we use \'etale descent to see that pullback along $f$ gives an equivalence between $\ell$-adic local systems on $G$ and $A$-equivariant local systems on $H$ {\it cf.\ } \cite{bernstein-luntz:equivariant_sheaves}*{Prop 8.1.1}. 
In particular, $f^*\cs{E}$ is the $A$-equivariant constant sheaf $V$ on $H$ with character $\psi$.
Since $f$ is a morphism of group schemes, the functor $f^*$ defines $\mu_\cs{E} : m^*\cs{E} \to \cs{E}\boxtimes\cs{E}$
from the isomorphism $m^*\psi \iso \psi \boxtimes\psi$ determined by $V\to V\otimes V$.
\end{proof}

We remark that, since $V$ is $1$-dimensional, the choice of $V \to V\otimes V$ is exactly the choice of an isomorphism $V\iso \EE$.

\subsection{The function-sheaf dictionary for connected algebraic groups over finite fields}\label{ssec:connected}

In general, $\bCS(G)$ is an essentially
proper subcategory of $\CS(G)$. 
However, if $G$ is connected and of finite type, the categories are equivalent, as we now show.

\begin{lemma}\label{lem:bounded_connected}
If $G$ is a connected commutative algebraic group over $\Fq$ then 
\[
B_G : \bCS(G) \to \CS(G)
\]
 is an equivalence of categories.
\end{lemma}

\begin{proof}
Choose any $\Fq$-rational point $g$ on $G$ and let $\bg$ be the geometric point on $G$ lying above $g$.
Recall that the \emph{Weil group} of $G$, which we will denote by $\Weil{}(G,\bg)$, is a subgroup of the \'etale
fundamental group defined by the following diagram:
\[
 \begin{tikzcd}
 1 \rar & \ar[equal]{d} \pi_1(\bG, \bg) \rar & \Weil{}(G,\bg) \rar \dar[hook] & \Weil{} \rar \dar[hook] & 1 \\
 1 \rar &  \pi_1(\bG, \bg) \rar & \pi_1(G,\bg) \rar & \Gal(\bFq/\Fq) \rar & 1.
 \end{tikzcd}
\]
The $\Fq$-rational point $g$ under the geometric point $\bg$ determines a splitting
$\Weil{}\to \Weil{}(G,\bg)$ of $\Weil{}(G,\bg)\to \Weil{}$.
%
  Since $G$ is connected, the geometric point $\bg$ determines
  an equivalence between the category of $\ell$-adic Weil local systems on $G$ and
  $\ell$-adic representations of $\Weil{}(G,\bg)$ \cite{deligne:80a}*{1.1.12}.
  
  Now let $(\gcs{L},\mu,\phi)$ be a character sheaf on $G$
  and let $\lambda : \Weil{}(G, \bg) \to \EEx$ be the character determined by $(\gcs{L},\phi)$.
  Composing with the splitting $\Weil{} \to \Weil{}(G,\bg)$ yields an $\ell$-adic character
  $\lambda_g : \Weil{} \to \EEx$, which is the same as the Trace of Frobenius defined in Section~\ref{ssec:Frob}, for every $\Fq$ rational point $g$ on $G$:
  $
  \lambda_g(\Frob{}) =  \trFrob{\cs{L}}(g).
  $
%
  On the other hand, we have already seen that $\trFrob{\cs{L}} : G(\Fq) \to \EEx$
  is a group homomorphism. 
  Since $G$ is an algebraic group over $\Fq$, $G(\Fq)$ is finite.
  Therefore $\trFrob{\cs{L}}(g) = \lambda_g(\Frob{})$ is a root of unity
  for every $g\in G(\Fq)$.  Since $\Weil{}$ is generated by
  $\Frob{}$ and $\lambda_g : \Weil{} \to \EEx$ is
  a character, it follows that the image of $\lambda_g$ is a finite group.
  Thus, $\lambda_g$ extends to an $\ell$-adic character of $\Gal(\bFq/\Fq)$,
  which we will also denote $\lambda_g$.
%
  We may now lift the $\ell$-adic character $\lambda_g : \Gal(\bFq/\Fq) \to \EEx$
  to an $\ell$-adic character $\pi_1(G,\bg) \to \EEx$ using the canonical topological group homomorphism
  $\pi_1(G,\bg) \to \Gal(\bFq/\Fq)$. 
 % 
  The $\Fq$ rational point $g$ also
  determines an equivalence between the category of $\ell$-adic
  representations of $\pi_1(G,\bg)$ and $\ell$-adic local systems on $G$. Let
  $\cs{E}$ be a local system on $G$ in the isomorphism class
  determined by this $\ell$-adic character of $\pi_1(G,\bg)$.
  Then $b_G^*\cs{E} \iso \gcs{L}$.
  
  Since $b_{G\times G}^*$ is full and faithful (again, see
\cite{deligne-katz:SGA7.2}*{Expos\'e XIII} or \cite{beilinson-bernstein-deligne:81a}*{Prop. 5.1.2}),
 \[
  b_{G\times G}^* : \Hom(m^*\cs{E},\cs{E}\boxtimes\cs{E}) \to \Hom({\bar m}^*\gcs{L},\gcs{L}\boxtimes\gcs{L})
 \]
  is a bijection
  (hom taken in the categories on constructible $\ell$-adic sheaves on
  $G\times G$ and ${\bar G}\times {\bar G}$ respectively,
  in which $\ell$-adic local systems sit as full subcategories).
  Let $\mu_\cs{E} : m^*\cs{E} \to \cs{E}\boxtimes\cs{E}$ be the isomorphism matching
  $\mu : {\bar m}^*\gcs{L} \to \gcs{L}\boxtimes\gcs{L}$,
  the latter appearing in the definition of $\cs{L}$.
  Then, as in Section~\ref{ssec:descentG}, $(\cs{E},\mu_\cs{E})$ is an object of $\bCS(G)$
  and $\cs{L} \ceq (\gcs{L},\mu,\phi)$ is isomorphic to $(b_G^*\cs{E},b_{G\times G}^*\mu_\cs{E})$ in $\CS(G)$.
  Thus, the full and faithful functor $B_G : \bCS(G) \to \CS(G)$ from Section~\ref{ssec:descentG}
  is also essentially surjective, hence an equivalence.
\end{proof}

Using this equivalence of categories, we may give a good description of $\CS(G)$ when $G$ is connected and finite type.

\begin{proposition}\label{prop:connected}
 If $G$ is a connected, commutative algebraic group over $\Fq$ then:
 \begin{enumerate}
 \labitem{(1)}{c1} $\TrFrob{G} : \CSiso{G} \to G(\Fq)^*$ is an isomorphism of groups;
 \labitem{(2)}{c2} every character sheaf on $G$ is isomorphic to one defined by a discrete isogeny;
 \labitem{(3)}{c3} $\Aut(\cs{L}) = 1$, for all character sheaves $\cs{L}$ on $G$.
 \end{enumerate}
 \end{proposition}
\begin{proof}
In Lemma~\ref{lem:bounded_connected}, we saw that every character sheaf $\cs{L}$ on $\bG$ descends to $G$; let $\cs{E}$ be an object of $\bCS(G)$ for which $B_G(\cs{E}) \iso \cs{L}$.
 Since the functor $B_G : \bCS(G) \to \CS(G)$ is full and faithful, $\Aut(\cs{L}) = \Aut(\cs{E})$.
From here, Deligne's function-sheaf dictionary for connected commutative algebraic groups over finite fields, as in \cite{deligne:SGA4.5}*{Sommes trig.} or \cite{laumon:87a}*{1.1.3}, gives us all we need, as we briefly recall. %\todo{Fix citelist} %\citelist{\cite{deligne:SGA4.5}*{Sommes trig.},\cite{laumon:87a}*{1.1.3}}

As in the proof of Proposition~\ref{prop:finite}, use \'etale descent to see that pullback by the Lang isogeny $\Lang : G\to G$ defines an equivalence of categories between local systems on $G$ and $G(\Fq)$-equivariant local systems on $G$. 
Under this equivalence, local systems $\cs{E}$ on $G$ arising from objects in $\bCS(G)$ are matched with $G(\Fq)$-equivariant constant local systems of rank-one on $G$, and therefore with one-dimensional representations of $G(\Fq)$. 
In the same way, pullback along the isogeny $\Lang\times\Lang : G\times G\to G\times G$ matches the extra structure $\mu_\cs{E} : m^*\cs{E} \to \cs{E}\boxtimes\cs{E}$ with an isomorphism $m^*V \to V\boxtimes V$ of one-dimensional representations of $G(\Fq)\times G(\Fq)$, which is exactly an isomorphism $V \to V\otimes V$ of one-dimensional representations, which is exactly the choice of an isomorphism $V\iso \EE$.
We see that $\bCS(G)$ is equivalent to the category of characters of $G(\Fq)$.
Let $\bg$ be a geometric point above $g \in G(\Fq)$.  If $\cs{E}$ matches $\psi : G(\Fq)\to \EEx$ under this equivalence, a simple calculation on stalks reveals that the action of Frobenius on $\cs{E}_{\bar g}$ is multiplication by $\psi(g)^{-1}$.
In other words, for every $\cs{E}$ in $\bCS(G)$, the trace of $\Lang^*\cs{E}$ is $\trFrob{\cs{E}}^{-1}$ as a representation of $G(\Fq)$, proving parts \ref{c1} and \ref{c2}.

To finish the proof of part \ref{c3}, suppose $\Lang^*\cs{E} = V$ with isomorphism $V \to V\otimes V$.  Observe that the equivalence above establishes a bijection between $\Aut(\cs{E})$ and the group of automorphisms of $\rho : V\to V$ for which 
\[
\begin{tikzcd}
\arrow{d}{} V \arrow{r}{\rho} & V\arrow{d}{}\\
V\otimes V \arrow{r}{\rho\otimes \rho} & V\otimes V
\end{tikzcd}
\]
commutes. 
Since the only such isomorphism $\rho$ is $\id_V$, it follows that $\Aut(\cs{E}) = 1$, completing the proof.
\end{proof}

We have just seen that, for a connected commutative algebraic group $G$ over $\Fq$, the category of character sheaves on $G$ is equivalent to the category of one-dimensional representations $V$ of $G(\Fq)$ equipped with an isomorphism $V\iso \EE$, and therefore equivalent to the category of characters $\psi$ of $G(\Fq)$.
We have also just seen that if the character of $\Lang^*\cs{E}$ is $\psi$ then the canonical isomorphism $m^*\psi \iso \psi \boxtimes \psi$ determines the isomorphism $\mu_\cs{E} : \cs{E} \to \cs{E}\boxtimes\cs{E}$.
This fact leads (back) to a perspective on the function-sheaf dictionary common in the literature in which one considers one-dimensional local systems $\cs{E}$ on $G$ for which \emph{there exists} an isomorphism $m^*\cs{E} \iso \cs{E} \boxtimes\cs{E}$ \cite{kamgarpour:09a}*{Intro}.
As a slight variation, one may also consider one-dimensional local systems $\gcs{L}$ on $\bG$ for which \emph{there exists} an isomorphism $\Frob{G}^*\gcs{L} \iso \gcs{L}$ and an isomorphism $\bm^*\gcs{L} \iso \gcs{L} \boxtimes\gcs{L}$.

Although the category $\CS(G)$ of character sheaves on $G$ specializes to $\bCS(G)$ when $G$ is connected and of finite type, this description is \emph{not} sufficient when extending the dictionary to smooth commutative group schemes, as we will see already in Section~\ref{sec:etale}.
In particular, for a given $\gcs{L}$ and $\phi$ there may be many $\mu$ so that $(\gcs{L},\mu,\phi)$ is a character sheaf.  For \'etale $G$, Proposition~\ref{prop:etale-iso} shows that $\Hh^2(\bG,\EEx)^{\Weil{}}$ measures the possibilities for $\mu$.  We will see in Section~\ref{sec:main} that $\Hh^2(\pi_0(\bG),\EEx)^{\Weil{}}$ plays an analogous role for general smooth commutative group schemes $G$.

\subsection{Examples}

\section{Character sheaves on \'etale commutative group schemes} \label{sec:etale}

In this section we find a complete characterization of the category of character sheaves on \'etale commutative group schemes over finite fields.

\subsection{Stalks of character sheaves}\label{ssec:stalks}

The equivalence $G \mapsto G(\bFq)$,
from the category of \'etale commutative group schemes over $\Fq$ to the category of commutative groups equipped
with a continuous action of $\Gal(\bFq/\Fq)$,
provides the following simple description of character sheaves.
%
A character sheaf $\cs{L}$ on an \'etale commutative group scheme $G$ over $\Fq$ is:
\begin{enumerate}
 \labitem{(cs.1)}{cs.1} an indexed set of one-dimensional
  $\EE$-vector spaces $\gcs{L}_x$, as $x$ runs over
  $G(\bFq)$;

 \labitem{(cs.2)}{cs.2} an indexed set of isomorphisms
  $\mu_{x,y} : \gcs{L}_{x+y} \xrightarrow{\iso} \gcs{L}_{x} \otimes\gcs{L}_{y}$,
  for all $x,y \in G(\bFq)$, such that
  \[
   \begin{tikzcd}[row sep=40]
    \gcs{L}_{x+y+z} \arrow{rr}{\mu_{x+y,z}} \arrow[swap]{d}{\mu_{x,y+z}}
    && \gcs{L}_{x+y}\otimes\gcs{L}_{z} \dar{\mu_{x,y} \tight{0.5}{\otimes}{1} \id} \\
    \gcs{L}_{x} \otimes\gcs{L}_{y+z} \arrow{rr}{\id \otimes\mu_{y,z}}
    && \gcs{L}_{x} \otimes\gcs{L}_{y} \otimes\gcs{L}_{z}
   \end{tikzcd}
  \]
  commutes, for all $x,y,z\in G(\bFq)$; and
 \labitem{(cs.3)}{cs.3} an indexed set of isomorphisms $\phi_{x} : \gcs{L}_{\Frob{}(x)} \to \gcs{L}_x$
  such that
  \[
   \begin{tikzcd}[row sep=40]
    \gcs{L}_{\Frob{}(x)+\Frob{}(y)} \arrow[swap]{d}{\phi_{x+y}} \arrow{rr}{\mu_{\Frob{}(x),\Frob{}(y)}}
    && \gcs{L}_{\Frob{}(x)}\otimes\gcs{L}_{\Frob{}(y)} \dar{\phi_x \tight{0}{\otimes}{0} \phi_y} \\
    \gcs{L}_{x+y} \arrow{rr}{\mu_{x,y}}
    && \gcs{L}_x \otimes\gcs{L}_y
   \end{tikzcd}
  \]
  commutes, for all $x,y\in G(\bFq)$.
\end{enumerate}
Under this equivalence, a morphism $\rho : \cs{L} \to \cs{L'}$ of character sheaves on $G$ is given by
\begin{enumerate}
 \labitem{(cs.4)}{cs.4} an indexed set $\brho_x : \gcs{L}_x \to \gcs{L'}_x$
  of linear transformations such that
  \[
   \begin{tikzcd}[column sep=40]
    \arrow[swap]{d}{\phi_x} \gcs{L}_{\Frob{}(x)} \rar{\brho_{\Frob{}(x)}} & \gcs{L'}_{\Frob{}(x)} \dar{\phi_x'}
    &\arrow[draw=none]{d}[pos=.4,description]{\text{\normalsize{and}}}
    & \arrow[swap]{d}{\mu_{x,y}} \gcs{L}_{x+y} \rar{\brho_{x+y}} & \gcs{L'}_{x+y} \dar{\mu'_{x,y}} \\
    \gcs{L}_x \rar{\brho_x} & \gcs{L'}_x
    & {} & \gcs{L}_x\otimes\gcs{L}_y \rar{\brho_x\otimes\brho_y} & \gcs{L'}_x \otimes\gcs{L'}_y
   \end{tikzcd}
  \]
  both commute, for all $x, y \in G(\bFq)$.
\end{enumerate}

We will see that $\TrFrob{G} : \CSiso{G} \to G(\Fq)^*$ may not provide complete
information about isomorphism classes of character sheaves on $G$ when $G$ is not a connected algebraic group.
Our main tool for understanding this phenomenon
is a group homomorphism $S_G: \CSiso{G} \to \Hh^2(E^\bullet_G)$ defined in Section~\ref{ssec:S}, for which the next two sections are preparation.

\subsection{A spectral sequence}\label{ssec:E}

Let $G$ be a smooth commutative group scheme over $\Fq$.
The zeroth page of the Hochschild-Serre spectral sequence
is a double complex $E^{\bullet, \bullet}$ defined by
\[
E^{i,j} = C^i(\Weil{}, C^j(G(\bFq), \EEx));
\]
see \cite{vakil:Algebraic_Geometry}*{\S 1.7}, expanding on \cite{weibel:Homological_Algebra}*{Ch 5 and \S 7.5}.
The standard derivative on cochains yields two derivatives,
\begin{align*}
d_G &: E^{i,j} \to E^{i,j+1} \quad \mbox{and} \\
d_{\Weil{}} &: E^{i,j} \to E^{i+1,j};
\end{align*}
we use the first as the derivative $d_0$ on the zeroth page, and the second to induce $d_1$.
Combining them also yields a derivative $d = d_G + (-1)^j d_{\Weil{}}$ on the total complex
\[
E^n_G = \bigoplus_{i+j=n} E^{i,j}.
\]
The machinery of spectral sequences gives us a sequence of pages $E_r^{i,j}$, converging to a page $E_{\infty}^{i,j}$. We summarize the key properties of this spectral sequence in the following proposition.

\begin{proposition} In the spectral sequence defined above,
\begin{enumerate}
\item the second page is given by $E_2^{i,j} = \Hh^i(\Weil{}, \Hh^j(G(\bFq), \EEx))$,
\item there is an isomorphism $\Hh^n(\Weil{} \ltimes G(\bFq), \EEx) \cong \Hh^n(E_G^\bullet)$, and
\item there is a filtration $\Hh^n(\Weil{} \ltimes G(\bFq), \EEx) = F_n \supset \cdots \supset F_{-1} = 0$, with $F_i / F_{i-1} \cong E_{\infty}^{i, n-i}$.
\end{enumerate}
\end{proposition}

Moreover, since $\Weil{} \cong \ZZ$ has cohomological dimension $1$, $E_2^{i,j} = 0$ for $i > 1$ and the sequence degenerates at the second page: $E_{\infty}^{i,j} = E_2^{i,j}$. We obtain the following corollary:

\begin{corollary}\label{cor:spectral_ses}
There is a short exact sequence
 \[
    0 \to
    \Hh^0(\Weil{},\Hh^2({\bar G},\EEx)) \to
    \Hh^2(E^\bullet_G) \to
    \Hh^1(\Weil{},\Hh^1({\bar G},\EEx)) \to
    0.
 \]
\end{corollary}

This sequence will play a key role in understanding the kernel of $\TrFrob{G}$, as described in the next few sections.
For this application, we need a good understanding of these maps to and from the total complex.

\begin{proposition} \label{prop:ses_desc}
Consider the short exact sequence in Corollary~\ref{cor:spectral_ses}.
\begin{enumerate}
\item Every class $[\alpha\oplus\beta\oplus\gamma] \in \Hh^2(E^\bullet_G)$ is cohomologous to one with $\gamma=0$.
\item The map $\Hh^2(E^\bullet_G) \to \Hh^1(\Weil{},\Hh^1({\bar G},\EEx))$ is given by $[\alpha\oplus\beta\oplus 0] \mapsto [\beta]$.
\item Suppose $a \in Z^2({\bar G}, \EEx)$ represents a class in $\Hh^2({\bar G},\EEx)$ fixed by Frobenius.
The map $\Hh^0(\Weil{},\Hh^2({\bar G},\EEx)) \to \Hh^2(E^\bullet_G)$ is given by $[a] \mapsto [a \oplus 0 \oplus 0]$.
\end{enumerate}
\end{proposition}
\begin{proof}
Since $\Hh^2(\Weil{}, C^0({\bar G}, \EEx)) = 0$, we may find a $\gamma_1 \in C^1(\Weil{}, C^0({\bar G}, \EEx))$ with $d_{\Weil{}}\gamma_1 = \gamma$.
Subtracting $d \gamma_1$ from $\alpha\oplus\beta\oplus\gamma$, we may assume that $\gamma = 0$.

The latter two claims follow from tracing through the definition of latter pages in the spectral sequence.
\end{proof}

\subsection{From character sheaves to the total complex}\label{ssec:S}

Let $G$ be a smooth commutative group scheme over $\Fq$.
In this section we define a group homomorphism
\[
S_G : \CSiso{G} \to \Hh^2(E^\bullet_G).
\] 
Let $\cs{L} = (\gcs{L},\mu,\phi)$ be a character sheaf on $G$.
For each geometric point $x\in {\bar G}$, choose a basis $\{ v_x \}$ for $\gcs{L}_x$.
Through this choice, $\cs{L}$ determines functions
\begin{align*}
a : {\bar G}\times {\bar G} &\to \EEx & b : {\bar G} &\to \EEx \\
\mu_{x,y}(v_{x+y}) &= a(x,y) v_x \otimes v_y & \phi_x(v_{\Frob{G}(x)}) &= b(x) v_x.
\end{align*}
Condition~\ref{CS.2} implies that
\begin{equation}\label{2-cocyle}
a(x+y,z) a(x,y) = a(x,y+z) a(y,z)
\end{equation}
for all $x,y,z\in {\bar G}$, so $a \in Z^2({\bar G},\EEx)$.  Similarly, condition~\ref{CS.3} gives
\begin{equation}\label{nohom}
\frac{a(\Frob{G}(x),\Frob{G}(y))}{a(x,y)} =  \frac{b(x+y)}{b(x) b(y)}
\end{equation}
for $x, y \in {\bar G}$.
Let $\alpha \in C^0(\Weil{},C^2({\bar G},\EEx)$ be the $0$-cochain corresponding to $a$ and let $\beta\in C^1(\Weil{},C^1({\bar G},\EEx)$ be the cocycle such that $\beta(\Frob{})$ is $b$.  We will write both $\alpha$ and $\beta$ additively.
Then
\[
d_G\alpha =0, \qquad\qquad
d_{\Weil{}} \alpha = d_{G} \beta,\qquad\qquad
d_{\Weil{}} \beta =0;
\]
in other words,
\[\alpha\oplus \beta \in Z^2(E^\bullet_G).\]
Although the cocycle $\alpha\oplus \beta$ is not well defined by $\cs{L}$, its class in $\Hh^2(E^\bullet_G)$ is.
To see this, let $\{ v'_x \in \gcs{L}_x^\times \tq x \in {\bar G}\}$ be another choice and let $\alpha'\oplus \beta' \in Z^2(E^\bullet_G)$ be defined by $\cs{L}$ and this choice, as above.
Now let $\delta \in C^0(\Weil{},C^1({\bar G},\EEx))$ correspond to the function $d : {\bar G}\to \EEx$ defined by $v'_x = d(x) v_x$.
Chasing through \ref{CS.2} and \ref{CS.3}, we find
\[
\alpha'\oplus\beta' = \alpha\oplus\beta + d\delta,
\]
so the class $[\alpha\oplus\beta]$ of $\alpha\oplus\beta$ in $\Hh^2(E^\bullet_G)$ is independent of the choice made above. It is also easy to see that $[\alpha\oplus\beta] = [\alpha_0\oplus\beta_0]$ when $\cs{L} \iso \cs{L}_0$,
which concludes the definition of the function
\begin{align*}
S_G : \CSiso{G} &\to \Hh^2(E^\bullet_G)\\
[\cs{L}] &\mapsto [\alpha\oplus \beta].
\end{align*}
It is also easy to see that $[\alpha_1\oplus\beta_1] + [\alpha_2\oplus\beta_2] = [\alpha_3\oplus\beta_3]$ when $\cs{L}_3 = \cs{L}_1\otimes \cs{L}_2$, so $S_G$ is a group homomorphism.

%\begin{remark}
%Although we will mainly use the group homomorphism $S_G : \CSiso{G}\to \Hh^2(E^\bullet_G)$ when $G$ is an \'etale commutative group scheme, the function is defined for all smooth commutative group schemes $G$;
%however, the next result does not extend to all smooth commutative group schemes.
%\end{remark}

\begin{proposition}\label{prop:SGiso}
If $G$ is \'etale then $S_G:  \CSiso{G} \to \Hh^2(E^\bullet_G)$ is an isomorphism.
\end{proposition}
\begin{proof}
Suppose $[\cs{L}] \in \CSiso{G}$ with $S_G([\cs{L}]) = [\alpha \oplus \beta] = 0$,
so that $\alpha \oplus \beta = d\sigma$ for some $\sigma \in C^0(\Weil{},C^1({\bar G},\EEx) = C^1({\bar G},\EEx)$.
For each $x\in {\bar G}$, define $\sigma_x : \gcs{L}_x \to \EE$ by $\sigma_x : v_x \mapsto \sigma(x)$.
Then the indexed set of isomorphisms $\{ \sigma_x : \gcs{L}_x \to \EE \tq x\in {\bar G}\}$
defines an isomorphism $\cs{L} \to (\EE)_G$.
Since $\cs{L} = 0 \in \CSiso{G}$, $S_G$ is injective.

To see that $S_G$ is surjective, begin with $\alpha\oplus\beta\oplus 0 \in Z^2(E^\bullet_G)$.
Since $d_{\Weil{}} \beta = 0$, we may define $a = \alpha \in C^2({\bar G},\EEx)$ and
$b = \beta(\Frob{}) \in C^1({\bar G},\EEx)$, which are related to $\alpha$ and $\beta$ as above.
Set $\gcs{L}_x = \EE$, define $\mu_{x,y} : \gcs{L}_{x+y} \to \gcs{L}_x\otimes\gcs{L}_y$
by $\mu_{x,y}(1) = a(x,y) (1\otimes 1)$ and $\phi_x : \gcs{L}_{\Frob{G}(x)} \to \gcs{L}_x$ by $\phi_x(1)= b(x)$.
Then \ref{CS.1} holds since $d_G \alpha =0$ and \ref{CS.2} holds since $d_{\Weil{}}\alpha =d_G \beta$.
Tracing the construction backward, we have defined a character sheaf $\cs{L}$ on $G$ with
$S_G(\cs{L}) = [\alpha\oplus\beta\oplus 0]$, showing that $S_G$ is surjective.
\end{proof}


\subsection{Objects in the \'etale case}\label{ssec:SandT}

In this section we fit the group homomorphisms $\TrFrob{G}$ and $S_G$ into a commutative diagram, determining the kernel and cokernel of the homomorphism $\TrFrob{G} : \CSiso{G} \to G(k)^*$ when $G$ is an \'etale commutative group scheme over $\Fq$.

%
We begin with a simple, general result relating duals, invariants and coinvariants.

\begin{lemma} \label{lem:dual-inv}
Let $X$ be an abelian group equipped with an action of $\Weil{}$.  Then
\begin{align*}
 (X^*)_{\Weil{}} &\to (X^{\Weil{}})^* \\
 [f] &\mapsto f|_{X^{\Weil{}}}
\end{align*}
is an isomorphism.
\end{lemma}

\begin{proof}
We can describe $X^{\Weil{}}$ as the kernel of the map $X \xrightarrow{\Frob{}-1} X$;
let $Y = (\Frob{}-1)X$ be the augmentation ideal.  Dualizing the sequence
\[
 0 \to X^{\Weil{}} \to X \to Y \to 0
\]
yields
\[
 0 \to Y^* \to X^* \to (X^{\Weil{}})^* \to \Ext^1_\ZZ(Y, \EEx).
\]
Since $\Ext^1_\ZZ(-,\EEx)$ vanishes, we get a natural isomorphism from the cokernel of $Y^* \xrightarrow{\Frob{}-1} X^*$ to $(X^{\Weil{}})^*$.
\end{proof}

\begin{proposition}\label{prop:sur_etale}
If $G$ is \'etale, then $\TrFrob{G} : \CSiso{G} \to G(\Fq)^*$ is surjective
and split.
\end{proposition}
\begin{proof}
Pick $\chi \in G(\Fq)^*$. 
Let $[\beta]\in \Hh^1(\Weil{},\bG^*)$ be the class corresponding to $\chi$ under Lemma~\ref{lem:dual-inv}.
Every representative cocycle $\beta \in Z^1(\Weil{},\bG^*)$ determines a homomorphism $\beta(\Frob{}) : G(\bFq)\to \EEx$ such that $\beta(\Frob{})\vert_{G(\Fq)} = \chi$.
Set $\gcs{L}_x = \EE$ for every $x\in G(\bFq)$.
Define $\mu_{x,y} : \gcs{L}_{x+y} \to \gcs{L}_x\otimes \gcs{L}_y$ by $\mu_{x,y}(1) = 1 \otimes 1$ and
$\phi_{x} : \gcs{L}_{\Frob{}(x)} \to \gcs{L}_x$ by $\phi_{x}(1) = \beta(\Frob{})(x)$.
Since $\beta(\Frob{}) : G(\bFq) \to \EEx$ is a group homomorphism,
condition \eqref{nohom} is satisfied with $a =1$.
So $\cs{L} = (\gcs{L}, \mu, \phi)$
is a character sheaf with $\trFrob{\cs{L}} = \chi$.
This shows that $\TrFrob{G}$ is surjective.

Now let $\beta' \in Z^1(\Weil{},\bG^*)$ be another representative for $[\beta]$
so $\beta-\beta' = d_{\Weil{}} \delta$ for some $\delta \in C^0(\Weil{},\bG^*)$ defining $d \in \Hom(G(\bFq),\EEx)$.
Let $\cs{L}'$ be the character sheaf on $G$ defined by $\beta'$, as above.
For each $x\in G(\bFq)$, define $\brho_x :\cs{L}_x\to \cs{L}'_x$ by $\brho_x(1) = d(x)$.
The collection of isomorphisms $\{ \brho_x \tq x\in G(\bFq)\}$ satisfies condition~\ref{CS.4}, so it defines a morphism $\rho : \cs{L}\to \cs{L}'$, which is clearly an isomorphism. 
%
We have now defined a section of $\TrFrob{G}$. 

Now suppose $\chi_1, \chi_2 \in G(\Fq)^*$. Pick cocycles $\beta_1,\beta_2\in Z^1(\Weil{},\bG^*)$ and construct character sheaves $\cs{L}_1$ and $\cs{L}_2$ on $G$ as above. Since $\cs{L}_1\otimes \cs{L}_2$ is exactly the character sheaf built from the cocycle $\beta_1\cdot \beta_2$, and since $\trFrob{\cs{L}_1\otimes \cs{L}_2} = \trFrob{\cs{L}_1}\cdot \trFrob{\cs{L}_2}$, the section of $\TrFrob{G}$ defined here is a homomorphism.
\end{proof}

\begin{proposition} \label{prop:etale-iso}
 If $G$ is \'etale then the map $S_G : \CSiso{G}\to \Hh^2(E^\bullet_G)$ induces an isomorphism of split short exact sequences
\[
\begin{tikzcd}
 0 \arrow{r} & \ker \TrFrob{G} \arrow{d} \arrow{r} & \CSiso{G}\arrow{d}{S_G} \arrow{r}{\TrFrob{G}} \arrow{r} & G(\Fq)^* \arrow{d} \arrow{r} & 0\\
  0 \arrow{r} & \Hh^0(\Weil{},\Hh^2({\bar G},\EEx)) \arrow{r} & \Hh^2(E^\bullet_G) \arrow{r} & \Hh^1(\Weil{},\Hh^1({\bar G},\EEx)) \arrow{r} & 0.
 \end{tikzcd}
 \]
\end{proposition}
\begin{proof}
This result follows from Propositions~\ref{prop:ses_desc}, \ref{prop:SGiso} and \ref{prop:sur_etale}.
\end{proof}

\subsection{Descent, revisited}\label{ssec:bS}

In this section we justify the appearance of Weil sheaves in Definition~\ref{def:CS}.

\begin{proposition}\label{prop:bounded-etale}
Let $G$ be a commutative etale group scheme over $\Fq$.
Then the image of $\bCS(G)$ under $\TrFrob{G} : \CS(G) \to G(\Fq)^*$ is $\Hom_\text{grp}(G(\Fq),\mathbb{\bar Z}_{\ell}^\times)$.
\end{proposition}

\begin{proof}
Objects in $\bCS(G)$ may be described by a small modification to the technique used in Sections~\ref{ssec:E} and \ref{ssec:S}. 
Set $F^{i,j} \ceq C^i_{\text{cts}} (\Gal(\bFq/\Fq), C^j(G(\bFq), \EEx))$.
Then the results of Section~\ref{ssec:E} adapt to give a short exact sequence in continuous Galois cohomology
 \[
    0 \to
    \Hh^0(\Fq,\Hh^2({\bar G},\EEx)) \to
    \Hh^2(F^\bullet_G) \to
    \Hh^1(\Fq,\Hh^1({\bar G},\EEx)) \to
    0,
 \]
for which the maps are given by the analogues of Proposition~\ref{prop:ses_desc}.
Moreover, using \cite{deligne-katz:SGA7.2}*{Expos\'e XIII, Rappel 1.1.3} we see that Proposition~\ref{prop:SGiso} adapts to provide an isomorphism $\bCSiso{G}\to \Hh^2(F^\bullet_G)$ compatible with $\bCS(G) \to \CS(G)$ and with the
the canonical map of exact sequences 
 \[
\begin{tikzcd}
    0 \arrow{r} &
    \Hh^0(\Fq,\Hh^2({\bar G},\EEx)) \arrow{r} \arrow{d} &
    \Hh^2(F^\bullet_G) \arrow{r} \arrow{d} &
    \Hh^1(\Fq,\Hh^1({\bar G},\EEx)) \arrow{r} \arrow{d} &
    0
\\
    0 \arrow{r} & 
    \Hh^0(\Weil{},\Hh^2({\bar G},\EEx)) \arrow{r} & 
    \Hh^2(E^\bullet_G) \arrow{r} & 
    \Hh^1(\Weil{},\Hh^1({\bar G},\EEx)) \arrow{r} &
    0.
\end{tikzcd}
 \]
In this way, Proposition~\ref{prop:bounded-etale} is now reduced to the claim
\[
\Hh^1(\Fq,\Hh^1({\bar G},\EEx)) = \Hom_\text{grp}(G(\Fq),\mathbb{\bar Z}_{\ell}^\times).
\]
To see that, one may argue as follows. 
Pick $i\in \pi_0(G)$ and let $G^i \hookrightarrow G$ be the corresponding connected component. 
Pick a geometric point ${\bar x}^i$ on $G^i$ and observe that since $G^i$ is connected as a $\Fq$-scheme, $G^i(\bFq)$ is canonically identified with the $\Gal(\bFq/\Fq)$-orbit of ${\bar x}^i$. 
We remark that while $G^i$ is defined over $\Fq$, the set $G^i(\Fq)$ is non-empty only when $G^i(\bFq) = \{ {\bar x}^i\}$.
Since $\Hh^1({\bar G},\EEx) = \Hom_\text{grp}({\bar G},\EEx)$, evaluation $\chi \mapsto \chi({\bar x}^i)$ defines $\Hh^1(\Fq,\Hh^1({\bar G},\EEx)) \to \Hh^1(\Fq,\EEx)$. 
By continuity, $\Hh^1(\Fq,\EEx) = \Hh^1(\Fq,\mathbb{\bar Z}_{\ell}^\times)$.
Letting $i$ range over $\pi_0(G)$ we conclude that $\Hh^1(\Fq,\Hh^1({\bar G},\EEx)) = \Hh^1(\Fq,\Hh^1({\bar G},\mathbb{\bar Z}_{\ell}^\times))$.
When adapted to abelian groups with continuous action of $\Gal(\bFq/\Fq)$, the strategy of the proof of Lemma~\ref{lem:dual-inv} gives $\Hh^1(\Fq,\Hh^1({\bar G},\mathbb{\bar Z}_{\ell}^\times)) =  \Hom_\text{grp}(G(\Fq),\mathbb{\bar Z}_{\ell}^\times)$, concluding the proof.
\end{proof}

\subsection{Morphisms in the \'etale case}\label{ssec:mor-etale}

While most of this paper focuses on the question of determining the isomorphism classes of objects, a complete understanding of the morphisms in $\CS(G)$ also requires a description of the automorphisms of an arbitrary character sheaf $\cs{L}$.

\begin{proposition}\label{prop:autornaught_etale}
Let $G$ be an \'etale commutative group scheme over $\Fq$.
If $\cs{L}$ and $\cs{L}'$ are character sheaves on $G$ then
every $\rho\in \Hom(\cs{L},\cs{L}')$ is either trivial or an isomorphism. Moreover, the trace map induces an isomorphism of groups
\[
\Aut_\CS{}(\cs{L}) \to \Hom(G(\bFq)_{\Weil{}}, \EEx).
\]
\end{proposition}

\begin{proof}
We have already seen, in Lemma~\ref{lem:autornaught}, that every $\rho\in \Hom(\cs{L},\cs{L}')$ is either trivial or an isomorphism.
Now suppose $\rho \in \Aut(\cs{L})$.
The second diagram in \ref{cs.4} shows that the association $x \mapsto \brho_x$ is a homomorphism from $G(\bFq)$ to $\EEx$ and the first diagram in \ref{cs.4} shows that it factors through $G(\bFq) \to G(\bFq)_{\Weil{}}$.  

Conversely, if $\rho : G(\bFq)_{\Weil{}} \to \EEx$ is any homomorphism, then defining $\brho_x$ as multiplication by $\rho(x)$ will define a morphism $\gcs{L} \to \gcs{L}'$ satisfy the two diagrams in \ref{cs.4}.  

Composition of morphisms corresponds to pointwise multiplication in this correspondence, showing that the resulting bijection is actually a group isomorphism.
\end{proof}

\subsection{Examples}

Let $G$ be the etale group scheme $\ZZ$ over $\Fq$ and consider the character $\chi : \ZZ \to \EEx$ of $G(\Fq)$ determined by $\chi(1) = 1+\ell$.
The stalk of $\cs{L}$ at the component $1$ is defined
by a pro-\'etale cover  . . . 

\section{The function-sheaf dictionary for smooth commutative group schemes}\label{sec:main}


\subsection{Restriction to the identity component} \label{ssec:restriction}

Consider the short exact sequence
defining the component group scheme for $G$:
\begin{equation}\label{eq:pi0}
\begin{tikzcd}
0 \rar & G^0 \arrow{r}{\iota_0} & G \arrow{r}{\pi_0} & \pi_0(G) \rar & 0.
\end{tikzcd}
\end{equation}
Since $\pi_0(G)$ is an \'etale commutative group scheme -- and thus smooth --
Lemma~\ref{lem:pullback} implies that \eqref{eq:pi0} defines a sequence of functors
\begin{equation}\label{eq:pi1}
\begin{tikzcd}
\CS(0) \rar & \CS(\pi_0(G)) \arrow{r}{\pi_0^*} & \CS(G) \arrow{r}{\iota_0^*} & \CS(G^0) \rar & \CS(0)
\end{tikzcd}
\end{equation}
and therefore, after passing to isomorphism classes, a sequence of abelian groups
\begin{equation}\label{eq:pi2}
\begin{tikzcd}
0 \rar &
\CSiso{\pi_0(G)} \arrow{r}{\pi_0^*} & \CSiso{G} \arrow{r}{\iota_0^*} & \CSiso{G^0} \rar & 0.
\end{tikzcd}
\end{equation}
 Note that we found the groups $\CSiso{\pi_0(G)}$ and $\CSiso{G^0}$
in Sections~\ref{ssec:SandT} and \ref{ssec:connected}, respectively.
We will shortly see that \eqref{eq:pi2} is exact.


\begin{lemma}\label{lemma:ext}
Every discrete isogeny to $G^0$ extends to a discrete
isogeny to $G$ inducing an isomorphism on component groups.
\end{lemma}

\begin{proof}
Let $\pi: B \to G^0$ be a discrete isogeny, and set $A \ceq \ker \pi$.
  We will find a discrete isogeny $f: H\to G$
  such that that $H^0 = B$, $f^0 =\pi$ and
  $\pi_0(f) : \pi_0(H)\to \pi_0(G)$ is an isomorphism of component
  groups.  Namely, we will fit $\pi$ into the following diagram,
  \begin{equation}\label{extension-diagram}
  \begin{tikzcd}
  A \arrow{r} \dar & A \dar \\
  B \rar \dar[swap]{\pi} & H \rar \dar[swap]{f} & \pi_0(H) \arrow{d}[below,rotate=90]{\sim}[swap]{\pi_0(f)} \\
  G^0 \rar & G \rar & \pi_0(G),
  \end{tikzcd}
  \end{equation}
  where all rows and columns are exact and all maps are defined over
  $\Fq$.  We will do so by passing back and forth between group
  schemes over $\Fq$ and their $\bFq$-points.

  Extensions of $G^0(\bFq)$ by $A(\bFq)$ with $\Weil{}$-equivariant maps, such as $B(\bFq)$,
  correspond to classes in $\Ext^1_{\ZZ[\Weil{}]}(G^0(\bFq), A(\bFq))$.
  Similarly, extensions of $G(\bFq)$ by $A(\bFq)$ with $\Weil{}$-equivariant maps correspond to
  classes in $\Ext^1_{\ZZ[\Weil{}]}(G(\bFq), A(\bFq))$.  The map
  $G^0(\bFq) \to G(\bFq)$ induces a homomorphism
  \[
  \Ext^1_{\ZZ[\Weil{}]}(G(\bFq), A(\bFq)) \to \Ext^1_{\ZZ[\Weil{}]}(G^0(\bFq), A(\bFq))
  \]
  fitting into the long exact sequence 
  \[
  \Ext^1_{\ZZ[\Weil{}]}(G(\bFq), A(\bFq)) \to \Ext^1_{\ZZ[\Weil{}]}(G^0(\bFq), A(\bFq)) \to \Ext^2_{\ZZ[\Weil{}]}(\pi_0(G)(\bFq), A(\bFq))
  \]
  derived from applying
  the functor $\Hom(\mbox{---}, A(\bFq))$ to $G^0(\bFq) \to G(\bFq) \to \pi_0(G)(\bFq)$.
  Since $\Weil{} \cong \ZZ$ has cohomological dimension $1$ \cite{brown:CohomologyGrps}*{Ex. 4.3},
  $\Ext^2_{\ZZ[\Weil{}]}(\pi_0(G)(\bFq), A(\bFq))$ vanishes \cite{cartan-eilenberg:HomologicalAlgebra}*{Thm. 2.6}.

  We therefore have the existence of diagram \eqref{extension-diagram}
  at the level of $\bFq$-points.  This expresses $H(\bFq)$ as a
  disjoint union of translates of $B(\bFq)$; by transport of structure
  we may take $H$ to be a group scheme over $\bFq$.  Similarly, the
  restriction of $f$ to each component of $H$ is a morphism of
  schemes, and thus $f$ is as well.  Finally, the whole diagram
  descends to a diagram of $\Fq$-schemes since the $\bFq$-points of
  the objects come equipped with continuous $\Gal(\bFq/\Fq)$-actions and the
  morphisms are $\Gal(\bFq/\Fq)$-equivariant.
\end{proof}

We now wish to apply the results of Section~\ref{ssec:connected} to the identity component of $G$; to do so, we must know that the identity component of $G$ is actually an algebraic group over $\Fq$.

\begin{lemma} \label{lem:G0alg-grp}
If $G$ is a commutative smooth group scheme over $\Fq$ then its identity component, $G^0$, is a connected algebraic group over $\Fq$.
\end{lemma}
\begin{proof}
 Since $G$ is a smooth group scheme over $\Fq$, its
 identity component $G^0$ is a connected smooth,
 group scheme of finite type over $\Fq$, reduced over some finite extension of $\Fq$
 \cite{vdGeer-Moonen:AbelianVarieties}*{3.17}.
 Since $\Fq$ is a finite field and hence perfect, $G^0$ is actually reduced over $\Fq$
 \cite{EGAIV2}*{Prop 6.4.1}.  Since every group scheme over a field is separated
 \cite{vdGeer-Moonen:AbelianVarieties}*{3.12},
 it follows that $G^0$ is a connected algebraic group.
\end{proof}

\begin{proposition}\label{prop:restriction}
The restriction functor $\iota_0^* : \CS(G)\to \CS(G^0)$ is essentially surjective.
\end{proposition}

\begin{proof}
  By Lemma~\ref{lem:G0alg-grp} and Proposition~\ref{prop:connected}, every
  character sheaf on $G^0$ is isomorphic to $(\pi_! \EE)_\psi$ for some discrete isogeny $\pi : B \to G^0$ and character $\psi : \ker \pi \to \EEx$.
  So to prove the proposition it suffices to show that $(\pi_! \EE)_\psi$ extends to a character sheaf on $G$.
%
 By Lemma~\ref{lemma:ext}, there is an extension of the
 discrete isogeny $\pi : B \to G^0$ to a discrete isogeny $f : H \to G$
 such that $\pi_0(f) : \pi_0(H)\to \pi_0(G)$ is an isomorphism.
 Then $(f_! \EE)_\psi$ is a character sheaf on $G$ and
 $(f_! \EE)_\psi\vert_{G^0} \iso (\pi_! \EE)_\psi$.
\end{proof}

\subsection{The component group sequence} \label{ssec:component}

\begin{lemma}\label{lem:extension}
The group homomorphism $\pi_0^*: \CSiso{\pi_0(G)} \to \CSiso{G}$ is injective.
\end{lemma}
\begin{proof}
Let $\cs{L}$ be a character sheaf on $\pi_0(G)$ and let $\rho : \pi_0^*\cs{L} \to (\EE)_{G}$ be an isomorphism in $\CS(G)$. 
For each $x\in \pi_0(\bG)$, set $\bG^x \ceq \pi_0^{-1}(x)$.
The restriction $\pi_0^*\gcs{L}\vert_{\bG^x}$ is the constant sheaf $(\gcs{L}_x)_{\bG^x}$ so the isomorphism $\brho\vert_{\bG^x} : (\gcs{L}_x)_{\bG^x} \to (\EE)_{\bG^x}$ determines an isomorphism $\brho_x : \gcs{L}_x \to (\EE)_x$. The collection $\{ \brho_x \tq x\in \pi_0(\bG) \}$ determines an isomorphism $\cs{L} \to (\EE)_{\pi_0(G)}$  in $\CS(\pi_0(G))$.
\end{proof}


\begin{proposition}\label{prop:middleexact}
 The sequence
 \[
  \begin{tikzcd}
  0 \rar & \CSiso{\pi_0(G)} \arrow{r}{\pi_0^*} & \CSiso{G} \arrow{r}{\iota_0^*} & \CSiso{G^0} \rar & 0.
  \end{tikzcd}
 \]
 is exact.
\end{proposition}

\begin{proof}
Exactness at $\CSiso{G^0}$ follows from Proposition~\ref{prop:restriction},
and at $\CSiso{\pi_0(G)}$ from Lemma~\ref{lem:extension}.
Here we show that it is also exact at $\CSiso{G}$.
First note that $\iota_0^* \circ \pi_0^*$ is trivial by Lemma~\ref{lem:pullback}.
So it suffices to show that if $\cs{L} = (\gcs{L},\mu,\phi)$ is a character sheaf on $G$
with $\cs{L}\vert_{G^0} = (\EE)_{G^0}$ then $\cs{L}$ is in the essential image of $\pi_0^*$.

As above, set $\bG^x \ceq \pi_0^{-1}(x)$ for $x\in \pi_0(\bG)$.
Let $g, g'$ be geometric points in the same
geometric connected component $\bG^x$.
Set $a = g^{-1}g'$ and note that $a$ is a geometric point in $\bG^0$.
Let $\mu_{g,a} : \gcs{L}_{ga} \to \gcs{L}_g \otimes \gcs{L}_a$
be the isomorphism of vector spaces obtained by restriction of
$\mu : m^*\gcs{L} \to \gcs{L} \boxtimes \gcs{L}$ to the
geometric point $(g,a)$ on $\bG^x \times \bG^0$.
Since $\cs{L}\vert_{G^0} = (\EE)_{G^0}$,
the stalk of $\gcs{L}$ at $a$ is $\EE$.
In this way the pair of geometric points $g, g' \in \bG^x$
determines an isomorphism $\varphi_{g,g'} \ceq \mu_{g,a}^{-1}$
from $\gcs{L}_{g}$ to $\gcs{L}_{g'}$.
%
The isomorphisms $\varphi_{g,g'}: \gcs{L}_{g} \to \gcs{L}_{g'}$ are canonical
in the following sense: if $g,g'\in \bG^x$ and $h,h'\in \bG^y$
then it follows from \ref{CS.2} and \ref{CS.3}
that
 \begin{equation}\label{eq:CS}
  \begin{tikzcd}[column sep=40]
   \gcs{L}_{gh} \arrow{r}{\varphi_{gh,g'h'}} \arrow[swap]{d}{\mu_{g,h}}
  & \gcs{L}_{g'h'} \arrow{d}{\mu_{g',h'}}
  &\arrow[draw=none]{d}[pos=.4,description]{\text{\normalsize{and}}}
  &  \gcs{L}_{\Frob{}(g)} \arrow{r}{\varphi_{\Frob{}(g),\Frob{}(g')}} \arrow[swap]{d}{\phi_{g}} & \gcs{L}_{\Frob{}(g')} \arrow{d}{\phi_{g'}} \\
  \gcs{L}_{g} \otimes \gcs{L}_{h} \arrow{r}{\varphi_{g,g'}\otimes \varphi_{h,h'}}
  & \gcs{L}_{g'} \otimes \gcs{L}_{h'}
  & {}
  & \gcs{L}_{g} \arrow{r}{\varphi_{g,g'}} & \gcs{L}_{g'}
  \end{tikzcd}
 \end{equation}
both commute.

For each $x\in \pi_0(\bG)$, pick $g(x)\in \bG^x$
and set $\gcs{E}_x \ceq \gcs{L}_{g(x)}$.
Let $\phi_x : \gcs{E}_{\Frob{}(x)} \to \gcs{E}_x$
be the isomorphism of $\EE$-vector spaces obtained by composing
$\varphi_{g(\Frob{}(x)),\Frob{}(g(x))} : \gcs{L}_{g(\Frob{}(x))} \to \gcs{L}_{\Frob{}(g(x))}$
with $\phi_{g(x)} : \gcs{L}_{\Frob{}(g(x))} \to \gcs{L}_{g(x)}$.
For each pair $x,y\in \pi_0(\bG)$
let $\mu_{x,y} : \gcs{E}_{x+y}\to \gcs{E}_x\otimes \gcs{E}_y$
be the isomorphism of $\EE$-vector spaces obtained by composing
$\varphi_{g(x+y),g(x)g(y)} : \gcs{L}_{g(x+y)} \to \gcs{L}_{g(x)g(y)}$
with $\mu_{g(x),g(y)} : \gcs{L}_{g(x)g(y)} \to \gcs{L}_{g(x)}\otimes \gcs{L}_{g(y)}$.
Using \eqref{eq:CS}, it follows that \ref{CS.1}, \ref{CS.2} and \ref{CS.3} are satisfied for
$\cs{E} \ceq (\gcs{E}_x, \mu_{x,y}, \phi_x)$, thus defining a character sheaf on $\pi_0(G)$.

The pullback $\pi_0^*(\cs{E})$ of $\cs{E}$ along $\pi_0 : G \to \pi_0(G)$ is constant
on geometric connected components, with stalks given by
$(\pi_0^* \cs{E})_g = \cs{E}_{x}$ for all $g\in \bG^x$.  Thus both $\pi_0^*\cs{E}$ and $\cs{L}$
are constant on geometric connected components of $G$.
The choices above define isomorphisms
$\gcs{L}\vert_{\bG^x} \to  (\gcs{E}_{x})_{\bG^x}$ for each $x\in \pi_0(\bG)$.
The resulting isomorphism $\gcs{L} \to \pi_0^* \gcs{E}$ satisfies \ref{CS.4},
thus defining an isomorphism $\cs{L} \to \pi_0^* \cs{E}$ in $\CS(G)$.
\end{proof}

\subsection{The dictionary}
\label{ssec:snake}

We saw in Proposition~\ref{prop:functorialG} that $\TrFrob{G} : \CSiso{G} \to G(\Fq)^*$ is a functorial group homomorphism.
In this section we find the image and kernel of $\TrFrob{G}$.

\begin{theorem}\label{thm:snake}
  If $G$ is a smooth commutative group scheme over $\Fq$ then
  \[
  \TrFrob{G} : \CSiso{G} \to G(\Fq)^*
  \]
is surjective with kernel canonically isomorphic to $\Hh^2(\pi_0({\bar G}),\EEx)^{\Weil{}}$.
\end{theorem}

\begin{proof}
  Let
  \begin{equation}\label{eq:pre-snake}
  \begin{tikzcd}[row sep=20, column sep=20]
    0 \rar & \CSiso{\pi_0(G)} \rar \dar{\TrFrob{\pi_0(G)}}
    & \CSiso{G} \rar \dar{\TrFrob{G}} & \CSiso{G^0} \rar \dar{\TrFrob{G^0}} & 0\\
    0 \rar & \pi_0(G)(\Fq)^* \rar
    &  G(\Fq)^* \rar & G^0(\Fq)^* \rar & 0
  \end{tikzcd}
  \end{equation}
  be the commutative diagram of abelian groups obtained by applying
  Lemma~\ref{lem:pullback} to \eqref{eq:pi0}.
 %
  The sequence of abelian groups
\[
  \begin{tikzcd}
    1 \rar & G^0(\Fq) \rar & G(\Fq) \rar & \pi_0(G)(\Fq) \rar & 0,
  \end{tikzcd}
\]
  is exact since $\Hh^1(\Fq,G^0) =0$ by Lemma~\ref{lem:G0alg-grp} and Lang's theorem on connected algebraic groups over finite fields \cite{lang:56a}.
  Since $\EEx$ is divisible, $\Hom(\ - \ ,\EEx)$ is exact and thus the dual sequence of
  character groups in \eqref{eq:pre-snake} is exact.
%
  The upper row in \eqref{eq:pre-snake} is exact by Proposition~\ref{prop:middleexact}.
  Now Lemma~\ref{lem:G0alg-grp} and Proposition~\ref{prop:connected}
  imply that $\ker \TrFrob{G^0} =0$ and $\coker \TrFrob{G^0}=0$,
  while Proposition~\ref{prop:etale-iso} gives $\ker \TrFrob{\pi_0(G)} \iso \Hh^0(\Weil{},\Hh^2(\pi_0({\bar G}),\EEx))$
  and $\coker \TrFrob{\pi_0(G)}=0$.
It now follows from the snake lemma
 \begin{equation}\label{eq:snake}
  \begin{tikzcd}[row sep=20, column sep=15]
    0 \arrow[dotted]{r} & \ker \TrFrob{\pi_0(G)} \arrow[dotted]{r} \dar & \arrow{d} \ker \TrFrob{G} \arrow[dotted]{r} & \ker \TrFrob{G^0} =0 \arrow[dotted, out=-10, in=170]{dddll} \dar & \\
    0 \rar & \CSiso{\pi_0(G)} \rar \dar{\TrFrob{\pi_0(G)}}
    & \CSiso{G} \rar \dar{\TrFrob{G}} & \CSiso{G^0} \rar \dar{\TrFrob{G^0}} & 0\\
    0 \rar & \pi_0(G)(\Fq)^* \rar \dar
    & \arrow{d} G(\Fq)^* \rar & G^0(\Fq)^* \rar \dar & 0\\
   &  \coker \TrFrob{\pi_0(G)} =0 \arrow[dotted]{r} & \coker \TrFrob{G} \arrow[dotted]{r} & \coker \TrFrob{G^0} =0 & 
  \end{tikzcd}
  \end{equation}
that $\coker \TrFrob{G} =0$
and $\ker \TrFrob{\pi_0(G)} \to \ker \TrFrob{G}$ is an isomorphism.
This gives the promised short exact sequence
\[
  \begin{tikzcd}
0 \arrow{r} & \Hh^2(\pi_0(\bG),\EEx)^{\Weil{}} \arrow{r} & \CSiso{G} \arrow{r}{\TrFrob{G}} & G(\Fq)^* \arrow{r} & 0. 
  \end{tikzcd}
\]
\end{proof}

\begin{remark}
Although $\TrFrob{\pi_0(G)}$ is split and $\TrFrob{G^0}$ is an isomorphism, it does not follow that $\TrFrob{G}$ is split. 
\end{remark}


\subsection{On the necessity of working with Weil sheaves}

\begin{proposition} \label{prop:bounded}
Let $G$ be a smooth commutative group scheme over $\Fq$. 
Then $\cs{L}\in \CS(G)$ descends to $G$ if and only if $\trFrob{\cs{L}} : G(\Fq) \to \EEx$ has bounded image.
\end{proposition}
\begin{proof} 
By Lemma~\ref{lem:G0alg-grp}, the identity component $G^0$ is a connected algebraic group over $\Fq$. 
It follows from Proposition~\ref{prop:connected} that the restriction of $\cs{L}$ to $G^0$ descends to $G$. 
Also, since $G^0(\Fq)$ is finite, the image of $\trFrob{\cs{L}} : G(\Fq) \to \EEx$ is a finite subgroup and therefore bounded image.  
%
If $\chi \in G(\Fq)^*$ then there is some finite-image character $\chi_0$
with the same restriction to $G^0(\Fq)$ since $G^0(\Fq)$ is lies inside the torsion part of
the finitely generated abelian group $G(\Fq)$.  Therefore $\chi$ is bounded 
if and only if $\chi \cdot \chi_0^{-1}$ is bounded.  But $\chi \cdot \chi_0^{-1}$ descends
to a character of $\pi_0(G)$.
Thus, it is enough to prove Corollary~\ref{prop:bounded}
for \'etale group schemes $G$, which is done in Proposition~\ref{prop:bounded-etale}.
\end{proof}

Proposition~\ref{prop:bounded} shows that the full subcategory
$\bCS(G) \subset \CS(G)$ is not an equivalence, for smooth commutative group schemes $G$.
This simple observation reveals the necessity of working with Weil sheaves in Definition~\ref{def:CS}: one cannot geometrize all characters of $G(\Fq)$ using local systems on $G$, for general smooth commutative groups schemes $G$. 

For example, consider the case when $G$ is the discrete \'etale group scheme $\ZZ$.
If $\chi : \ZZ \to \EEx$ is the character of $G(\Fq)$ determined by $\chi(1) = \ell$
and if $\cs{L}$ is a character sheaf on $G$ in the isomorphism class
corresponding to $\chi$ under Theorem~\ref{thm:snake},
then $\cs{L}$ does not descend to $G$ since the image of $\chi$ is not
bounded.

\subsection{Morphisms of character sheaves} \label{ssec:CSmor}

\begin{theorem}\label{thm:autornaught}
Let $G$ be a smooth commutative group scheme over $\Fq$.
There is a canonical isomorphism
\[
\Aut(\cs{L}) \iso \Hom(\pi_0(\bG)_{\Weil{}},\EEx).
\]
\end{theorem}
\begin{proof} 
Fix $\cs{L} = (\gcs{L},\mu,\phi)$ and consider the group homomorphism from $\Aut(\cs{L})$ to $\Hom(\bG_{\Weil{}},\EEx)$ defined in the proof of Proposition~\ref{prop:autornaught_etale}.
This homomorphism is injective because morphisms of sheaves are determined by the linear transformations induced on stalks.  
Homomorphisms in the image of $\Aut(\cs{L}) \to \Hom(\bG_{\Weil{}},\EEx)$ are continuous when $\bG$ is viewed as the base of the espace \'etal\'e attached to $\gcs{L}$.
Since $\ell$ is invertible in $\Fq$, it follows that the image of $\Aut(\cs{L}) \to \Hom(\bG_{\Weil{}},\EEx)$ is contained in $\Hom(\pi_0(\bG_{\Weil{}}),\EEx)$. 
We also have $\pi_0(\bG_{\Weil{}})=\pi_0(\bG)_{\Weil{}}$. 
To see that $\Aut(\cs{L}) \to \Hom(\pi_0(\bG)_{\Weil{}},\EEx)$ is surjective, begin with $\theta\in\Hom(\pi_0(\bG)_{\Weil{}},\EEx)$ and, for each $[x] \in \pi_0(\bG)_{\Weil{}}$ define $\bar\rho^y: \gcs{L}^y \to \gcs{L}^y$ by scalar multiplication by $\theta([x])\in \EEx$ for each $y\in [x]$.
This defines an isomorphism $\bar\rho : \gcs{L}\to \gcs{L}$ of local systems on $\bG$ compatible with $\mu$ and $\phi$, and thus an isomorphism $\rho :\cs{L}\to \cs{L}$ which maps to $\theta$ under $\Aut(\cs{L}) \to \Hom(\pi_0(\bG)_{\Weil{}},\EEx)$.
\end{proof}

\subsection{Base change}\label{ssec:basechange}

When using character sheaves to study characters, it is useful to understand
how character sheaves behave under change of fields.
Let $k'$ be a finite extension of $k$. Then $k \hookrightarrow k'$ induces a group homomorphism
$i_{k'/k} : G(k) \hookrightarrow G(k')$ and thus a homomorphism
\begin{align*}
i_{k'/k}^* : G(k')^* &\to G(k)^* \\
\chi &\mapsto \chi\circ i_{k'/k}.
\end{align*}
We can interpret this operation on characters in terms of character sheaves:

\begin{proposition} \label{prop:csbe}
Set $G_{k'} \ceq G\times_\Spec{k} \Spec{k'}$ and let
\[
\CS(\Res_{k'/k}(G_{k'})) \xrightarrow{\iota^*} \CS(G)
\]
be the functor obtained by pullback along the canonical closed immersion 
\[\iota : G \hookrightarrow \Res_{k'/k}(G_{k'})\] of $k$-schemes.
The following diagram commutes:
\[
\begin{tikzcd}
\CSiso{\Res_{k'/k}(G_{k'})} \arrow[two heads]{r}{\iota^*} \dar[swap]{\TrFrob{\Res_{k'/k}(G_{k'})}} & \CSiso{G} \dar{\TrFrob{G}} \\
G(k')^* \arrow[two heads]{r}{i_{k'/k}^*} & G(k)^*.
\end{tikzcd}
\]
\end{proposition}
\begin{proof}
The closed immersion $\iota : G \hookrightarrow \Res_{k'/k}(G_{k'})$ is given by \cite{bosch-lutkebohmert-reynaud:NeronModels}*{\S 7.6}.
Proposition~\ref{prop:csbe} follows immediately from Lemma~\ref{lem:pullback} together with the identifications
\[
\Res_{k'/k}(G_{k'})(k) \cong G_{k'}(k') \cong G(k')
\]
from the definitions of Weil restriction and base change.
\end{proof}

In the opposite direction, let $\Nm : G(k') \to G(k)$ be the norm map and consider the group homomorphism:
\begin{align*}
\Nm^* : G(k)^* &\to G(k')^* \\
\chi &\mapsto \chi\circ \Nm.
\end{align*}
We can also interpret this operation in terms of character sheaves.

If $\cs{L} \ceq (\gcs{L}, \mu, \phi)$ is a character sheaf on $G$, we define
$\cs{L}' \ceq (\gcs{L}, \mu, \phi_{k'})$ on the base change
$G_{k'}$ of $G$ to $k'$ by setting
\[
\phi_{k'} \ceq \phi \circ \Frob{G}^*(\phi) \circ \cdots \circ (\Frob{G}^{n-1})^*(\phi).
\]
The commutativity of the diagram (CS.3) for $\phi_{k'}$
follows from the fact that $\Frob{G_{k'}} = \Frob{G}^n$.
Note that we may also think about the construction of $\phi_{k'}$ from $\phi$
as restricting the action $\varphi$ of $\Weil{k}$ on $\gcs{L}$,
defined in Section~\ref{ssec:category}, to $\Weil{k'}$.

\begin{proposition}\label{prop:basechange}
With notation above,
the rule $\nu_{k'/k}: (\gcs{L}, \mu, \phi) \mapsto (\gcs{L}, \mu, \phi_{k'})$
 defines a monoidal functor $\CS(G) \to \CS(G_{k'})$
 such that the following diagram commutes:
\[
\begin{tikzcd}[column sep=60]
\CSiso{G} \rar{\nu_{k'/k}} \dar{\TrFrob{G}} & \CSiso{G_{k'}} \dar{\TrFrob{G_{k'}}} \\
G(k)^*  \rar{\Nm^*} & G(k')^*.
\end{tikzcd}
\]
\end{proposition}

\begin{proof}
Let $\cs{L} \ceq (\gcs{L}, \mu, \phi) \in \CS(G)$ and write $F$ for $\Frob{G}$.  For any $x \in G(k')$,
we may compute the value of $\TrFrob{G_{k'}}(\nu_{k'/k}\cs{L})(x)= t_{\nu_{k'/k}\cs{L}}(x)$ as the trace of $\phi_{k'}$ on $\gcs{L}_x$,
and the value of $\Nm^*(\TrFrob{G}(\cs{L}))(x)$ as the trace of $\phi$ on $\gcs{L}_{\Nm(x)}$.
Applying \ref{CS.3} to the stalk of $\gcs{L}^{\boxtimes n}$ at the point $(x, \Frob{}(x), \ldots, \Frob{}^{n-1}(x))$ yields a diagram
\[
\begin{tikzcd}
\gcs{L}_{\Nm(x)} \rar \dar{\phi_{\Nm(x)}} & \gcs{L}_{F(x)} \otimes \gcs{L}_{F^2(x)} \otimes \cdots \otimes \gcs{L}_x
\dar{\phi_x \otimes (F^*\phi)_x \otimes \cdots \otimes ((F^{n-1})^*\phi)_x} \\
\gcs{L}_{\Nm(x)} \rar & \gcs{L}_x \otimes \gcs{L}_{F(x)} \otimes \cdots \otimes \gcs{L}_{F^{n-1}(x)}.
\end{tikzcd}
\]
Choose a basis vector $v$ for $\gcs{L}_{\Nm(x)}$ and write $v_0 \otimes v_1 \otimes \cdots \otimes v_{n-1}$ for the image of $v$ under the
bottom map,
for $v_i \in \gcs{L}_{\Frob{}^i(x)}$.  By \ref{CS.2}, $v$ maps to
$v_1 \otimes v_2 \otimes \cdots \otimes v_0$ along the top of the diagram.
Let $\alpha_i \in \EEx$ represent $((F^i)^*\phi)_x$ with respect to these bases and let $\alpha$ be
the trace of $\phi_{\Nm(x)}$.  We may now equate the trace $\alpha$ of $\phi$ on $\gcs{L}_{\Nm(x)}$
with the product $\alpha_0 \cdots \alpha_{n-1}$, which is the trace of $\phi_{k'}$ on $\gcs{L}_x$.
\end{proof}

Finally, let $G'$ be a smooth commutative group scheme over $k'$.
We explain how to geometrize the canonical isomorphism between characters of $G'(k')$ and of $(\Res_{k'/k}G')(k)$.
We may decompose the base change $(\Res_{k'/k}G')_{k'}$ of $\Res_{k'/k}G'$ to $k'$
into a product of copies of $G'$, indexed by elements of $\Gal(k'/k)$:
\[
(\Res_{k'/k}G')_{k'} \cong \prod_{\Gal(k'/k)} G'.
\]
Since products and coproducts agree for group schemes we have a natural inclusion of $k'$-schemes
\[
G' \hookrightarrow (\Res_{k'/k}G')_{k'},
\]
mapping $G'$ into the summand corresponding to $1 \in \Gal(k'/k)$.  Composing $\nu_{k'/k}$
from Proposition~\ref{prop:basechange} with pullback along this map yields a functor
\[
\rho : \CS(\Res_{k'/k}G') \to \CS(G').
\]

\begin{proposition}
Let $k'/k$ be a finite extension and let $G'$ be a smooth commutative group scheme over $k'$.
Then the functor 
\[
\rho : \CS(\Res_{k'/k}G') \to \CS(G'),
\]
defined above, induces
\[
\begin{tikzcd}
\CSiso{\Res_{k'/k} G'} \dar{\TrFrob{\Res_{k'/k} G'}} \rar{\rho} & \CSiso{G'} \dar{\TrFrob{G'}}\\
G'(k')^* \rar & G'(k')^*,
\end{tikzcd}
\]
where the bottom map is the identity.
\end{proposition}
\begin{proof}
By Lemma~\ref{lem:pullback} the pullback part of the definition of $\rho$ corresponds to the map
\[
(\Res_{k'/k}G')(k')^* \to G'(k')^*
\]
induced by $g \mapsto (g, 1, \ldots, 1)$.  Since the action of $\Gal(k'/k)$ on
\[
(\Res_{k'/k}G')_{k'} \cong \prod_{\Gal(k'/k)} G'
\]
is given by permuting coordinates, composition with the norm map yields the identity on $G'(k')$.
\end{proof}

\subsection{The dictionary for commutative algebraic groups over finite fields}


\subsection{Examples}


\section{Applications to quasicharacters of $p$-adic tori and Abelian varieties}\label{sec:applications}

Let $K$ be a local field with finite residue field $\Fq$; we will now denote the group $\Weil{}$ by $\Weil{\Fq}$.
We continue to assume that $\ell$ is invertible in $\Fq$. 
In this section we consider connected commutative algebraic groups over $K$ that admit a N\'eron model, by which we mean a locally finite type N\'eron model.
By \cite{bosch-lutkebohmert-reynaud:NeronModels}*{\S 10.2, Thm 2}, these are precisely the connected commutative algebraic groups over $K$ that contain no subgroup isomorphic to $\mathbb{G}_\text{a}$.
Write $\mathcal{N}_K$ for the full subcategory of the category of algebraic groups consisting of such objects.  This category is additive, and includes all algebraic tori over $K$, abelian varieties over $K$ and unipotent $K$-wound groups.
We write $\mathcal{N}$ for the category of Neron models that arise in this way; in particular, $\mathcal{N}$ is a full subcategory of the category of smooth commutative group schemes over $R$.

\subsection{Quasicharacters}\label{ssec:quasicharacters}

Let $R$ be the ring of integers of $K$.
Write $\m$ for the maximal ideal of $R$ and set $R_n = R/\m^{n+1}$ for every non-negative integer $n$.
Suppose $X_K \in \mathcal{N}_K$ has N\'eron model $X$ over $R$.
Note that $X(K) = X(R)$.
A \emph{quasicharacter of $X(K)$} is a group homomorphism $X(K) \to \EEx$ that factors through $X(R) \to X(R_n)$ for some non-negative integer $n$.
We note that this definition is compatible with \cite{cassels-frohlich:AlgebraicNumberTheory}*{Ch XV, \S 2.3}.
The group of quasicharacters of $X(K)$ will be denoted by 
$\Hom_\text{}(X(K),\EEx)$
 and the subgroup of those that factor through $X(R_n)$ will be denoted by $\Hom_n(X(K),\EEx)$.
In this section we will see how to geometrize and categorify quasicharacters of $X(K)$ using character sheaves.

\subsection{Review of the Greenberg transform} \label{ssec:rev_Greenberg}

Let $K$, $R$ and $R_n$ be as above.
For each $n \in \NN$, the Greenberg functor maps schemes over $R_n$ to schemes over $\Fq$.
See \cite{bertrapelle-gonzales:Greenberg} for the definition and fundamental properties of the Greenberg functor as we use it; other useful references include
\cite{greenberg:61}, \cite{greenberg:63a},
\cite{demazure-gabriel:GroupesAlgebriques}*{V, \S 4, no. 1},
\cite{bosch-lutkebohmert-reynaud:NeronModels}*{Ch. 9, \S 6} and
\cite{nicaise-sebag:motivicSerre}*{\S 2.2}. %citelist
%
For any non-negative integer $n$ we will write
\[
\Gr^R_n : \Sch{R} \to \Sch{\Fq}
\]
for the functor produced by composing pullback along $\Spec{R_n} \to \Spec{R}$ with the Greenberg functor. 
This functor respects open immersions, closed immersions, \'etale morphisms, smooth morphisms and geometric components.  Moreover, there is a canonical isomorphism
\[
\Gr^R_n(X)(\Fq) \iso X(R_n)
\]
for any scheme $X$ over $R$.

For any $n\leq m$,  the surjective ring homomorphism $R_{m} \to R_n$ determines a
natural transformation 
\[
\varrho^R_{n\leq m} : \Gr^R_{m} \to \Gr^R_n
\]
between additive functors.
Crucially, $\varrho^R_{n\leq m}(X): \Gr^R_{m}(X)\to \Gr^R_n(X)$ is an affine morphism of $\Fq$-schemes, for every $R$-scheme $X$ and every $n\leq m$ \cite{bertrapelle-gonzales:Greenberg}*{Prop 4.3}.
This observation is key to the proof that, for any scheme $X$ over $R$, the projective limit 
\[
\Gr_R(X) \ceq \varprojlim_{n\in \NN} \Gr^R_n(X),
\]
taken with respect to the surjective morphisms $\varrho^R_{n\leq m}(X) : \Gr^R_{m}(X) \to \Gr^R_n(X)$,
exists in the category of group schemes over $\Fq$;
see \cite{EGAIV3}*{\S 8.2}.
This leads to the definition of what we shall call the {\it Greenberg transform}:
\[
\Gr_R : \Sch{R} \to \Sch{\Fq}.
\]
By construction, the $\Fq$-scheme $\Gr_R(X)$ comes equipped with morphisms 
\[
\varrho^R_n(X) : \Gr^R(X) \to \Gr^R_n(X),\qquad \forall n\in \NN.
\]

\subsection{Quasicharacter sheaves}
%{Character sheaves on the Greenberg transform of a N\'eron model} 
\label{ssec:CS_on_GN}
 
Let $X \in \mathcal{N}$ be as above.
Set $S = \Spec{R}$ and $S_n = \Spec{R_n}$;
note that $S_0 = \Spec{\Fq}$ is the special fibre of $S$.

For every non-negative integer $n$, $\Gr^R_n(X)$ is a {\it smooth} commutative group scheme over $S_0$
with component group scheme $\pi_0(X)_{\Fq} \ceq \pi_0(X) \times_S S_0$ and $\Fq$-rational points $X(R_n).$
%
The Greenberg transform $\Gr_R(X)$ of $X$ is a commutative group scheme over $\Fq$
with component group scheme $\pi_0(X)_{\Fq}$
and $\Fq$-rational points $X(K).$
However, $\Gr_R(X)$ is not locally of finite type and therefore not smooth.
%
The morphism of $\Fq$-schemes $\varrho^R_n(X) : \Gr_R(X) \to \Gr^R_n(X)$ induces a functor
\[
\varrho^R_n(X)^* : \CS(\Gr^R_n(X)) \to \CS(\Gr_R(X)),
\]
as in Lemma~\ref{lem:pullback}.
%This functor is faithful, so we may view $\CS(\Gr^R_n(X))$ as a subcategory of $\CS(\Gr_R(X))$.

We now define a category which will geometrize quasicharacters of $X(K)$.
We begin by considering the rigid monoidal category $\CS(G)$ defined in Section~\ref{ssec:category} without insisting that the commutative group $\Fq$-scheme $G$ is smooth, so that we can replace it with $\Gr_R(X)$ for smooth group $R$-scheme $X$.

\begin{definition}
Let $X$ be a smooth group scheme over $R$, with $R$ as above.
A {\it quasicharacter sheaf for $X$} is an object of
the following rigid monoidal subcategory of $\CS(\Gr_R(X))$, denoted by $\QCS(X)$:
\begin{enumerate}
\item
objects in $\QCS(X)$ are the $\ell$-adic sheaves $\varrho^R_n(X)^*\cs{L}$, for $n\in \NN$ and $\cs{L} \in \CS(\Gr^R_n(X))$; 
\item
morphisms $\varrho^R_n(X)^*\cs{L} \to \varrho^R_m(X)^*\cs{L}'$ in $\QCS(X)$ are those morphisms in $\CS(\Gr_R(X))$ which take the form $\varrho^R_m(X)^*\rho$ for $\rho \in \Hom(\varrho^R_{n\leq m}(X)^*\cs{L},\cs{L}')$ when $n\leq m$, and $\varrho^R_n(X)^*\rho$ for $\rho \in \Hom(\cs{L},\varrho^R_{m\leq n}(X)^*\cs{L}')$ when $m\leq n$.
\end{enumerate}

\end{definition}


\begin{remark}
If $\varrho^R_n(X)^* : \CS(\Gr^R_n(X)) \to \CS(\Gr_R(X))$ is full then the construction above can be improved by forming $\QCS(X)$ from the essential images of the functors $\varrho^R_n(X)^*$; however, we do not know if $\varrho^R_n(X)^*$ is full.
\end{remark}

\begin{remark}
We offer the following alternate construction of $\QCS(X)$, which does not make use of subcategories.
For objects, one considers triples 
\[
\cs{F} \ceq (n, \{\cs{F}_i\}_{n\leq i}, \{\alpha_{i \le j}\}_{n\le i \le j}),
\] where $n$
is a non-negative integer, each $\cs{F}_i$ is a character sheaf on $\Gr^R_i(X)$ and each 
\[
\alpha_{i \le j} : \cs{L}_j \to \varrho^R_{i \le j}(X)^* \cs{L}_i
\]
 is an isomorphism; here $\alpha_{i \le i}$ is the identity and the $\alpha_{i \le j}$ are compatible with each other.  
If $\cs{F} \ceq (n, \{\cs{F}_i\}, \{\alpha_{i \le j}\})$
and $\cs{F}' \ceq (m, \{\cs{F}'_i\}, \{\alpha'_{i \le j}\})$ are objects then $\Hom(\cs{F}, \cs{F}')$ is the set
of equivalence classes of pairs $(k, \{\beta_i\}_{k \le i})$, where $n,m \le k$ and the $\beta_i : \cs{F}_i \to \cs{F}'_i$ are
morphisms of character sheaves so that
\[
\begin{tikzcd}
\cs{F}_j \rar{\alpha_j} \dar{\beta_i} & \varrho^R_{i \le j}(X)^* \cs{F}_i \dar{f_{i \le j}^*\beta_j} \\
\cs{F}'_j \rar{\alpha_j} & \varrho^R_{i \le j}(X)^* \cs{L}'_i
\end{tikzcd}
\]
commutes for all $k\le i\le j$; we identify two such pairs $(k, \{\beta_i\})$ and $(l, \{\gamma_i\})$ if $\beta_i = \gamma_i$
for sufficiently large $i$.  Identities and composites are defined in the natural way.
\end{remark}


\begin{theorem}\label{thm:CSXK}
Let $K$ be a local field with residue field $\Fq$, in which $\ell$ is invertible; 
let $R$ be the ring of integers of $K$.
Fix a separable closure ${\bar K}$ of $K$ and let $\tilde{K}$ be the maximal unramified extension of $K$ in ${\bar K}$; write $\tilde{R}$ for the ring of integers of $\tilde{K}$ and set $\tilde{S} = \Spec{\tilde{R}}$.
\begin{enumerate}
\labitem{(0)}{x0}
The trace of Frobenius provides a natural transformation between the additive functors
\[
%F_{\QCS}: 
X \mapsto \QCSiso{X}
\qquad\text{and}\qquad
%F_{\text{}}: 
X \mapsto \Hom_{\text{}}(X(K),\EEx)
\]
as functors from $\mathcal{N}$ to the category of commutative groups.
\end{enumerate} 
Regarding this natural transformation, for every $X \in \mathcal{N}$:
\begin{enumerate}[resume]
\labitem{(1)}{x1} there is a canonical short exact sequence of commutative groups 
\[
0 \to \Hh^2(\pi_0(X)_{\bFq},\EEx)^{\Weil{\Fq}} \to \QCSiso{X} \to \Hom_\text{}(X(K),\EEx) \to 0;
\] 
\labitem{(2)}{x2} for all quasicharacter sheaves $\cs{F}$, $\cs{F}'$ on $\Gr_R(X)$, and for every $\rho \in \Hom(\cs{F},\cs{F}')$, either $\rho$ is trivial or $\rho$ is an isomorphism;
\labitem{(3)}{x3} for all quasicharacter sheaves $\cs{F}$ for $X$, there is a canonical isomorphism
\[
\Aut(\cs{F}) \iso \Hom(\pi_0(X \times_S \tilde{S})_{\Weil{\Fq}},\EEx)
\]
\end{enumerate}
\end{theorem}

\begin{proof}
%\begin{enumerate}
To prove \ref{x0}, use: Proposition~\ref{prop:functorialG} with $G = \Gr^R_n(X)$; the fact that N\'eron models are unique up to isomorphism; the fact that every $\CS(\Gr^R_n(X))$ is a full subcategory of $\QCS(X)$; and the observation that every object in $\QCS(X)$ is in the essential image of $\CS(\Gr^R_n(X))$ for some $n$.
To prove \ref{x1}, use Theorem~\ref{thm:snake} with $G = \Gr^R_n(X)$ and then argue as in part \ref{x0}.
To prove \ref{x2}, argue as in the proof of Lemma~\ref{lem:autornaught}.
To prove \ref{x3}, use: the fact that the component group of $\Gr^R_n(X)$ is independent of $n$; Theorem~\ref{thm:autornaught} with $G = \Gr^R_n(X)$, in which case $\pi_0(G) = \pi_0(X \times_S S_0)$ and $\pi_0(\bG) = \pi_0(X \times_S \tilde{S})$; and then argue as in part \ref{x0}.
%\end{enumerate}
\end{proof}

\subsection{Quasicharacter sheaves for $p$-adic tori} \label{ssec:CS_tori}

As we explained in the Introduction, our original motivation for this paper was to find a geometrization and categorification of quasicharacters of $p$-adic tori. 
This is now provided by the following adaptation of Theorem~\ref{thm:CSXK} in the case when $T\in \mathcal{N}$ is a Neron model for an algebraic torus over $K$.

\begin{corollary}\label{cor:CS_tori}
Let $T$ be a Neron model for an algebraic torus over $K$.
The following is a commutative diagram of exact sequences.
\[
  \begin{tikzcd}[row sep=20, column sep=18]
{}  & 0 \arrow{d} & 0 \arrow{d} &  & \\ 
   & \Hh^2(X_*(T_K)_{\mathcal{I}_K},\EEx)^{\Weil{\Fq}}  \arrow{d} & \arrow{d} \Hh^2(X_*(T_K)_{\mathcal{I}_K},\EEx)^{\Weil{\Fq}} & 0 \arrow{d} & \\
    0 \rar & \QCSiso{\pi_0(T)} \arrow{r}{\pi_0^*} \dar{\TrFrob{\pi_0(T)}}
    & \QCSiso{T} \arrow{r}{\iota_0^*}  \arrow{d}{\TrFrob{\Gr_R(T)}} & \QCSiso{T^0} \arrow{r} \arrow{d}{\TrFrob{\Gr_R(T)^0}} & 0\\
    0 \arrow{r} & \Hom(\pi_0(T)(\Fq), \EEx) \arrow{r}{\text{infl'n}} \arrow{d}{\iso}
    & \arrow{d}{\iso} \Hom_\text{}(T(K),\EEx) \arrow{r}{\text{rest'n}} & \Hom_\text{}(T^0(R),\EEx) \arrow{r} \arrow{d}{\iso} & 0\\
 &  0  & 0 & 0 & 
  \end{tikzcd}
 \]
 \end{corollary}
\begin{proof} 
The horizontal sequence of groups coming from categories of quasicharacter sheaves is exact by Proposition~\ref{prop:middleexact}, together with the observation that the functors $\pi_0^*$ and $\iota^*$ perverse limits.
It is elementary that the horizontal sequence of quasicharacters is exact.
The diagram commutes by Lemma~\ref{lem:pullback}.
Accordingly, by Theorem~\ref{thm:CSXK}, the kernel of $\TrFrob{T}$ is $\Hh^2(\pi_0(T)_{\bFq},\EEx)^{\Weil{}}$. 
By \cite{bitan:discriminant}*{Eq 3.1}, the special fibre of the component group scheme for $T$ is given by
\[
 \pi_0(T)_{\bFq} = X_*(T_K)_{\mathcal{I}_K},
\]
where $X_*(T_K)$ is the cocharacter lattice of $T_K$ and $\mathcal{I}_K$ is the inertia group for $K$.
Thus,
\[
\Hh^2(\pi_0(T)_{\bFq},\EEx)^{\Weil{\Fq}} = \Hh^2(X_*(T_K)_{\mathcal{I}_K},\EEx)^{\Weil{\Fq}}.
\] 
which coincides with the kernel of $\TrFrob{\pi_0(T)}$ since $\pi_0(\pi_0(T))_{\bFq} = \pi_0(T)_{\bFq}$.
\end{proof}

\begin{corollary}
Let $T$ be a Neron model for an algebraic torus over $K$.
For $\cs{E}\in \QCS(\pi_0(T))$, $\cs{F}\in {\QCS(T)}$ and $\cs{F}^0\in {\QCS(T^0)}$, there are canonical isomorphisms
\[
\Aut(\cs{E})\iso (\check{T}_\ell)^{\Weil{K}},
\quad
\Aut(\cs{F})\iso (\check{T}_\ell)^{\Weil{K}},
\quad
\Aut(\cs{F}^0)\iso 1,
\]
 where $\check{T}_\ell \ceq \Hom(X_*(T_K),\EEx)$, the $\ell$-adic dual torus to $T_K$.
\end{corollary}

\begin{proof}
We already know $\Aut(\cs{E}) =1$ from Proposition~\ref{prop:connected}, part \ref{x3}.
By Theorem~\ref{thm:CSXK}, 
\[\Aut(\cs{F})  \iso \Hom((\pi_0(X \times_S \tilde{S})_{\Weil{\Fq}},\EEx),\] canonically, for $X = T$ and $X=\pi_0(T)$ (and for $X= T^0$, though that case has already been treated by a different argument, above).
By \cite{bitan:discriminant}*{Eq 3.1} again, 
\[
\Hom((\pi_0(T)_{\bFq})_{\Weil{}},\EEx) \iso
\Hom(X_*(T_K)_{\Weil{K}},\EEx),
\]
canonically.
But $\Hom(X_*(T_K)_{\Weil{K}},\EEx) \iso \Hom(X_*(T_K),\EEx)^{\Weil{K}}$, canonically.
So, for any quasicharacter sheaf $\cs{F}$ for $T$,
\[
\Aut(\cs{F}) \iso (\check{T}_\ell)^{\Weil{K}},
\]
canonically.
The case $X= \pi_0(T)$ recovers the same automorphism group since $\pi_0(\pi_0(T))_{\bFq} = \pi_0(T)_{\bFq}$.
\end{proof}

\subsection{Geometric Langlands correspondence for $p$-adic tori}\label{ssec:LLCT}

\begin{lemma}\label{lem:LLCT}
The following diagram commutes. 
\[
\begin{tikzcd}
   0 \arrow{r} & \Hom(\pi_0(T)(\Fq), \EEx) \arrow{r}{\text{infl'n}} \arrow{d}{\iso}
    & \arrow{d}{\iso} \Hom_\text{}(T(K),\EEx) \arrow{r}{\text{rest'n}} & \Hom_\text{}(T^0(R),\EEx) \arrow{r} \arrow{d}{\iso} & 0\\    
    0 \arrow{r}  
 & \Hh^1(\Weil{\Fq}, \check{T}_\ell^{\mathcal{I}_K}) \arrow{r}{\text{infl'n}}
 & \Hh^1(\Weil{K}, \check{T}_\ell) \arrow{r}{\text{rest'n}} 
 & \Hh^1(\mathcal{I}_K, \check{T}_\ell)^{\Weil{\Fq}} \arrow{r}  
 & 0
\end{tikzcd}
\] 
\end{lemma}
\begin{proof}
Since $\pi_0(T)_{\bFq} = X_*(T_K)_{\mathcal{I}_K}$ by \cite{bitan:discriminant}*{Eq 3.1}, the isomorphism on the left can be seen directly:
\[\Hom(\pi_0(T)(\Fq), \EEx)
= \Hom((X_*(T)_{\mathcal{I}_K})^{\Weil{\Fq}}, \EEx)
= \Hom(X_*(T)_{\mathcal{I}_K}, \EEx)_{\Weil{\Fq}} 
%= \Hh^1(\Weil{}, \Hom(X_*(T), \EEx)^{\mathcal{I}_K}) 
= \Hh^1(\Weil{\Fq}, \check{T}_\ell^{\mathcal{I}_K}).
\]
The middle isomorphism is the Langlands correspondence for $p$-adic tori, itself a fairly direct consequence of local class field theory; see \cite{yu:09a}, for example. The isomorphism 
\[
\Hom_\text{}(T^0(R),\EEx) \iso \Hh^1(\mathcal{I}_K, \check{T}_\ell)^{\Weil{\Fq}}
\] 
and the commutativity of the diagram is then a consequence of the inflation-restriction exact sequence and the fact that $\Weil{}$ has cohomological dimension $1$.
\end{proof}

\begin{remark}
Combining Corollary~\ref{cor:CS_tori} with Lemma~\ref{lem:LLCT} now produces the following commutative diagram of exact sequences:
\begin{equation}\label{eq:GLCT}
  \begin{tikzcd}[row sep=20, column sep=15]
{}  & 0 \arrow{d} & 0 \arrow{d} &  & \\ 
   & \Hh^2(X_*(T_K)_{\mathcal{I}_K},\EEx)^{\Weil{\Fq}}  \arrow{d} & \arrow{d} \Hh^2(X_*(T_K)_{\mathcal{I}_K},\EEx)^{\Weil{\Fq}} & 0 \arrow{d} & \\
    0 \rar & \QCSiso{\pi_0(T)} \arrow{r}{\pi_0^*} \arrow{d}
    & \QCSiso{T} \arrow{r}{\iota_0^*}  \arrow{d} & \QCSiso{T^0} \arrow{r} \arrow{d} & 0\\
   0 \arrow{r}  
 & \Hh^1(\Weil{\Fq}, \check{T}_\ell^{\mathcal{I}_K}) \arrow{r}{\text{infl'n}} \arrow{d}
 & \Hh^1(\Weil{K}, \check{T}_\ell) \arrow{r}{\text{rest'n}} \arrow{d}
 & \Hh^1(\mathcal{I}_K, \check{T}_\ell)^{\Weil{\Fq}} \arrow{r} \arrow{d}  
 & 0\\
 &  0  & 0 & 0 & 
  \end{tikzcd}
 \end{equation}
It is natural to ask if the vertical surjections can be defined directly, without making use of local class field theory.  
The case $T_K = \Gm{K}$ is already very interesting, in which case \eqref{eq:GLCT} becomes
\begin{equation}\label{eq:Gm}
\begin{tikzcd}
    0 \rar & \QCSiso{\ZZ} \arrow{r}{\pi_0^*} \arrow{d}{\iso}
    & \QCSiso{T} \arrow{r}{\iota_0^*}  \arrow{d}{\iso} & \QCSiso{\Gm{R}} \arrow{r} \arrow{d}{\iso} & 0\\
    0 \arrow{r}  
 & \Hom(\Weil{\Fq}, \EEx) \arrow{r}{\text{infl'n}}
 & \Hom(\Weil{K}, \EEx) \arrow{r}{\text{rest'n}} 
 & \Hom(\mathcal{I}_K, \EEx) \arrow{r}  
 & 0
\end{tikzcd}
\end{equation}
We have seen that the isomorphism $\QCS(\ZZ)\iso \Hom(\Weil{\Fq}, \EEx)$ is elementary. 
The isomorphism $\QCSiso{\Gm{R}}\iso \Hom(\mathcal{I}_K, \EEx)$ may be deduced directly from the class field theory of Serre and Hezewinkel as revisited beautifully in \cite{suzuki-yoshida:12a}; see also \cite{Suzuki:Neron}.
The monoidal category of character sheaves on $\Gr^R_n(\Gm{R})$ is canonically equivalent to the category of $\ell$-adic characters of the quotient of the etale fundamental group $\pi_1(\Gr^R_n(\Gm{R}),\bFq)$ determined by the Lang isogeny for $\Gr^R_n(\Gm{R})$.
Accordingly, quasicharacter sheaves on $\Gm{R}$ may be apprehended as $\ell$-adic characters of
the inverse limit of these quotients, as $n$ ranges over $\NN$.
By \cite{serre:isogenies}, there is a canonical isomorphism between this inverse limit and $\mathcal{I}_K^\text{ab}$;
this isomorphism is {\it not} restricted to the case when $K$ and $\Fq$ have equal characteristic.
This provides the isomorphism $\QCSiso{\Gm{R}}\iso \Hom(\mathcal{I}_K, \EEx)$ in \eqref{eq:Gm}. 
Since the sequences are split, this gives $\QCSiso{T}\iso \Hom(\Weil{K}, \EEx)$.
\end{remark}

\begin{remark}
This paper shows that the \'etale site on $\Gr_R(X)$ is rich enough to geometrize all quasicharacters of $X(K)$ as $\ell$-adic local systems on $\Gr_R(X)$, where $X$ is a Neron model for an algebraic torus or an abelian variety over a local field $K$. 
It is natural to ask if the \'etale site on the generic fibre $X_K$ would have sufficed. 
This seems unlikely, since the geometric etale fundamental group of $\Gm{K}$ is ${\hat \ZZ}$, though limited results in this direction were established in \cite{Cunningham-Kamgarpour} when $K = \mathbb{Q}_p$.
\end{remark}

\subsection{Weil restriction and quasicharacter sheaves}\label{ssec:wrK}

We expect that the general case of \eqref{eq:GLCT}, where $K$ is any local field and $T_K$ is any torus over $K$, may be deduced from the case \eqref{eq:Gm}.
Some work in this direction appears in \cite{Suzuki:Neron}.
In this section we develop a tool for further work in that direction.

Let $K'/K$ be a finite Galois extension of local fields and
let $k'/k$ be the corresponding extension of residue fields.
Let $R$ and $R'$ be the rings of integers of $K$ and $K'$, respectively.
Let $X_K \in \mathcal{N}$ have N\'eron model $X$, set $X_{K'} \ceq X_K \times_\Spec{K} \Spec{K'}$
and let $X'$ be a N\'eron model for $X_{K'}$.

\begin{proposition}\label{prop:wrK}
The canonical closed immersion 
\[
X_K \hookrightarrow \Res_{K'/K} X_{K'}
\]
of $K$-group schemes
induces a map of $\Fq$-group schemes 
\[
f : \Gr_R(X) \to \Res_{k'/k} \Gr_{R'}(X')
\] 
which, through quasicharacter sheaves, induces
\[
\begin{tikzcd}[column sep=60]
\Hom_{\text{}}(X_K(K'), \EEx) \arrow{r}{\chi \mapsto \chi\vert_{X(K)}} &\Hom_{\text{}}(X_K(K), \EEx).
\end{tikzcd}
\]
\end{proposition}

\begin{proof}
Adapting the argument in \cite{bosch-lutkebohmert-reynaud:NeronModels}*{\S 7.6, Prop 6} to include locally finite type N\'eron models,
one may show that $\Res_{R'/R}(X')$ is a N\'eron model for $\Res_{K'/K}(X_{K'})$.
Thus, by the N\'eron mapping property, the canonical closed immersion
\[
X_K\hookrightarrow \Res_{K'/K}(X_{K'})
\]
 extends uniquely to a morphism
\begin{equation}\label{me}
X\to \Res_{R'/R}(X')
\end{equation}
 of smooth $R$-group schemes.
%
Applying the functor $\Gr^R_{n}$ to \eqref{me}
and using \cite{bertrapelle-gonzales:Greenberg}*{Thm 1.1} defines the morphism of smooth group schemes
\begin{equation}\label{men}
f_n: \Gr_{n-1}^R(X) \to \Res_{k'/k} \Gr_{en-1}^{R'}(X'),
\end{equation}
where $e$ is the ramification index of $K'/K$.
Using Lemma~\ref{lem:pullback}, \eqref{men} induces a functor 
\begin{equation}\label{men-pulled}
f_n^* : \CS(\Res_{k'/k} \Gr_{en-1}^{R'}(X'))\to \CS(\Gr_{n-1}^{R}(X)).
\end{equation}
Since 
\[
\left(\Res_{k'/k} \Gr_{en-1}^{R'}(X') \right)(\Fq) = \left(\Gr_{en-1}^{R'}(X')\right)(k'),
\]
it follows from Lemma~\ref{lem:pullback} that the pullback functor \eqref{men-pulled} actually induces
\[ 
\Hom_{en-1}(X'(R'),\EEx) \to \Hom_{n-1}(X(R)),\EEx)
\]
Since $X'$ is a N\'eron model for $X_{K'}$ and $X$ is a N\'eron model for $X_K$,
 this may be re-written as
 \[ 
\Hom_{en-1}(X_{K'}(K'),\EEx)= \Hom_{en-1}(X_{K}(K'),\EEx) \to \Hom_{n-1}(X_K(K)),\EEx)
\]
Passing to limits now defines
\[ 
\Hom_\text{}(X_K(K'),\EEx) \to \Hom_\text{}(X_K(K),\EEx)
\]
Argue as in Proposition~\ref{prop:csbe} to see that this is indeed restriction of characters.
\end{proof}

\subsection{Transfer of quasicharacters}\label{ssec:transfer}

Let $K$ and $L$ be local fields with rings of integers $\OK$ and $\OL$, respectively. 
Pick uniformizers $\varpi_K$ and $\varpi_L$ for $\OK$ and $\OL$, respectively;
what follows will not depend on these choices.
Suppose $\ell$ is invertible in the residue fields of $K$ and $L$.

We begin with $X_K\in \mathcal{N}_K$ with N\'eron model $X$ and $Y_{L}\in \mathcal{N}_L$ with N\'eron model $Y$.
Suppose $m$ is a positive integer such that 
\[
\OK/\varpi_K^{m}\OK \iso \OL/\varpi_L^{m}\OL.
\]
Suppose also that
\begin{equation}\label{eq:schematic_transfer}
X \times_{\Spec{\OK}} \Spec{\OK/\varpi_K^{m}\OK} \iso Y \times_{\Spec{\OL}} \Spec{\OL/\varpi_L^{m}\OL}
\end{equation}
as smooth group schemes over $\OK/\varpi_K^{m}\OK$. 
Then
\[
\Gr^{\OK}_{m-1}(X) \iso \Gr^{\OL}_{m-1}(Y)
\]
as smooth group schemes over $\Fq$.  Accordingly, by Lemma~\ref{lem:pullback}, the isomorphism above determines an equivalence of categories
\begin{equation}\label{eq:categorical_transfer}
\CS(\Gr^{\OK}_{m-1}(X)) \cong \CS(\Gr^{\OL}_{m-1}(Y))
\end{equation}
which induces an isomorphism
\begin{equation}\label{eq:transfer}
\Hom_{m-1}(X(K),\EEx)  \iso  \Hom_{m-1}(Y(L),\EEx).
\end{equation}
The isomorphism \eqref{eq:transfer} is an instance of {\it transfer} of (certain) quasicharacters between $X(K)$ and $Y(L)$. 
We now recognize this transfer of quasicharacters as a consequence of the equivalence of categories of quasicharacter sheaves \eqref{eq:categorical_transfer}.

The isomorphism \eqref{eq:schematic_transfer} can indeed exist between quasicharacters of algebraic tori over local fields, even when the characteristics of $K$ and $L$ differ.
Suppose $T_K$ and $M_L$ are tori over local fields $K$ and $L$,
splitting over $K'$ and $L'$, respectively.
Then $T_K$ and $M_L$ are said to be \emph{$n$-congruent} \cite{chai-yu:01a}*{\S 2} if there are isomorphisms
 \begin{align*}
  \alpha : \OO{K'}/\varpi_K^n \OO{K'} &\to \OO{L'}/\varpi_{L}^n \OO{L'} \\
  \beta : \Gal(K'/K) &\to \Gal(L'/L) \\
  \phi : X^*(T_K) &\to X^*(M_L)
 \end{align*}
 satisfying the conditions
 \begin{enumerate}
  \item $\alpha$ induces an isomorphism $\OK/\varpi_K^n \OK \to \OO{L}/\varpi_{L}^n \OO{L}$,
  \item $\alpha$ is $\Gal(K'/K)$-equivariant relative to $\beta$, and
  \item $\phi$ is $\Gal(K'/K)$-equivariant relative to $\beta$.
 \end{enumerate}
If $T_K$ and $M_L$ are $n$-congruent then $\alpha$, $\beta$ and $\phi$ determine an isomorphism 
\begin{equation}\label{transfer}
  \Hom_{n-1}(T_K(K), \EEx) \iso \Hom_{n-1}(M_L(L),\EEx).
\end{equation}
Note that if $T_K$ and $M_L$ are $n$-congruent, then they are $m$-congruent for every
$m \leq n$.

Now let $T$ be a N\'eron model for $T_K$ and let $M$ be a N\'eron model for $M_L$.
One of the main results of \cite{chai-yu:01a} gives an isomorphism of group schemes 
\[
T \times_{\Spec{\OK}} \Spec{\OK/\varpi_K^m\OK} \iso M \times_{\Spec{\OL}} \Spec{\OL/\varpi_L^m\OL}
\] 
assuming that $T_K$ and $M_L$ are sufficiently congruent.
They define a quantity $h$ (the smallest integer so that $\varpi^h$ lies in the
Jacobian ideal associated to a natural embedding of $T_K$ into an induced torus \cite{chai-yu:01a}*{\S 8.1}) and show  that if $n > 3h$ and $T_K$ and $M_L$ are $n$-congruent then there is a canonical isomorphism of smooth group schemes
 $
\Gr_{n-3h-1}(T) \to \Gr_{n-3h-1}(M)
 $
 determined by $\alpha, \beta$ and $\phi$ \cite{chai-yu:01a}*{Thm. 8.5}.
Combining this with the paragraph above gives the following instance of the geometrization and categorification of the transfer of quasicharacters.

\begin{proposition}\label{prop:transfer}
 With notation above, suppose that $T_K$ and $M_L$ are $n$-congruent and $n > 3h$.  Set $m = n-3h$.
 Then there is a canonical equivalence of categories
 \[
 \CS(\Gr^{\OK}_{m-1}(T)) \iso \CS(\Gr^{\OL}_{m-1}(M))
 \]
 determined by $\alpha, \beta$ and $\phi$ inducing an isomorphism
 \[
\Hom_{m-1}(T(K), \EEx) \iso  \Hom_{m-1}(M(L), \EEx)
 \]
compatible with \eqref{transfer}.
\end{proposition}


\subsection{Examples}

\bibliography{bibliography/Biblio}   

\end{document}