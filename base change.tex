% !TEX encoding = UTF-8 Unicode
\documentclass[11pt]{amsart}
\title[Geometrization leftovers]{Geometrization leftovers}
\usepackage[british]{babel}
\usepackage{datetime}
\date{\today}
\author{Clifton Cunningham}
\address{University of Calgary}
\email{cunning@math.ucalgary.ca}
\author{David Roe}
\address{Pacific Institute for the Mathematical Sciences at the University of Calgary}
\email{roed.math@gmail.com}
\usepackage[utf8]{inputenc}
\renewcommand{\baselinestretch}{1.2}
\usepackage[notcite,color]{showkeys}
\usepackage{hyperref}
\usepackage{geometry}
\usepackage{amsthm}
\usepackage{amsmath}
\usepackage{amssymb}
%\usepackage[shortalphabetic]{amsrefs}
\usepackage[alphabetic]{amsrefs}
\renewcommand\MR{\relax}
\usepackage{xypic}
\usepackage{textcomp}
\usepackage{mathrsfs}
\usepackage{yfonts}
\newcommand{\mathswab}[1]{\operatorname{\textswab{#1}}}
\usepackage[T1]{fontenc}
\usepackage{tikz}
\usetikzlibrary{shapes,arrows,calc,matrix}
\usepackage{tikz-cd}
\usepackage{manfnt}

%\include{definitions}
\theoremstyle{plain}
      \newtheorem{theorem}{Theorem}[section]
      \newtheorem{proposition}[theorem]{Proposition}
      \newtheorem{lemma}[theorem]{Lemma}
      \newtheorem{corollary}[theorem]{Corollary}
      
      \theoremstyle{definition}
      \newtheorem{definition}[theorem]{Definition}
      
      \theoremstyle{remark}
      \newtheorem{remark}[theorem]{Remark}

%% Global TikZ settings

\tikzset{every picture/.style={>=stealth},label/.style={font=\footnotesize}}

%%%% Some Named Categories

%%%% Functors

\newcommand{\gal}[1]{{\operatorname{Gal}\hskip-1pt\left( {\bar #1}/#1 \right)}}
\newcommand{\Spec}[1]{{\operatorname{Spec}\hskip-1pt( #1 )}}

\newcommand{\FF}{{\mathbb{F}}}
\newcommand{\ZZ}{{\mathbb{Z}}}
\newcommand{\NN}{{\mathbb{N}}}
\newcommand{\CC}{{\mathbb{C}}}
\newcommand{\QQ}{{\mathbb{Q}}}
\newcommand{\RR}{{\mathbb{R}}}
\newcommand{\EE}{\mathbb{\bar Q}_\ell}
\newcommand{\OK}{\mathcal{O}_K}
\newcommand{\pK}{\mathfrak{p}_K}
\newcommand{\OL}{\mathcal{O}_L}
\newcommand{\OO}[1]{\mathcal{O}_{#1}}
\newcommand{\Zp}{\mathbb{Z}_p}
\newcommand{\Qp}{\mathbb{Q}_p}
%\newcommand{\Fp}{\mathbb{F}_p}
\newcommand{\bFq}{\bar{k}}
\newcommand{\Fq}{k}
\newcommand{\Fqm}{k_m}
%\newcommand{\bFp}{{\mathbb{\bar F}_p}}

\newcommand{\EEx}{\EE^\times}

\DeclareMathOperator{\Gal}{Gal}
\DeclareMathOperator{\W}{W}
\newcommand{\Frob}{{\operatorname{Fr}}}
\DeclareMathOperator{\Aut}{Aut}
\DeclareMathOperator{\Hom}{Hom}
\DeclareMathOperator{\ord}{ord}
\DeclareMathOperator{\coker}{coker}
\DeclareMathOperator{\Gr}{Gr}
\DeclareMathOperator{\Irrep}{Irrep}
\DeclareMathOperator{\Pic}{Pic}
\DeclareMathOperator{\id}{id}
\DeclareMathOperator{\Ext}{Ext}
\DeclareMathOperator{\Hh}{H}
\DeclareMathOperator{\Res}{Res}
\DeclareMathOperator{\Nm}{Nm}

\newcommand{\cdef}[1]{{{\color{cyan}#1}\index{#1}}} 
\newcommand{\sheafHom}{{\mathscr{H}\hskip-4pt{\it o}\hskip-2pt{\it m}}}
\newcommand{\abs}[1]{{\vert #1 \vert}}
\newcommand{\ceq}{{\, :=\, }}
\newcommand{\tq}{{\ \vert\ }}
\newcommand{\iso}{{\ \cong\ }}
\newcommand{\obj}{{\text{obj}\, }}
\newcommand{\Gm}[1]{\mathbb{G}_{\hskip-1pt\textbf{m},#1}}
\newcommand{\GN}[1]{\mathswab{#1}}
\newcommand{\bGN}[1]{{\bar{\mathswab{#1}}}}
\newcommand{\TT}{\underline{T}}
\newcommand{\TL}{\underline{T_L}}
\newcommand{\invlim}[1]{\lim\limits_{\overleftarrow{#1}}}
\newcommand{\dirlim}[1]{\lim\limits_{\overrightarrow{#1}}}
\newcommand{\limit}[1]{\mathop{\textsc{lim}}\limits_{#1}}
\newcommand{\colimit}[1]{\mathop{\textsc{colim}}\limits_{#1}}
\newcommand{\cs}[1]{{\mathcal{#1}}}
\newcommand{\gcs}[1]{{\mathcal{\bar #1}}}
\newcommand{\dualgcs}[1]{\gcs{#1}^\dagger}
\newcommand{\dualcs}[1]{\cs{#1}^\dagger}
\newcommand{\CS}{{\mathcal{C\hskip-1.8pt S}}}
\newcommand{\GC}{{\mathcal{G\hskip-0.8pt C}}}
\newcommand{\CSiso}[1]{\CS(#1)_{/\textit{iso}}}
\newcommand{\CSbiso}[1]{\CS_0(#1)_{/\text{iso}}}
\newcommand{\CE}{{\mathcal{C\hskip-1.8pt E}}}
\newcommand{\Lgroup}[1]{{\,^L\hskip-1pt{#1}}}
\newcommand{\dualgroup}[1]{{\check{#1}}}
\newcommand{\Lang}{{\operatorname{Lang}}}
\newcommand{\image}{{\operatorname{im}}}
\newcommand{\Weil}[1]{\mathcal{W}_{#1}}
\newcommand{\level}[1]{(\mathrm{level} N)}

\makeatletter
\newcommand{\labitem}[2]{%
\def\@itemlabel{\textbf{#1}}
\item
\def\@currentlabel{#1}\label{#2}}
\makeatother

\renewcommand{\bf}{\bar{f}}
\newcommand{\bg}{\bar{g}}
\newcommand{\bm}{\bar{m}}
\newcommand{\bG}{\bar{G}}
\newcommand{\bH}{\bar{H}}
\newcommand{\tight}[3]{\hspace{-#1pt}{#2}\hspace{-#3pt}}
\newcommand{\GxG}{\text{$G \tight{1}{\times}{1} G$}}
\newcommand{\bGxG}{\text{$\bar{G} \tight{1}{\times}{1} \bar{G}$}}
\newcommand{\bfxf}{\text{$\bar{f} \tight{1}{\times}{1} \bar{f}$}}
\newcommand{\GxxG}{\text{$G \tight{1}{\times}{1} G$}}
\newcommand{\LxL}{\text{$\gcs{L} \tight{0}{\boxtimes}{0} \gcs{L}$}}
\newcommand{\ExE}{\text{$\cs{E}\tight{0}{\boxtimes}{0}\cs{E}$}}
\newcommand{\bExE}{\text{$\gcs{E}\tight{0}{\boxtimes}{0}\gcs{E}$}}
\newcommand{\AxA}{\text{$A \tight{1}{\times}{1} A$}}
\newcommand{\BxB}{\text{$B \tight{1}{\times}{1} B$}}
\newcommand{\GzxGz}{\text{$G^0 \tight{1}{\times}{1} G^0$}}

\newcommand\todo[1]{\ \vspace{5mm}\par \noindent\framebox{\begin{minipage}[c]{0.95 \textwidth} \tt #1\end{minipage}} \vspace{5mm} \par}
\newcommand\Clifton[1]{\marginpar{\smaller\smaller CC: #1}}
\newcommand\David[1]{\marginpar{\smaller\smaller DR: #1}}

\begin{document}

\section{Restriction of scalars}

In the opposite direction, there is a canonical closed immersion of $k$-schemes
$G \hookrightarrow \Res_{k'/k}(G_{k'})$ \cite[\S 7.6]{BLR}.  Pullback yields a functor
\[
\CS(\Res_{k'/k}(G_{k'})) \xrightarrow{\iota} \CS(G).
\]
\begin{proposition} \label{prop:csbe}
The following diagram commutes, where the bottom map is induced by the inclusion $G(k) \hookrightarrow G(k')$:
\[
\begin{tikzcd}
\CSiso{\Res_{k'/k}(G_{k'})} \arrow[two heads]{r}{\iota} \dar{t_{\Res_{k'/k}(G_{k'})}} & \CSiso{G} \dar{t_G} \\
\Hom(G(k'), \EEx) \arrow[two heads]{r} & \Hom(G(k), \EEx).
\end{tikzcd}
\]
\end{proposition}

Now suppose that $G'$ is an abelian group scheme over $k'$, satisfying the conditions of Theorem \ref{thm:snake}.
There is a canonical isomorphism between $G'(k')$ and $(\Res_{k'/k}G')(k)$;
we now describe the corresponding functor from
$\CS(G')$ to $\CS(\Res_{k'/k}G')$.  The base change $(\Res_{k'/k}G')_{k'}$ of $\Res_{k'/k}G'$ to $k'$
decomposes into a product of copies of $G'$, indexed by elements of $\Gal(k'/k)$:
\[
(\Res_{k'/k}G')_{k'} \cong \prod_{\Gal(k'/k)} G'.
\]
The component corresponding to the identity element yields a natural inclusion 
\[
G' \hookrightarrow (\Res_{k'/k}G')_{k'}
\]
of $k'$-schemes.  The definition of base change yields a map of $k$-schemes
\[
(\Res_{k'/k}G')_{k'} \to \Res_{k'/k}G'.
\]
Consecutive pullback along these maps yields a functor
\[
\CS(G') \to \CS(\Res_{k'/k}G').
\]

\begin{proposition}
  Then the following diagram commutes:
\[
\begin{tikzcd}
\CSiso{G'} \dar{t_{G'}} \rar & \CSiso{\Res_{k'/k} G'} \dar{t_{\Res_{k'/k} G'}} \\
 \Hom(G'(k'), \EEx) \rar & \Hom((\Res_{k'/k}G')(k), \EEx)
\end{tikzcd}
\]
\end{proposition}

\section{Base change over the local field}

We have already seen that the trace of Frobenius defines a canonical isomorphism 
$t_T : \CSiso{\GN{T}} \to \Hom_\text{ad}(T(K),\EEx)$. 
In this section we expand on the canonicity of this group isomorphism 
by showing that it is a natural transformation. We also show 

\begin{proposition}
The isomorphism $t_T : \CSiso{\GN{T}} \to \Hom_\text{ad}(T(K),\EEx)$ 
is a natural transformation between the following two contravariant functors 
from the category of algebraic tori over $K$ to the category of abelian groups: 
\begin{enumerate}
\item $T \mapsto \CSiso{\GN{T}}$
\item $T \mapsto \Hom_\text{ad}(T(K),\EEx)$.
\end{enumerate}
\end{proposition}

\Clifton{I think I can prove this proposition without too much difficulty.}
\begin{proof}
\end{proof}

Let $L/K$ be a finite Galois extension and let $K'/K$ be the maximal unramified subextension, with residue field $k'$.  If $T$ is a torus over $K$, write $\TL$ for the N\'eron model of the base change $T_L$.  Then there is a canonical isomorphism
\[
\Res_{k'/k} \Gr(\TL) \cong \Gr(\underline{\Res_{L/K}(T_L)}).
\]
The natural inclusion $T \hookrightarrow \Res_{L/K} T_L$ induces a map of $k$-schemes
\[
\GN{T} \hookrightarrow \Res_{k'/k} \Gr(\TL)
\]
and thus a functor on character sheaves through pullback.

\begin{proposition}
The following diagram commutes:
\[
\begin{tikzcd}
\CSiso{\Res_{k'/k} \Gr(\TL)} \arrow[two heads]{r} \dar & \CSiso{\GN{T}} \dar\\
\Hom(T(L), \EEx) \arrow[two heads]{r} & \Hom(T(K), \EEx)
\end{tikzcd}
\]
\end{proposition}

We now translate the norm map $T(L) \to T(K)$ to character sheaves.  There is an open immersion
\[
\TT \times_{\OK} \OL \to \TL
\]
induced by the canonical isomorphism of generic fibers \cite[p. 1]{Chai}.  After taking Greenberg transforms,
we obtain a diagram
\[
\begin{tikzcd}
\Gr(\TT) \dar & \Gr(\TT) \times_{\Gr(\OK)} \Gr(\OL) \lar{\alpha} \rar{\beta} \dar & \Gr(\TL) \dar \\
\Spec{k} & \Spec{k'} \lar \rar & \Spec{k'}.
\end{tikzcd}
\]
In the case that $L/K$ is unramified, the left hand square is a pullback square and the right hand horizontal maps are isomorphisms.  In general, we may define a functor
\begin{align*}
\CS(\GN{T}) &\to \CS(\Gr(\TL)) \\
\cs{L} &\mapsto \beta_* \alpha^* \cs{L}.
\end{align*}

\begin{proposition}
The following diagram commutes:
\[
\begin{tikzcd}
\CSiso{\GN{T}} \rar{\beta_* \circ \alpha^*} \dar & \CSiso{\Gr(\TL)} \dar \\
\Hom(T(K), \EEx) \rar{\Nm} & \Hom(T(L), \EEx)
\end{tikzcd}
\]
\end{proposition}

\begin{bibdiv}
\begin{biblist}

\bib{BLR}{book}{
   author={Bosch, Siegfried},
   author={L{\"u}tkebohmert, Werner},
   author={Raynaud, Michel},
   title={N\'eron models},
   series={Ergebnisse der Mathematik und ihrer Grenzgebiete (3) [Results in
   Mathematics and Related Areas (3)]},
   volume={21},
   publisher={Springer-Verlag},
   place={Berlin},
   date={1990},
%   pages={x+325},
%   isbn={3-540-50587-3},
%   review={\MR{1045822 (91i:14034)}},
}

\bib{Chai}{article}{
   author={Chai, C.L.},
   title={N\'eron models for semiabelian varieties: congruence and change of base field.},
   journal={Asian J. Math.},
   volume={4},
   number={4},
   date={2000},
   pages={715--736}}

\end{biblist}
\end{bibdiv}

\end{document}
