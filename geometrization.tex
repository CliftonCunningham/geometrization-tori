% !TEX encoding = UTF-8 Unicode
%\documentclass{compositio}
\documentclass{amsart}
\pdfoutput=1

%%%%%%%%%%%%%%%%%% HEADING %%%%%%%%%%%%%%%%%%
\title[Geometrization of the local Langlands correspondence for tori]{Geometrization of the local Langlands correspondence for algebraic tori}
\date{\today}
\author{Clifton Cunningham}
\email{cunning@math.ucalgary.ca}
\address{University of Calgary}
\author{David Roe}
\email{roed.math@gmail.com}
\address{Pacific Institute for the Mathematical Sciences at the University of Calgary}
%\classification{14L15 (primary), 11G25, 14F05, 14D24, 11S31 (secondary)}
%\keywords{quasicharacters, quasicharacter sheaves, algebraic tori, Greenberg transforms, N\'eron models, local fields, group schemes}

%%%%%%%%%%%%%%%%% PACKAGES %%%%%%%%%%%%%%%%%
% Encodings
\usepackage[utf8]{inputenc}
\usepackage[T1]{fontenc}
% AMS packages
\usepackage{amsmath, amsthm, amssymb}
\usepackage[alphabetic]{amsrefs}
\usepackage{todonotes}
% Fonts
\usepackage{mathrsfs, yfonts}
% TikZ
\usepackage{tikz}
\usetikzlibrary{shapes,arrows,calc,matrix}
\usepackage{tikz-cd}
% Hyperrefs
\usepackage{hyperref}

%%%%%%%%%%%%%%% THEOREM STYLES %%%%%%%%%%%%%%%
\theoremstyle{plain}
      \newtheorem{theorem}{Theorem}[section]
      \newtheorem{proposition}[theorem]{Proposition}
      \newtheorem{lemma}[theorem]{Lemma}
      \newtheorem{corollary}[theorem]{Corollary}

      \theoremstyle{definition}
      \newtheorem{definition}[theorem]{Definition}

      \theoremstyle{remark}
      \newtheorem{remark}[theorem]{Remark}

%%%%%%%%%%%%%%%% TIKZ SETTINGS %%%%%%%%%%%%%%%%
\tikzset{every picture/.style={>=stealth},label/.style={font=\footnotesize}}

%%%%%%%%%%%%%%% RINGS AND GROUPS %%%%%%%%%%%%%%%
\newcommand{\FF}{{\mathbb{F}}}
\newcommand{\ZZ}{{\mathbb{Z}}}
\newcommand{\NN}{{\mathbb{N}}}
\newcommand{\CC}{{\mathbb{C}}}
\newcommand{\QQ}{{\mathbb{Q}}}
\newcommand{\RR}{{\mathbb{R}}}
\newcommand{\EE}{\mathbb{\bar Q}_\ell}
\newcommand{\Zl}{\mathbb{\bar Z}_\ell}
\newcommand{\OK}{\mathcal{O}_K}
\newcommand{\pK}{\mathfrak{p}_K}
\newcommand{\OKp}{\mathcal{O}_{K'}}
\newcommand{\pKp}{\mathfrak{p}_{K'}}
\newcommand{\OL}{\mathcal{O}_L}
\newcommand{\OO}[1]{\mathcal{O}_{#1}}
\newcommand{\Zp}{\mathbb{Z}_p}
\newcommand{\Qp}{\mathbb{Q}_p}
\newcommand{\bFq}{\bar{k}}
\newcommand{\Fq}{k}
\newcommand{\Fqm}{k_m}
\newcommand{\WW}{\mathbb{W}}
\newcommand{\EEx}{\EE^\times}
\newcommand{\Zlx}{\mathbb{\bar Z}_\ell^\times}
\newcommand{\Weil}[1]{\mathcal{W}_{#1}}
\newcommand{\comp}{\Pi} % Component group

%%%%%%%%%%%%%%% ALGEBRAIC GROUPS %%%%%%%%%%%%%%%
\newcommand{\mathswab}[1]{\operatorname{\textswab{#1}}}
\newcommand{\Gm}[1]{\mathbb{G}_{\hskip-1pt\textbf{m},#1}}
\newcommand{\GN}[1]{\mathswab{#1}}
\newcommand{\bGN}[1]{{\bar{\mathswab{#1}}}}
\newcommand{\TT}{\underline{T}}
\newcommand{\TL}{\underline{T_L}}
\newcommand{\TTp}{\underline{T}'}

%%%%%%%%%%%%%%% NAMED OPERATORS %%%%%%%%%%%%%%%
\DeclareMathOperator{\Gal}{Gal}
\DeclareMathOperator{\W}{W}
\newcommand{\Frob}[1]{\operatorname{F}_{#1}}
\DeclareMathOperator{\Aut}{Aut}
\DeclareMathOperator{\Hom}{Hom}
\DeclareMathOperator{\ord}{ord}
\DeclareMathOperator{\coker}{coker}
\DeclareMathOperator{\Gr}{Gr}
\DeclareMathOperator{\Irrep}{Irrep}
\DeclareMathOperator{\Pic}{Pic}
\DeclareMathOperator{\id}{id}
\DeclareMathOperator{\Ext}{Ext}
\DeclareMathOperator{\Hh}{H}
\DeclareMathOperator{\Res}{Res}
\DeclareMathOperator{\Nm}{Nm}
\DeclareMathOperator{\trace}{Tr}
\DeclareMathOperator{\obj}{obj}
\DeclareMathOperator{\mor}{mor}
\DeclareMathOperator{\Lang}{Lang}
\DeclareMathOperator{\image}{im}
\DeclareMathOperator{\Loc}{Loc}
\newcommand{\gal}[1]{{\operatorname{Gal}\hskip-1pt\left( {\bar #1}/#1 \right)}}
\newcommand{\Spec}[1]{{\operatorname{Spec}\hskip-1pt( #1 )}}

%%%%%%%%%%%% MISCELLANEOUS OPERATORS %%%%%%%%%%%%
\newcommand{\sheafHom}{{\mathscr{H}\hskip-4pt{\it o}\hskip-2pt{\it m}}}
\newcommand{\abs}[1]{{\vert #1 \vert}}
\newcommand{\ceq}{{\, :=\, }}
\newcommand{\tq}{{\ \vert\ }}
\newcommand{\iso}{{\ \cong\ }}
\newcommand{\dual}[1]{{\hat{#1}}}
%% Limits
\newcommand{\invlim}[1]{\lim\limits_{\overleftarrow{#1}}}
\newcommand{\dirlim}[1]{\lim\limits_{\overrightarrow{#1}}}
\newcommand{\limit}[1]{\mathop{\textsc{lim}}\limits_{#1}}
\newcommand{\colimit}[1]{\mathop{\textsc{colim}}\limits_{#1}}
%% Fonts for quasicharacter sheaves
\newcommand{\qcs}[1]{{\mathcal{#1}}}
\newcommand{\gqcs}[1]{{\mathcal{\bar #1}}}
\newcommand{\dualgcs}[1]{\gqcs{#1}^\dagger}
\newcommand{\dualcs}[1]{\qcs{#1}^\dagger}
%% Categories
\newcommand{\QC}{{\mathcal{Q\hskip-0.8pt C}}}
\newcommand{\QCb}{{\QC_0}}
\newcommand{\QCf}{{\QC_f}}
\newcommand{\QCiso}[1]{\QC(#1)_{/\text{iso}}}
\newcommand{\QCbiso}[1]{\QCb(#1)_{/\text{iso}}}
\newcommand{\QCfiso}[1]{\QCf(#1)_{/\text{iso}}}
\newcommand{\trFrob}[1]{t_{#1}}
\newcommand{\TrFrob}[1]{T_{#1}}
%% amsmath replacements for \atop
\newcommand{\latop}[2]{\genfrac{}{}{0pt}{0}{#1}{#2}}
\newcommand{\partop}[2]{\genfrac{(}{)}{0pt}{0}{#1}{#2}}
%% Labeled items
\makeatletter
\newcommand{\labitem}[2]{
\def\@itemlabel{\textbf{#1}}
\item
\def\@currentlabel{#1}\label{#2}}
\makeatother
%% Shorthand for bars
\renewcommand{\bf}{\bar{f}}
\newcommand{\bg}{\bar{g}}
\newcommand{\bm}{\bar{m}}
\newcommand{\bG}{\bar{G}}
\newcommand{\bH}{\bar{H}}
%% Spacing control
\newcommand{\tight}[3]{\hspace{-#1pt}{#2}\hspace{-#3pt}}
\newcommand{\GxG}{\text{$G \tight{1}{\times}{1} G$}}
\newcommand{\bGxG}{\text{$\bar{G} \tight{1}{\times}{1} \bar{G}$}}
\newcommand{\bfxf}{\text{$\bar{f} \tight{1}{\times}{1} \bar{f}$}}
\newcommand{\GxxG}{\text{$G \tight{1}{\times}{1} G$}}
\newcommand{\LxL}{\text{$\gqcs{L} \tight{0}{\boxtimes}{0} \gqcs{L}$}}
\newcommand{\ExE}{\text{$\qcs{E}\tight{0}{\boxtimes}{0}\qcs{E}$}}
\newcommand{\bExE}{\text{$\gqcs{E}\tight{0}{\boxtimes}{0}\gqcs{E}$}}
\newcommand{\AxA}{\text{$A \tight{1}{\times}{1} A$}}
\newcommand{\BxB}{\text{$B \tight{1}{\times}{1} B$}}
\newcommand{\GzxGz}{\text{$G^0 \tight{1}{\times}{1} G^0$}}
%% Margin notes
\newcommand\Clifton[1]{\marginpar{\smaller\smaller CC: #1}}
\newcommand\David[1]{\marginpar{\smaller\smaller DR: #1}}
%% Hyphenation override
\hyphenation{quasi-character}

%%%%%%%%%%%% BEGIN DOCUMENT %%%%%%%%%%%%
\begin{document}

\begin{abstract}
For any non-archimedean local field $K$
  and any algebraic torus $T$ over $K$, we use the
  function--sheaf dictionary for smooth group schemes over finite fields
  to geometrize quasicharacters of $T(K)$
  as quasicharacter sheaves on the Greenberg transform of the Néron
  model of $T$.
 We offer two applications of this geometrization:
 a categorical transfer of quasicharacters between characteristic-$p$ tori and
 characteristic-$0$ tori; and a geoemtric proof of the local Langlands correspondence for algebraic tori.
  \end{abstract}

\maketitle

\section*{Introduction}



In this paper we explain how to geometrize quasicharacters of algebraic tori over local fields.
%
Let $K$ be a non-archimedian local field
with finite residue field $\Fq$; we place no restriction on the characteristic of $K$.
Write $\OK$ for the ring of integers of $K$ and $\pK$ for the maximal ideal in $\OK$.
Let $T$ be a torus defined over $K$.  The N\'eron model $\TT$ of $T$
is a smooth group scheme over $\OK$ with generic fibre $T$ and a canonical isomorphism
$\TT(\OK) \cong T(K)$.  For each positive integer $n$, let $\GN{T}_n$ be the
Greenberg transform of $\TT \times_{\Spec{\OK}} \Spec{\OK/\pK^n}$. Then $\GN{T}_n$ is a smooth commutative group scheme over $\Fq$; as such, it carries the category $\QC(\GN{T}_n)$ of quasicharacter sheaves, introduced in \cite{cunningham-roe:dictionary}. The main result in this paper, Theorem~\ref{thm:application}, provides a functorial translation of characters $T(K)$ of depth less than $n$ into quasicharacter sheaves on $\GN{T}_n$:
\[
\QCiso{\GN{T}_n}
\iso
\Hom_{< n}(T(K),\EEx).
\]
We also describe a pro-group scheme $\GN{T}$
simultaneously geometrizing all quasicharacters ({\it i.e.}, finite-depth characters) of $T(K)$.

\[
\begin{tikzcd}
\ & \arrow[swap]{dl}{\trFrob{G}} \QCiso{\GN{T}} \arrow[dashed]{dr} & \\
\Hom_{\text{qc}}(T(K),\EEx) \arrow{rr}{\text{rec}_T} && H^1(K,\hat{T}_\ell)
\end{tikzcd}
\]


While we have not yet fully explored the applications of such a geometrization, we want to highlight
some of the potential benefits of Theorem~\ref{thm:application} here.
Characters on $T$ have limited
functoriality: they can be pulled back along maps to $T$, but it is less clear what to do with them
in the presence of maps from $T$ to other groups.  Sheaves, on the other hand, come
equipped with a host of functors, including pushforward.
We hope to use this to our advantage as we work toward an adaptation of
Lusztig's theory of characters sheaves to the study of characters of
admissible representations of reductive $p$-adic groups that extends beyond the case of $p$-adic tori.

DeBacker and Reeder have recently applied Deligne-Lusztig
representations to part of the depth-zero part of the
local Langlands correspondence \cite{reeder-debacker:09a}.
Deligne-Lusztig theory also plays an important role in
Waldspurger's work on nilpotent distributions \cite{waldspurger:01}.
After developing Deligne-Lusztig representations,
the next step Lusztig took in studying representations
of finite groups of Lie type was to introduce character sheaves for reductive groups over $\Fq$.
We hope that our quasicharacter sheaves
can provide a stepping stone to a similar theory for
representations of $p$-adic groups, especially
in the positive-depth case.

The scope of the Langlands program has grown explosively in the past several decades, and
different parts of it apply to different kinds of representations.  In particular, the classic local Langlands
correspondence addresses complex (or $\ell$-adic) representations of real and $p$-adic groups, while
the geometric Langlands program aims at representations of groups in characteristic $p$.  The fact that
characteristic $p$ group schemes arise as the space underlying quasicharacter sheaves, regardless of
the characteristic of $K$, provides a potential bridge allowing the methods of the geometric Langlands
program to be applied to the classic context.

Finally, we are tantalized by the possibility of providing an alternate perspective on local class field theory
through quasicharacter sheaves.  The local Langlands correspondence for tori provides an isomorphism
$
\Hh^1(\Weil{K}, \hat{T}_\ell) \cong \Hom(T(K), \EEx)
$,
where $\Weil{K}$ is the Weil group for $K$.
A geometric description of the resulting isomorphism $\Hh^1(\Weil{K}, \hat{T}_\ell) \cong \QCiso{\GN{T}}$
would yield a different description of the Artin reciprocity map.

We close the introduction with a more detailed description of the structure of this paper.
In Section~\ref{sec:GN} we review the Greenberg transform of N\'eron models of tori, which produces
group schemes over $\Fq$ from algebraic tori $T$ over $K$.
Section~\ref{sec:bdchar} then uses \cite{cunningham-roe:dictionary} to give a geometrization of characters of $T$
of bounded depth.
We give a reinterpretation of this result in Section~\ref{sec:quasichar}, using a limit process to simultaneously
geometrize all quasicharacters of $T$.  Sections~\ref{sec:bdchar}, \ref{sec:quasichar} and \ref{sec:wrK} also include
functoriality results that we hope to use in a future discussion of
local class field theory.
In Section~\ref{sec:transfer} we discuss a transfer principle which allows
a geometric transfer of characters between characteristic $p$ tori and characteristic $0$ tori.
We close the paper with Section~\ref{sec:reciprocity}, in which we suggest that quasicharacter
sheaves may play a key role in a geometric and categorical form of the local Langlands
correspondence for algebraic tori over local fields.

%\begin{acknowledgements}
\subsection*{Acknowledgements}
%We would like to thank Masoud Kamgarpour, Pramod Achar, and Hadi Salmasian for allowing us to hijack much of our Research in Teams meeting at BIRS into a discussion of quasicharacter sheaves; their knowledge and help have been invaluable.
We thank Alessandra Bertrapelle and Cristian Gonz\'ales-Avil\'es for disabusing us of a misapprehension concerning the Greenberg realization functor and for pointing us to a proof of Lemma~\ref{lemma:wrK}.

We also gratefully acknowledge the financial support of the Pacific Institute for the Mathematical Sciences
and the National Science and Engineering Research Council.
%as well the hospitality of the Banff International Research Station during our weeklong stay in May 2012.

%\end{acknowledgements}

\tableofcontents


\section{Quasicharacters of algebraic tori over local fields}

Let $\Fq$ be a finite field and let $K$ be a non-archimedean local field with residue field $\Fq$.
We place no restriction on the characteristic of $K$.
We write $\OK$ for the ring of integers of $K$ and $\pK$ for the maximal ideal in $\OK$.
Let $T$ be an algebraic torus over $K$; the filtrations defined by Moy-Prasad and Yu \cite{yu:03a}*{\S 5}
allow us to define the depth of character $\chi$ as the infimum of the $r$ so that $\chi$ vanishes on $T(K)_r$.
We write
\[
\Hom_{< n}(T(K),\EEx)
\]
for the group of characters of $T(K)$ depth less than $n$
and
\[
\Hom_{\text{qc}}(T(K),\EEx)
\]
for the group of characters of $T(K)$ of arbitrary finite depth;
we call these quasicharacters of $T(K)$, following \cite{tate:67a}*{2.3}.

\section{Greenberg of N\'eron}
\label{sec:GN}

A N\'eron model for $T$ is a smooth group scheme $\TT$ over $\OK$ with generic fiber $T$
that satisfies the N\'eron mapping property: if $S$ is any smooth scheme over $\OK$ then every morphism
$S_\eta \to T$ over $K$ extends uniquely to a morphism $S \to \TT$.  N\'eron models exist for any torus
$T$ \cite{bosch-lutkebohmert-reynaud:NeronModels}*{Ch 10.1, Thm 6} and are unique up to isomorphism;
$\TT$ is separated, commutative, locally finite type over $\OK$ and satisfies $\TT(\OK) = T(K)$.

We will also need the Greenberg functor \citelist{\cite{greenberg:61} \cite{greenberg:63a}}:
a functor from the category of schemes locally of finite type
over an Artinian local ring with perfect residue field $k$
to the category of schemes locally of finite type over $k$.
This functor respects open immersions, closed immersions, \'etale morphisms, smooth morphisms and geometric components.
See
\citelist{
\cite{demazure-gabriel:GroupesAlgebriques}*{V, \S 4, no. 1}
\cite{bosch-lutkebohmert-reynaud:NeronModels}*{Ch. 9, \S 6}
\cite{nicaise-sebag:motivicSerre}*{\S 2.2}
\cite{bertrapelle-gonzales:Greenberg}*{\S 5}
}
for the definition and fundamental properties of the Greenberg functor as it is used here.

\begin{definition}\label{def:Grn}
For any local field $K$ with residue field $\Fq$ and for any positive integer $n$, let
\[
\Gr_n : \partop{\text{Schemes}}{\text{lft over $\OK$}} \to \partop{\text{Schemes}}{\text{lft over $\Fq$}}
\]
be the functor produced by precomposing pullback along $\OK \to \OK/\pK^n$ with the Greenberg functor.
\end{definition}

The surjective ring homomorphism $\OK/\pK^{n+1} \to \OK/\pK^n$ determines a
natural transformation $P_{n} : \Gr_{n+1} \to \Gr_n$ between additive functors.
For a given Néron model $\TT$ of $T$, set
\begin{align*}
 \GN{T}_n &\ceq \Gr_n(\TT), \\
 f_n &\ceq P_n(\TT).
\end{align*}

We list some fundamental properties of $f_n$ and $\GN{T}_n$ for further use:
 \begin{enumerate}
  \labitem{(GN.0)}{truncation}
  $f_n : \GN{T}_{n+1} \to \GN{T}_n$ is a surjective morphism of group schemes over $\Fq$
  \citelist{\cite{bosch-lutkebohmert-reynaud:NeronModels}*{p. 276} \cite{bertrapelle-gonzales:Greenberg}*{Prop. 7.2}}.
  We remark that $f_n : \GN{T}_{n+1} \to \GN{T}_n$ is not induced from a morphism of Néron models.
  \labitem{(GN.1)}{smooth}
  $\GN{T}_n$ is a $\QC$-scheme.
  Note that $\underline{T}$ is a smooth group scheme over $\OK$, by construction.
  Since smoothness is preserved under base change, the group scheme
  $\TT_n \ceq \underline{T} \times_\Spec{\OK} \Spec{\OK/\pK^n}$
  is smooth over an Artinian ring with perfect residue field.
  The Greenberg functor preserves smoothness,
  so $\GN{T}_n$ is a smooth group scheme locally of finite type over $\Fq$.
  The Greenberg functor takes the geometric identity component of $\TT_n$
  to the geometric identity component of $\GN{T}_n$.
  In fact, $\pi_0(\GN{T}_n)$ is independent of $n$, equal to $\comp \ceq \pi_0(\TT) \times_{\Spec{\OK}} \Spec{\Fq}$.
  Moreover, $\Pi$ is finitely generated \cite{xarles:93a}*{2.18}.
  \labitem{(GN.2)}{identity} $\GN{T}_n^\circ$ is affine of finite type over $\Fq$.
  \labitem{(GN.3)}{points} $\GN{T}_n(\Fq) = \TT(\OK/\pK^n)$, canonically.
\end{enumerate}

\begin{proposition}
The projective limit of the system
\[
\cdots \xrightarrow{f_n} \GN{T}_{n+1} \xrightarrow{f_{n-1}} \cdots \xrightarrow{f_1} \GN{T}_1
\]
exists in the category of group schemes over $\Fq$.
\end{proposition}
\begin{proof}
Since the component group $\comp$ is independent of $n$, the projective system induces a sequence of affine schemes
\begin{equation} \label{eq:Ti}
\cdots \xrightarrow{f^i_n} \GN{T}^i_{n+1} \xrightarrow{f^i_{n-1}} \cdots \xrightarrow{f^i_1} \GN{T}^i_1
\end{equation}
for each $i \in \comp$.
Since the category of affine schemes over $k$ is equivalent to the category of $k$-algebras, which has direct limits \todo{citation},
the limit of \eqref{eq:Ti} exists in the category of affine group schemes over $k$ for each $i \in \comp$.  Write $\GN{T}^i$ for the limit,
and define
\[
\GN{T} \ceq \coprod_{i \in \comp} \GN{T}^i
\]
as a $k$-scheme.  We claim that $\GN{T}$ is a projective limit of the original system, in the category of $k$-schemes.  Suppose that $Z$ is a $k$-scheme
with a system of morphisms $z_n : Z \to \GN{T}_n$.  Let $\alpha : \pi_0(Z) \to \comp$ be the induced map on component groups, which is
independent of $n$.  \todo{the components of $Z$ are not necessarily affine; is the projective limit in the category of affine schemes still the limit in the category of schemes?}


The multiplication maps $m_n : \GN{T}_n \times \GN{T}_n \to \GN{T}_n$ induce maps of affine schemes
$m_n^{i,j} : \GN{T}_n^i \times \GN{T}_n^j \to \GN{T}_n^{i+j},$ inducing maps $m^{i,j} : \GN{T}^i \times \GN{T}$ [under construction]
\end{proof}

\section{The function--sheaf dictionary for quasicharacters of bounded depth}\label{sec:qcs}

In \cite{cunningham-roe:dictionary} we extended the function--sheaf dictionary of Deligne  \cite{deligne:SGA4.5}*{sommes trig.} to smooth commutative group schemes locally of finite type over finite fields for which the geometric component group is finitely generated.
Since we will use this dictionary here, we briefly review the main notation and result from that paper.

Let $G$ be a smooth commutative group schemes locally of finite type over $\Fq$ for which the geometric component group is finitely generated. A quasicharacter sheaf on $G$ is a triple $\qcs{L}\ceq
(\gqcs{L},\mu,\phi)$ where $\gqcs{L}$ is an $\ell$-adic local system on $\bG$ of rank~one and $\mu: m^*
\gqcs{L} \to \LxL$ and $\phi : \Frob{G}^*\gqcs{L} \to \gqcs{L}$ are isomorphisms satisfying natural compatibilty conditions; see \cite{cunningham-roe:dictionary}.
%
Quasicharacter sheaves on $G$ form a rigid monoidal category, denoted by $\QC(G)$
and the trace of Frobenius
\begin{equation}\label{snake}
  \trFrob{G} : \QCiso{G} \to G(\Fq)^*
\end{equation}
  is a functorial isomorphism of groups \cite{cunningham-roe:dictionary}*{Thm. 11.1}.


%\section{Application to quasicharacters of bounded depth} \label{sec:bdchar}

%We now apply the results of Section~\ref{sec:qcs} to characters of $T(K)$.

\begin{theorem}\label{thm:application}
Let $n$ be a positive integer.
For every algebraic torus $T$ over $K$, the trace of Frobenius
  \[
\trFrob{T,n}:  \QCiso{\GN{T}_n} \to \Hom_{< n}(T(K),\EEx)
  \]
is an isomomorphism.
Moreover, the isomorphism $\trFrob{T,n}$ is a natural transformation
between the additive functors
\begin{enumerate}
\item[$F_1$:] $T \mapsto \QCiso{\GN{T}_n}$
\item[$F_2$:] $T \mapsto \Hom_{<n}(T(K),\EEx)$
\end{enumerate}
from the additive category of algebraic tori over $K$ to the category of abelian groups.
\end{theorem}

\begin{proof}
  By \ref{smooth}, we may use \cite{cunningham-roe:dictionary}*{Thm. 11.1}
   to give the functorial isomorphism
  $\QCiso{\GN{T}_n} \iso \GN{T}_n(\Fq)^*$.
  By \ref{points}, $\GN{T}_n(\Fq) = \TT(\OK/\pK^n)$.
  Then we recall that the filtration of $T(K)$ used to
  define depth has the property that $\TT(\OK/\pK^n) \cong T(K) / T(K)_n$
  \cite{yu:03a}*{Prop 5.2}, so characters of $\GN{T}_n(\Fq)$
  are in bijection with characters of $T(K)$ vanishing on $T(K)_n$.
  This shows that $\trFrob{T,n}$ is an isomorphism.

  The definition of $F_1$ depends
  on the choice of a Néron model for every algebraic torus $T$.
  However, since Néron models are unique up to unique isomorphism,
  the isomorphism class of $\GN{T}_n$ is uniquely determined by $T$
  and the group of isomorphism classes of objects in $\QC(\GN{T}_n)$
  is uniquely determined by $T$.
  To define $F_1$ on morphisms one uses the Néron mapping property.

  Finally, we show that $\trFrob{T,n}$ is functorial.
The first part of \cite{cunningham-roe:dictionary}*{Prop. 4.1} shows that $F_1$ is a functor,
while the second part shows that trace of Frobenius is a natural transformation
$t_{ - ,n} : F_1 \to F_2$. When further combined with \cite{cunningham-roe:dictionary}*{Prop. 4.2},
we see that $F_1$ is an additive functor and $t_{ - ,n}: F_1 \to F_2$ is a
natural transformation between additive functors.
\end{proof}


\section{A function--sheaf dictionary for quasicharacters} \label{sec:quasichar}

In this section we package the isomorphisms of Theorem~\ref{thm:application} into
a single dictionary.

\begin{definition}\label{def:objGN}
The \emph{Greenberg transform of the N\'eron model of $T$}
is the pro-group scheme over $\Fq$ given by
$\GN{T} \ceq \varprojlim_{n\in \ZZ_{>0}} \GN{T}_{n}$,
with reference to the family of  surjective morphisms of
$\Fq$-group schemes $f_n : \GN{T}_{n+1} \to \GN{T}_n$.
\end{definition}

For any  ordered pair of positive integers $n< m$ define
$f_{n\leq m} : \GN{T}_{m} \to \GN{T}_{n}$
by $f_{n\leq m}\ceq f_n \circ f_{n+1} \circ \cdots \circ f_{m-1}$;
also set $f_{n\leq n} \ceq \id_{\GN{T}_n}$.
Note that $f_{n\leq m}:  \GN{T}_{m} \to \GN{T}_{n}$ is a surjective morphism
of group schemes over $\Fq$.
Let
\[
f_{n\leq m}^* : \QC(\GN{T}_n) \to \QC(\GN{T}_{m})
\]
be the functor produced by applying \cite{cunningham-roe:dictionary}*{Prop. 4.1} to $f_{n\leq m}$.
We now define a category which allows us to extend the notion
of quasicharacter sheaf to the pro-group scheme $\GN{T}$.

\begin{definition}\label{def:catGN}
An object in $\QC(\GN{T})$ is a triple $(n, \{\qcs{L}_i\}_{i \ge n}, \{\alpha_{i \le j}\}_{j \ge i \ge n})$, where $n$
is a positive integer, each $\qcs{L}_i$ is a quasicharacter sheaf on $\GN{T}_i$ and each
$\alpha_{i \le j} : \qcs{L}_j \to f^*_{i \le j} \qcs{L}_i$ is an isomorphism.  We require $\alpha_{i \le i}$ to be
the identity and that the $\alpha_{i \le j}$ are compatible with each other.  If $\qcs{L} \ceq (n, \{\qcs{L}_i\}, \{\alpha_{i \le j}\})$
and $\qcs{L}' \ceq (m, \{\qcs{L}'_i\}, \{\alpha'_{i \le j}\})$ are objects then $\Hom(\qcs{L}, \qcs{L}')$ is the set
of equivalence classes of pairs $(k, \{\beta_i\}_{i \ge k})$.  Here $k \ge m, n$ and the $\beta_i : \qcs{L}_i \to \qcs{L}'_i$ are
morphisms of quasicharacter sheaves so that
\[
\begin{tikzcd}
\qcs{L}_j \rar{\alpha_j} \dar{\beta_i} & f^*_{i \le j} \qcs{L}_i \dar{f_{i \le j}^*\beta_j} \\
\qcs{L}'_j \rar{\alpha_j} & f^*_{i \le j} \qcs{L}'_i
\end{tikzcd}
\]
commutes for all $j \ge i \ge k$.  We identify two such pairs $(k, \{\beta_i\})$ and $(l, \{\gamma_i\})$ if $\beta_i = \gamma_i$
for sufficiently large $i$.  Identities and composites are defined in the natural way.

If $n$ is a positive integer and $\qcs{L}_n \in \QC(\GN{T}_n)$ then we may define an object
\[
[\qcs{L}_n] \ceq (n, \{f_{n \le i} \qcs{L}_n\}_{i \ge n}, \{1_{\qcs{L}_j}\}_{j \ge i \ge n}).
\]
Every object in $\QC(\GN{T})$ is isomorphic to such a $[\qcs{L}_n]$.  Similarly, any morphism is equal to one of the form
\[
[\beta_k] \ceq (k, \{f_{k \le i}^* \beta_k\}_{i \ge k}).
\]

We may define a tensor product on $\QC(\GN{T})$ by setting
\[
 \qcs{L} \otimes \qcs{L}' \ceq (k, \{\qcs{L}_i \otimes \qcs{L}'_i\}, \{\alpha_{i \le j} \otimes \alpha'_{i \le j}\}),
\]
where $k = \max(m, n)$.
Similarly, the dual of an object $\qcs{L}$ is defined by
\[
\qcs{L}^\vee \ceq (n, \{\qcs{L}_i^\vee\}, \{ \alpha_{i \le j}^\vee \}).
\]
We refer to objects in the rigid monoidal category
$\QC(\GN{T})$ as \emph{quasicharacter sheaves on $\GN{T}$}
or \emph{quasicharacter sheaves for $T$}.
\end{definition}

\begin{corollary}\label{cor:application}
For every algebraic torus $T$ over $K$, the trace of Frobenius
  \[
\trFrob{T}:  \QCiso{\GN{T}} \to \Hom_{\text{qc}}(T(K),\EEx)
  \]
is an isomomorphism.
Moreover, the isomorphism $\trFrob{T}$ is a natural transformation
between the additive functors
\begin{enumerate}
\item[$F_{\QC}$:] $T \mapsto \QCiso{\GN{T}}$
\item[$F_{\text{qc}}$:] $T \mapsto \Hom_{\text{qc}}(T(K),\EEx)$
\end{enumerate}
from the additive category of algebraic tori over $K$ to the category of abelian groups.
\end{corollary}

\begin{proof}
We may define full and faithful functors $F_n : \QC(\GN{T}_n) \to \QC(\GN{T})$
for each $n$ by $F_n(\qcs{L}) \ceq [\qcs{L}_n]$ and $F_n(\alpha_n) \ceq [\alpha_n]$.
For positive integers $n \le m$ these functors fit together into a diagram
\[
\begin{tikzcd}
\ & & & \QC(\GN{T})\\
\arrow{urrr}{F_n} \QC(\GN{T}_n) \arrow[swap]{rr}{f_{n\leq m}^*} && \QC(\GN{T}_m) \arrow[swap]{ur}{F_m}. &
\end{tikzcd}
\]
Passing to isomorphism classes of objects and invoking the
universal property of direct limits gives a group isomorphism
$\theta: \varinjlim_{n\in \ZZ_{>0}} \QCiso{\GN{T}_n} \to \QCiso{\GN{T}}$
with inverse $[\qcs{L}]_{/\text{iso}} \mapsto [\qcs{L}_{/\text{iso}}]$.
Now use Theorem~\ref{thm:application}, \cite{cunningham-roe:dictionary}*{Prop. 4.1}
and the fact that  $\Hom_{\text{qc}}(T(K),\EEx) = \varinjlim_{n\in \ZZ_{>0}} \GN{T}_n(k)^*$,
where this direct limit is taken with respect to the family of group homomorphisms
dual to $f_{n} : \GN{T}_{n+1}(\Fq) \to \GN{T}_n(\Fq)$.
\end{proof}


\section{Restriction of scalars over local fields}
\label{sec:wrK}

As indicated in the Introduction, we are tantalized
by the idea that quasicharacter sheaves for $T$
may play a key role in a new geometric, categorial version
of the local Langlands correspondence for algebraic tori
over non-archimedean local fields.
As with the local Langlands correspondence for algebraic tori over local fields,
it should be possible to reduce the essential points of the argument to the case $T= \Gm{K}$,
provided that the category of quasicharacter sheaves behaves properly with regard to
Weil restriction, as in \cite{yu:09a}*{Thm 7.5 (2)}.
The next lemma shows that the category of quasicharacter sheaves for
algebraic tori does indeed behave properly with regard to Weil restriction.

\begin{lemma}\label{lemma:wrK}
Weil restriction is compatible with the Greenberg transform in the following sense:
if $e$ is the ramification degree of $K'/K$ and $k'$ is the residue field of $K'$ then,
for every positive integer $n$,
\begin{equation}
\Gr_{n}\left( \Res_{\OKp/\OK}(\TTp)\right)
\iso
\Res_{k'/k} \Gr_{en}(\TTp).
\end{equation}
\end{lemma}

\begin{proof}
First, note that since $\Res_{\OKp/\OK}(\TTp)$
is a scheme over $\OK$,
the Greenberg transform $\Gr_{n}\left( \Res_{\OKp/\OK}(\TTp)\right)$
is obtained by base change along $\OK \to \OK/\pK^n$
followed by the Greenberg functor over the Artinian ring $\OK/\pK^n$.
On the other hand, since $\TTp$ is a scheme over $\OKp$,
the Greenberg transform $\Gr_{en}(\TTp)$ is defined by base change along
$\OKp \to \OKp/\pKp^{en}$
followed by the Greenberg functor over the Artinian ring $\OKp/\pKp^{en}$.
Note also that
\begin{equation}\label{nem}
\OKp \otimes_{\OK} \OK/\pK^n \iso \OKp/\pKp^{en}.
\end{equation}
Now Lemma~\ref{lemma:wrK} follows from \cite{bertrapelle-gonzales:Greenberg}*{Prop. 6.3} after observing that  $\TTp \times_\Spec{\OKp} \Spec{\OKp/\pKp^{en}}$ is admissible relative to $\Spec{\OKp/\pKp^{en}} \to \Spec{\OK/\pK^n}$, cf. \cite{bertrapelle-gonzales:Greenberg}*{Rem 2.9 (a)}.
\end{proof}

\begin{proposition}\label{prop:wrK}
Let $K'/K$ be a finite Galois extension and
let $k'/k$ be the corresponding extension of residue fields.
Let $T$ be an algebraic torus over $K$ and
set $T' \ceq T \times_\Spec{K} \Spec{K'}$.
The canonical closed immersion $T \hookrightarrow \Res_{K'/K} T'$
induces a map of group pro-schemes $\GN{T} \to \Res_{k'/k} \GN{T}'$ which,
through the pullback functor and the trace of Frobenius, induces the homomorphism
\[
\begin{tikzcd}[column sep=60]
\Hom_{\text{qc}}(T(K'), \EEx) \arrow{r}{\chi \mapsto \chi\vert_{T(K)}} &\Hom_{\text{qc}}(T(K), \EEx).
\end{tikzcd}
\]
\end{proposition}

\begin{proof}
Adapting the argument in \cite{bosch-lutkebohmert-reynaud:NeronModels}*{\S 7.6, Prop 6} to locally finite type Néron models,
one may show that $\Res_{\OKp/\OK}(\TTp)$ is a Néron model for $\Res_{K'/K}(T')$.
By the Néron mapping property, the canonical closed immersion
$T\hookrightarrow \Res_{K'/K}(T')$ extends uniquely to a morphism
$\underline{T}\to \Res_{\OKp/\OK}(\TTp)$ of smooth $\OK$-group schemes.
Applying the Greenberg transform $\Gr_{n}$ to this morphism
and using Lemma~\ref{lemma:wrK} defines the morphism of $\Fq$-group schemes
\begin{equation}\label{men}
\GN{T}_n \to \Res_{k'/k} \GN{T}'_{en}.
\end{equation}
Passing to the pro-group schemes in Definition~\ref{def:objGN}
the family of morphisms \eqref{men} determines a morphism of pro-group schemes
\[
\GN{T} \to \Res_{k'/k} \GN{T}'.
\]
Since $\left(\Res_{k'/k} \GN{T}'_{en} \right)(\Fq) = \left(\GN{T}'_{en}\right)(k')$,
the pullback functor $\QC(\Res_{k'/k} \GN{T}'_{en})\to \QC(\GN{T}_n)$
obtained by applying \cite{cunningham-roe:dictionary}*{Prop. 4.1} to \eqref{men} gives
\[
\Hom_{< en}(T'(K'),\EEx) \to \Hom_{< n}(T(K),\EEx).
\]
Passing to inverse limits defines
\[
\Hom_{\text{qc}}(T(K'),\EEx) \to \Hom_{\text{qc}}(T(K),\EEx).
\]
Argue as in \cite{cunningham-roe:dictionary}*{Prop. 5.1} to see that this is indeed restriction of characters.
\end{proof}

If we specialize Proposition~\ref{prop:wrK} to the case when
$T = \Gm{K}$ then we see how to geometrize and categorify the restriction of quasicharacters from $(K')^\times$ to $K^\times$.


\section{Transfer of quasicharacter sheaves} \label{sec:transfer}

The techniques in this paper apply to all non-archimdean local fields $K$ with residue field $k$
and to all algebraic tori $T$ over $K$, without placing any restrictions on the characteristic of $K$
or on the ramification of $T$.
We may therefore ask for a comparison of
quasicharacter sheaves on $T$ and on $T'$ whenever there is a relationship
between characters of $T(K)$ and of $T'(K')$.
Such a relationship can exist between characters of $T(K)$ and $T'(K')$ even
when the characteristics of $K$ and $K'$ differ
\citelist{\cite{chai-yu:01a} \cite{deligne:84a}*{\S 1}}.

We recall the notion of $N$-congruent tori from \cite{chai-yu:01a}*{\S 2}.
Suppose $T$ and $T'$ are tori over non-archimedian local fields $K$ and $K'$,
splitting over $L$ and $L'$ respectively.
Then $T$ and $T'$ are said to be \emph{$N$-congruent} if there are isomorphisms
 \begin{align*}
  \alpha : \OL/\pi_K^N \OL &\to \OO{L'}/\pi_{K'}^N \OO{L'} \\
  \beta : \Gal(L/K) &\to \Gal(L'/K') \\
  \phi : X^*(T) &\to X^*(T')
 \end{align*}
 satisfying the conditions
 \begin{enumerate}
  \item $\alpha$ induces an isomorphism $\OK/\pi_K^N \OK \to \OO{K'}/\pi_{K'}^N \OO{K'}$,
  \item $\alpha$ is $\Gal(L/K)$-equivariant relative to $\beta$, and
  \item $\phi$ is $\Gal(L/K)$-equivariant relative to $\beta$.
 \end{enumerate}
If $T$ and $T'$ are $N$-congruent then $\alpha$, $\beta$ and $\phi$ determine an isomorphism
\begin{equation}\label{transfer}
  \Hom_{<N}(T(K), \EEx) \iso \Hom_{<N}(T'(K'),\EEx).
\end{equation}
Note that if $T$ and $T'$ are $N$-congruent, then they are $N'$ congruent for every
$N' \leq N$, the isomorphism \eqref{transfer} is depth-preserving at integers less than $N$.
In this section we show that the isomorphism \eqref{transfer} comes from an equivalence of
categories of quasicharacter sheaves on $\GN{T}_n$ and those on $\GN{T}'_n$.

One of the main results of \cite{chai-yu:01a} gives an isomorphism of group schemes between
$\TT_n$ and $\TT'_n$ assuming that $T$ and $T'$ are sufficiently congruent.
They define a quantity $h$ as the smallest integer so that $\pi^h$ lies in the
Jacobian ideal associated to a natural embedding of $T$ into an induced torus \cite{chai-yu:01a}*{\S 8.1}.
Then it follows from \cite{chai-yu:01a}*{Thm. 8.5} that if $N > 3h$ and $T$ and $T'$ are $N$-congruent
then there is a canonical isomorphism of smooth group schemes
 $
  \TT_{N-3h} \to \TT'_{N-3h}
 $
 determined by $\alpha, \beta$ and $\phi$.
Combining this with Theorem~\ref{thm:application} gives the geometrization and categorification of the transfer of characters.

\begin{corollary}\label{cor:transfer}
 With notation above, suppose that $T$ and $T'$ are $N$-congruent and $N > 3h$.  Set $n \ceq N-3h$.
 Then there is a canonical equivalence of categories
 \[
  \QC(\GN{T}'_{n}) \to \QC(\GN{T}_{n})
 \]
 determined by $\alpha, \beta$ and $\phi$ inducing the isomorphism
 \[
  \Hom_{<n}(T'(K'), \EEx) \to \Hom_{<n}(T(K), \EEx)
 \]
 through the trace of Frobenius.
\end{corollary}


\section{Toward geometric reciprocity for tori over local fields}
\label{sec:reciprocity}

We have given a functorial relationship between quasicharacter sheaves
on the Greenberg transform of the Néron model of $T$
and quasicharacters of $T(K)$.
When coupled with the reciprocity map $\text{rec}_T $ for $T$,
which enjoys the same functorial properties,
it follows that quasicharacter sheaves for $T$
functorially correspond to Langlands parameters for $T$.
We suspect that it is possible to extract Langlands parameters
from quasicharacter sheaves directly and that the reciprocity map
can then be viewed as a matching between quasicharacters and Langlands parameters coming
 from the same quasicharacter sheaf.
 \[
\begin{tikzcd}
\ & \arrow[swap]{dl}{\trFrob{G}} \QCiso{\GN{T}} \arrow[dashed]{dr} & \\
\Hom_{\text{qc}}(T(K),\EEx) \arrow{rr}{\text{rec}_T} && H^1(K,\hat{T}_\ell)
\end{tikzcd}
\]
 Results in this paper provide tools for reduction of the problem to the case $T=\Gm{K}$.
 In that case, our suspicions are all but confirmed
 by the class field theory of Serre-Hazewinkel
 \citelist{\cite{serre:60a} \cite{serre:61a} \cite{demazure-gabriel:GroupesAlgebriques}*{Appendice}},
 especially as revisited in \cite{suzuki-yoshida:12a}*{\S 2}.
 What remains to be done is a careful comparison of
 quasicharacter sheaves for $\Gm{K}$ and the local systems on
  the fpqc-site of the pro-quasi-algebraic varieties appearing in \cite{suzuki-yoshida:12a}.
In this regard, a comparison between the results in this paper and those in \cite{cunningham-kamgarpour:geometrization} might also be illuminating.

\bibliographystyle{amsalpha}
\bibliography{Biblio}

\end{document}
%sagemathcloud={"zoom_width":90}